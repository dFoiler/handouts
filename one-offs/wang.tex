\documentclass{article}
\usepackage[utf8]{inputenc}

\newcommand{\nirpdftitle}{KAKEYA IN \texorpdfstring{$\RR^3$}{R3}}
\usepackage{import}
\inputfrom{../../notes}{nir}
\usepackage[backend=biber,
    style=alphabetic,
    sorting=ynt
]{biblatex}
\setcounter{tocdepth}{2}

\pagestyle{contentpage}

\title{Kakeya Sets in \texorpdfstring{$\RR^3$}{R3}}
\author{Hong Wang}
\date{2 April 2025}
\usepackage{graphicx}

\begin{document}

\maketitle

\tableofcontents

\section{The Kakeya Conjecture}
This talk was given on April 2nd, given by Hong Wang, describing joint work with Zahl. We are interested in Kakeya sets.
\begin{definition}[Kakeya set]
	A \textit{Kakeya set} is a compact subset $K\subseteq\RR^n$ that contains a unit line segment in every direction.
\end{definition}
\begin{example}
	The unit disk in the plane is a Kakeya set.
\end{example}
A natural question is how small a Kakeya set can be.
\begin{theorem}[Besicovitch 1919] \label{thm:besico}
	For every $n\ge0$, there is a Kakeya set in $\RR^n$ of Lesbegus measure $0$.
\end{theorem}
\begin{proof}
	One can explicitly produce such a set via a tree-like construction.
\end{proof}
This allows us to make the following definition.
\begin{definition}[Besicovitch]
	A \textit{Besicovitch set} is a Kakeya set with Lesbesgue measure $0$.
\end{definition}
\begin{remark}[Fefferman 1970]
	Such sets were used to give a counterexample to the ball multiplier conjecture. This provides a connection
\end{remark}
\begin{remark}
	There are also connections between Kakeya sets and number theory and homogeneous dynamics.
\end{remark}
In spite of \Cref{thm:besico}, one still expects Kakeya sets to be fairly large, made rigorous by the following conjectuer
\begin{conj}[Kakeya]
	Any Kakeya set in $\RR^n$ has Hausdorff dimension $n$.
\end{conj}
The definition of Hausdorff dimension is somewhat technical, but it will not be totally necessary for this talk.

Let's give some historical remarks. For $n=2$, the result is known due to Davies. For $n=3$, Bourgain (1995) showed that Kakeya sets have Hausdorff dimension at least $2+\frac13$, and Wolff (1995) showed $2+\frac12$; both these results use combinatorial methods. A team of Katz--Laba--Tao (2000) showed dimension $2+\frac12+\varepsilon$ using heavy machinery. Earlier this year, the full result was proved.
\begin{remark}
	Let's describe why going beyond $2+\frac12$ is difficult. The reason, roughly speaking, is that there is a Heisenberg group $H\subseteq\CC^3$ which has dimension $2+\frac12$ but behaves in many ways like a Kakeya set.
\end{remark}
The team Katz--Laba--Tao showed that having counterexamples to the Kakeya conjecture of dimension $2+\frac12$ will have many remarkable properties.

\section{The Proof}
A few years ago, Wang and Zahl showed the following.
\begin{theorem}[Wang--Zahl 2022]
	A sticky Kakeya set in $\RR^3$ has Hausdorff dimension $3$.
\end{theorem}
Here, stickiness basically means that the unit line segments in the same direction are close together in some sense. Notably, this proof must rule out the Heisenberg group $H\subseteq\CC^3$, which is sticky. Let's describe how this is done. For this, we introduce projection theory, which is a branch of geometric measure theory which wants to understand how the dimension of a set behaves under orthogonal projection to hyperplanes. However, this machinery is only well-developed for sticky and almost Alfors regular sets, which are hypotheses that allow one to apply induction of scales.

In this situation, one can use Bourgain's sum product theorem, which is the analytic input able to distinguish between $\RR$ and $\CC$. Approximately speaking, Borugain's sum product theorem explains that reasonably large subsets $A\subseteq\RR$ have $A+A$ or $A\cdot A$ much larger than $A$ itself.

We now turn to general Kakeya sets. Approximately speaking, our chief difficulty will be figuring out how to apply an induction of scales. One can translate our game into the following: for given $\delta>0$ small, let $\mathbb T$ be a collection of $\delta$-tubes in $\RR^3$, and then we would like to control the measure $\left|\bigcup_{T\in\mathbb T}T\right|$. In particular, the following is enough to prove the Kakeya conjecture.
\begin{conj}[Kakeya, discretized] \label{conj:kakeya-conj-discrete}
	For given $\delta>0$, let $\mathbb T$ be a set of $\delta$-tubes containing a $\delta$-tube in every $\delta$-separated direction. Then
	\[\left|\bigcup_{T\in\mathbb T}T\right|\approx1.\]
\end{conj}
This conjecture is hard because it does not permit induction of scales: namely, even if we know the conjecture for some medium $\delta$, we cannot achieve the result for smaller $\delta$ because $\mathbb T$ at small scales might have the $\delta$-tubes in similar directions too far apart to be combined into single $\delta$-tubes.

Thus, we are interested in replacing our hypothesis on having $\delta$-tubes in enough directions with a better list of hypotheses implied by this. We will work with the Wolff axioms.
\begin{definition}
	For $\delta>0$ small, let $\mathbb T$ be a set of $\delta$-tubes.
	\begin{itemize}
		\item $\mathbb T$ satisfies the \textit{Katz--Tao Wolff axiom} if and only if all convex sets $U$ with
		\[\mathbb T[U]\coloneqq\{T\in\mathbb T:T\subseteq U\}\]
		has $\#\mathbb T[U]\ll\left|U\right|/\left|T\right|$.
		\item $\mathbb T$ satisfies the \textit{Frostman Wolff axiom} if and only if all convex sets $U$ have $\#\mathbb T[U]\ll\left|U\right|\cdot\#\mathbb T$.
	\end{itemize}
\end{definition}
\begin{remark} \label{rem:kakeya-to-wolff}
	If $\mathbb T$ contains a $\delta$-tube in every $\delta$-separated direction, then one can show that $\mathbb T$ satisfies Katz--Tao Wolff and Frostman Wolff axioms. Note that this holds only in $\RR^3$. In higher dimensions such as $\RR^4$, one must work with a more complicated set of axioms; for example, the hypersurface cut out by $xy-uw=1$ in $\RR^4$ fails to satisfy the above axioms despite being Kakeya.
\end{remark}
These hypotheses now permit induction on scales, so we are ready to state the main result.
\begin{theorem}[Wong--Zahl 2025] \label{thm:get-kakeya}
	Let $\mathbb T$ be a collection of $\delta$-tubes.
	\begin{listalph}
		\item If $\mathbb T$ satisfies the Katz--Tao Wolff axiom, then there are not too many tubes, and they are roughly disjoint; i.e., $\#\mathbb T\ll\delta^{-2}$ and $\left|\bigcup_{T\in\mathbb T}T\right|\approx\left|T\right|\cdot\#\mathbb T$.
		\item If $\mathbb T$ satisfies the Frostman Wolff axiom, then there are many tubes, and they are not too close together; i.e., $\#\mathbb T\gg\delta^{-2}$ and $\left|\bigcup_{T\in\mathbb T}T\right|\approx1$.
	\end{listalph}
\end{theorem}
Via \Cref{rem:kakeya-to-wolff}, this implies \Cref{conj:kakeya-conj-discrete}.

Let's say something about the proof \Cref{thm:get-kakeya}. We may assume that $\mathbb T$ is not sticky, which means that there is a small $\rho>\delta$ such that any collection $\mathbb T_\rho$ of distinct $\rho$-tubes which contains $\mathbb T$ but has many tubes $\#\mathbb T_\rho\gg\rho^{-2}$. To prove our result, it turns out that we would like to bound the multiplicity of $\mathbb T$.
\begin{definition}[multiplicity]
	Let $\mathbb T$ be a collection of $\delta$-tubes. Then the \textit{multiplicity} $\mu(\mathbb T)$ is the number of tubes $T\in\mathbb T$ which pass through a typical point in $\bigcup_{T\in\mathbb T}T$.
\end{definition}
Now, for any $\mathbb T$ satisfying the Katz--Tao Wolff axiom, we find $\mu(\mathbb T)\ll\#\mathbb T^\sigma$ for some explicit $\sigma>0$. In the non-sticky case, we expect to be able to achieve an improvement $\mu(\mathbb T)\ll\#T^{\sigma-\varepsilon}$ for some positive $\varepsilon$. The rough idea is that the non-sticky condition means that working at a larger scale $\rho$ forces us to introduce more tubes than expected.

In a few more words, one picks up points with many $\delta$-tubes intersecting through them, which we label as grains $G$. Then one is able to produce a bound of the form
\[\mu(\mathbb T)\approx\mu(\mathbb T[T_\rho])\cdot\mu(G).\]
Note that experts thus far have considered bounds of the type $\mu(\mathbb T)\ll\mu(\mathbb T[T_\rho])\cdot\mu(\mathbb T_\rho)$, which holds by some combinatorial scaling argument. Working with grains is able to produce a stronger bound, but we are now left with the task of finding structure in these grains. Here is the key input.
\begin{proposition}
	Let $\mathbb T$ be a set of $\delta$-tubes. Then there is a subset $\mathbb T'\subseteq\mathbb T$ and a set $\mc W$ of $A\times B\times1$ boxes with $\delta\le A\le B\le1$ satisfying the following.
	\begin{listroman}
		\item $\mc W$ satisfies the Katz--Tao Wolff axiom.
		\item For all $W\in\mathbb W$, the set $\mathbb T'[W]$ satisfies a version of the Frostman Wolff axiom. 
	\end{listroman}
\end{proposition}
For example, if $G$ is Katz--Tao, then one proceeds by writing
\[\mu(\mathbb T)\approx\mu(\mathbb T[T_\rho])\cdot\mu(G)\ll(\rho-2)\sigma\ll\#\mathbb T_\rho^\sigma,\]
which produces the required improvement.

\end{document}
\documentclass{article}
\usepackage[utf8]{inputenc}

\newcommand{\nirpdftitle}{Student Arithmetic Geometry Seminar}
\usepackage{import}
\inputfrom{../../notes}{nir}
\usepackage[backend=biber,
    style=alphabetic,
    sorting=ynt
]{biblatex}
\setcounter{tocdepth}{2}

\pagestyle{contentpage}

\title{Student Arithmetic Geometry Seminar}
\author{Nir Elber}
\date{Spring 2024}
\usepackage{graphicx}

\begin{document}

\maketitle

\tableofcontents

\section{January 19: Martin Olsson}
Today, we will provide an introduction to what this seminar will be about.

\subsection{Motivating Paper}
We are motivated by Deligne--Mumford's paper ``The irreducibility of moduli of curves of given genus.'' The main theorem is as follows.
\begin{theorem}[Deligne--Mumford] \label{thm:dm}
	Fix a nonnegative integer $g\ge2$ and a field $k$. Then the moduli space $\mc M_g$ of curves of genus $g$ over $k$ is irreducible.
\end{theorem}
There is quite a bit which goes into this proof.
\begin{itemize}
	\item Curve theory.
	\item The formalism of moduli problems.
	\item The Semistable reduction theorem.
	\item Jacobian varieties.
	\item The notion of stable curves and the compactification $\ov{\mc M}_g$.
	\item Deformation theory.
	\item The theory of stacks and coarse moduli spaces.
	\item An analytic description to get from $\CC$.
\end{itemize}
From these ingredients, there are other directions to go in.
\begin{itemize}
	\item Finer geometry of $\mc M_g$.
	\item Moduli of curves with marked points $\mc M_{g,n}$.
	\item Higher-dimensional directions, such as abelian varieties.
\end{itemize}
Let's go ahead and provide a sketch.
\begin{proof}[Sketch of \Cref{thm:dm}]
	The result is known over $\CC$ via some analytic methods. We want to get to other fields. We begin with the following exercise.
	\begin{proposition} \label{prop:reduce-conn}
		Fix a complete discrete valuation ring $(R,\mf m)$ with fraction field $K$ and residue field $\kappa\coloneqq R/\mf m$. Fix a smooth proper morphism $f\colon X\to\Spec R$. If $X_{\ov K}$ is connected, then $X_{\ov\kappa}$ is also connected.
	\end{proposition}
	\begin{proof}[Sketch]
		One can assume that $\kappa$ is algebraically closed by some argument. Then the Formal functions theorem tells us that
		\[H^0(X,\OO_X)=\limit_nH^0(X_n,\OO_{X_n}),\]
		where $X_n\coloneqq X\times_V\Spec V/\mf m^n$. Now, suppose for the sake of contradiction that $X_\kappa=X^1_\kappa\sqcup X^2_\kappa$. Then by taking nilpotent thickenings, we see that $X_n=X^1_n\sqcup X^2_n$, so
		\[\limit_nH^0(X_n,\OO_{X_n})=\limit_nH^0(X_n^1,\OO_{X_n^1})\times\limit_nH^0(X_n^2,\OO_{X_n^2}).\]
		So we are receiving a product of rings $A_1\times A_2$, which are flat over $R$, so by viewing the global sections back in $H^0(X,\OO_X)$, which should be $\ov K$ upon algebraic closure, we will receive our contradiction.
	\end{proof}
	Now, the (coarse) moduli space $\mc M_g$ fails to be either smooth or proper, but some theory of stacks allows us to reduce to this case. Namely, the point is to find some compactification $\mc M_g\into\ov{\mc M}_g$ where $\ov{\mc M}_g$ is a smooth proper $\ZZ$-stack with $\mc M_g\subseteq\ov{\mc M}_g$ dense in each fiber. As such, we can imagine using \Cref{prop:reduce-conn} to pull back the result from $\ov{\mc M}_g$ to $\mc M_g$.
\end{proof}

\subsection{Thinking about the Moduli Spaces}
Let's describe what $\mc M_g$ is. By the functor of points description, we merely need to describe maps $S\to\mc M_g$, which we declare to be in natural bijection with genus-$g$ curves $\pi\colon C\to S$; i.e., $\pi$ is a proper, flat, and smooth morphism whose geometric fibers are genus-$g$ curves.
\begin{example}
	There is a family of curves over a field $k$ given by the equations
	\[y^2=(x-a_1)\cdots(x-a_n),\]
	where $a_1,\ldots,a_n$ are allowed to vary. Viewing the $a_\bullet$s as giving a point in affine space, we see that we are (approximately speaking) producing a rational map $\AA^n\to\mc M_g$, where perhaps we need to check that we have a curve of the correct genus.
\end{example}
Now, using a functor of points description, smoothness of $\mc M_g$ over $\Spec\ZZ$ is requiring the following (in the sense of formal smoothness): for any surjection $A'\onto A$ with kernel $J$ such that $J^2=0$, any morphism $\Spec A\to\mc M_g$ induces a unique dashed arrow.
% https://q.uiver.app/#q=WzAsNCxbMCwwLCJcXFNwZWMgQSJdLFswLDEsIlxcU3BlYyBBJyJdLFsxLDAsIlxcbWMgTV9nIl0sWzEsMSwiXFxTcGVjXFxaWiJdLFsxLDNdLFswLDFdLFswLDJdLFsyLDNdLFsxLDIsIiIsMSx7InN0eWxlIjp7ImJvZHkiOnsibmFtZSI6ImRhc2hlZCJ9fX1dXQ==&macro_url=https%3A%2F%2Fraw.githubusercontent.com%2FdFoiler%2Fnotes%2Fmaster%2Fnir.tex
\[\begin{tikzcd}
	{\Spec A} & {\mc M_g} \\
	{\Spec A'} & \Spec\ZZ
	\arrow[from=2-1, to=2-2]
	\arrow[from=1-1, to=2-1]
	\arrow[from=1-1, to=1-2]
	\arrow[from=1-2, to=2-2]
	\arrow[dashed, from=2-1, to=1-2]
\end{tikzcd}\]
As such, we can unwind the functor of points description of $\mc M_g$ to prove something like smoothness, which is quite remarkable.

Analogously, we can describe $\ov{\mc M}_g$ via the functor of points: maps $S\to\ov{\mc M}_g$ are ``stable curves'' $\pi\colon C\to S$, which will be a proper flat morphism whose geometric fibers by $\ov s\to S$ satisfy the following.
\begin{itemize}
	\item The geometric fibers are nodal curves, meaning that the completion at any closed point is $\kappa(\ov s)\bb x$ or $\kappa(\ov s)\bb{x,y}/(xy)$.
	\item Every rational component (namely, irreducible component whose normalization is $\PP^1$) has three distinguished points. (These three points are desirable, for example, to ensure that its automorphism group is trivial.)
\end{itemize}
Now, we can also check being proper via the functor of points description, using the valuative criterion. Namely, for a complete discrete valuation ring $R$ with fraction field $K$, we would like to know that any map $\Spec K\to\ov{\mc M}_g$ induces a unique dashed arrow.
% https://q.uiver.app/#q=WzAsNCxbMCwxLCJcXFNwZWMgUiJdLFsxLDEsIlxcU3BlY1xcWloiXSxbMSwwLCJcXG92e1xcbWMgTX1fZyJdLFswLDAsIlxcU3BlYyAgSyJdLFswLDFdLFszLDBdLFszLDJdLFsyLDFdLFswLDIsIiEiLDEseyJzdHlsZSI6eyJib2R5Ijp7Im5hbWUiOiJkYXNoZWQifX19XV0=&macro_url=https%3A%2F%2Fraw.githubusercontent.com%2FdFoiler%2Fnotes%2Fmaster%2Fnir.tex
\[\begin{tikzcd}
	{\Spec  K} & {\ov{\mc M}_g} \\
	{\Spec R} & \Spec\ZZ
	\arrow[from=2-1, to=2-2]
	\arrow[from=1-1, to=2-1]
	\arrow[from=1-1, to=1-2]
	\arrow[from=1-2, to=2-2]
	\arrow["{!}"{description}, dashed, from=2-1, to=1-2]
\end{tikzcd}\]
And again, we can directly turn this into a geometry problem of curves by tracking through the moduli interpretation, which is the Semistable reduction theorem.

\section{January 26th: Martin Olsson}

Today we discuss the setup for our moduli problems.

\subsection{The Yoneda Lemma}
Here is the statement.
\begin{theorem}[Yoneda] \label{thm:yo}
	Fix a category $\mc C$, and let $\mathrm{PSh}(\mc C)$ denote the category of presheaves on $\mc C$.
	\begin{listalph}
		\item For $X\in\mc C$, the functor $h_X\colon A\mapsto\op{Mor}(A,X)$ is a presheaf $h_X\colon\mc C\opp\to\mathrm{Set}$.
		\item There is a natural bijection
		\[\op{Mor}_{\mathrm{PSh}(\mc C)}(h_X,F)\to FX.\]
		\item The construction $h_\bullet$ forms a fully faithful embedding $h_\bullet\colon\mc C\to\mathrm{PSh}(\mc C)$.
	\end{listalph}
\end{theorem}
\begin{proof}
	We won't bother to show (a). For (b), the forward map sends $\eta\colon h_X\Rightarrow F$ to $\eta_X({\id_X})$, and one can check that this is a bijection. To show (c), we note that we need to show
	\[\op{Mor}_\mc C(X,Y)\simeq\op{Mor}_{\mathrm{Pre}(\mc C)}(h_X,h_Y),\]
	but this simply follows by taking $F=h_Y$ in (b).
\end{proof}
\begin{remark}
	Most of the time, we will take $\mc C=\mathrm{Sch}(S)$ for a fixed base scheme $S$.
\end{remark}
Anyway, we can now make the following definition.
\begin{definition}[representable]
	Fix a category $\mc C$. A functor $F\colon\mc C\opp\to\mathrm{Set}$ is \textit{representable} if and only if $F\simeq h_X$ for some object $X\in\mc C$. In the sequel, we may want to fix the isomorphism $F\simeq h_X$, which can be specified by an element $\xi\in FX$.
\end{definition}
Here are some examples.
\begin{example}
	Take $\mc C\coloneqq\mathrm{Sch}$, and consider the functor $F\colon\mc C\opp\to\mathrm{Set}$ defined by $FY\coloneqq\Gamma(Y,\OO_Y)^n$. We claim $F$ is represented by $\AA^n_\ZZ=\Spec\ZZ[x_1,\ldots,x_n]$. Indeed, we see
	\[\op{Mor}_{\mathrm{Sch}}(Y,\AA^n_\ZZ)\simeq\op{Hom}_{\mathrm{Ring}}(\ZZ[x_1,\ldots,x_n],\OO_Y(Y))\simeq\OO_Y(Y)^n,\]
	as desired. To specify this isomorphism, \Cref{thm:yo} tells us that it is enough to track through the identity map in $\op{Mor}(\AA^n_\ZZ,\AA^n_\ZZ)\simeq\Gamma(\AA^n_\ZZ,\OO_{\AA^n_\ZZ})^n$, which we can track through is $(x_1,\ldots,x_n)$. Of course, we could choose other isomorphisms, such as determined by the element $(x_n,\ldots,x_1)\in\Gamma(\AA^n_\ZZ,\OO_{\AA^n_\ZZ})^n$.
\end{example}
\begin{example}
	Take $\mc C\coloneqq\mathrm{Sch}$, and consider the functor $F\colon\mc C\opp\to\mathrm{Set}$ defined by
	\[FY\coloneqq\left\{(f_\bullet)\in\Gamma(Y,\OO_Y)^n:(f_1,\ldots,f_n)=\Gamma(Y,\OO_Y)\right\}.\]
	Then $F$ is represented by $\AA^n_\ZZ\setminus\{0\}$.
\end{example}
\begin{example} \label{ex:moduli-pn}
	Take $\mc C\coloneqq\mathrm{Sch}$, and consider the functor $F\colon\mc C\opp\to\mathrm{Set}$ defined by setting $FY$ to the collection of surjections $\OO_Y^{\oplus n}\onto\mc L$ up to isomorphism, where $\mc L/Y$ is a line bundle. Then $F$ is represented by $\PP^n_\ZZ$. The representing element in $F\left(\PP^n_\ZZ\right)$ is given by the surjection $(x_0,\ldots,x_n)\colon\OO_{\PP^n_\ZZ}\onto\OO_{\PP^n_\ZZ}(1)$.
\end{example}
\begin{remark}
	By identifying a line with the corresponding quotient space, we see that the above interpretation of $\PP^n_k$ agrees with the interpretation as ``lines in $k^{n+1}$.''
\end{remark}
\begin{example}
	Fix a field $k$ and homogeneous polynomial $f\in k[x_0,\ldots,x_n]$ of degree $N$, and consider $V(f)\subseteq\PP^n_k$. A map $Y\to V(f)$ will certainly produce a map $Y\to\PP^n_\ZZ$, so our answer should be a subfunctor of \Cref{ex:moduli-pn}. But then we want to determine which surjections $(s_0,\ldots,s_n)\colon\OO_Y^{\oplus n}\onto\mc L$ (which really consists of $n+1$ global sections of $\mc L$ globally generating), but then to live in $V(f)$, we request that $f(s_0,\ldots,s_n)=0$.
\end{example}
\begin{remark}
	Note $\AA^n_\ZZ\subseteq\PP^n_\ZZ$ (e.g. going to the standard affine open $x_0\ne0$). So we expect to have an inclusion $h_{\AA^n_\ZZ}\subseteq h_{\PP^n_\ZZ}$, and one can track through that it simply corresponds to the provided map $(s_0,\ldots,s_n)\colon\OO_Y^{\oplus(n+1)}\onto\mc L$ having $s_0$ be an isomorphism. In this case, after identifying $\OO_Y$ with $\mc L$ via $s_0$, we see that the other sections are indeed providing an element of $\Gamma(Y,\OO_Y)^n$.
\end{remark}
\begin{example}[Hilbert scheme]
	Fix a flat separated $X$-scheme $S$. Then the Hilbert functor assigns a $Y$-scheme $S$ to flat proper (locally of finite presentation) subschemes $Z\subseteq X\times_SY$. It turns out that this functor is representable if $X$ is quasi-projective, which is a result due to Grothendieck.
\end{example}

\subsection{Functors Not Representable}
Representable functors have the tendency to be sheaves. For the Zariski topology, here is the relevant definition.
\begin{definition}[Zariski sheaf]
	A functor $F\colon\mc C\opp\to\mathrm{Set}$ is a \textit{Zariski sheaf} if and only if we have the usual equalizer diagram
	\[FY\to\prod_{\alpha\in\kappa}FY_\alpha\twoheadleftarrow\prod_{\alpha,\beta\in\kappa}F(Y_\alpha\cap Y_\beta),\]
	where $\{Y_\alpha\}_{\alpha\in\kappa}$ is a Zariski open cover of $Y$.
\end{definition}
\begin{example}
	If $F\simeq h_X$ is representable, then $F$ is a Zariski sheaf because morphisms glue.
\end{example}
\begin{nex}
	Define the functor $F$ as taking a scheme $Y$ and taking the quotient of the set of $(n+1)$-tuples $(s_0,\ldots,s_n)\in\Gamma(Y,\OO_Y)$ by multiplication by $\Gamma(Y,\OO_Y^\times)$. This is not a sheaf, but if we do sheafification, we will recover \Cref{ex:moduli-pn}. Thus, this functor is not representable.
\end{nex}
As another remark, we have the following.
\begin{lemma} \label{lem:field-extension-rep}
	If $X$ is a scheme, and $L/K$ is a field extension, then the map $X(\Spec K)\to X(\Spec L)$ is injective.
\end{lemma}
\begin{proof}
	Indeed, a map $\Spec K\to X$ is simply a point $x\in X$ together with an inclusion $\kappa(x)\to K$, which by a similar description for $\Spec L\to X$ is uniquely determined by that map $\Spec L\to X$.
\end{proof}
\begin{example}
	Define the functor $F$ by taking a scheme $Y$ and returning elliptic curves over $Y$ (up to isomorphism). But there are distinct elliptic curves over $\QQ$ which become isomorphic over $\ov\QQ$. For example, $y^2=x^3+D$ and $y^2=x^3-D$ is one such example, which become isomorphic over $\QQ(i)$ where we can send $(x,y)\mapsto(-x,iy)$. Thus, $F$ is not representable by \Cref{lem:field-extension-rep}.
\end{example}
\begin{remark}
	One can argue similarly for curves of genus $g\ge2$.
\end{remark}

\section{February 16th: Martin Olsson}
I missed one week, and then one week was cancelled. I am a little too tired to take detailed notes.
\begin{remark}
	Recall that the Yoneda lemma asserts that representing a functor $F\colon\mathrm{Sch}_S\opp\to\mathrm{Set}$ amounts to specifying an object $X$ in addition to an element $\xi\in FX$ so that the induced map $h_X\Rightarrow F$ (given by $\xi$) is representable.
\end{remark}
\begin{example}
	The Hilbert functor $\op{Hilb}_{\PP^n}$ is represented by some scheme $H$. In particular, we are provided an isomorphism $\sigma\colon h_H\Rightarrow\op{Hilb}_{\PP^n}$, so passing $\id_H$ through implies we are being given a flat proper subscheme $\mc Z\subseteq\PP^n_H$ so that $\op{Hilb}_H(S)$ is merely given by $h_H(S)$, which consists of pullbacks of $\mc Z$ along the morphisms $f\colon S\to H$. The point is that we can really see the element of $\op{Hilb}_H(S)$ provided to us by some $f\in h_H(S)$ as $f^*\mc Z$.
\end{example}
\begin{example}
	The functor $\mc M_{0,n}$ takes a scheme $S$ and then asks for curves $f\colon C\to S$ (i.e., $f$ is proper and flat where all fibers are isomorphic to $\PP^1$), and we mark $n$ points (in fact, $S$-points) somewhere on $C$. The issue here is that we are considering these up to isomorphism, but it turns out that an automorphism of $C$ is basically determined by where it sends three points. When $S=\Spec k$, this is classical; in general, one needs to make some more global argument. So, for example, one finds that $\mc M_{0,3}$ is $\Spec\ZZ$ (where the universal curve is $\PP^1_\ZZ$ with the marked points $\{0,1,\infty\}$).
\end{example}
\begin{example}
	Continuing from the above example, we see $\mc M_{0,4}$ is $\PP^1_\ZZ\setminus\{0,1,\infty\}$ basically by trying to mark a fourth point of $\PP^1_\ZZ$ (outside $\{0,1,\infty\}$). Here are universal curve is given by
	\[\PP^1\times\left(\PP^1_\ZZ\setminus\{0,1,\infty\}\right).\]
	Then our first three marked points are $\{0\}\times\mc M_{0,4}$ and $\{1\}\times\mc M_{0,4}$ and $\{\infty\}\times\mc M_{0,4}$, and the last marked point is given by what we chose in $\mc M_{0,4}$ to begin with, so it is given by the diagonal embedding.
\end{example}
\begin{remark}
	One takes the compactification $\ov{\mc M}_{0,n}$ of $\mc M_{0,n}$ by taking nodal curves (without automorphisms) instead of just curves. Using this one can realize $\ov{\mc M}_{0,4}$ by $\PP^1$, and we can track through what the points $\{0,1,\infty\}$ mean. For example, our universal curve is $\PP^1\times\PP^1$ blown up at three points in $\{(0,0),(1,1),(\infty,\infty)\}$.
\end{remark}
\begin{remark}
	Continuing, it turns out that $\ov{\mc M}_{0,5}$ ought to be the universal curve over $\ov{\mc M}_{0,4}$. The point is that choosing $5$ points amounts to choosing $4$ points first (which is $\mc M_{0,4}$) and then the last point goes up a dimension. This continues inductively.
\end{remark}

\end{document}
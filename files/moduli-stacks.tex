\documentclass{article}
\usepackage[utf8]{inputenc}

\newcommand{\nirpdftitle}{Student Arithmetic Geometry Seminar}
\usepackage{import}
\inputfrom{../../notes}{nir}
\usepackage[backend=biber,
    style=alphabetic,
    sorting=ynt
]{biblatex}
\setcounter{tocdepth}{2}

\pagestyle{contentpage}

\title{Student Arithmetic Geometry Seminar}
\author{Nir Elber}
\date{Spring 2024}
\usepackage{graphicx}

\begin{document}

\maketitle

\tableofcontents

\section{January 19: Martin Olsson}
Today, we will provide an introduction to what this seminar will be about.

\subsection{Motivating Paper}
We are motivated by Deligne--Mumford's paper ``The irreducibility of moduli of curves of given genus.'' The main theorem is as follows.
\begin{theorem}[Deligne--Mumford] \label{thm:dm}
	Fix a nonnegative integer $g\ge2$ and a field $k$. Then the moduli space $\mc M_g$ of curves of genus $g$ over $k$ is irreducible.
\end{theorem}
There is quite a bit which goes into this proof.
\begin{itemize}
	\item Curve theory.
	\item The formalism of moduli problems.
	\item The Semistable reduction theorem.
	\item Jacobian varieties.
	\item The notion of stable curves and the compactification $\ov{\mc M}_g$.
	\item Deformation theory.
	\item The theory of stacks and coarse moduli spaces.
	\item An analytic description to get from $\CC$.
\end{itemize}
From these ingredients, there are other directions to go in.
\begin{itemize}
	\item Finer geometry of $\mc M_g$.
	\item Moduli of curves with marked points $\mc M_{g,n}$.
	\item Higher-dimensional directions, such as abelian varieties.
\end{itemize}
Let's go ahead and provide a sketch.
\begin{proof}[Sketch of \Cref{thm:dm}]
	The result is known over $\CC$ via some analytic methods. We want to get to other fields. We begin with the following exercise.
	\begin{proposition} \label{prop:reduce-conn}
		Fix a complete discrete valuation ring $(R,\mf m)$ with fraction field $K$ and residue field $\kappa\coloneqq R/\mf m$. Fix a smooth proper morphism $f\colon X\to\Spec R$. If $X_{\ov K}$ is connected, then $X_{\ov\kappa}$ is also connected.
	\end{proposition}
	\begin{proof}[Sketch]
		One can assume that $\kappa$ is algebraically closed by some argument. Then the Formal functions theorem tells us that
		\[H^0(X,\OO_X)=\limit_nH^0(X_n,\OO_{X_n}),\]
		where $X_n\coloneqq X\times_V\Spec V/\mf m^n$. Now, suppose for the sake of contradiction that $X_\kappa=X^1_\kappa\sqcup X^2_\kappa$. Then by taking nilpotent thickenings, we see that $X_n=X^1_n\sqcup X^2_n$, so
		\[\limit_nH^0(X_n,\OO_{X_n})=\limit_nH^0(X_n^1,\OO_{X_n^1})\times\limit_nH^0(X_n^2,\OO_{X_n^2}).\]
		So we are receiving a product of rings $A_1\times A_2$, which are flat over $R$, so by viewing the global sections back in $H^0(X,\OO_X)$, which should be $\ov K$ upon algebraic closure, we will receive our contradiction.
	\end{proof}
	Now, the (coarse) moduli space $\mc M_g$ fails to be either smooth or proper, but some theory of stacks allows us to reduce to this case. Namely, the point is to find some compactification $\mc M_g\into\ov{\mc M}_g$ where $\ov{\mc M}_g$ is a smooth proper $\ZZ$-stack with $\mc M_g\subseteq\ov{\mc M}_g$ dense in each fiber. As such, we can imagine using \Cref{prop:reduce-conn} to pull back the result from $\ov{\mc M}_g$ to $\mc M_g$.
\end{proof}

\subsection{Thinking about the Moduli Spaces}
Let's describe what $\mc M_g$ is. By the functor of points description, we merely need to describe maps $S\to\mc M_g$, which we declare to be in natural bijection with genus-$g$ curves $\pi\colon C\to S$; i.e., $\pi$ is a proper, flat, and smooth morphism whose geometric fibers are genus-$g$ curves.
\begin{example}
	There is a family of curves over a field $k$ given by the equations
	\[y^2=(x-a_1)\cdots(x-a_n),\]
	where $a_1,\ldots,a_n$ are allowed to vary. Viewing the $a_\bullet$s as giving a point in affine space, we see that we are (approximately speaking) producing a rational map $\AA^n\to\mc M_g$, where perhaps we need to check that we have a curve of the correct genus.
\end{example}
Now, using a functor of points description, smoothness of $\mc M_g$ over $\Spec\ZZ$ is requiring the following (in the sense of formal smoothness): for any surjection $A'\onto A$ with kernel $J$ such that $J^2=0$, any morphism $\Spec A\to\mc M_g$ induces a unique dashed arrow.
% https://q.uiver.app/#q=WzAsNCxbMCwwLCJcXFNwZWMgQSJdLFswLDEsIlxcU3BlYyBBJyJdLFsxLDAsIlxcbWMgTV9nIl0sWzEsMSwiXFxTcGVjXFxaWiJdLFsxLDNdLFswLDFdLFswLDJdLFsyLDNdLFsxLDIsIiIsMSx7InN0eWxlIjp7ImJvZHkiOnsibmFtZSI6ImRhc2hlZCJ9fX1dXQ==&macro_url=https%3A%2F%2Fraw.githubusercontent.com%2FdFoiler%2Fnotes%2Fmaster%2Fnir.tex
\[\begin{tikzcd}
	{\Spec A} & {\mc M_g} \\
	{\Spec A'} & \Spec\ZZ
	\arrow[from=2-1, to=2-2]
	\arrow[from=1-1, to=2-1]
	\arrow[from=1-1, to=1-2]
	\arrow[from=1-2, to=2-2]
	\arrow[dashed, from=2-1, to=1-2]
\end{tikzcd}\]
As such, we can unwind the functor of points description of $\mc M_g$ to prove something like smoothness, which is quite remarkable.

Analogously, we can describe $\ov{\mc M}_g$ via the functor of points: maps $S\to\ov{\mc M}_g$ are ``stable curves'' $\pi\colon C\to S$, which will be a proper flat morphism whose geometric fibers by $\ov s\to S$ satisfy the following.
\begin{itemize}
	\item The geometric fibers are nodal curves, meaning that the completion at any closed point is $\kappa(\ov s)\bb x$ or $\kappa(\ov s)\bb{x,y}/(xy)$.
	\item Every rational component (namely, irreducible component whose normalization is $\PP^1$) has three distinguished points. (These three points are desirable, for example, to ensure that its automorphism group is trivial.)
\end{itemize}
Now, we can also check being proper via the functor of points description, using the valuative criterion. Namely, for a complete discrete valuation ring $R$ with fraction field $K$, we would like to know that any map $\Spec K\to\ov{\mc M}_g$ induces a unique dashed arrow.
% https://q.uiver.app/#q=WzAsNCxbMCwxLCJcXFNwZWMgUiJdLFsxLDEsIlxcU3BlY1xcWloiXSxbMSwwLCJcXG92e1xcbWMgTX1fZyJdLFswLDAsIlxcU3BlYyAgSyJdLFswLDFdLFszLDBdLFszLDJdLFsyLDFdLFswLDIsIiEiLDEseyJzdHlsZSI6eyJib2R5Ijp7Im5hbWUiOiJkYXNoZWQifX19XV0=&macro_url=https%3A%2F%2Fraw.githubusercontent.com%2FdFoiler%2Fnotes%2Fmaster%2Fnir.tex
\[\begin{tikzcd}
	{\Spec  K} & {\ov{\mc M}_g} \\
	{\Spec R} & \Spec\ZZ
	\arrow[from=2-1, to=2-2]
	\arrow[from=1-1, to=2-1]
	\arrow[from=1-1, to=1-2]
	\arrow[from=1-2, to=2-2]
	\arrow["{!}"{description}, dashed, from=2-1, to=1-2]
\end{tikzcd}\]
And again, we can directly turn this into a geometry problem of curves by tracking through the moduli interpretation, which is the Semistable reduction theorem.

\end{document}
\documentclass{article}
\usepackage[utf8]{inputenc}
\usepackage{amsfonts,amsmath}

\usepackage{comment}
\usepackage[margin=0.75in]{geometry}
\addtolength{\topmargin}{-0.8in}
\addtolength{\textheight}{1.5in}
\usepackage{asymptote}
\usepackage{hyperref}    % mailto
\usepackage{anyfontsize} % For the section size
\usepackage{multicol}
\usepackage{worldflags}

\usepackage{marvosym}    % \Telefon \Letter
\usepackage{fontawesome} % \faMapMarker \faGithub

% I heavily abuse document behavior and commands for my bidding.
\usepackage{xcolor}
\setcounter{secnumdepth}{0}
\setlength{\parindent}{0em}
\setlength{\parskip}{0.8em}
\renewcommand{\familydefault}{\sfdefault}
\pagenumbering{gobble}

\definecolor{headercolor}{HTML}{3e80b0}
\renewcommand{\section}[1]{{\color{headercolor}\LARGE
    \textbf{#1\phantom{p}}\hrule}}
    % Insert the phantom p to make the bottom hrule just below the text.

\newcommand{\entry}[5]{{\large\textbf{#1}}\hfill\textit{#3}\\
    #2\hfill\\
    \textcolor{gray}{#5}}

\newcommand{\award}[3]{{\large\textbf{#1}}\hfill\textit{#2}\\
    \textcolor{gray}{#3}}

\begin{document}

\hspace{-8.15em}
\begin{asy}
    size(24cm);
    real h = 3.5;
    fill((0,0)--(0,h)--(21.6,h)--(21.6,0)--cycle, hsv(205,0.1,1));
\end{asy}

\vspace{-2.4cm}


\begin{center}
    \textcolor{headercolor}{\Huge\textbf{Nir Elber}}
    
    \textcolor{gray}{
        % \textbf{\scalebox{0.2}{\worldflag{US}} US Citizen} \\[0.1em]
        \faMapMarker~\,Berkeley, CA\hspace{1.3em}$\bullet$\hspace{1em}
        {\fontsize{14}{1}\selectfont\Telefon}~\,(512) 634-7121\hspace{1.3em}$\bullet$\hspace{1em}
        {\fontsize{13}{1}\Letter}~\,\href{mailto:nire@berkeley.edu}{nire@berkeley.edu}
        \hspace{1.1em}$\bullet$\hspace{1.4em}{\fontsize{11}{1}\faGithub}~\,\href{https://github.com/dFoiler}{dFoiler}
    }
    
    \vspace{-0.5em}
\end{center}

\section{Education}

\award{University of California, Berkeley}{Aug. 2021--May 2025}
{Undergraduate pursuing a Bachelor of Arts in mathematics. Current GPA: 4.0/4.0. Current math GPA: 4.0/4.0.}

% \award{Program in Mathematics for Young Scientists (PROMYS)}{July--Aug. 2019, 2020, 2021}
% {Student in the six-week summer program immersing students in an accelerated undergraduate number theory course. As \textbf{junior counselor} (2021), facilitated discussions in the Galois theory and hyperelliptic curves seminars. Also oversaw student work, organized social events, gave talks to students and counselors, suggested program improvements.}

% \award{Liberal Arts and Science Academy (LASA)}{2017--2021}
% {Public magnet high school, ranked 34th nationally by \textit{U.S. News \& World Report}. GPA: 3.87/4.0. Rank: 11/310.}
%Top 10\% every year.

\iffalse
    \entry{MORPH}{Student}{June--Aug. 2020}{3 hours/week}
    {Worked through some of \textit{Problems in Analytic Number Theory} as directed reading, wrote over 100 pages of solutions, wrote introductory handout for the Prime Number Theorem.}
\fi

\award{Programming}{}{Python (SageMath, Jupyter), \LaTeX{}, C++}{}

% Maybe mention Spring 2021 DRP?

\section{Research Experience}

\award{University of Michigan REU}{June~2023--August~2023}
{Attached invariants to triples of certain representations of $\operatorname{GL}_2(\mathbb F_q)$ and examined properties of these invariants. Advisors: Jialiang Zou and Elad Zehlinger.}

\award{University of Michigan REU}{May~2022--July~2022}
{Used group cohomology to explicitly compute the Tate canonical class. Advisors: Alexander Bertoloni Meli, Patrick Daniels, and Peter Dillery.}

% \award{Program in Mathematics for Young Scientists (PROMYS)}{July--Aug. 2019, 2020, 2021}
% {In 2021, studied patterns in graphical properties of prime sums. In 2020, studied asymptotics associated to Thue's lemma. In 2019, participated in directed reading on group-theoretic properties of fractional linear transformations.
% As \textbf{junior counselor} (2021), facilitated discussions in the Galois theory and hyperelliptic curves seminars. Also oversaw student work, organized social events, gave talks to students and counselors, suggested program improvements. Also student (2019).
% }

\section{Preprints}

{\large\bfseries Interests:} {\large\color{gray}Algebraic number theory and algebraic geometry.}

% \award{Degenerate Principal Series Representations of Symplectic Groups over Finite Fields}{}
% {Studied the reducibility and eigenvalues of the intertwining operator of the degenerate principal series representations of the groups $\mathrm{Sp}_{2n}(\mathbb F_q)$ and $\mathrm{GSp}_{2n}(\mathbb F_q)$. In progress.}

\award{Generalized Periodicity in Group Cohomology}{}
{To appear in \textit{Communications in Algebra}. Studied a certain generalization of periodic cohomology with computational applications. Published (\underline{\href{https://arxiv.org/abs/2302.06160}{arXiv:2302.06160}}).}

\award{Explicit Computations of Fundamental Classes}{}
{Provided an explicit computation of the local fundamental class in various cases. Gave applications to Artin reciprocity and computations of the Tate canonical class. Preprint (\underline{\href{https://arxiv.org/abs/2302.06163}{arXiv:2302.06163}}).}

% \award{Prime Sums}{}
% {Studied graphical properties of certain prime sums and their generalizations to $\mathbb Z[i]$. With Anupam Datta, Raymond Feng, and Henry Xie. Preprint (\texttt{\href{https://arxiv.org/abs/2111.02795}{arXiv:2111.02795}}).}

\section{Talks}

\award{Gamma Factors for Representations of General Linear Groups over Finite Fields}{Jan.~2023}{Joint Mathematics Meeting}

\award{One $\mathbf{\mathrm{GL}_2}$, Two $\mathbf{\mathrm{GL}_2}$, Red $\mathbf{\mathrm{GL}_2}$, Blue $\mathbf{\mathrm{GL}_2}$}{July 2023}{Summer Undergraduate Michigan Mathematics Research Conference}

\award{Group Laws for Galois Gerbs}{Aug.~2022}{Young Mathematicians Conference}

\award{Groups of Wrath}{July 2022}{University of Michigan REU Seminar}

\section{Outreach}

\iffalse
    \entry{Technical Director}{Hyde Park Baptist Church Media Ministry}{Summer 2017--Present}{3 hours/week}
    {Operated the video switcher as final check before broadcast. Also occasional cameraperson. Served for weekly broadcasts, weddings, and funerals.}
\fi

\entry{Academic Officer}{Math Undergraduate Student Society (MUSA)}{Dec.~2022--Present}{}
{Organized ``Math Mondays,'' a weekly undergraduate seminar. Also Vice President (Dec.~2022--Dec.~2023), in which I ran the bureaucratic side of MUSA, in writing emails, delegating appropriately, and setting up meetings.}

\entry{Problem Writer}{Berkeley Math Tournament (BMT)}{Sept.~2021--Present}{}
{Collaborated making problems (writing and solving) for BMT, a high-school math contest with \raisebox{0.5ex}{\texttildelow}1200 participants in 2021. Attended weekly problem-writing meetings and socials. Also graded proof-based questions for the US and China contests.}

% \award{Private Tutor}{Sept. 2021--Present}
% {Tutored online for Discrete Structures I (COMP 1805) for Carleton University.}

% \entry{Peer Tutor}{Breakthrough Central Texas}{Sept.~2019--May 2021}{3 hours/week}
% {Aided students (one-on-one and in small groups) through physics and math after school as a part of Breakthrough's Central Texas chapter at the Liberal Arts and Science Academy.}


\newpage

\addtolength{\topmargin}{0.8in}
\vspace{10em}


\section{Teaching}

\entry{Facilitator}{MUSA 154}{Fall 2023}{}
{MUSA 154 is a student-lead course discussing topics in Diophantine equations: continued fractions, Pell's equations, Dirichlet's unit theorem, Hensel's lemma, elliptic curves, SageMath. As facilitator, wrote \underline{\href{https://dfoiler.github.io/notes/154/notes.pdf}{course notes}}, wrote and graded homework problems, and lectured. \href{https://drive.google.com/file/d/1rFrRkBsHJudkTDIoJZ76rk3sxnTt2ntE/view}{\underline{Syllabus}}}.

\entry{Facilitator}{MUSA 74}{Spring 2023, Fall 2023}{}
{MUSA 74 is a student-lead course to help math students transition to proof-based upper-division courses. As facilitator, rewrote and reformatted \underline{\href{https://musa-berk.github.io/MUSA-74/notes.pdf}{course notes}}, wrote and graded homework problems, and lectured. \href{https://musa-berk.github.io/MUSA-74/syllabus.pdf}{\underline{Syllabus}}.}

\entry{Peer Tutor}{Student Learning Center}{Mar.~2022--Dec.~2023}{4 hours/week}
{Worked one-on-one in drop-in environment for real analysis and abstract algebra and lower-division math as needed. Also co-wrote, organized, and delivered lecture-style content reviews for final exam preparation.}

% \entry{Junior Counselor}{PROMYS}{July--Aug. 2021}{}
% {Participated in the PROMYS program with both student and counselor duties: attended lectures on Galois theory and hyperelliptic curves, gave talks to students and seminars to counselors, organized some social events.}

\iffalse
    \entry{Peer Tutor}{LASA Math Department}{Sept.~2018--Present}{3 hours/week}
    {Formed relationships while tutoring one-on-one Precalculus, Linear Algebra, and Logic. Conveyed some metamathematics.}
\fi

\iffalse
    \entry{VBS Teen Helper}{Hyde Park Baptist Church Children's Ministry}{June 2017--2019}{40 hours/week}
    {For two weeks annually, worked along with one or two adults responsible for the behavior of about 15 elementary-school children. Similarly helped with KAMP (July 2017) for one week, 30 hours.}
\fi

\iffalse
    \entry{Operation Christmas Child}{Samaritan's Purse}{November 2017--2019}{4 hours/week}
    {Packed and labeled shoeboxes into larger boxes for shipping. One week a year.}
\fi

% \section{Leadership}

% \entry{President}{Liberal Arts and Science Academy Math Club}{Dec.~2019--Mar.~2021}{3 hours/week}
% {Spearheaded activity planning, curated email and website, prepared handouts, lectured. Member Sept. 2017--Dec. 2019.}

% \entry{Head Problem Writer}{Austin Math Circle Practice MATHCOUNTS}{June 2019--Jan.~2021}{1 hour/week}
% {Wrote for the written and unwritten tests, formatted the written test, organized problem placement, planned some contest details. Also proctored, graded, and sorted student exams.}

% \entry{Officer of Outreach}{LASA Christian Club}{Sept.~2019--Present}{1 hour/week}
% {Ran prayer time, organized other officers, curated email, led club activities. Member Sept. 2018--May 2019.}

% \entry{Teacher's Aide}{LASA}{Spring 2020}{4 hours/week}
% {Taught AP Calculus BC during teacher's paternity leave. Lectured, graded papers, helped office hours.}

% \section{Selected Awards}

% \award{Capture the Flag Contests (CTFs) --- 1st in angstromCTF 2019}{2018--2020}
% {Participated in CTFs, intense multi-day cybersecurity contests. Specialized in cryptography and heap exploitation.
% % Co-founder of my team, the Callipygian Consortium of Cryptography, ranked 11th nationally on CTFtime in 2020. \\
% Placed 1st in angstromCTF 2019, 2nd in HSCTF 7, 5th in picoCTF 2019, top 50 in plaidCTF 2020 and GoogleCTF 2020.}

% \award{AIME Qualifier}{2018--2021}
% {Qualified from the American Mathematics Contest as top 10\% participant. Scored 6 in 2018, 10 in 2020, 8 in 2021. Also scored 132 on the AMC 10 in 2018, with Distinguished Honor Role.}

\iffalse
    \award{Texas A\&M University High School Contest}{Oct. 2018}
    {State contest. Placed 5th in Best Student Closed exam, 10th in EF exam.}
\fi

% \award{USACO Gold}{Mar. 2020}
% {Placed in gold division in 2020 US Open. Studied and implemented computer-theoretic algorithms.}

\iffalse
    \award{TLU Bulldog Calculus Showdown}{Mar. 2018, 2019}
    {State contest. Led team to place 1st consecutively for the final round.}
\fi

\iffalse
    \award{National AP Scholar}{2020}
    {Scored 4 or higher on eight total AP exams.}
\fi

% \newpage
\section{Coursework}
Course numbers beginning with 2 are graduate coursework. A * signals currently taking. Courses are listed thematically.
\begin{multicols}{2}
    \begin{itemize}
        \item 104: Real Analysis
        \item 185: Complex Analysis
        \item 202A: Topology and Measure Theory
        \item 215A: Algebraic Topology
        \item 214*: Smooth Manifolds
        \item 225A: Model Theory
        \item 198: Category Theory (student-led)
        \item 250A: Groups, Rings, and Fields
        \item 250B: Commutative Algebra
        \item 191: Analytic Number Theory
        \item 199: Arithmetic Statistics (reading course)
        \item 199: Modular Forms (reading course)
        \item 191: Automorphic Forms (reading course)
        \item 256A: Algebraic Geometry, Schemes
        \item 256B: Algebraic Geometry, Cohomology
        \item 191*: \'Etale Cohomology (reading course)
        \item 254B: Rational Points on Varieties
        \item 254B*: Complex Multiplication of Abelian Varieties
    \end{itemize}
\end{multicols}

\end{document}
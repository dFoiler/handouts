\documentclass{article}
\usepackage[utf8]{inputenc}

\newcommand{\nirpdftitle}{Student Automorphic Forms Seminar}
\usepackage{import}
\inputfrom{../../notes}{nir}
\usepackage[backend=biber,
    style=alphabetic,
    sorting=ynt
]{biblatex}
\setcounter{tocdepth}{2}

\pagestyle{contentpage}

\title{Student Automorphic Forms Seminar}
\author{Nir Elber}
\date{Fall 2024}
\usepackage{graphicx}

\begin{document}

\maketitle

\tableofcontents

\section{September 4th: Nir Elber}
Today we're talking about the local theory of Tate's thesis.

\subsection{A Little Global Theory}
In order to not lose perspective in the Fourier analysis that makes up the body of this talk, we discuss a little global theory. The goal of Tate's thesis is to derive analytic properties of $L$-functions such as the following.
\begin{definition}
	We define the \textit{Riemann $\zeta$-function} as
	\[\zeta(s)\coloneqq\prod_{p\text{ prime}}\frac1{1-p^{-s}}.\]
\end{definition}
\begin{definition}
	Given a Dirichlet character $\chi\colon(\ZZ/N\ZZ)^\times\to\CC^\times$, we define the \textit{Dirichlet $L$-function} by
	\[L(s,\chi)\coloneqq\prod_{p\text{ prime}}\frac1{1-\chi(p)p^{-s}}.\]
\end{definition}
\begin{definition}
	For a number field $K$, we define the Dedekind $\zeta$-function of a number field $K$
\[\zeta_K(s)\coloneqq\prod_{\substack{\mf p\subseteq\OO_K\\\mf p\text{ prime}}}\frac1{1-\op N(\mf p)^{-s}}.\]
\end{definition}
For now, ``analytic properties'' means deriving a meromorphic continuation, which in practice means deriving a functional equation.
\begin{remark}
	There is a common generalization of the above two $L$-functions called a ``Hecke $L$-function,'' but streamlined definitions would require a discussion of the ad\'eles, which we are temporarily avoiding.
\end{remark}
In fact, Hecke had proven functional equations before Tate, but Tate's arguments modernize better, which is why we will talk about them.

By way of example, we state the functional equation for the Riemann $\zeta$-function $\zeta(s)$.
\begin{theorem}[Riemann] \label{thm:riemann-fe}
	Define the completed Riemann $\zeta$-function by
	\[\Xi(s)\coloneqq\pi^{-s/2}\Gamma(s)\zeta(s).\]
	Then $\Xi(s)$ admits an analytic continuation to all of $\CC$ and satisfies the functional equation $\Xi(s)=\Xi(1-s)$.
\end{theorem}
In some sense, the real goal of Tate's thesis is to explain the presence of the mysterious factor $\pi^{-s/2}\Gamma(s)$ which permits the functional equation $\Xi(s)=\Xi(1-s)$. To understand this, we write
\[\Xi(s)=\pi^{-s/2}\Gamma(s)\prod_{p\text{ prime}}\frac1{1-p^{-s}},\]
and the idea is that we should view this product over all places of $\QQ$: the factor $\pi^{-s/2}\Gamma(s)$ belongs to the archimedean place of $\QQ$!
\begin{idea}[Tate] \label{idea:l-function-places}
	Completed $L$-functions should be products over all places.
\end{idea}
Very roughly, \Cref{idea:l-function-places} allows us to reduce global functional equations into products of local ones.

\subsection{Local \texorpdfstring{$Z$}{ Z}-integrals}
We are interested in proving equations of the type $Z(s)\approx Z(1-s)$ for some suitable function $Z$. We will emply the following trick: we will show that both $Z(s)$ and $Z(1-s)$ live in the same one-dimensional space of functions and then study the scale factor between the two functions later.

To define our function $Z$, we take motivation from the definition of
\[\Gamma(s)=\int_{\RR^+}e^{-t}t^s\,\frac{dt}t.\]
We know that this should correspond to the archimedean place of $\QQ$, but we would like to extend this definition to the finite places. As such, we make the following observations.
\begin{itemize}
	\item $\RR$ is the completion of $\QQ$ with respect to the archimedean place.
	\item The function $t\mapsto e^{-t}$ is an additive character $\RR\to\CC^\times$.
	\item The function $t\mapsto t^{-s}$ is a multiplicative character $\RR^\times\to\CC^\times$.
	\item The measure $dt/t$ is a Haar measure of $\RR^+$.
\end{itemize}
To begin our generalizations, we recall the definition of a local field, which will place $\RR$ in the correct context.
\begin{definition}[local field]
	A \textit{local field} is a locally compact nondiscrete topological field. It turns out that local fields of characteristic zero are exactly the finite extensions of $\RR$ and $\QQ_p$.
\end{definition}
Next up, to place $dt/t$ in the correct context, we should define a Haar measure.
\begin{definition}[Haar measure]
	Fix a locally compact topological group $G$. Then a \textit{left-invariant Haar measure} $d\mu_\ell(g)$ is a Radon measure such that $\mu_\ell(gS)=\mu_\ell(S)$ for each $g\in G$ and measureable set $S$. In terms of intergrals, this is equivalent to
	\[\int_Gf(gh)\,dh=\int_Gf(h)\,dh\]
	for each $g\in G$ and integrable function $f$. It turns out that the Haar measure is unique up to scalar.
\end{definition}
\begin{remark}
	In general, a left-invariant Haar measure need not be right-invariant. However, this is frequently true: for example, if $G$ is abelian or ``reductive'' (such as $G=\op{GL}_n$), then left-invariant Haar measures are right-invariant.
\end{remark}
\begin{example}
	The Lebesgue measure $dt$ is a Haar measure on $\RR$. The measure $dt/\left|t\right|$ is a Haar measure on $\RR^+$ and $\RR^\times$.
\end{example}
\begin{example}
	Fix a prime $p$. There is a unique Haar measure $\mu$ on $\QQ_p$ such that $\mu(\ZZ_p)=1$. For example, we find that $\mu(a+p\ZZ_p)=\frac1p$ for each $a\in\QQ_p$.
\end{example}
\begin{remark}
	Local fields turn out be normed, though this is not immediately obvious from the definition. Even though $\RR$, $\CC$, and $\QQ_p$ all have natural norms (extended from $\QQ$), here is a hands-free way to obtain this norm from a local field $K$: choosing a Haar measure $dt$ on $K$, we define the norm $\left|a\right|$ of some $a\in K$ as the scalar such that
	\[d(at)=\left|a\right|dt.\]
	It is not at all obvious that $\left|\cdot\right|$ is a norm (in particular, why does it satisfy the triangle inequality?), but it is true. As an example, $\left|\cdot\right|$ is the square of the Euclidean norm on $\CC$. For $\QQ_p$, we find $\left|p\right|=1/p$.
\end{remark}
\begin{example}
	We are now able to say that $dt/\left|t\right|$ is a Haar measure of $K^\times$ for any local field $K$.
\end{example}
At this point, we may expect that our generalization of $\Gamma(s)$ to a generic local field $K$ to be
\[Z(\psi,\omega)=\int_K\psi(t)\omega(t)\,\frac{dt}{\left|t\right|},\]
where $\psi\colon K\to\CC^\times$ and $\omega\colon K^\times\to\CC^\times$ are characters. However, this is a little too rigid for our purposes.

Notably, by taking linear combinations of additive characters, Fourier analysis explains that understanding $\Gamma$ very well should permit understanding integrals of the form
\[\int_{\RR^+}f(t)t^s\,\frac{dt}t,\]
where $f\colon\RR\to\CC$ is some sufficiently nice function. It will help to have the flexibility that this extra $f$ permits.
\begin{definition}
	Fix a local field $K$. For a nice enough function $f$ and character $\omega\colon K^\times\to\CC^\times$, we define the \textit{local $Z$-integral}
	\[Z(f,\omega)\coloneqq\int_{K^\times}f(t)\omega(t)\,\frac{dt}{\left|t\right|}.\]
\end{definition}
We will not dwell on this, but perhaps we should explain what is required by ``nice enough'' function $f$. The keyword is ``Schwartz--Bruhat.''
\begin{definition}[Schwartz--Bruhat]
	Fix a local field $K$.
	\begin{itemize}
		\item If $K\in\{\RR,\CC\}$, we say $f\colon K\to\CC$ is \textit{Schwartz--Bruhat} if and only if it is infinitely differentiable and for all of its derivatives to decay rapidly (namely, $f(t)p(t)\to0$ as $\left|t\right|\to\infty$ for any polynomial $p$).
		\item For other $K$, we say $f\colon K\to\CC$ is \textit{Schwartz--Bruhat} if and only if it is locally constant and compactly supported.
	\end{itemize}
	We let $S(K)$ denote the vector space of Schwartz--Bruhat functions on $K$, and we let $S(K)'$ denote its dual (i.e., the vector space of distributions).
\end{definition}
\begin{example}
	The function $t\mapsto e^{-t^2}$ is a Schwartz--Bruhat function $\CC\to\CC$.
\end{example}
\begin{example}
	Fix a prime $p$. Then the indicator function $1_{\ZZ_p}$ on $\QQ_p$ is Schwartz--Bruhat.
\end{example}
Importantly, the definition of Schwartz--Bruhat will promise that 

\subsection{Fourier Analysis}
We now take a moment to review where we are standing. We were hoping to prove a statement like $Z(s)\approx Z(1-s)$, where $Z(s)$ perhaps has some kind analytic properties. However, we currently have a function $Z(f,\omega)$, so it's not at all obvious how to replace $f$ and $\omega$ with ``dual'' entries or how to make sense of analytic properties. In this subsection, we address both of these concerns; they are both related to character theory.

Going in order of difficulty, it is a little easier to explain how to add analytic structure to $Z$. After taking norms, we can find some $s\in\CC$ such $\left|\omega\right|=\left|\cdot\right|^s$ so that $\eta\coloneqq\omega\left|\cdot\right|^{-s}$ outputs to $S^1$, and
\[\omega=\eta\left|\cdot\right|^s.\]
Thus, we see that we can decompose characters $K^\times\to\CC^\times$ into a unitary part $\eta$ and an ``unramified'' part $\left|\cdot\right|^s$, and the unramified part now has a complex parameter $s\in\CC$. Namely, $\omega\left|\cdot\right|^{-s}$ now outputs to $S^1$; i.e., this character is unitary.
Fix a local field $K$.

Thus, we can view $Z(f,\eta)$ as having three parameters as $Z(f,\eta,s)\coloneqq Z\left(f,\eta\left|\cdot\right|^s\right)$, where we now require that $\eta$ is unitary. Because we already have some notion of smoothness in the parameter $s\in\CC$, it remains to understand smoothness in the parameter of unitary character $\eta$.
\begin{example}
	In the archimedean case, the parameter $\eta$ is not so interesting.
	\begin{itemize}
		\item The characters $\RR^\times\to S^1$ take the form $t\mapsto t^{-a}\left|t\right|^s$ where $a\in\{0,1\}$ and $s\in\CC$.
		\item The characters $\CC^\times\to S^1$ take the form $z\mapsto z^a\ov z^b\left|z\right|^s$ where $a,b\in\ZZ$ and $s\in\CC$.
	\end{itemize}
\end{example}
We now understand that our functional equation for $Z$ should arise from $Z(f,\eta,s)$. We hope to take $s\mapsto1-s$, and it seems reasonable (by looking at functional equations for $L(s,\chi)$) to replace $\eta$ with $\eta^{-1}$.

However, we still need to replace $f$ with some dual function. In the archimedean case, we expect this to be the Fourier transform defined by
\[\widehat f(s)\coloneqq\int_\RR f(t)e^{2\pi ist}\,dt.\]
Such a definition will more or less carry through for arbitrary local fields, but we will have to go through approximately the same dictionary that defined $Z(f,\omega)$ from $\Gamma(s)$. In particular, the functions $t\mapsto e^{2\pi ist}$ list the characters of $\RR$, so the above is the integration of our function $f$ against the list of characters. To this end, we pick up the following theorem.
\begin{theorem} \label{thm:local-self-dual}
	Fix a local field $K$. Then there exists a nontrivial character $\psi\colon K\to S^1$. Any choice of $\psi$ defines a bijection $K\to\widehat K$ by sending $a\in K$ to the character $\psi_a(t)\coloneqq\psi(at)$.
\end{theorem}
\begin{remark} \label{rem:lca-duality}
	For a general locally compact abelian group $G$, one can give $\widehat G\coloneqq\op{Hom}(G,\CC)$ a locally compact topology. With this topology in hand, one actually finds that the map $K\to\widehat K$ given by $a\mapsto\psi_a$ is an isomorphism of locally compact abelian groups. However, we do not currently have the need to work in this level of generality.
\end{remark}
We are now ready to define our Fourier transform.
\begin{definition} \label{def:local-fourier}
	Fix a local field $K$, and choose a nontrivial character $\psi\colon K\to S^1$. For $f\in S(K)$, we define the \textit{Fourier transform}
	\[\mc F_\psi f(t)\coloneqq\int_Kf(t)\psi(st)\,dt.\]
\end{definition}
\begin{example}
	For $K=\RR$, choose $\psi(t)\coloneqq e^{2\pi it}$. Then the Fourier transform (up to normalization) agrees with the usual one
	\[\mc F_\psi f(t)\coloneqq\int_\RR f(t)e^{2\pi ist}\,dt.\]
\end{example}
It turns out that one has the usual properties for the Fourier transform, such as $\mc F_\psi\mc F_\psi f(t)=f(-t)$.

We are now ready to state the local functional equation.
\begin{theorem} \label{thm:local-fe}
	Fix a local field $K$, and choose a nontrivial character $\psi\colon K\to S^1$. For every $f\in S(K)$, the function $Z(f,\omega,s)$ has a meromorphic continuation to $s\in\CC$ (with well-understood poles) and satisfies a functional equation of the form
	\[Z(f,\omega,s)=\gamma(\psi,\omega,s,dx)Z\left(\mc F_\psi f,\omega^{-1},1-s\right),\]
	where $\gamma(\psi,\omega,s,dx)$ is meromorphic in $s\in\CC$ (with well-understood poles).
\end{theorem}
Tate's original proof of \Cref{thm:local-fe} was more or less by an explicit computation: by some argument interchanging integrals, one can relate $Z(f,\omega,s)$ with $Z(g,\omega,s)$ for separate $f,g\in S(K)$, which allows us to reduce the proof to a single $f$. Then one chooses some $f$ such that $\mc F_\psi f=f$ (e.g., one chooses a Gaussian when $K=\RR$) and does an explicit computation to check the result.

\subsection{Multiplicity One}
We are not going to prove \Cref{thm:local-fe} in detail, but we will gesture towards an argument which uses a bit more functional analysis and less explicit computation. The key point is that $Z(-,\omega)\colon S(K)\to\CC$ is a distribution with the curious property that
\[\int_{K^\times}f(at)\omega(t)\,\frac{dt}t=\omega^{-1}(a)\int_{K^\times}f(t)\omega(t)\,\frac{dt}{\left|t\right|}.\]
Thus, we see that $Z(-,\omega)$ is an eigendistribution of sorts. To make this more explicit, we let $K^\times$ act on $S(K)$ by translation: for $a\in K$ and $f\in S(K)$, we define $(r(a)f)(t)\coloneqq f(at)$. Then $K^\times$ acts on the distributions $S(K)'$ accordingly: for $a\in K$ and $\lambda\in S(K)'$ and $f\in S(K)$, we can define
\[\langle r'(a)\lambda,f\rangle\coloneqq\left\langle\lambda,r(a)^{-1}f\right\rangle.\]
Tracking everything through, we see that $Z(-,\omega)\in S(K)'$ is an eigendistribution for the action of $K^\times$ on $S(K)'$ with eigenvalue given by the character $\omega$.
\begin{notation}
	Let $S(K)'(\omega)$ denote the space of $K^\times$-eigendistributions with eigenvalue $\omega$.
\end{notation}
Now, we see that $Z(-,\omega)\in S(K)'(\omega)$, and one can check that $Z\left(\mc F_\psi-,\omega^{-1}\right)$ is also in $S(K)'(\omega)$. Thus, a statement like \Cref{thm:local-fe} will follow from the following ``representation-theoretic'' multiplicity one result.
\begin{theorem} \label{thm:mult-one}
	Fix a local field $K$ and character $\omega\colon K\to\CC^\times$. Then
	\[\dim S(K)'(\omega)=1.\]
\end{theorem}
Let's explain the sort of inputs that go into proving \Cref{thm:mult-one}, but we will not say more.
\begin{itemize}
	\item One can verify that $Z(-,\omega)$ provides some vector in $S(K)'(\omega)$, so this space is at least nonempty.
	\item Let $C_c^\infty(K)$ denote the set of compactly supported Schwartz functions on $K$. Duality provides an exact sequence
	\[0\to S(K)'_0\to S(K)'\to C_c^\infty(K)'\to0,\]
	where $S(K)'_0$ denotes the distributions supported at $0$. Taking eigenspaces, we get an exact sequence
	\[0\to S(K)'_0(\omega)\to S(K)'(\omega)\to C_c^\infty(K)'(\omega).\]
	\item One can show that $C_c^\infty(K)'(\omega)$ is one-dimensional with basis given by $f\mapsto\int_Kf(t)\omega(t)\,dt/\left|t\right|$.
	\item Understanding $S(K)'_0$ comes down to some casework. For example, we look at the nonarchimedean case. If $\omega$ is trivial, we have the distribution $f\mapsto f(0)$; otherwise, $S(K)'_0(\omega)$ is zero.
\end{itemize}

\section{September 11: Reed Jacobs}
Today we are talking about the global theory of Tate's thesis. This will mostly be an excuse to discuss the ad\'eles.

\subsection{Ad\'eles}
For today, $K$ is a global field; in particular, it is a number field (in characteristic $0$) or a finite extension of $\FF_p(t)$ (in postive characteristic $p>0$). Here is our main character for today.
\begin{definition}[ad\'ele ring]
	Fix a global field $K$, and let $V_K$ denote the set of places of $K$. Then the \textit{ad\'ele ring $\AA_K$} is the restricted direct product
	\[\AA_K\coloneqq\prod_{v\in V_K}(K_v,\OO_v).\]
	Here, this notation means that $\AA_K$ consists of infinite tuples $(a_v)_v\in\prod_{v\in V_K}K_v$ where $a_v\in \OO_v$ for all but finitely many places $v\in V_K$.
\end{definition}
\begin{remark}
	Even though $\OO_v$ may not have a definition for infinite $v$, we see that this does not matter because there are only finitely many infinite places anyway.
\end{remark}
\begin{remark}
	We see that $\AA_K$ becomes a ring under the pointwise operations. In fact, it becomes a topological ring when given the subspace topology of the product topology on $\prod_vK_v$. In particular, upon taking the intersection, we see that our basic open sets look like
	\[\prod_{v\in V_K}U_v,\]
	where $U_v\subseteq K_v$ is open, but $U_v=\OO_v$ for all but finitely many $v\in V_K$.
\end{remark}
Here are some quick properties of the topology.
\begin{proposition}
	Fix a global field $K$. Then $\AA_K$ is a locally compact topological ring.
\end{proposition}
\begin{proof}
	The fact that the ring operations are continuous can be checked pointwise on the level of the product $\prod_vK_v$. Perhaps we should also check that $\AA_K$ is Hausdorff, for which we note that this follows from being a subspace of $\prod_vK_v$ again.
	
	To be locally compact, we have to do a little more work. This follows more or less  Fix some $x\in\AA_K$, and we want some open neighborhood $U\subseteq\AA_K$ of $x$ with compact closure. By translation, we may assume that $x=0$. Then the open subset
	\[\prod_{\substack{v\in V_K\\v\text{ infinite}}}B_v(0,1)\times\prod_{\substack{v\in V_K\\v\text{ finite}}}\OO_v,\]
	where $B_v(0,1)$ is the open ball around $0\in K_v$. This is open by construction, and its closure $\prod_{v\in V_K}\overline{B_v(0,1)}$ is a product of compact sets and hence compact.
\end{proof}
\begin{remark}
	Do note that $\prod_vK_v$ fails to be locally compact because the open subsets are too big! This is one reason why we want to work with the ad\'eles instead: once has a chance of doing some reasonable topology on $\AA_K$.
\end{remark}
To misquote Tate, we remark that one can really only extract arithmetic information from $\AA_K$ by putting $K$ inside it. Here is this embedding.
\begin{proposition}
	Fix a global field $K$. Then the embedding $i\colon K\into\AA_K$ defined by
	\[i\colon a\mapsto(a)_v\]
	is discrete and cocompact.
\end{proposition}
\begin{proof}
	We will not prove cocompactness; it is equivalent to the finiteness of the class group and Dirichlet's unit theorem. Alternatively, one can use Minkowski theory to show this directly.

	Discreteness is a little easier to show. For $a\in K$, we must show that $i(a)$ has an open neighborhood disjoint from the rest of $i(K)$. By translating, we may take $a=0$. Now, define the open neighborhood of $0$ given by
	\[U\coloneqq\prod_{\substack{v\in V_K\\v\text{ infinite}}}B_v(0,1)\times\prod_{\substack{v\in V_K\\v\text{ finite}}}\OO_v.\]
	Now, for any $b\in K^\times$, the product formula asserts that $\prod_v\left|b\right|_v=1$, but $\prod_v\left|b_v\right|_v<1$ for any $(b_v)_v\in U$, so $U\cap i(K)=\{0\}$, as required.
\end{proof}
The above cocompactness tells us that the quotient $\AA_K/K$ will be interesting to us; similarly, the group $\AA_K^\times/K^\times$ will be interesting to us.

\subsection{Some Pontryagin Duals}
We have some time, so let's expand the discussion of \Cref{rem:lca-duality}.
\begin{definition}
	Fix a locally compact abelian group $G$. Then we define the group $G\coloneqq\op{Hom}\left(G,S^1\right)$ to be the \textit{Pontryagin dual}, which we turn into a topological group by giving it the compact-open topology. It turns out that $\widehat G$ is a locally compact abelian group.
\end{definition}
\begin{example}
	\Cref{thm:local-self-dual} explains that local fields $K_v$ are self-dual.
\end{example}
\begin{example}
	One can check that a character $\ZZ\to S^1$ has equivalent data to an element of $S^1$, which essentially proves $\widehat{\ZZ}\cong S^1$.
\end{example}
\begin{example}
	On the other hand, a continuous homomorphism $S^1\to S^1$ must be exponentiation by some integer, so $\widehat{S^1}\cong\ZZ$.
\end{example}
\begin{remark}
	It is a general fact that the Pontryagin dual of $\widehat G$ is isomorphic to $G$. There is at least a map $G\to\op{Hom}\left(\widehat G,S^1\right)$ given by $g\mapsto\op{ev}_g$, where $\op{ev}_g\colon\widehat G\to S^1$ is given by $\op{ev}_g(\chi)\coloneqq\chi(g)$.
\end{remark}
We even have a nice duality, as in \Cref{thm:local-self-dual}.
\begin{theorem} \label{thm:adele-self-dual}
	Fix a global field $K$. Then the locally compact abelian group $\AA_K$ is self-dual. More precisely, there exists a nontrivial character $\psi\colon\AA_K\to S^1$. Then for any choice of $\psi$, then the map $\AA_K\to\widehat{\AA_K}$ given by
	\[a\mapsto(\psi_a\colon t\mapsto\psi(at))\]
	is an isomorphism of locally compact abelian groups.
\end{theorem}
\begin{proof}[Sketch]
	This follows roughly from \Cref{thm:local-self-dual}, essentially by trying to take a product of the self-duality results for each local field in the restricted direct product. A detailed proof would require understanding the topology of the Pontryagin dual in more detail, so we will avoid it.
\end{proof}
Here is an application.
\begin{corollary} \label{cor:dual-global}
	Fix a global field $K$. Then the Pontryagin dual of $K$ is $\AA_K/K$.
\end{corollary}
\begin{proof}
	Fix a nontrivial character $\psi\colon\AA_K/K\to S^1$ (note the quotient!), and we construct that the composite
	\[K\into\AA_K\stackrel\psi\cong\widehat{\AA_K},\]
	where the last map is given by \Cref{thm:adele-self-dual}. Now, this composite actually outputs to $\widehat{\AA_K/K}$: any $a\in K$ produces a character $\psi_a$ satisfying $\psi_a(t)=\psi(at)=1$ for each $t\in K$, so $\psi_a$ descends to a character of $\AA_K/K$. We claim that the above map is an isomorphism, which we do by combining three observations.
	\begin{itemize}
		\item We can check that $\widehat{\AA_K/K}$ is a vector space over $K$.
		\item Note $\AA_K/K$ is compact, so $\widehat{\AA_K/K}$ is discrete by some fact of Pontryagin duals.
		\item Note $\widehat{\AA_K/K}/K\subseteq\widehat{\AA_K}/K\cong\AA_K/K$, and this last space is compcat.
	\end{itemize}
	The last two observations imply $\widehat{\AA_K/K}$ is finite, but then the first observation requires it to be trivial.
\end{proof}

\subsection{A Poisson Summation Formula}
The presence of a duality permits a well-behaved Fourier analysis. For our technicalities, we explain what our Schwartz functions are.
\begin{definition}[Schwartz]
	A \textit{Schwartz--Bruhat} function $f$ on the quotient $\AA_K/K$ is a function of the form $\prod_\nu f_\nu$ such that each $f_\nu$ is Schwartz, and $f_\nu=1_{\OO_\nu}$ for all but finitely many $\nu$. We also permit finite $\CC$-linear combinations of these functions to be Scwhartz--Bruhat.
\end{definition}
And here is our corresponding Fourier transform; it is our global analogue of \Cref{def:local-fourier}.
\begin{definition}
	Fix a global field $K$, and choose a nontrivial character $\psi\colon\AA_K/K\to S^1$. For any Schwartz--Bruhat function $f$, then we define the \textit{Fourier transform} to be
	\[\mc F_\psi f(y)\coloneqq\int_{x\in\AA_K}f(x)\psi(xy)\,dx.\]
\end{definition}
\begin{remark} \label{rem:general-fourier-trans}
	For the discussion which follows shortly, it will be worth our time to write out what the Fourier transform is in general for a locally compact abelian group $G$. Given a nice enough function $f\colon G\to\CC$ (such as Schwartz--Bruhat), the Fourier transform $\widehat f$ is a function on $\widehat G$ defined by
	\[\widehat f(\chi)\coloneqq\int_Gf(g)\chi(g)\,dg.\]
	For example, the identification of characters of $\AA_K$ with $\AA_K$ recovers the above definition. One finds that taking the Fourier transform twice almost recovers the original function.
\end{remark}
Notably, the Fourier transform depends on a choice of Haar measure. These can be found by taking the product of the Haar measures on the local fields; by scaling, we can get the usual identity $\mc F_\psi\mc F_\psi f(x)=f(-x)$.

Riemann's original proof of \Cref{thm:riemann-fe} has the result more or less follow from some functional equation coming from a Poisson summation formula. As such, we desire a Poisson summation formula in our context as well. The point of a Poisson summation formula is to relate sums for a function $f$ over the discrete cocompact subgroup $\ZZ\subseteq\QQ$ with the same sums of the Fourier transform. Because $K\subseteq\AA_K$ is discrete and cocompact, it is natural to have the following statement.
\begin{theorem}[Poisson summation] \label{thm:adelic-p-s}
	Fix a global field $K$, and choose a nontrivial character $\psi\colon\AA_K/K\to S^1$. For any Schwartz--Bruhat function $f$, we have
	\[\sum_{\gamma\in K}f(\gamma)=\sum_{\gamma\in K}\mc F_\psi f(\gamma).\]
\end{theorem}
\begin{proof}
	Define $F\colon\AA_K\to\CC$ by $F(x)\coloneqq\sum_{\gamma\in K}f(x+\gamma)$. Note that $F$ descends to a function on $\AA_K/K$. Quickly, note that the convergence of these sums follows from our definition of Schwartz--Bruhat; the convergence is fast enough to ensure that both sides define a continuous function on $\AA_K/K$. We will spend the rest of the proof ignoring convergence issues.
	
	The point is to compute the Fourier transform of $F$ and apply Fourier inversion. However, $F$ has descended to a function on $\AA_K/K$, so its Fourier transform should be a function on the Pontryagin dual of $\AA_K/K$, which is $K$ by \Cref{cor:dual-global}. As such, we let $D\subseteq\AA_K$ be a fundamental domain for $\AA_K/K$, which we give a volume of $1$ to fix our Haar measure, and we compute
	\begin{align*}
		\mc F_\psi F(y) &= \int_{\AA_K/K}F(x)\psi(xy)\,dx \\
		&= \int_DF(x)\psi(xy)\,dx \\
		&= \int_D\sum_{\gamma\in K}f(x+\gamma)\psi(xy)\,dx \\
		&= \sum_{\gamma\in K}\int_Df(x+\gamma)\psi(xy)\,dx \\
		&= \sum_{\gamma\in K}\int_Df(x)\psi((x-\gamma)y)\,dx \\
		&\stackrel*= \sum_{\gamma\in K}\int_Df(x)\psi(xy)\,dx \\
		&= \mc F_\psi f(y),
	\end{align*}
	where $\stackrel*=$ holds because $\psi|_K=1$. The general theory of Pontryagin duality allows us to apply a Fourier inversion formula for $\widehat K\cong\AA_K/K$ to see $F(x)$ equals
	\[\sum_{y\in K}\widehat F(y)\overline{\psi(xy)}=\sum_{y\in K}\mc F_\psi f(y)\overline{\psi(xy)}.\]
	(Note that this is basically the Fourier transform of the Fourier transform, where we are plugging into the general definition of \Cref{rem:general-fourier-trans}.) Plugging in $x=0$ completes the proof.
\end{proof}
\begin{remark}[Reed]
	In the case where $K$ is the function field of a curve $C$ over a finite field $\FF_q$, one is able to turn \Cref{thm:adelic-p-s} into the Riemann--Roch theorem for $C$. We will not explain this in detail.
\end{remark}
\begin{remark}
	One can basically mimic the proof of \Cref{thm:riemann-fe} to prove a functional equation for ``global $Z$-integrals'' which look like
	\[Z(f,\chi)\coloneqq\int_{\AA_K^\times}f(x)\chi(x)\,d^\times x,\]
	where $d^\times x$ is some suitably defined Haar measure on $\AA_K^\times$. The key analytic input into the proof of \Cref{thm:riemann-fe} is a Poisson summation formula used to produce the symmetry; analogously, the key input into the global functional equation is the ``ad\'elic Poisson summation formula'' \Cref{thm:adelic-p-s} used to produce the symmetry.
\end{remark}

\end{document}
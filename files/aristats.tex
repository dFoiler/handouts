\documentclass{article}
\usepackage[utf8]{inputenc}

\newcommand{\nirpdftitle}{Arithmetic Statistics}
\usepackage{import}
\inputfrom{../notes}{nir}
% \usepackage[backend=biber,
%     style=alphabetic,
%     sorting=ynt
% ]{biblatex}
% \addbibresource{bib.bib}
\setcounter{tocdepth}{2}

\pagestyle{contentpage}

\title{Arithmetic Statistics}
\author{Nir Elber}
\date{Fall 2022}
\usepackage{graphicx}
\lhead{}
\rhead{\textit{Arithmetic Statistics}}

\begin{document}

\maketitle

\tableofcontents

\section{September 14th}
Recall the following definition.
\begin{definition}
	We define $D(X)$ to be the set of number fields $K/\QQ$ of discriminant up to $X$, up to isomorphism.
\end{definition}
Last time we were interested in counting the number of fields in $D(X)$, perhaps with some added Galois or local constraints.

For today, we restrict to looking at imaginary quadratic fields, so we will quickly define the notation $D_{\textrm{imq}}(X)$ to be the set of imaginary quadratic fields $K/\QQ$ of discriminant upper-bounded by $X$, and we will be interested in studying
\[\lim_{X\to\infty}\frac{\#\{K\in D_{\textrm{imq}}(X):\op{Cl}(K)_p\cong\Gamma\}}{\#D_{\textrm{imq}}(X)}.\]
Note that we are interested in studying these sorts of class group questions as distributions, but we could also care about various moment problems.

It turns out that these class group questions are related to counting number fields. For example, where $K$ is an imaginary quadratic field, define $M$ to be the set of surjections $\op{Cl}(K)\onto\ZZ/3\ZZ$; analogously define, $P$ to be the set of ordered pairs $(L,\psi)$ where $L/K$ is an unramified degree-$3$ extension of $K$, and $\psi$ is a choice of isomorphism $\op{Gal}(L/K)\to\ZZ/3\ZZ$. Then class field theory provides a natural isomorphism
\[M\simeq P.\]
One can extend this to work in the other direction.

\end{document}
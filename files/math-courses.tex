\documentclass{article}
\usepackage[utf8]{inputenc}

\usepackage{import}
\inputfrom{../../notes}{nir}

\title{List of Mathematics Coursework}
\author{Nir Elber \\\\ Math GPA: 4.0/4.0}
\date{Spring 2023}

\usepackage[dvipsnames]{xcolor}
\setcounter{secnumdepth}{0}
\setlength{\parindent}{0.0em}
\setlength{\parskip}{0.8em}
% \renewcommand{\familydefault}{\sfdefault}
\pagenumbering{gobble}

\definecolor{headercolor}{HTML}{3e80b0}
\renewcommand{\section}[1]{{\color{headercolor}\LARGE
    \textbf{#1\phantom{p}}}}
    % Insert the phantom p to make the bottom hrule just below the text.
\renewcommand{\subsection}[1]{{\color{headercolor}\Large
    \textbf{#1\phantom{p}}}}

\newcommand{\entry}[5]{{\large\textbf{#1}, \textit{#2}}\\
    #3\hfill\\
    \textcolor{gray}{#4}}

\newcommand{\award}[3]{{\large\textbf{#1}}\hfill\textit{#2}\\
    \textcolor{gray}{#3}}

\begin{document}

\maketitle

\section{University of California, Berkeley}

* Note that Math 198 and 199 courses are graded on Pass/No Pass basis. A course beginning with ``2'' indicates graduate coursework.

\subsection{Fall 2021}

\entry{Math 104}{Introduction to Analysis}
{Grade: A+}{The real number system. Sequences, limits, and continuous functions in R and R. The concept of a metric space. Uniform convergence, interchange of limit operations. Infinite series. Mean value theorem and applications. The Riemann integral. \\
Book: Ross's \textit{Elementary Analysis: The Theory of Calculus}.}


\entry{Math 250A}{Groups, Rings, and Fields}
{Grade: A+}{Group theory, including the Jordan-Holder theorem and the Sylow theorems. Basic theory of rings and their ideals. Unique factorization domains and principal ideal domains. Modules. Chain conditions. Fields, including fundamental theorem of Galois theory, theory of finite fields, and transcendence degree. \\
Book: Lang's \textit{Algebra}.}


\subsection{Spring 2022}

\entry{Math 185}{Introduction to Complex Analysis}
{Grade: A+}{Analytic functions of a complex variable. Cauchy's integral theorem, power series, Laurent series, singularities of analytic functions, the residue theorem with application to definite integrals. Some additional topics such as conformal mapping. \\
Book: Brown and Churchill's \textit{Complex Variables and Applications}.}


\entry{Math 250B}{Commutative Algebra}
{Grade: A+}{Development of the main tools of commutative and homological algebra applicable to algebraic geometry, number theory and combinatorics. \\
Book: Eisenbud's \textit{Commutative Algebra: with a View Toward Algebraic Geometry}.}


\entry{Math 198*}{Category Theory}
{Grade: Pass}{Introduction to category theory. Functors, natural transformations, equivalence, Yoneda lemma, limits, adjoints, units, counits. \\
Book: Riehl's \textit{Category Theory in Context}.}


\entry{Math 199*}{Modular Forms Reading Course}
{Grade: Pass}{First introduction to modular forms, with a focus on number-theoretic and algebro-geometric applications. We are considering doing a final project on sphere-packing in eight dimensions. \\
Books: Serre's \textit{A Course in Arithmetic} and Milne's ``Modular Functions and Modular Forms'' notes.}


\subsection{Fall 2022}

\entry{Math 202A}{Introduction to Topology and Measure Theory}
{Grade: A+}{Metric spaces and general topological spaces. Compactness and connectedness. Characterization of compact metric spaces. Theorems of Tychonoff, Urysohn, Tietze. Function spaces; Arzela-Ascoli and Stone-Weierstrass theorems. Locally compact spaces; one-point compactification. Introduction to measure and integration. Sigma algebras of sets. Measures and outer measures. Lebesgue measure on the line and $\mathbb R^n$. Construction of the integral. Dominated convergence theorem. \\
Book: Very loosely Lang's \textit{Real and Functional Analysis}.}


\entry{Math 256A}{Algebraic Geometry}
{Grade: A+}{Affine and projective algebraic varieties. Theory of schemes and morphisms of schemes. Smoothness and differentials in algebraic geometry. \\
Book: Chapter II of Hartshorne's \textit{Algebraic Geometry}.}


\entry{Math 199*}{Arithmetic Statistics Reading Course, Fall 2022}
{Grade: Pass}{Introduction to the field of arithmetic statistics. Counted number fields with prescribed Galois group and local conditions. Touched a few other topics, such as counting elliptic curves with isogenies. \\
Book: Melanie Matchett Wood's Arizona Winter School 2014 course notes in arithmetic statistics.}


\subsection{Spring 2023}

\entry{Math 191}{Analytic Number Theory}
{Grade: A+}{Dirichlet's Theorem on primes in progressions. Dirichlet $L$-functions, class numbers. Analytic theory: analytic continuation, functional equations and the Prime Number Theorem. Non-vanishing of $L(1, \chi_d)$, Siegel's theorem and its ineffectivity. The Bombieri-Vinogradov theorem. Elementary Sieve. Vinogradov's Theorem and his bilinear sieve method. Primality testing. The Burgess bound. Elementary counting in finite fields. \\
Book: Davenport's \textit{Multiplicative Number Theory}.}


\entry{Math 254B}{Rational Points on Varieties}
{Grade: A}{Quadratic forms, the Hasse--Minkowski theorem, Galois cohomology, the Hasse norm theorem. Elliptic curves, height functions, the Morderll--Weil theorem. The Brauer--Manin obstruction. \\
Books: Milne's class field theory notes, Silverman's \textit{The Arithmetic of Elliptic Curves}, and Poonen's \textit{Rational Points on Varieties}.}


\entry{Math 256B}{Cohomology of Schemes}
{Grade: A}{Coherent sheaves and their cohomology. Riemann-Roch theorem and selected applications. \\
Books: Chapters III and IV of Hartshorne's \textit{Algebraic Geometry}.}


\subsection{Fall 2023}

\entry{Math 215A}{Algebraic Topology}
{Grade: A+}{Fundamental group and covering spaces, simplicial and singular homology theory with applications, cohomology theory, duality theorem. \\
Book: Hatcher's \textit{Algebraic Topology}.}


\entry{Math 225A}{Model Theory}
{Grade: A}{Completeness and compactness theorems. Quantifier elimination, applications to algebra. Basic type theory. \\
Book: Chapters 1--4 of Markers's \textit{Model Theory: An Introduction}.}


\entry{Math 199*}{Automorphic Forms Reading Course}
{Grade: Pass}{Linea algebraic groups, reductive groups. Lie groups and Lie algebras, $(\mf g,K)$-modules. Ad\'eles, id\'eles, hyperspecial subgroups, strong approximation in algebraic groups. Representations of locally compact groups. Discrete automorphic forms. \\
Book: Chapters 1--4 of Markers's \textit{Model Theory: An Introduction}.}


\subsection{Spring 2024}

\entry{Math 214}{Differential Topology}
{Grade: Concurrent}{Smooth manifolds and maps, tangent and normal bundles. Sard's theorem and transversality, Whitney embedding theorem. differential forms, Stokes' theorem, Frobenius theorem. Basic degree theory. Flows, Lie derivative, Lie groups and algebras. \\
Book: Lee's \textit{Smooth Manifolds}.}


\entry{Math 254B}{Complex Multiplication of Abelian Varieties}
{Grade: Concurrent}{Basics of Abelian Varieties, the complex analytic and algebro-geometric theories. The main theorem of complex multiplication. Statements of class field theory. \\
Book: Milne's \textit{Complex Multiplication}.}


\entry{Math 199*}{\'Etale Cohomology}
{Grade: Concrurrent}{\'Etale morphisms and covers, \'etale fundamental group. Grothendieck topologies, cohomology of curves, proper base change, Poincar\'e duality. \\
Book: Deligne's \textit{SGA $4\frac12$}.}

\end{document}
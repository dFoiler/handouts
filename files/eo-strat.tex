\documentclass{article}
\usepackage[utf8]{inputenc}

\newcommand{\nirpdftitle}{Student Number Theory Seminar}
\usepackage{import}
\inputfrom{../../notes}{nir}
\usepackage[backend=biber,
    style=alphabetic,
    sorting=ynt
]{biblatex}
\setcounter{tocdepth}{2}

\pagestyle{contentpage}

\title{Student Number Theory Seminar}
\author{Nir Elber}
\date{Spring 2024}
\usepackage{graphicx}

\begin{document}

\maketitle

\tableofcontents

\section{January 25: Ekedahl--Oort Stratification}
We're going to talk about the Ekedahl--Oort stratification.

\subsection{Dieudonn\'e Modules}
We begin with some motivation. Fix a perfect field $k$ of positive characteristic $p\coloneqq\op{char}k$. There are three possibilities for an elliptic curve $E/k$.
\begin{itemize}
	\item Ordinary: $E[p]\left(\ov k\right)\cong\ZZ/p\ZZ$.
	\item Supersingular: $E[p]\left(\ov k\right)=0$.
\end{itemize}
Notably, $E[p]$ should still have rank $p^2$ (as a finite flat group scheme). It turns out to be productive to use the theory of Dieudonn\'e modules, which is somehow a linearization of the problem (analogous to how Lie algebras linearizes Lie groups).
\begin{definition}[Dieudonn\'e ring]
	Fix a perfect field $k$ of positive characteristic, and let $W(k)$ denote the ring of Witt vectors. Then the \textit{Dieudonn\'e ring} $D_k$ is the non-commutative $W(k)$-algebra generated by $F$ and $V$ satisfying the relations
	\[FV=VF=p\qquad\text{and}\qquad Fw=w^\sigma\qquad\text{and}\qquad wV=Vw^\sigma,\]
	where $(-)^\sigma$ is the Frobenius. A \textit{Dieudonn\'e module} is a $D_k$-module.
\end{definition}
Here is why we care.
\begin{theorem}
	Fix a perfect field $k$ of positive characteristic. There is an additive anti-equivalence of categories from finite commutative $p$-group schemes over $k$ and $D_k$-modules of finite $W(k)$-length. Given such a group scheme $G$, we will let $\mathbb DG$ denote the $D_k$-module.
\end{theorem}
Here are some examples.
\begin{example}
	One has $\mathbb D(\underline{\ZZ/p\ZZ})\cong k$ with $F$ being the Frobenius and $V=0$.
\end{example}
\begin{example}
	One has $\mathbb D(\mu_{p,k})\cong k$ with $F=0$ and $V$ being the inverse Frobenius.
\end{example}
\begin{example}
	Let $\alpha_p$ denote the kernel of the $p$th-power map $\mathbb G_a\to\mathbb G_a$. Then $\mathbb D(\alpha_p)\cong k$ with $F=V=0$.
\end{example}
\begin{example}
	Fix a perfect field $k$ of positive characteristic, and let $A$ be an abelian $k$-variety. Then we have $\mathbb D(A[p])\cong H^1_{\mathrm{dR}}(A)$. (This isomorphism goes through the crystalline site.) In fact, there is an isomorphism of short exact sequences as follows.
	% https://q.uiver.app/#q=WzAsMTAsWzAsMCwiMCJdLFsxLDAsIkheMChBLFxcT21lZ2Ffe0Eva30pIl0sWzIsMCwiSF4xX3tcXG1hdGhybXtkUn19KEEpIl0sWzMsMCwiSF4xKEEsXFxPT19BKSJdLFs0LDAsIjAiXSxbMSwxLCIoayxcXHNpZ21hXnstMX0pXFxvdGltZXNfa1xcbWF0aGJiIEQoQVtGXSkiXSxbMiwxLCJcXG1hdGhiYiBEKEFbcF0pIl0sWzMsMSwiXFxtYXRoYmIgRChBW1ZdKSJdLFs0LDEsIjAiXSxbMCwxLCIwIl0sWzksNV0sWzUsNl0sWzYsN10sWzcsOF0sWzAsMV0sWzEsMl0sWzIsM10sWzMsNF0sWzEsNV0sWzIsNl0sWzMsN11d&macro_url=https%3A%2F%2Fraw.githubusercontent.com%2FdFoiler%2Fnotes%2Fmaster%2Fnir.tex
	\[\begin{tikzcd}
		0 & {H^0(A,\Omega_{A/k})} & {H^1_{\mathrm{dR}}(A)} & {H^1(A,\OO_A)} & 0 \\
		0 & {(k,\sigma^{-1})\otimes_k\mathbb D(A[F])} & {\mathbb D(A[p])} & {\mathbb D(A[V])} & 0
		\arrow[from=2-1, to=2-2]
		\arrow[from=2-2, to=2-3]
		\arrow[from=2-3, to=2-4]
		\arrow[from=2-4, to=2-5]
		\arrow[from=1-1, to=1-2]
		\arrow[from=1-2, to=1-3]
		\arrow[from=1-3, to=1-4]
		\arrow[from=1-4, to=1-5]
		\arrow[from=1-2, to=2-2]
		\arrow[from=1-3, to=2-3]
		\arrow[from=1-4, to=2-4]
	\end{tikzcd}\]
	Here, $(k,\sigma^{-1})$ denotes
\end{example}
So here is another characterization of an elliptic curve $E$ being supersingular.
\begin{itemize}
	\item Ordinary: $F^*\colon H^1(E,\OO_E)\to H^1(E,\OO_E)$ is nonzero; equivalently, $V^*\colon H^0(E,\Omega_{E/k})\to H^0(E,\Omega_{E/k})$ is nonzero.
	\item Supersingular: otherwise.
\end{itemize}
For example, suppose $E/k$ is ordinary. Note that $V$ vanishes on $\mathbb D(E[V])$, so we get $\mathbb D(E[V])=\mathbb D(\underline{\ZZ/p\ZZ})$. Similarly, $F$ vanishes on $\mathbb D(A[F])$, so we get $\mathbb D(\mu_p)$. Thus, we get a short exact sequence
\[0\to\mathbb D(\mu_p)\to\mathbb D(E[p])\to\mathbb D(\underline{\ZZ/p\ZZ})\to0,\]
which upon reversing $\mathbb D$ produces
\[0\to\ZZ/p\ZZ\to E[p]\to\mu_p\to0.\]
This splits at $\ZZ/p\ZZ\to E[p]$ by the Frobenius, so $E[p]\cong\mu_p\oplus\ZZ/p\ZZ$.

On the other hand, the supersingular case will end up producing a short exact sequence
\[0\to\alpha_p\to E[p]\to\alpha_p\to0,\]
which now need not split.

\subsection{\texorpdfstring{$F$}{ F}-zips}
Let $X/k$ be a smooth proper $k$-scheme. As a technical hypothesis, we want the Hodge to de Rham spectral sequence degenerates at $E_1$, though I'm not totally sure what that means. In this situation, we get two filtration.
\begin{itemize}
	\item Hodge filtration: $H^1_{\mathrm{dR}}(X)\supseteq\mathrm{Fil}^1_H\supseteq\mathrm{Fil}^2_H\cdots\supseteq0$. Set $C_i\coloneqq\mathrm{Fil}^i_H$ for brevity.
	\item Conjugate filtration: there is an analogous filtration $H^1_{\mathrm{dR}}(X)\supseteq\ov{\mathrm{Fil}^1_H}\supseteq\ov{\mathrm{Fil}^2_H}\cdots\supseteq0$. Set $D_i\coloneqq\ov{\mathrm{Fil}}_H^{n-i}$ for brevity.
\end{itemize}
In this situation, we will get a Cartier isomorphism $\sigma^*(C^i/C^{i+1})\to(D_i/D_{i-1})$.
\begin{example}
	Let $A/k$ be an abelian variety.
	\begin{itemize}
		\item We have $\mathbb D(A[p])=H^1_{\mathrm{dR}}(A)$.
		\item The first filtration: $H^1_{\mathrm{dR}}(A)\supseteq\ker F\supseteq0$.
		\item The second filtration: $0\subseteq\ker V\subseteq H^1_{\mathrm{dR}}(A)$.
		\item The Cartier isomorphism: $\im F=\ker V$ and $\ker F=\im V$.
	\end{itemize}
\end{example}
We now package all this data into an $F$-zip.
\begin{definition}[$F$-zip]
	Fix an $\mathbb F_q$-scheme $S$. Then an \textit{$F$-zip} over $S$ is a tuple $(M, C^\bullet, D_\bullet,\varphi_\bullet)$ satisfying some coherence conditions. We define its \textit{type} as the map $\tau\colon\ZZ\to\ZZ_{\ge0}$ by $\tau(i)\coloneqq\dim_k\left(C^i/C^{i+1}\right)$.
\end{definition}
We now want to understand $F$-zips.
\begin{refcontext}
	Continue with $A/k$ as an abelian variety. Then a polarization on $A$ induces a symplectic form on $H^1_{\mathrm{dR}}(A)$.
\end{refcontext}
So actually we want to understand $F$-zips with this extra symplectic structure.
\begin{definition}[symplectic $F$-zip]
	Fix everything as above. A \textit{symplectic $F$-zip} is an $F$-zip $(M,C^\bullet,D_\bullet,\varphi_\bullet)$ such that there is a symplectic form $\psi$ on $M$, with some coherence conditions. For example, we want $C^\bullet$ and $D_\bullet$ to be symplectic flags (i.e., the symplectic dual spaces of an element of $C^\bullet$ lives in $C^\bullet$, and similar for $D_\bullet$).
\end{definition}
So here is a classification result.
\begin{theorem}
	Let $k$ be algebraically closed, and let $(V,\psi)$ be a symplectic $k$-vector space and let $G=\op{Sp}(V,\psi)$ with Weyl group $(W,I)$. Let $\tau$ be an ``admissible type'' (namely, on the type of our $F$-zips). Then there is a bijection between isomorphism classes of symplectic $F$-zips of type $\tau$ and $W_j\backslash W$.
\end{theorem}
The point is that $F$-zips can be understood from ``combinatorial data'' from the Weyl group.

\end{document}
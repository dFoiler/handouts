\documentclass{article}
\usepackage[utf8]{inputenc}

\newcommand{\nirpdftitle}{Student Number Theory Seminar}
\usepackage{import}
\inputfrom{../../notes}{nir}
\usepackage[backend=biber,
    style=alphabetic,
    sorting=ynt
]{biblatex}
\setcounter{tocdepth}{2}

\pagestyle{contentpage}

\title{Student Number Theory Seminar}
\author{Nir Elber}
\date{Spring 2024}
\usepackage{graphicx}

\begin{document}

\maketitle

\tableofcontents

\section{January 25: Sean Gonzales}
We're going to talk about the Ekedahl--Oort stratification.

\subsection{Dieudonn\'e Modules}
We begin with some motivation. Fix a perfect field $k$ of positive characteristic $p\coloneqq\op{char}k$. There are three possibilities for an elliptic curve $E/k$.
\begin{itemize}
	\item Ordinary: $E[p]\left(\ov k\right)\cong\ZZ/p\ZZ$.
	\item Supersingular: $E[p]\left(\ov k\right)=0$.
\end{itemize}
Notably, $E[p]$ should still have rank $p^2$ (as a finite flat group scheme). It turns out to be productive to use the theory of Dieudonn\'e modules, which is somehow a linearization of the problem (analogous to how Lie algebras linearizes Lie groups).
\begin{definition}[Dieudonn\'e ring]
	Fix a perfect field $k$ of positive characteristic, and let $W(k)$ denote the ring of Witt vectors. Then the \textit{Dieudonn\'e ring} $D_k$ is the non-commutative $W(k)$-algebra generated by $F$ and $V$ satisfying the relations
	\[FV=VF=p\qquad\text{and}\qquad Fw=w^\sigma\qquad\text{and}\qquad wV=Vw^\sigma,\]
	where $(-)^\sigma$ is the Frobenius. A \textit{Dieudonn\'e module} is a $D_k$-module.
\end{definition}
Here is why we care.
\begin{theorem}
	Fix a perfect field $k$ of positive characteristic. There is an additive anti-equivalence of categories from finite commutative $p$-group schemes over $k$ and $D_k$-modules of finite $W(k)$-length. Given such a group scheme $G$, we will let $\mathbb DG$ denote the $D_k$-module.
\end{theorem}
Here are some examples.
\begin{example}
	One has $\mathbb D(\underline{\ZZ/p\ZZ})\cong k$ with $F$ being the Frobenius and $V=0$.
\end{example}
\begin{example}
	One has $\mathbb D(\mu_{p,k})\cong k$ with $F=0$ and $V$ being the inverse Frobenius.
\end{example}
\begin{example}
	Let $\alpha_p$ denote the kernel of the $p$th-power map $\mathbb G_a\to\mathbb G_a$. Then $\mathbb D(\alpha_p)\cong k$ with $F=V=0$.
\end{example}
\begin{example}
	Fix a perfect field $k$ of positive characteristic, and let $A$ be an abelian $k$-variety. Then we have $\mathbb D(A[p])\cong H^1_{\mathrm{dR}}(A)$. (This isomorphism goes through the crystalline site.) In fact, there is an isomorphism of short exact sequences as follows.
	% https://q.uiver.app/#q=WzAsMTAsWzAsMCwiMCJdLFsxLDAsIkheMChBLFxcT21lZ2Ffe0Eva30pIl0sWzIsMCwiSF4xX3tcXG1hdGhybXtkUn19KEEpIl0sWzMsMCwiSF4xKEEsXFxPT19BKSJdLFs0LDAsIjAiXSxbMSwxLCIoayxcXHNpZ21hXnstMX0pXFxvdGltZXNfa1xcbWF0aGJiIEQoQVtGXSkiXSxbMiwxLCJcXG1hdGhiYiBEKEFbcF0pIl0sWzMsMSwiXFxtYXRoYmIgRChBW1ZdKSJdLFs0LDEsIjAiXSxbMCwxLCIwIl0sWzksNV0sWzUsNl0sWzYsN10sWzcsOF0sWzAsMV0sWzEsMl0sWzIsM10sWzMsNF0sWzEsNV0sWzIsNl0sWzMsN11d&macro_url=https%3A%2F%2Fraw.githubusercontent.com%2FdFoiler%2Fnotes%2Fmaster%2Fnir.tex
	\[\begin{tikzcd}
		0 & {H^0(A,\Omega_{A/k})} & {H^1_{\mathrm{dR}}(A)} & {H^1(A,\OO_A)} & 0 \\
		0 & {(k,\sigma^{-1})\otimes_k\mathbb D(A[F])} & {\mathbb D(A[p])} & {\mathbb D(A[V])} & 0
		\arrow[from=2-1, to=2-2]
		\arrow[from=2-2, to=2-3]
		\arrow[from=2-3, to=2-4]
		\arrow[from=2-4, to=2-5]
		\arrow[from=1-1, to=1-2]
		\arrow[from=1-2, to=1-3]
		\arrow[from=1-3, to=1-4]
		\arrow[from=1-4, to=1-5]
		\arrow[from=1-2, to=2-2]
		\arrow[from=1-3, to=2-3]
		\arrow[from=1-4, to=2-4]
	\end{tikzcd}\]
	Here, $(k,\sigma^{-1})$ denotes
\end{example}
So here is another characterization of an elliptic curve $E$ being supersingular.
\begin{itemize}
	\item Ordinary: $F^*\colon H^1(E,\OO_E)\to H^1(E,\OO_E)$ is nonzero; equivalently, $V^*\colon H^0(E,\Omega_{E/k})\to H^0(E,\Omega_{E/k})$ is nonzero.
	\item Supersingular: otherwise.
\end{itemize}
For example, suppose $E/k$ is ordinary. Note that $V$ vanishes on $\mathbb D(E[V])$, so we get $\mathbb D(E[V])=\mathbb D(\underline{\ZZ/p\ZZ})$. Similarly, $F$ vanishes on $\mathbb D(A[F])$, so we get $\mathbb D(\mu_p)$. Thus, we get a short exact sequence
\[0\to\mathbb D(\mu_p)\to\mathbb D(E[p])\to\mathbb D(\underline{\ZZ/p\ZZ})\to0,\]
which upon reversing $\mathbb D$ produces
\[0\to\ZZ/p\ZZ\to E[p]\to\mu_p\to0.\]
This splits at $\ZZ/p\ZZ\to E[p]$ by the Frobenius, so $E[p]\cong\mu_p\oplus\ZZ/p\ZZ$.

On the other hand, the supersingular case will end up producing a short exact sequence
\[0\to\alpha_p\to E[p]\to\alpha_p\to0,\]
which now need not split.

\subsection{\texorpdfstring{$F$}{ F}-zips}
Let $X/k$ be a smooth proper $k$-scheme. As a technical hypothesis, we want the Hodge to de Rham spectral sequence degenerates at $E_1$, though I'm not totally sure what that means. In this situation, we get two filtration.
\begin{itemize}
	\item Hodge filtration: $H^1_{\mathrm{dR}}(X)\supseteq\mathrm{Fil}^1_H\supseteq\mathrm{Fil}^2_H\cdots\supseteq0$. Set $C_i\coloneqq\mathrm{Fil}^i_H$ for brevity.
	\item Conjugate filtration: there is an analogous filtration $H^1_{\mathrm{dR}}(X)\supseteq\ov{\mathrm{Fil}^1_H}\supseteq\ov{\mathrm{Fil}^2_H}\cdots\supseteq0$. Set $D_i\coloneqq\ov{\mathrm{Fil}}_H^{n-i}$ for brevity.
\end{itemize}
In this situation, we will get a Cartier isomorphism $\sigma^*(C^i/C^{i+1})\to(D_i/D_{i-1})$.
\begin{example}
	Let $A/k$ be an abelian variety.
	\begin{itemize}
		\item We have $\mathbb D(A[p])=H^1_{\mathrm{dR}}(A)$.
		\item The first filtration: $H^1_{\mathrm{dR}}(A)\supseteq\ker F\supseteq0$.
		\item The second filtration: $0\subseteq\ker V\subseteq H^1_{\mathrm{dR}}(A)$.
		\item The Cartier isomorphism: $\im F=\ker V$ and $\ker F=\im V$.
	\end{itemize}
\end{example}
We now package all this data into an $F$-zip.
\begin{definition}[$F$-zip]
	Fix an $\mathbb F_q$-scheme $S$. Then an \textit{$F$-zip} over $S$ is a tuple $(M, C^\bullet, D_\bullet,\varphi_\bullet)$ satisfying some coherence conditions. We define its \textit{type} as the map $\tau\colon\ZZ\to\ZZ_{\ge0}$ by $\tau(i)\coloneqq\dim_k\left(C^i/C^{i+1}\right)$.
\end{definition}
We now want to understand $F$-zips.
\begin{refcontext}
	Continue with $A/k$ as an abelian variety. Then a polarization on $A$ induces a symplectic form on $H^1_{\mathrm{dR}}(A)$.
\end{refcontext}
So actually we want to understand $F$-zips with this extra symplectic structure.
\begin{definition}[symplectic $F$-zip]
	Fix everything as above. A \textit{symplectic $F$-zip} is an $F$-zip $(M,C^\bullet,D_\bullet,\varphi_\bullet)$ such that there is a symplectic form $\psi$ on $M$, with some coherence conditions. For example, we want $C^\bullet$ and $D_\bullet$ to be symplectic flags (i.e., the symplectic dual spaces of an element of $C^\bullet$ lives in $C^\bullet$, and similar for $D_\bullet$).
\end{definition}
So here is a classification result.
\begin{theorem} \label{thm:classify-symp-f-zip}
	Let $k$ be algebraically closed, and let $(V,\psi)$ be a symplectic $k$-vector space and let $G=\op{Sp}(V,\psi)$ with Weyl group $(W,I)$. Let $\tau$ be an ``admissible type'' (namely, on the type of our $F$-zips). Then there is a bijection between isomorphism classes of symplectic $F$-zips of type $\tau$ and $W_j\backslash W$.
\end{theorem}
The point is that $F$-zips can be understood from ``combinatorial data'' from the Weyl group, which are what produce the Ekedahl--Oort stratification.

\section{January 31st: Sean Gonzales}
Today we're going to define a Shimura datum. To review, let's do an example using \Cref{thm:classify-symp-f-zip}.
\begin{example}
		As usual, fix a perfect field $k$ of positive characteristic $p$, and let $E$ be an elliptic $k$-curve. Then $W={\op{GSp}_2}={\op{GL}_2}$, where our vector space is $H^1_{\mathrm{dR}}(E)\cong k^2$. Fixing a basis $\{e_1,e_2\}$ corresponding to the action, our $F$-zip can come in two forms.
		\begin{itemize}
			\item Ordinary: $C^\bullet\colon0\subseteq ke_1\subseteq k^2$ and $D_\bullet\colon 0\subseteq ke_2\subseteq k^2$.
			\item Supersingular: $C^\bullet\colon 0\subseteq ke_1\subseteq k^2$ and $D_\bullet\colon 0\subseteq ke_1\subseteq k^2$.
		\end{itemize}
		Notably, ordinary is $(1,2)\in W$, and supersingular is $\id$.
\end{example}

\subsection{Shimura Datum Examples}
A Shimura datum will consist of a pair $(G,X)$. Instead of giving a precise definition now, we write out some examples.
\begin{example}
	Elliptic curves over $\CC$ can be written as $\CC/\Lambda$, where $\Lambda=\ZZ\oplus\ZZ\tau$ is a lattice, where $\tau\in\HH$. Equivalently, we can imagine fixing $\Lambda\coloneqq\ZZ^2$ and choose an embedding $j\colon\RR^2\to\CC$. The point is that choice of $\tau\in\HH$ then defines the map $\RR^2\to\CC$ given by $(0,1)\mapsto\tau$, which is equivalently defining a map $\CC^\times\to\op{GL}_2(\RR)$.

	The point of thinking this way is that the map $\CC^\times\to\op{GL}_2(\RR)$ is really a map $h\colon\mathbb S\to\op{GL}_{2,\RR}$ where $\mathbb S\coloneqq\op{Res}_{\CC/\RR}(\mathbb G_{m,\CC})$ is the Deligne torus. With this viewpoint, $(\Lambda,h)$ is a $\ZZ$-Hodge structure: $\Lambda\otimes_\CC$ has basis given by $\tau$ and something else, where the point is that $h$ acts by conjugation on one basis vector and identity on the other one.

	Anyway, taking $X$ to be the conjugacy class of a particular $h$ (namely, $i\mapsto\begin{bsmallmatrix}
		0 & -1 \\ 1 & 0
	\end{bsmallmatrix}$) has $\op{Sh}({\op{GL}_2},X)$ being the needed modular curve. This Shimura datum ``explains'' how $2$-dimensional $\ZZ$-Hodge structures correspond to elliptic curves.
\end{example}
\begin{example}
	Abelian varieties over $\CC$ can be written as $\CC^g/\Lambda$ with a Riemann form $\psi\colon\Lambda\times\Lambda\to\ZZ$. Again, we can imagine this as fixing $\Lambda\coloneqq\ZZ^{2g}$ and then choosing an embedding $\RR^{2g}\cong\CC^g$, but this is equivalent to choosing a map $h\colon\CC^\times\to\op{GSp}_{2g}(\psi)$. Then one can tell much the same story, producing a Shimura datum $\op{Sh}({\op{GSp}_{2g}(\psi)},X)$.
\end{example}
\begin{example} \label{ex:non-split-sh}
	Let's try to parameterize elliptic curves $E$ over $\CC$ with an embedding $i\colon\ZZ[i]\to\op{End}_\CC(E)$. The elliptic curve itself becomes $2$-dimensional Hodge structure, but we should now have some additional $\ZZ[i]$-module structure. Notably, it's not even clear what our group is.

	Well, set $\Lambda\coloneqq\ZZ^2$ as usual, and provide it with $\ZZ[i]$-action in the usual way by $i\mapsto\begin{bsmallmatrix}
		0 & -1 \\ 1 & 0
	\end{bsmallmatrix}$. So our group $G$ should have $G(R)$ be the automorphisms of $\Lambda\otimes_\ZZ R$ commuting with the given action of $\ZZ[i]\otimes_\ZZ R$, which is approximately $R[i]^\times$. So our group ought to be $\op{Res}_{\QQ(i)/\QQ}(\mathbb G_{m,\QQ(i)})$. Notably, this group isn't even split!
\end{example}
We are approaching the end of the talk, so we may as well define something.
\begin{definition}[reflex field]
	Fix $(G,X)$. Then the \textit{reflex field} $E$ of $(G,X)$ is the fixed field of the subgroup of $\op{Gal}(\ov{\QQ}/\QQ)$ which fixes the conjugacy class of the map $z\mapsto h_\CC(z,1)$. (This is algebraic over $\QQ$ for reasons we will not explain.)
\end{definition}
For good enough primes $p$ (for example, we want $G$ to be unramified at $p$, i.e. split over $\QQ_p$), one can reduce $\op{Sh}(G,X)$ modulo $\mf p\mid p\OO_E$, where $\mf p\in V(E)$.
\begin{example}
	We continue \Cref{ex:non-split-sh}. Odd primes $p$ are good enough. Quickly, note that we have a reductive model of $G$ over $\ZZ_p$ given by
	\[G(R)\coloneqq\op{GL}_{\ZZ_p[i]\otimes R}\left(\ZZ_p^2\otimes R\right).\]
	Thus, for example if $p\equiv1\pmod4$, then $\ZZ_p[i]$ splits into $\ZZ_p\times\ZZ_p$, so we are looking at $\op{GL}_{R^2}\left(R^2\right)$, which is $R^\times\times R^\times$. This is $\mathbb G_m\times\mathbb G_m$, which reduces$\pmod p$ just fine. Going back to the moduli problem, one can track back through to see that we are looking for elliptic $\FF_p$-curves $E$ equipped with a map $\ZZ[i]\to\op{End}(E)$, which is equivalent to being ordinary!
\end{example}

\section{February 7th: Sean Gonzales}
Today we will talk about Hasse invariants and their generalizations. Throughout today, $k$ is a field of positive characteristic $p\coloneqq\op{char}k>0$.

\subsection{Hasse Invariant}
Here is a definition, which is perhaps not so helpful.
\begin{definition}[Hasse invariant]
	Fix an elliptic curve $E$ over $k$. Then the \textit{Hasse invariant} $h_E$ of $E$ is $0$ if $E$ is supersingular and is $1$ if $E$ is ordinary.
\end{definition}
Here is a more advanced definition, which is the same upon defining supersingular and ordinary.
\begin{definition}[Hasse invariant]
	Fix an elliptic curve $E$ over $k$. Let $F\colon E\to E$ denote the absolute Frobenius. Then the \textit{Hasse invariant} $h_E$ of $E$ is $0$ if $F^*\colon H^1(E,\OO_E)\to H^1(E,\OO_E)$ is the zero map and $1$ if $F^*$ is nonzero (i.e., an isomorphism because $H^1(E,\OO_E)\cong k$).
\end{definition}
So we've introduced some cohomology. In fact, by using Serre duality, we can move everything into cohomology.
\begin{definition}[Hasse invariant]
	Fix an elliptic curve $E$ over $k$, and let $\omega$ be a basis of $H^0(E,\Omega_E)$, which corresponds canonically to a basis element $\eta\in H^1(E,\OO_E)^\lor$ by Serre duality. Letting $F\colon E\to E$ be the absolute Frobenius, then the \textit{Hasse invariant} is the unique $h_{E,\omega}\in k$ such that
	\[F^*\eta=h_{E,\omega}\eta.\]
\end{definition}
\begin{remark}
	Adjusting $\omega\mapsto\lambda\omega$ makes $\eta\mapsto\lambda^{-1}\eta$, so $h_{E,\lambda\omega}=\lambda^{1-p}h_{E,\omega}$. Thus, one can view $h_{E,\omega}$ has a level $1$, weight $p-1$ modular form$\pmod p$.
\end{remark}
\begin{remark}
	Locally, we can think about $h_{E,\omega}$ as an element of $H^0\left(E,\underline\omega_E^{\otimes(p-1)}\right)$.
\end{remark}
We would like to understand these ``modular forms'' as sections of some line bundles. I need to focus on the exposition, so I am going to stop taking notes.

\section{February 14th: Connor James Halleck-Dube}
Today we're talking about trace formulae and orbital integrals.

\subsection{The Trace Formula}
Fix locally compact topological group $G$. Representation theory might be interested in the decomposition of $L^2(G)$. It turns out $L^2(G)$ is pretty big, so perhaps we might have access to a discrete subgroup $\Gamma\subseteq G$, and we'll ask for representations living $L^2(\Gamma\backslash G)$. If $\Gamma\backslash G$ is in fact compact, then $L^2(\Gamma\backslash G)$ will decompose as a Hilbert space into a sum of irreducible representations, as one expects from representation theory of finite groups.
\begin{example} \label{ex:r-mod-z}
	The representations of $\RR$ living in $L^2(\ZZ\backslash\RR)$ is really asking for representations to $S^1$, which is basically Fourier analysis. Explicitly, one can argue that
	\[L^2(\ZZ\backslash\RR)\cong L^2(\ZZ)=\widehat{\bigoplus_{n\in\ZZ}}\RR_n,\]
	where $\RR$ acts on $\RR_n$ via the character $r\mapsto\exp(2\pi inr)$. As a remark, one can even decompose $L^2(\RR)$, but the decomposition is more complicated because it is no longer ``discrete.''
\end{example}
\begin{example} \label{ex:finite-group}
	Fix a finite group $G$ and $\Gamma\subseteq G$ to be any subgroup. Then we are asking for irreducible representations of $G$ when $\Gamma=1$, and in general we are asking to decompose $\op{Ind}_\Gamma^G1$.
\end{example}
\begin{example} \label{ex:adele}
	Fix a global field $F$, and let $\AA_F$ be the ad\'ele ring. Given a reductive $F$-group $G$, one has a discrete inclusion $G(F)\to G(\AA_F)$. The representation theory here is related to the story of modular forms, where $G$ is a real reductive Lie group and $\Gamma\subseteq G$ an arithmetic subgroup.
\end{example}
In our representation theory, one often wants to pick up some tools to do functional analysis.
\begin{itemize}
	\item One can upgrade the $G$-action on $L^2(\Gamma\backslash G)$ to a map $R\colon C_c(G)\to\op{End}L^2(\Gamma\backslash G)$ in a way that agrees with the earlier $G$-action if we view $G$ in $C_c(G)$ as Dirac $\delta$s. Explicitly, given $f\in C_c(G)$, we have
	\[(Rf)(\varphi)(g)\coloneqq\int_Gf(h)\varphi(gh)\,dh.\]
	The point is that if we imagine $f$ as an indicator for some $g\in G$, then we are looking at the desired right translation. Notably, our Haar measure permits
	\[(Rf)(\varphi)(g)=\int_Gf\left(g^{-1}h\right)\varphi(h)\,dh.\]
	Now, $\varphi$ is really a function on $\Gamma\backslash G$, so we can first integrate over the quotient as
	\[(Rf)(\varphi)(g)=\int_{\Gamma\backslash G}\Bigg(\sum_{\gamma\in\Gamma}f\left(g^{-1}\gamma h\right)\Bigg)\varphi(h)\,dh.\]
	So we are looking at an integral operator via the kernel $K_f(g,h)\coloneqq\sum_{\gamma\in\Gamma}f\left(g^{-1}\gamma h\right)$. Note that this is in fact a finite sum because $\op{supp}f$ is compact and $\Gamma$ is discrete.

	\item Now that we have some ring homomorphism, we can define some traces. For some reason, the kernel $K(g,h)$ is approximately analogous to some matrix entries, so we attempt to define
	\[\tr Rf\coloneqq\int_{\Gamma\backslash G}K_f(h,h)\,dh.\]
	When $\Gamma\subseteq G$ is cocompact, this integral will always exist, but in general it is a hypothesis on the function that we can take its trace as above.

	Expanding, we can do the same sort of conjugacy class dependence of $f$ to see
	\[\tr Rf=\int_{\Gamma\backslash G}\Bigg(\sum_{\gamma\in\Gamma}f\left(h^{-1}\gamma h\right)\Bigg)dh=\sum_{\text{class }[\gamma]}\op{vol}(\Gamma_\gamma\backslash G_\gamma)\int_{G_\gamma\backslash G}f\left(h^{-1}\gamma h\right)\,dh.\]
	Here $G_\gamma$ and $\Gamma_\gamma$ are stabilizers. This is called the ``geometric expansion'' of the trace formula, and the integrals are ``orbital integrals'' because we are integrating over orbits.

	\item On the other side, under the hypothesis that $\Gamma\subseteq G$ is cocompact, we decompose
	\[L^2(\Gamma\backslash G)=\widehat{\bigoplus_{\alpha\in\kappa}}\pi_\alpha^{m_\alpha},\]
	so by decomposing the action of some $Rf$ here, we receive
	\[\tr Rf=\sum_{\alpha\in\kappa}m_\alpha\tr R(f)|_{\pi_\alpha}.\]
	This is called the ``spectral side.''
\end{itemize}
So we have proven the following.
\begin{theorem}[Trace formula] \label{thm:tr-form}
	Fix a cocompact discrete subgroup $\Gamma$ of a locally compact topological group $G$. Given $f\in C_c(G)$, one has
	\[\sum_{\text{class }[\gamma]}\op{vol}(\Gamma_\gamma\backslash G_\gamma)\int_{G_\gamma\backslash G}f\left(h^{-1}\gamma h\right)\,dh=\sum_{\alpha\in\kappa}m_\alpha\tr R(f)|_{\pi_\alpha}.\]
\end{theorem}
\begin{proof}
	Both sides equal $\tr R(f)$.
\end{proof}
The point is that we can try to relate some ``geometric'' orbital integrals with ``spectral'' information on irreducible representations.
\begin{example}
	Continue from \Cref{ex:r-mod-z}. For some $f\in C_c(\ZZ\backslash\RR)$, computing \Cref{thm:tr-form} produces
	\[\sum_{n\in\ZZ}f(n)=\sum_{n\in\ZZ}\widehat f(n).\]
	The geometric side is direct, and the spectral side depends on the Fourier expansion of $f$.
\end{example}
\begin{example}
	Continue from \Cref{ex:finite-group}. Then one recovers Burnside's lemma or Frobenius reciprocity.
\end{example}
\begin{example}
	Selberg was able to make \Cref{thm:tr-form} work in the case where $\op{SL}_2(\ZZ)\subseteq\op{SL}_2(\RR)$. Selberg then used this along with the fact that geodesics of quotients $\Gamma\backslash\HH$ with conjugacy classes in $\Gamma$, where (say) $\Gamma$ is an arithmetic subgroup; comparing this with \Cref{thm:tr-form} allows one to count geodesics by understanding the relevant representation theory.
\end{example}
From here, Arthur extended \Cref{thm:tr-form} to allow $G(F)\subseteq G(\AA_F)$ where $G$ is reductive. For example, if $G$ has any parabolic subgroup or any split torus, we are horribly not compact, so the geometric side of \Cref{thm:tr-form} can become infinite. The idea was to modify \Cref{thm:tr-form} via adding in some alternating sum of parabolic subgroups in order to cancel out some infinities.

\subsection{Orbital Integrals}
Let's examine our orbital integrals
\[O_\gamma(f)\coloneqq\int_{G_\gamma(\AA_F)\backslash G(F)}f\left(g^{-1}\gamma g\right)\,dg,\]
where $\gamma\in\Gamma$ and $f\in C_c^\infty(G(\AA_F))$ (where the $\infty$ means some smoothness that we will not be precise about). Here, $F$ is a global field, so working over $\AA_F$ allows us to perhaps decompose the integral into a product of local orbital integrals.
\begin{example}
	Take $G\coloneqq\op{GL}_2$ and $F_v\coloneqq\QQ_p$ and $\gamma\coloneqq\begin{bsmallmatrix}
		a \\ & b
	\end{bsmallmatrix}$ for $a-b\in\OO_v^\times$. Then we can ask for the integral
	\[O_\gamma=\int_{G_\gamma(F_v)\backslash G(F_v)}1_{g^{-1}\gamma g\in G(\OO_v)}\,dg,\]
	which we can compute via some explicit combinatorics. For example, it turns out to be a rational function in $p$.
\end{example}

\section{March 20th: Seewoo Lee}
Today we're talking about linear programming. In short, linear programming is optimizing a linear functional over linear constructs.
\begin{remark}
	Integer linear programming is NP-complete; in particular, it cannot be so easily solved.
\end{remark}

\subsection{Sphere Packing}
For a dimension $d$, let $\Delta_d$ be the optimal density for $\RR^d$.
\begin{itemize}
	\item One can fully tile $\RR$, so $\Delta_1=1$.
	\item Thue proved that $\Delta_2=\pi/(2\sqrt3)$, given by a hexagonal lattice.
	\item Kepler conjectured and Hales proved that $\Delta_3=\pi/(3\sqrt2)$. One difficulty here is that there are uncountably many different packings with the same density, which one must do by hand.
	\item We have a conjecture for $\Delta_4$ but no current proof.
	\item Viazovska in 2016 shows that the $E_8$ lattice was optimal for $8$ dimensions, proving $\Delta_8=\pi^4/384$.
	\item Then Cohn, Kumar, Miller, Radchenko, and Viazovska showed that the Leech lattice is optimal in $24$ dimensions, proving $\Delta_{24}=\pi^{12}/12!$.
\end{itemize}
The bottom two results used ``linear programming bounds'' due to Cohn and Elkies.
\begin{theorem}
	Fix a Schwartz function $f\colon\RR^d\to\RR$ with the following properties.
	\begin{itemize}
		\item $f(x)\le0$ for $\norm x\ge1$.
		\item $\widehat f(y)\ge0$ always. Here, $\widehat f$ denotes the Fourier transform $\widehat f(y)\coloneqq\int_{\RR^d}f(x)e^{i\langle x,y\rangle}\,dx$.
	\end{itemize}
	Then the center density (i.e., the number of permitted centers per unit volume) is bounded above by
	\[\frac1{2^d}\cdot\frac{f(0)}{\widehat f(0)}.\]
\end{theorem}
In practice, one finds that we may assume that the function is radial, so a reasonable choice is
\[f(x)=p(\norm x)e^{-\alpha\norm x^2},\]
where $p$ is some polynomial. Then we are basically trying to maximize the quantity in the inclusion subject to the constraints given by the hypotheses.

One can also prove the following variant of the above.
\begin{theorem}
	Fix a Schwartz function $f\colon\RR^d\to\RR$ and real number $r>0$ with the following properties.
	\begin{itemize}
		\item $f(x)=\widehat f(0)>0$.
		\item $f(x)\le0$ whenever $\norm x\ge r$.
		\item $\widehat f(y)\ge0$ always.
	\end{itemize}
	Then
	\[\Delta_d\le\op{vol}B(0,r/2).\]
\end{theorem}
\begin{remark}
	Viazovska in her work was able to construct functions providing the optimal bound for $d\in\{8,24\}$. It is expected that one works for $d=2$, but it is known yet.
\end{remark}
\begin{proof}[Proof for lattice packings]
	Let $\Lambda$ be the lattice, and let $r$ be the smallest nonzero distance between any two vectors in $\Lambda$, which then gives our packing. Then Poisson summation implies
	\[\sum_{x\in\Lambda}f(x)=\frac1{\op{vol}\left(\RR^d/\Lambda\right)}\widehat f(y).\]
	The left-hand side is bounded above by $f(0)$, and the right-hand side is bounded below by $\widehat f(0)/\op{vol}\left(\RR^d/\Lambda\right)$. This inequality produces the result.
\end{proof}
Let's discuss how one might hope to construct the best possible function $f$. The proof showed that we need $f(x)=0$ for nonzero $x\in\Lambda$ (based on us just throwing out those terms) and $\widehat f(y)=0$ for nonzero $y$ in the dual lattice (for the same reason). But we also need sign conditions, so almost all these roots need to be double roots. In theory these sorts of conditions tell us what the best possible function $f$ will look like. Indeed, Viazovska managed to construct functions like this as $f=f_++f_-$ where $\widehat f_+=f_+$ and $\widehat f_-=-f_-$ and 
\[f_\pm(x)=\sin^2\left(\frac{\pi\norm x^2}2\right)\int_0^\infty\psi_\pm(it)e^{-\pi\norm x^2t}\,dt.\]
It turns out that $\psi_\pm$ being a Fourier eigenfunction essentially corresponds to $\psi_\pm$ being a (quasi-)modular form with explicit weight, depth, and level. The space of such (quasi-)modular forms is finite-dimensional, and then one can solve some system of equations to optimize.
\begin{remark}
	One can use quadratics instead of linear programming to pack spheres, basically by adding in another positive-definite function $f_3$. This is able to provide better bounds than linear programming; indeed, one recovers linear programming bounds by setting $f_3\equiv0$. It is conjectured that this is optimal at $d=4$.
\end{remark}
\begin{remark}
	There is a ``dual'' linear programming bound. Techniques like these are able to show that linear programming is provably suboptimal for $d\in\{3,4,5,6,12,16\}$.
\end{remark}

\end{document}
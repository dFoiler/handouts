\documentclass{article}
\usepackage[utf8]{inputenc}

\newcommand{\nirpdftitle}{Student Arithmetic Geometry Seminar}
\usepackage{import}
\inputfrom{../../notes}{nir}
\usepackage[backend=biber,
    style=alphabetic,
    sorting=ynt
]{biblatex}
\setcounter{tocdepth}{2}

\pagestyle{contentpage}

\title{Student Arithmetic Geometry Seminar}
\author{Nir Elber}
\date{Fall 2024}
\usepackage{graphicx}

\begin{document}

\maketitle

\tableofcontents

\section{August 30: Martin Olsson}
Today is an organizational meeting. There is no paper list yet (but soon), but almost all dates for talks have been taken anyway. The papers that Professor Olsson has in mind are along the lines of ``how to do birational geometry with stacks.''

\subsection{Geometric Invariant Theory}
There is a classical book by Mumford and Fogarty on geometric invariant theory. More recently there is some hope to do this theory over a more general base by Seshadri and some theory of ``adequate'' moduli spaces by Alper.

For today, we will over a Noetherian ring $R$, and let $G$ be a smooth, affine, connected group scheme with reductive geometric fibers, which we may just call a reductive group scheme over $R$.
\begin{example}
    The group $G=\mathrm{GL}_{n,R}$ and the other classical groups are an example, but $\mathbb G_{a,R}$ is not.
\end{example}
\begin{example}
    We won't define reductive, but here is one way to access the notion: examples of reductive group schemes are the linearly reductive group schemes whose category of representations is semisimple, and these are all the examples in characteristic $0$ (but not in characteristic $p$).
\end{example}
In Alper's story, a linearly reductive group scheme corresponds to a ``good'' moduli space, but a reductive group scheme corresponds to an ``adequate'' moduli space. (We have not said what ``corresponds'' means.)

For our affine story of geometric invariant theory, one has an affine $R$-scheme $X=\Spec A$ equipped with a $G$-action, which amounts to a morphism $G\times X\to X$ with some special properties. Everything in sight is affine, so we can also think about this as a morphism $A\to A\otimes\OO_G$ with some special properties. In general, there is a quotient map $X\to [X/G]$, where $[X/G]$ is some stack, and then $[X/G]$ maps onto $Y\coloneqq\Spec A^G$, and $\Spec A^G$ is perhaps the ``affine quotient.'' Here is the visual.
\[X\to[X/G]\stackrel\pi\to Y.\]
Geometric invariant theory now roughly divides into two steps.
\begin{enumerate}
    \item Find a substack $\mc X^s$ of $[X/G]$ with finite diagonal. (This roughly corresponds to finding the points with closed orbits by the $G$-action.)
    \item Find the course moduli space of $\mc X^s$.
\end{enumerate}
Let's see an example.
\begin{example}
    We work over a field $k$; fix some integers $a_0,\ldots,a_n\in\ZZ$. Now, we let $\mathbb G_m$ act on $\AA_k^{n+1}=\Spec k[x_0,\ldots,x_n]$ by
    \[u*x_i\coloneqq u^{a_i}x_i.\]
    Note that having $a_i>0$ for all $i$ implies that $A^G=k$, so $A^G$ is quite small!
\end{example}
We are able to execute the first step above, which we do in steps.
\begin{enumerate}
    \item Given a geometric point $\ov x$ of $[X/G]$, let $G_{\ov x}$ denote the stabilizer (which is some subgroup scheme).
    \item Then we let $\mc U\subseteq[X/G]$ be the maximal open subscheme containing the $\ov x$ for which $G_{\ov x}$ is finite.
    \item It turns out that $\pi(\mc X\setminus\mc U)\subseteq Y$ is closed; we let this subset be $Z$.
    \item It now turns out that $\mc X^s\coloneqq\pi^{-1}(Y\setminus Z)$ will do the trick.
\end{enumerate}
Let's work through an example. Continue with a field $k$, and now take two integers $a,b\in\ZZ$, and we are able to let $\mathbb G_m$ act on $A\coloneqq\Spec k[x,y]$ by $u*x\coloneqq u^ax$ and $u*y\coloneqq u^by$. On rings, this map is as follows.
\[\arraycolsep=1.4pt\begin{array}{cccccc}
    k[x,y] &\otimes& k[u,1/u] &\from& k[x,y] \\
    x &\otimes& u^a &\mapsfrom& x \\
    y &\otimes& u^b &\mapsfrom& y
\end{array}\]
Let's do some cases.
\begin{listalph}
    \item Suppose $a=0$ and $b\ne0$; the case $a\ne0$ and $b=0$ is symmetric. Now, it turns out we can only check monomials, and we find that $A^{\mathbb G_m}=k[x]$. But there is some extra stacky information because having $y$ nonzero makes our point have stabilizer $\mu_b$ (which is the $b$th roots of unity); when $y=0$, our stabilizer is actually a full $\mathbb G_m$! Thus, we can compute that $\mc U$ is $[\Spec k[x,y,1/y]/\mathbb G_m]$, so we will find that $\mc X^s$ is empty!

    \item Suppose $a>0$ and $b<0$, and set $g\coloneqq\gcd(a,b)$. On monomials, we find
    \[u*x^\alpha y^\beta=u^{a\alpha+b\beta}x^\alpha y^\beta,\]
    so one can calculate that $A^{\mathbb G_m}=\Spec k[w]$, where $w\coloneqq x^{b/g}y^{a/g}$.

    Now, if $\alpha=0$ or $\beta=0$, then our stabilizer is small (something like $\mu_a$ or $\mu_b$ again), and our orbit fails to be closed. But if both are nonzero, then we are cutting out some subvariety which looks like
    \[x^{b/g}y^{a/g}=\alpha\beta,\]
    which is closed with stabilizer small. As a result, our $\mc U$ is everything minus the origin, so $Z=0$, so we find that $\mc X^s$ consists of the points not on the axes modulo $\mathbb G_m$.
\end{listalph}
\begin{remark}
    We close by making a quick remark on how to add a character to this story. With our group $G$, we may equip a character $\chi\colon G\to\mathbb G_m$, and then we can try to understand
    \[A_\chi\coloneqq\{g\in A:g*f=\chi(g)f\}.\]
\end{remark}

\section{September 6th: Martin Olsson}
Today we are giving an introduction of algebraic stacks ``for the working mathematician.''

\subsection{Algebraic Stacks}
We work over a base scheme $S$. Here is an okay definition.
\begin{definition}[algebraic stack]
    An \textit{algebraic stack} is a functor $\mf X\colon\mathrm{Sch}\opp\to\mathrm{Groupoids}$ satisfying the following.
    \begin{listalph}
        \item Descent: $\mf X$ is a sheaf for the \'etale topology.
        \item The diagonal of $\mf X$ is representable by a scheme.
        \item $\mf X$ admits a smooth cover by a scheme.
    \end{listalph}
\end{definition}
\begin{remark}
    Here, $\mathrm{Groupoids}$ is a category of categories (where all morphisms are isomorphisms). One must be a rather careful to explain what a functor valued in groupoids actually is.
\end{remark}
\begin{remark}
    The ``algebraic'' part of ``algebraic stack'' arises from (b) and (c).
\end{remark}
Intuitively, a stack should be thought of as a scheme with some ``stacky points'' that have some larger automorphism group; for example, quotient stacks can be thought of in this way. (This is somewhat similar to orbifolds in differential topology.) Our goal is to turn the above definition into this intuition.

One way to produce stacks is by group actions.
\begin{definition}[principal homogeneous space]
    Fix an affine group scheme $G$ over $S$. Then a \textit{principal homogeneous space} under $G$ is a flat surjective scheme $P\to S$ with $G$-action such that the induced map $G\times_SP\to P\times_SP$ given by $(g,x)\mapsto(gx,x)$ is an isomorphism.
\end{definition}
\begin{example}
    There is an equivalence of groupoid categories between invertible sheaves over $S$ and principal homogeneous spaces under $\mathbb G_m$. In one direction, we take the line bundle $\mc L$ to the scheme representing the functor $\op{Isom}(\mc L,\mc O_S)$, where the $\mathbb G_m$-action arises from its action on $\OO_S$. One can check that $\op{Isom}(\mc L,\OO_S)$ is isomorphic to
    \[\Spec_S\left(\bigoplus_{n\in\ZZ}\mc L^{\otimes n}\right),\]
    where the $\mathbb G_m$-action on $\mc L^{\otimes n}$ is given by the $n$th power of $\mathbb G_m$ on $\mc L$.
\end{example}
\begin{example}
    Given a $G$-action on a scheme $U$, let's try to make sense of the algebraic stack $\mf X\coloneqq[U/G]$. Well, given a test scheme $T\to S$, we produce the groupoid of diagrams
    % https://q.uiver.app/#q=WzAsNCxbMCwwLCJQIl0sWzAsMSwiVCJdLFsxLDEsIlMiXSxbMSwwLCJVIl0sWzEsMl0sWzMsMl0sWzAsMywiXFx2YXJwaGkiXSxbMCwxLCJHIiwyXV0=&macro_url=https%3A%2F%2Fraw.githubusercontent.com%2FdFoiler%2Fnotes%2Fmaster%2Fnir.tex
    \[\begin{tikzcd}
        P & U \\
        T & S
        \arrow["\rho", from=1-1, to=1-2]
        \arrow["G"', from=1-1, to=2-1]
        \arrow[from=1-2, to=2-2]
        \arrow[from=2-1, to=2-2]
    \end{tikzcd}\]
    where $P\to T$ is a principal homogeneous space under $G$. We will write down our objects as pairs $(P,\rho)$. Intuitively, one can think of the object $(P,\rho)$ as being related to its image in $U$, which is approximately a $G$-orbit in $U$.
\end{example}
Let's explain this last example a little more.
\begin{listalph}
    \item Descent follows by some kind of faithfully flat descent for affine schemes.
    \item Approximately speaking, (b) is asking for the functor $\op{Isom}_{\mf X}((P,\rho),(P',\rho'))\colon\op{Sch}_T\opp\to\mathrm{Set}$ to be representable. Here, our isomorphisms need to be isomorphisms of the principal homogeneous spaces (namely, commuting with the $G$-action) also commuting with the $\rho$s.
    \item Quickly, note that there is a ``tautological'' principal homogeneous space $G\times U\to U$, which we will label $(P_0,\rho_0)$. If $G$ is smooth, then (c) is the statement that the map
    \[\op{Isom}_{U\times_ST}((P_0,\rho_0),(P,\rho))\to T\]
    is smooth and surjective, which is not obvious but true.
\end{listalph}
Here is are some more examples.
\begin{example}
    We attempt to take a quotient stack $\mf X\coloneqq[U/\mathbb G_m]$. Then our objects in some groupoid $\mf X(T)$ are principal homogeneous spaces over $\mathbb G_m$, which we now understand to be line bundles. Thus, for example, taking two line bundles $\mc L$ and $\mc L'$, (b) is asking for
    \[\op{Isom}_T(\mc L,\mc L')\cong\op{Isom}_T(\mc L'\otimes\mc L^\lor,\OO_T)\]
    to be representable. One can check this sometimes, I suppose.
\end{example}
\begin{example}
    Take $G=\mu_p$ over a base field $k$ of positive characteristic $p>0$. We try to consider $B\mu_p\coloneqq[(\Spec k)/\mu_p]$. Notably, $\mu_p$ is flat but not smooth, so checking (c) may be trickier. The main point is to use the Kummer exact sequence
    \[1\to\mu_p\to\mathbb G_m\stackrel{(\cdot)^p}\to\mathbb G_m\to1.\]
    As such, principal homogeneous spaces over $\mu_p$ turn out to be pairs $(\mc L,\lambda)$, where $\mc L$ is an invertible sheaf, and $\lambda\colon\mc L^{\otimes p}\to\OO$ is an isomorphism. One can check that this identification shows $B\mu_p$ is the same as $[\mathbb G_m/\mathbb G_m]$, where $\mathbb G_m$ acts on $\mathbb G_m$ by $u*v\coloneqq u^pv$. The point is that we are now taking a quotient by a smooth scheme, so we have checked (c)! In general, one can always do this kind of trick when our group $G$ is flat.
\end{example}
\begin{example}
    One can enforce points with automorphisms by basically making our line bundles in our groupoids. For example, let's say we want to put a $\mu_2$ stabilizer on $0\in\PP^1$ and put a $\mu_3$ stabilizer on $\infty\in\PP^1$. Then our functor $\mf X$ should assign test schemes $T$ to a map $T\to\PP^1$ so that the produced line bundle $\mc L$ has assigned isomorphisms $\mc L_0^{\otimes2}\to\OO_{\PP^1,0}$ and $\mc L_\infty^{\otimes3}\to\OO_{\PP^1,\infty}$. It is not obvious if we can realize $\mf X$ as a quotient, though it turns out that we can.
\end{example}

\section{September 13: Martin Olsson}
A paper list has been released. I'm too tired to take notes today.

\section{September 20: Rose Lopez}
Today we're talking about birational geometry of stacks.

\subsection{Classical Birational Geometry}
Here is the central definition.
\begin{definition}[birational]
    A \textit{rational map} $f\colon X\to Y$ is a morphism on dense open subsets of $X$ and $Y$. An isomorphism in this category (of varieties equipped with rational maps as morphisms) is a \textit{birational map}.
\end{definition}
One can always factor birational maps.
\begin{theorem}[weak factorization] \label{thm:weak-factorization}
    Fix a birational map $f\colon X\to Y$ of smooth proper varieties over an algebraically closed field $k$ of characteristic $0$. Suppose that $f$ is an isomorphism on $U\subseteq X$. Then $f$ factors as
    \[X_0\to X_1\to\cdots\to X_n=Y,\]
    where each map $X_i\to X_{i+1}$ is either a blow up or the reverse of a blow up $X_{i+1}\to X_i$, where the blow-ups are all disjoint from $U$.
\end{theorem}
To do our geometry, it is helpful to keep track of some algebra.
\begin{definition}[Burnside ring]
    Fix a field $k$ and nonnegative integer $n$. Then the \textit{Burnside ring} $\mathrm{Burn}_{k,n}$ is the free abelian group generated by the isomorphism classes of finitely generated field extensions of $k$ of transcendence degree $n$.
\end{definition}
Notably, given a smooth projective irreducible $k$-variety $X$, we can produce a class $[X]$ in the Burnside ring as $[k(X)]$. By summing over all transcendence degrees, we get a graded Burnside ring $\op{Burn}$. Given $U\subseteq X\setminus D$ for a divisor $D=D_1\cup\cdots\cup D_\ell$, one can define $[U]$ basically by subtracting out the classes of the divisors $D_i\times\PP^1$ and adding back in $(D_i\cap D_j)\times\PP^2$.

With our larger ring, we can find relations add $[X]+[Y]=[X\cup Y]$ and $[X]\cdot[Y]=[X\times Y]$ and $[U]=[U']$ if and only if there is an isomorphism between them coming from a birational projective morphism. The moral of the story is that $\op{Burn}_n$ turns out to basically be generated by these $U$ modulo certain relations given by cutting out other dense open subsets.

\subsection{Stacky Birational Geometry}
Given stacks $\mc X$ and $\mc Y$, a rational map $f\colon\mc X\to\mc Y$ is a morphism defined on a dense open subscheme of $\mc X$. Then an isomorphism on the dense opens is \textit{birational}; however, our definition of birationally equivalent requires that the birational map is proper (defined by valuative criterion) and representable (by schemes). One can tell a lot of the same story; for example, there are blow-ups. Now that we are looking at schemes, we can also take root stacks in a way that Professor Olsson described last week. The moral of the story is that one can prove an analogue of \Cref{thm:weak-factorization}, where we need to allow blow-ups and these root stacks.

\section{September 27: Xiangru Zeng}
Today we are discussing valuative criteria for the existence of moduli stacks. We will work over an algebraically closed field $k$ of characteristic $0$.

\subsection{Valuative Criteria}
Roughly speaking, one can frequently show that a moduli space is at least an algebraic stack. However, we will often want to upgrade this stack to a more geometric object, such as an algebraic space or scheme. Here is an example of one such result.
\begin{theorem}[Keel--Mori]
    Fix a Deligne--Mumford stack $\mf X$ of finite type and separated over $k$. Then there is a coarse moduli space $X$ equipped with a projection $\pi\colon\mf X\to X$; furthermore, one can guarantee that $\pi_*\OO_{\mf X}=\OO_X$ and that the functor $\pi_*\colon\mathrm{QCoh}(\mf X)\to\mathrm{QCoh}(X)$ is exact.
\end{theorem}
\begin{remark}
    The conclusion the above theorem defines what is called a ``tame coarse moduli space.'' There is also a relative notion over a scheme $S$. The above theorem more or less says all our finite type separated moduli spaces are tame in characteristic $0$.
\end{remark}
To improve the above result, we would like to remove the requirement that the stack is separated, which more or less is a requirement of having finite automorphism groups.

For example, if a linearly reductive group $G$ acts on some open quasiprojective variety $U\subseteq\PP^n$ semistably, then geometric invariant theory still permits us to produce a quotient
\[U/G=\op{Proj}\bigoplus_{n\ge0}\Gamma(U,\OO_U(n))^G.\]
Combining this example with the previous theorem, we produce the following definition.
\begin{definition}
    A map $\pi\colon\mf X\to X$ from an algebraic stack to an algebraic space is a \textit{good moduli space} if and only if the following hold.
    \begin{itemize}
        \item $\pi$ is quasiseparated and quasicompact.
        \item $\pi_*\OO_{\mf X}\cong\OO_X$.
        \item $\pi_*\colon\mathrm{QCoh}(\mf X)\to\mathrm{QCoh}(X)$ is exact.
    \end{itemize}
\end{definition}
It is not clear that this is a good definition, but here are some properties.
\begin{theorem}
    Fix a good moduli space $\pi\colon\mf X\to X$.
    \begin{listalph}
        \item $\pi$ is surjective and universally closed.
        \item For two points $\mf x_1,\mf x_2\in\mf X(k)$, we have $\pi(\mf x_1)=\pi(\mf x_2)$ if and only if $\overline{\{\mf x_1\}}\cap\overline{\{\mf x_2\}}\ne\emp$.
        \item If $\mf X$ is Noetherian, then $\pi$ is universal (with respect to being a good moduli space).
    \end{listalph}
\end{theorem}

\end{document}
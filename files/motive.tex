\documentclass{article}
\usepackage[utf8]{inputenc}

\newcommand{\nirpdftitle}{Motives}
\usepackage{import}
\inputfrom{../../notes}{nir}
\usepackage[backend=biber,
    style=alphabetic,
    sorting=ynt
]{biblatex}
\setcounter{tocdepth}{2}

\pagestyle{contentpage}

\title{Motives}
\author{Nir Elber}
\date{Spring 2024}
\usepackage{graphicx}

\begin{document}

\maketitle

\tableofcontents

\section{Introduction}
Here is the statement of the conjecture.
\begin{conj} \label{conj:desired-conjecture}
	Fix an abelian motive $A$ over a number field $K$, and let $G(A)$ denote the motivic Galois group of $A$. Suppose $A$ has good reduction at a prime $\mf p$ of $K$. Then there exists a class $F\in\op{Conj}G(A)(\QQ)$ such that
	\[F=[\rho_\ell(\mathrm{Frob}_\mf p)]\]
	for each rational prime $\ell\nmid\mf p$, where $\rho_\ell\colon\op{Gal}(\ov K/K)\to\op{GL}\left(H^1_{\mathrm{\acute et}}(A;\QQ_\ell)\right)$ is the $\ell$-adic Galois representation.
\end{conj}

\section{Background on Motives}
Throughout, $X$ denotes a smooth proper variety over a field $k$, which is possibly but not definitely algebraically closed.

\subsection{Tannakian Formalism}
Approximately speaking, the theory of (pure) motives takes the category of smooth proper varieties and attempts to give this category a long list of desirable properties. Tannakian formalism enumerates the desiderata. Our exposition follows \cite{milne-tannakian} and \cite[Chapters~2 and~6]{andre-motive}.
\begin{warn}
	We will not need any proofs from the theory of Tannakian formalism, so we will not provide them.
\end{warn}
Intuitively, a Tannakian category is one that looks like the category $\op{Rep}_k(G)$ of finite-dimensional representations of an affine $k$-group $G$. An important property of $\op{Rep}_k(G)$ is the ability to take tensor products, so we codify how useful tensor products are.
\begin{definition}[monoidal]
	A \textit{monoidal category} or \textit{$\otimes$-category} is a category $\mc C$ equipped with a bifunctor $\otimes\colon\mc C\times\mc C\to\mc C$ and identity object $1\in\mc C$ with the following identities.
	\begin{itemize}
		\item Associativity: there is a natural isomorphism $\alpha\colon((-\otimes-)\otimes-)\Rightarrow(-\otimes(-\otimes-))$.
		\item Identity: there are natural isomorphisms $(1\otimes-)\Rightarrow-$ and $(-\otimes1)\Rightarrow1$.
	\end{itemize}
	These isomorphisms satisfy certain coherence properties ensuring that one can associate and apply identity naturally in any suitable situation.
\end{definition}
In fact, $\op{Rep}_k(G)$ has a symmetry property.
\begin{definition}[symmetric monoidal]
	A \textit{symmetric monoidal category} is a monoidal category $\mc C$ further equipped with a symmetry isomorphism $(-\otimes-)\Rightarrow(-\otimes-)$ such that the composite
	\[(A\otimes B)\to(B\otimes A)\to(A\otimes B)\]
	is the identity.
\end{definition}
The reason we restricted $\op{Rep}_k(G)$ to finite-dimensional representations is so that we can take duals.
\begin{definition}[rigid]
	A rigid symmetric monoidal category is a symmetric monoidal category $\mc C$ further equipped with a natural isomorphism $(-)^\lor\colon\mc C\to\mc C\opp$ such that each $A\in\mc C$ makes $(-\otimes A^\lor)$ is left adjoint to $(-\otimes A)$, and $(A^\lor\otimes-)$ is right adjoint to $(A\otimes-)$.
\end{definition}
\begin{remark}
	Rigidity permits a notion of dimension of an object $A\in\mc C$ as the composite
	\[1\to A^\lor\otimes A\to A\otimes A^\lor\to1.\]
\end{remark}
Lastly, $\op{Rep}_k(G)$ has a forgetful functor to $\op{Vec}_k$, akin to the forgetful functor $\op{Set}(G)\to\op{Set}$ which appears in Grothendieck's Galois theory (used to define the \'etale fundamental group).
\begin{definition}[fiber functor]
	Fix an abelian rigid symmetric monoidal category $\mc C$ such that $F\coloneqq\op{End}(1)$ is a field. A \textit{fiber functor} is a faithful exact $\otimes$-functor $\omega\colon\mc C\to\op{Vec}_k$ for some finite field extension $k$ of $F$. If $k=F$, then we say that $\mc C$ is \textit{neutral Tannakian over $k$}.
\end{definition}
What is remarkable is that it turns out that one can recover the affine $k$-group $G$ from the (forgetful) fiber functor $\omega\colon\op{Rep}_k(G)\to\op{Vec}_k$ as ``$\underline{\op{Aut}}^\otimes(\omega)$.'' Explicitly, for a $k$-algebra $R$, an element of $\underline{\op{Aut}}^\otimes(\omega)(R)$ is a collection of automorphisms $(g_X)_{X\in\op{Rep}_k(G)}$ where $g_X$ is an $R$-linear automorphism of $\omega(X)\otimes_kR$, and these automorphisms are natural in $G$-linear maps $X\to Y$.

This process can in general recover a group $G$ from a neutral Tannakian category.
\begin{theorem}
	Fix a neutral Tannakian category $\mc C$ over a field $k$ equipped with fiber functor $\omega\colon\mc C\to\op{Vec}_k$.
	\begin{listalph}
		\item The functor $\underline{\op{Aut}}^\otimes(\omega)$ (defined analogously as above) is represented by an affine $k$-group $G$.
		\item The fiber functor $\omega$ then upgrades to a $\otimes$-equivalence $\mc C\to\op{Rep}_k(G)$.
	\end{listalph}
\end{theorem}
\begin{proof}
	See \cite[Theorem~2.11]{milne-tannakian}.
\end{proof}
It will be helpful to have some more concrete ways to understand $G$ from its Tannakian category. For example, if the Tannakian category is small, then $G$ should also be small. The following two propositions examine two versions of smallness.
\begin{definition}[$\otimes$-subcategory]
	Fix an abelian rigid symmetric monoidal category $\mc C$. Then the \textit{full $\otimes$-subcat\-egory} generated by a subset $S\subseteq\mc C$ of objects, denoted $\langle S\rangle^{\otimes}$ is the smallest abelian rigid monoidal subcategory.
\end{definition}
\begin{proposition}
	Fix an affine $k$-group $G$.
	\begin{listalph}
		\item Then $G$ is finite if and only if there is an object $X$ such that every object of $\op{Rep}_k(G)$ is a subquotient of $X^{\oplus n}$ for some nonnegative $n$.
		\item Then $G$ is algebraic (namely, finite type over $k$) if and only if $\op{Rep}_k(G)$ equals $\langle X\rangle^{\otimes}$ for some object $X$.
	\end{listalph}
\end{proposition}
\begin{proof}
	See \cite[Proposition~2.20]{milne-tannakian}.
\end{proof}
\begin{proposition}
	Fix a field $k$ of characteristic $0$ and an affine $k$-group $G$. Then $G^\circ\subseteq G$ is a projective limit of reductive $k$-groups if and only if $\op{Rep}_k(G)$ is semisimple.
\end{proposition}
\begin{proof}
	See \cite[Remark~2.28]{milne-tannakian}.
\end{proof}
Lastly, we will also want some functoriality. Approximately speaking, we expect surjections/injections of groups to correspond to ``surjections/injections'' of categories.
\begin{proposition}
	Fix a morphism $f\colon G\to G'$ of affine $k$-groups $G$, and let $\omega\colon\op{Rep}_k(G')\to\op{Rep}_k(G)$ be the corresponding functor.
	\begin{listalph}
		\item Suppose $\op{Rep}_k(G)$ is semisimple and that $k$ has characteristic $0$. Then $f$ is faithfully flat if and only if the following holds: for given $X'\in\op{Rep}_k(G')$, every subobject of $\omega(X')$ is isomorphic to $\omega(Y')$ for some subobject $Y'$ of $X'$.
		\item Then $f$ is a closed embedding if and only if every object $X\in\op{Rep}_k(G)$ is isomorphic to a subquotient of $\omega(X')$ for some $X'\in\op{Rep}_k(G')$.
	\end{listalph}
\end{proposition}
\begin{proof}
	Combine \cite[Remark~2.29]{milne-tannakian} with \cite[Proposition~2.21]{milne-tannakian}.
\end{proof}

\subsection{Review of Cohomology}
In this subsection, we review various cohomology theories, approximately following \cite[Section~1]{deligne-hodge}. We begin by discussing what is expected from a cohomology theory.
\begin{definition}[Weil cohomology]
	Let $\mc P(k)$ denote the category of smooth proper $k$-varieties. \todo{}
\end{definition}

\subsection{Hodge Structures}
The previous subsection mentioned that the cohomology $H^\bullet(X,\CC)$ of a complex projective variety $X$ admits a ``Hodge structure'' meaning that one has a decomposition
\[H^n(X,\CC)\cong\bigoplus_{p+q=n}H^{p,q}\]
where $H^{p,q}=\ov{H^{q,p}}$. What is interesting about this situation is that we begin with a $\QQ$-vector space $H^n(X,\CC)$, which then inherits the above decomposition only after base-change to $\CC$. This structure is what makes our complex-analytic cohomology interesting, so we give it a name.
\begin{definition}[Hodge structure]
	A \textit{$\QQ$-Hodge structure} of weight $m\in\ZZ$ is a finite-dimensional vector space $V\in\op{Vec}_\QQ$ such that $V_\CC$ admits a decomposition
	\[V_\CC=\bigoplus_{p+q=m}V^{p,q}_\CC\]
	where $V^{p,q}_\CC=\ov{V^{q,p}_\CC}$. We let $\op{HS}_\QQ$ denote the category of $\QQ$-Hodge structures, where a morphism of Hodge structures is a linear map preserving the decomposition over $\CC$.
\end{definition}
\begin{example}
	Give the Tate twist $\QQ(1)=2\pi i\QQ$ a Hodge structure of weight $-2$ where $\QQ(1)^{-1,-1}=\QQ(1)$ is nonzero.
\end{example}
The category $\op{HS}_\QQ$ becomes a faithful rigid tensor abelian subcategory of $\op{Vec}_\QQ$, where the forgetful functor is able to act as a fiber functor. As such, so we expect $\op{HS}_\QQ$ should arise from representations of some group. Let's explain how this is done.
\begin{notation}[Deligne torus]
	Let $\mathbb S\coloneqq\op{Res}_{\CC/\RR}\mathbb G_{m,\CC}$ denote the Deligne torus. We also let $w\colon\mathbb G_{m,\RR}\to\mathbb S$ denote the \textit{weight cocharacter} given by $w(r)\coloneqq r\in\CC$ on $\RR$-points.
\end{notation}
\begin{remark} \label{rem:concrete-deligne-torus}
	One can realize $\mathbb S$ more concretely as
	\[\mathbb S(R)=\left\{\begin{bmatrix}
		a & b \\ -b & a
	\end{bmatrix}\in{\op{GL}_2(R)}:a^2+b^2\in R^\times\right\},\]
	where $R$ is an $\mathbb R$-algebra. Indeed, there is a ring isomorphism from $R\otimes_\RR\CC$ to $\left\{\begin{bsmallmatrix}
		a & b \\ -b & a
	\end{bsmallmatrix}:a,b\in R\right\}$ by sending $1\otimes1\mapsto\begin{bsmallmatrix}
		1 \\ & 1
	\end{bsmallmatrix}$ and $1\otimes i\mapsto\begin{bsmallmatrix}
		1 \\ & -1
	\end{bsmallmatrix}$.
\end{remark}
We now explain how a representation of $\mathbb S$ converts to a Hodge structure.
\begin{lemma}
	Fix some $V\in\op{Vec}_\QQ$. Then a Hodge structure on $V$ has equivalent data to a representation $h\colon\mathbb S\to\op{GL}(V)_\RR$.
\end{lemma}
\begin{proof}
	\Cref{rem:concrete-deligne-torus} informs us that the character group $X^*(\mathbb S)$ of group homomorphisms $\mathbb S\to\mathbb G_m$ is a rank-$2$ free $\ZZ$-module generated by $z\colon\begin{bsmallmatrix}
		a & b \\ -b & a
	\end{bsmallmatrix}\mapsto a+bi$ and $\ov z\colon\begin{bsmallmatrix}
		a & b \\ -b & a
	\end{bsmallmatrix}\mapsto a-bi$ on $\CC$-points.\footnote{Alternatively, note one has an isomorphism $(\CC\otimes_\RR\CC)^\times\cong\CC^\times\times\CC^\times$ by sending $(z,w)\mapsto z\otimes w$. Then these two characters are $(z,w)\mapsto z$ and $(z,w)\mapsto w$.} Without too many details, upon passing to the Hopf algebra, one is essentially looking for units in $\RR\left[a,b,\left(a^2+b^2\right)^{-1}\right]$, of which there are not many. Note that there is a Galois action by $\op{Gal}(\CC/\RR)$ on these two characters $\{z,\ov z\}$, given by swapping them. Let $\iota\in\op{Gal}(\CC/\RR)$ denote complex conjugation, for brevity.

	Now, a representation $h\colon\mathbb S\to\op{GL}(V)_\RR$ must have $V_\CC$ decompose into eigenspaces according to the characters $X^*(\mathbb S)$, so one admits a decomposition
	\[V_\CC=\bigoplus_{\chi\in X^*(\mathbb S)}V_\CC^\chi.\]
	However, one also needs $V_\CC^{\iota\chi}=\ov{V_\CC^\chi}$ because $\iota$ swaps $\{\chi,\iota\chi\}$. By Galois descent, this is enough data to (conversely) define a representation $h\colon\mathbb S\to\op{Gal}(V)_\RR$.

	To relate the previous paragraph to Hodge structures, we recall that $X^*(\mathbb S)$ is a rank-$2$ free $\ZZ$-module, so write $\chi_{p,q}\coloneqq z^{-p}\ov z^{-q}$ so that $\iota\chi_{p,q}=\chi_{q,p}$. Setting $V_\CC^{p,q}\coloneqq V_\CC^{\chi_{p,q}}$ now explains how to relate the previous paragraph to a Hodge structure, as desired.
\end{proof}
\begin{remark}
	The weight of a Hodge structure on some $V\in\op{HS}_\QQ$ can be read off of $h$ as follows: note the weight cocharacter $h\circ w$ equals the $(-m)$th power map if and only if the weight is $m$. 
\end{remark}
Thus, we see that one has tensor products and duals of Hodge structures by tracking through the representation of $h$. For example, if $V\in\op{HS}_\QQ$ has $V^\lor$ inherit a Hodge structure by $(V^\lor)^{p,q}\coloneqq (V^{-p,-q})^\lor$. In particular, $\op{HS}_\QQ$ becomes Tannakian.\todo{}

% polarizable HS

\subsection{The Mumford--Tate Group}
We are now ready to define the main character of the present subsection, which is the Mumford--Tate group.\todo{}
\begin{definition}[Mumford--Tate group]
	For some $V\in\op{HS}_\QQ$, we define the \textit{Mumford--Tate group} $\op{MT}(V)$ as the smallest algebraic $\QQ$-group containing the image of the corresponding representation $h\colon\mathbb S\to\op{GL}(V)_\RR$.
\end{definition}
\begin{remark}
	Because $\mathbb S$ is connected, we see that $h$ is also connected. Namely, $\op{MT}(V)^\circ\subseteq\op{MT}(V)$ will be an algebraic $\QQ$-group containing the image of $h$ if $\op{MT}(V)$ does too, so equality is forced.
\end{remark}
It will turn out that $\op{MT}(V)$ is the algebraic group corresponding the full Tannakian subcategory $\langle V\rangle^{\otimes}$ of $\op{HS}_\QQ$. Unraveling the formalism, the key point is the following proposition.
\begin{proposition}
	Fix $V\in\op{HS}_\QQ$. Suppose $T\in\op{HS}_\QQ$ can be written as
	\[T=\bigoplus_{i=1}^N\left(V^{\otimes m_i}\otimes (V^\lor)^{\otimes n_i}\right)(p_i),\]
	where $m_i,n_i\ge0$ are nonnegative integers and $p_i\in\ZZ$. Then $W\subseteq T$ is a Hodge substructure if and only if the action of $\op{MT}(V)$ on $T$ stabilizes $W$.
\end{proposition}
\begin{proof}
	For each vector space in $\op{HS}_\QQ$, we let $h_\bullet$ denote the corresponding representation. Quickly, note that $h_T$ In the backwards direction, we note that $\op{MT}(V)$ stabilizing $W$ implies that $h(s)$ stabilizes $W_\RR$ for any $s$. We can thus view $W_\RR\subseteq T_\RR$ as a subrepresentation of $\mathbb S$, so taking eigenspaces reveals that $W$ can be given the structure of a Hodge substructure of $T$.

	The converse will have to use the construction of $T$. Indeed, suppose that $W\subseteq T$ is a Hodge substructure, and let $M\subseteq\op{GL}(V)$ be the smallest algebraic $\QQ$-group stabilizing $W\subseteq T$. We would like to show that $\op{MT}(V)\subseteq M$. By definition of $\op{MT}(V)$, it is enough to show that $h$ factors through $M_\RR$, meaning we must show that $h(s)$ stabilizes $W$ for each $s\in\mathbb S$. Well, $h(s)$ will act by characters on the eigenspaces $W^{p,q}_\CC\subseteq W_\CC$, so $h(s)$ does indeed stabilize $W$.
\end{proof}
\begin{corollary}
	Fix $V\in\op{HS}_\QQ$. Then $\op{MT}(V)$ is the group corresponding to the Tannakian subcategory $\langle V\rangle^{\otimes}$ of $\op{HS}_\QQ$.
\end{corollary}
\begin{proof}
	
\end{proof}

% \section{Background on Motives}
% Throughout, $X$ denotes a smooth proper variety over a field $k$, which is possibly but not definitely algebraically closed.

\section{Basic Cases}
In this subsection, we work out some basic cases.
\begin{proposition} \label{prop:cm}
	Fix an abelian variety $A$ over a number field $K$ with CM by $E$. Then \Cref{conj:desired-conjecture} holds for $A$.
\end{proposition}
\begin{proof}
	The main point is that we are able to lift $\op{Frob}_\mf p$ to become an endomorphism of $A$.

	Let $\mc A$ be the N\'eron model of $A$ over $\OO_{K_\mf p}$, and let $\kappa\coloneqq\OO_K/\mf p$ be the residue field. The N\'eron mapping property implies $\op{End}_E(A)^\circ=\op{End}_E^\circ(\mc A)$, which then has a natural reduction map to $\op{End}_E^\circ(\mc A_\kappa)$. An argument on the Tate module tells us that
	\[\op{End}_E^\circ(A)\to\op{End}_E^\circ(\mc A_\kappa)\]
	is injective, but we see that both sides are free $E$-modules of rank $1$. Thus, this reduction map is an isomorphism, so $\op{Frob}_\mf p$ lifts from an endomorphism on $\mc A_\kappa$ to an endomorphism on $A$.

	We now note that the diagram
	% https://q.uiver.app/#q=WzAsNCxbMCwwLCJIXjFfQihBO1xcUVEpXFxvdGltZXNfXFxRUVxcUVFfXFxlbGwiXSxbMSwwLCJIXjFfQihBO1xcUVEpXFxvdGltZXNfXFxRUVxcUVFfXFxlbGwiXSxbMCwxLCJIXjFfe1xcbWF0aHJte1xcYWN1dGUgZXR9fShBO1xcUVFfXFxlbGwpIl0sWzEsMSwiSF4xX3tcXG1hdGhybXtcXGFjdXRlIGV0fX0oQTtcXFFRX1xcZWxsKSJdLFswLDEsIkheMV9CKEYpIl0sWzIsMywiXFxyaG9fXFxlbGwoXFxvcHtGcm9ifV9cXG1mIHApIl0sWzAsMl0sWzEsM11d&macro_url=https%3A%2F%2Fraw.githubusercontent.com%2FdFoiler%2Fnotes%2Fmaster%2Fnir.tex
	\[\begin{tikzcd}
		{H^1_B(A;\QQ)\otimes_\QQ\QQ_\ell} & {H^1_B(A;\QQ)\otimes_\QQ\QQ_\ell} \\
		{H^1_{\mathrm{\acute et}}(A;\QQ_\ell)} & {H^1_{\mathrm{\acute et}}(A;\QQ_\ell)}
		\arrow["{H^1_B(F)}", from=1-1, to=1-2]
		\arrow[from=1-1, to=2-1]
		\arrow[from=1-2, to=2-2]
		\arrow["{\rho_\ell(\mathrm{Frob}_\mf p)}", from=2-1, to=2-2]
	\end{tikzcd}\]
	commutes by the functoriality of the applied comparison isomorphism (and the definition of $F$), so the result follows.
\end{proof}
\begin{proposition}
	Fix an elliptic curve $A$ over a number field. Then \Cref{conj:desired-conjecture} holds for $A$.
\end{proposition}
\begin{proof}
	If $A$ has complex multiplication, we are done by \Cref{prop:cm}. This leaves us with two cases.
	\begin{itemize}
		\item Suppose $A_\CC$ still has no complex multiplication. Then $\mathrm{MT}(A)$ is $\op{GL}_{2,\QQ}$, so the result follows from classical considerations.
		\item Suppose $A_\CC$ is CM so that $A$ has potential CM. For brevity, define $V\coloneqq H^1(A;\QQ)$. Note that $A_L$ has CM for some quadratic extension $L$ of $K$, so we produce a short exact sequence
		\[1\to\op{MT}(A)\to G(A)\to\op{Gal}(L/K)\to1.\]
		Note $\op{MT}(A)$ is a torus, so $V_\CC$ decomposes into two eigenspaces $V_\CC=V^1_\CC\oplus V^2_\CC$; considering the rank of $\op{MT}(A)$, we see that $\sigma\in\op{MT}(A)$ if and only if $\sigma_\CC\colon V_\CC\to V_L$ sends $V^1_\CC$ and $V^2_\CC$ to themselves. Thus, choosing some $c\in G(A)$ to lift the generator of $\op{Gal}(L/K)$, we see that $c$ must normalize $\op{MT}(A)$ while not actually living in $\op{MT}(A)$, and the only way for this to happen is for $c_\CC$ to swap $V^1_\CC$ and $V^2_\CC$ (possibly adding a scalar in the process to ensure that $c$ is defined over $\QQ$).
		
		This will be enough to complete the proof. Letting $q$ be the cardinality of $\OO_K/\mf p$, we know that $\rho_\ell(\mathrm{Frob}_\mf p)$ is semisimple with characteristic polynomial $P_\mf p(x)\in\QQ[x]$ not depending on $\ell$. The point is that the eigenvalues $\alpha_{1,\ell}$ and $\alpha_{2,\ell}$ of $\rho_\ell(\op{Frob}_\mf p)$ on $V^1_\CC$ and $V^2_\CC$ may not be determined up to order, but the set of eigenvalues $\{\alpha_{1,\ell},\alpha_{2,\ell}\}$ is independent of $\ell$. Conjugation by $c$ is able to swap the two eigenspaces, so we see that the conjugacy class of $\rho_\ell(\op{Frob}_\mf p)$ is now independent of $\ell$.
		\qedhere
	\end{itemize}
\end{proof}
\begin{remark}
	It may appear that one can upgrade this second proof to work for arbitrary abelian varieties with potential CM, but this is not the case. Indeed, the given proof only functions because $\op{Gal}(L/K)$ acts simply transitively on the eigenspaces of $\op{MT}(A)$ acting on $V_\CC$. However, $G(A)\subseteq\op{GSp}_{2\dim A}(\QQ)$, so one cannot hope for the normalizer of a torus to be large enough in general.
\end{remark}

\end{document}

% now that we know the frobenii live in the same connected component, one might be able to upgrade noot's weakly neat to merely require that some prescribed power is weakly neat

% galois commutes with complex multiplication only over the reflex field
% reason: complex multiplication is defined over the reflex field, so conjugating by Galois won't do anything!
\documentclass{article}
\usepackage[utf8]{inputenc}

\newcommand{\nirpdftitle}{Notes on Bump}
\usepackage{import}
\inputfrom{../../notes}{nir}
\usepackage[backend=biber,
    style=alphabetic,
    sorting=ynt
]{biblatex}
\setcounter{tocdepth}{2}

\pagestyle{contentpage}

\title{Notes on Bump}
\author{Nir Elber}
\date{Spring 2024}
\usepackage{graphicx}

\begin{document}

\maketitle

\tableofcontents

\section{June 7th}
We plan on covering \S1.1--1.3.

\subsection{Dirichlet \texorpdfstring{$L$}{ L}-Functions}
We begin by defining Dirichlet characters.
\begin{definition}[Dirichlet character]
	Fix a positive integer $N$. A \textit{Dirichlet character$\pmod N$} is a character $\chi\colon(\ZZ/N\ZZ)^\times$ extended to $\ZZ$ by declaring $\chi(n)=0$ whenever $\gcd(n,N)>1$. If $N\mid M$ where $N<M$, then a Dirichlet character $\chi\pmod N$ \textit{induces} a Dirichlet character$\pmod M$ by the canonical projection $\ZZ/M\ZZ\onto\ZZ/N\ZZ$. If $\chi$ is not induced by any other character, then $\chi$ is \textit{primitive}; otherwise, $\chi$ is \textit{imprimitive}.
\end{definition}
Dirichlet characters $\chi\pmod N$ have two important attached invariants.
\begin{definition}[$L$-function]
	Fix a Dirichlet character $\chi\pmod N$. Then we define the \textit{Dirichlet $L$-function} by
	\[L(s,\chi)\coloneqq\sum_{n=1}^\infty\frac{\chi(n)}{n^s}.\]
\end{definition}
\begin{remark}
	We note that $L(s,\chi)$ converges absolutely for $\Re s>1$. If $\chi$ is not induced by the trivial character, then one sees $L(s,\chi)$ actually converges for $\Re s>0$ uniformly on compacts. If $\chi=1$, then $L(s,\chi)=\zeta(s)$, and one can use a summation-by-parts argument to show that $\zeta(s)$ has an integral representation valid for $\Re s>0$.
\end{remark}
\begin{remark}
	The usual argument with unique prime factorization implies $L(s,\chi)$ admits an Euler product
	\[L(s,\chi)=\prod_p\frac1{1-\chi(p)p^{-s}}.\]
\end{remark}
Our goal for the time being is to show that $L(s,\chi)$ admits a meromorphic continuation and functional equation. To this end, we introduce the second invariant of a Dirichlet character.
\begin{definition}[Gauss sum]
	Fix a primitive Dirichlet character $\chi\pmod N$. Then we define the \textit{Gauss sum}
	\[\tau(\chi)\coloneqq\sum_{n\in\ZZ/N\ZZ}\tau(n)e^{2\pi in/N}.\]
\end{definition}
We may like to adjust the character $n\mapsto e^{2\pi in/N}$. To this end, we have the following lemma.
\begin{lemma}
	Fix a primitive Dirichlet character $\chi\pmod N$. Then
	\[\sum_{n\in\ZZ/N\ZZ}\chi(n)e^{2\pi inm/N}=\ov\chi(m)\tau(\chi).\]
\end{lemma}
\begin{proof}
	If $\gcd(m,N)=1$, then this is a matter of rearranging the sum. Otherwise, the right-hand side vanishes by definition of $\chi$, and one shows that the left-hand side vanishes essentially because the ``periods'' of $\chi$ and $n\mapsto e^{2\pi inm/N}$ differ.
\end{proof}
We will want to know that $\tau(\chi)$ is nonzero. As is common in harmonic analysis, it will be easier to compute the norm.
\begin{lemma}
	Fix a primitive Dirichlet character $\chi\pmod N$. Then $\left|\tau(\chi)\right|^2=N$.
\end{lemma}
\begin{proof}
	Some rearranging reveals that
	\[\tau(\chi)\ov{\tau(\chi)}=\frac1{\varphi(N)}\sum_{m\in\ZZ/N\ZZ}\sum_{n_1,n_2\in(\ZZ/N\ZZ)^\times}\chi(n_1)\ov\chi(n_2)e^{2\pi i(n_1-n_2)m/N}.\]
	(The point is that each $m$ produces the same value.) Summing over $m$, we see that we only care about terms where $n_1\equiv n_2\pmod N$, from which the result follows.
\end{proof}
Our proof of the functional equation requires the Poisson summation formula. Thus, we introduce a little more harmonic analysis.
\begin{definition}[Fourier transform]
	For a Schwartz function $f\colon\RR\to\CC$, we define its \textit{Fourier transform} by
	\[\mc Ff(x)\coloneqq\int_\RR f(y)e^{2\pi ixy}\,dy.\]
\end{definition}
\begin{example} \label{ex:fourier-transform-gaussian}
	For $t\in\RR$, define $f_t(x)\coloneqq e^{-\pi tx^2}$. Then one can compute that $\mc Ff_t=t^{-1/2}f_{1/t}$. Bump includes a proof using contour integration, but of course other proofs exist.
\end{example}
\begin{example} \label{ex:fourier-transform-almost-gaussian}
	For $t\in\RR$, define $g_t(x)\coloneqq xe^{-\pi tx^2}$. Integrating by parts and using \Cref{ex:fourier-transform-gaussian}, one finds that $\mc Fg_t=it^{-3/2}g_{1/t}$.
\end{example}
\begin{proposition}[Poisson summation] \label{prop:poisson-sum}
	For a Schwarz function $f\colon\RR\to\CC$, we have
	\[\sum_{n\in\ZZ}f(n)=\sum_{n\in\ZZ}\mc Ff(n).\]
\end{proposition}
\begin{proof}
	The trick is to consider the periodic function
	\[F(x)\coloneqq\sum_{n\in\ZZ}f(x+n).\]
	Because $f$ is Schwartz, $F$ is infinitely differentiable, so it admits a Fourier series. A computation of the Fourier coefficients then reveals that
	\[F(x)=\sum_{m\in\ZZ}\mc Ff(m)e^{2\pi imx},\]
	from which the result follows by taking $m=0$.
\end{proof}
\begin{corollary} \label{cor:twisted-poisson-sum}
	For a Schwarz function $f\colon\RR\to\CC$ and primitive Dirichlet character $\chi\pmod N$, we have
	\[\sum_{n\in\ZZ}\chi(n)f(n)=\frac{\tau(\ov\chi)}N\sum_{n\in\ZZ}\widehat f(n/N).\]
\end{corollary}
\begin{proof}
	Apply Poisson summation to the function
	\[g(x)\coloneqq\left(\frac{\tau(\ov\chi)}N\sum_{m\in\ZZ/N\ZZ}\ov\chi(m)e^{2\pi ixm/N}\right)f(x).\]
	For example, the left-hand side equals $\sum_{n\in\ZZ}g(n)$ because the big factor equals $\chi(n)$ when $x=n$ is an integer.
\end{proof}
We now move towards our proof of the functional equation. Our functional equation for Dirichlet $L$-func\-tions will be boot\-strapped from the functional equation for certain $\theta$-functions.
\begin{proposition} \label{prop:theta-functional-eq}
	Fix a primitive Dirichlet character $\chi\pmod N$. Say $\chi(-1)=(-1)^\varepsilon$, where $\varepsilon\in\{0,1\}$. Define the $\theta$-function
	\[\theta_\chi(t)\coloneqq\frac12\sum_{n\in\ZZ}n^\varepsilon\chi(n)e^{-\pi n^2t}.\]
	Then
	\[\theta_\chi(t)=\frac{(-i)^\varepsilon\tau(\chi)}{N^{1+\varepsilon}t^{\varepsilon+1/2}}\theta_{\ov\chi}\left(\frac1{N^2t}\right).\]
\end{proposition}
\begin{proof}
	Doing casework on $\varepsilon$, combine \Cref{cor:twisted-poisson-sum} with \Cref{ex:fourier-transform-gaussian,ex:fourier-transform-almost-gaussian}.
\end{proof}
At long last, here is our result.
\begin{theorem}
	Fix a primitive Dirichlet character $\chi\pmod N$. Say $\chi(-1)=(-1)^\varepsilon$, where $\varepsilon\in\{0,1\}$. Then the completed $L$-function
	\[\Lambda(s,\chi)\coloneqq\pi^{-(s+\varepsilon)/2}\Gamma\left(\frac{s+\varepsilon}2\right)L(s,\chi)\]
	has a meromorphic continuation to $\CC$ and satisfies the functional equation
	\[\Lambda(s,\chi)=(-i)^\varepsilon\tau(\chi)N^{-s}\Lambda(1-s,\ov\chi).\]
\end{theorem}
\begin{proof}
	It is enough to show the functional equation. A $u$-substitution proves
	\[\int_{\RR^+}e^{-\pi tn^2}t^{(s+\varepsilon)/2}\,\frac{dt}t=\pi^{-(s+\varepsilon)/2}\Gamma\left(\frac{s+\varepsilon}2\right)n^{-s-\varepsilon}.\]
	Summing over $n\ge0$ reveals
	\[\Lambda(s,\chi)=\int_{\RR^+}\theta_\chi(t)t^{(s+\varepsilon)/2}\,\frac{dt}t.\]
	\Cref{prop:theta-functional-eq} completes the proof.
\end{proof}

\subsection{The Modular Group}
The natural action of $\op{SL}_2(\CC)$ on $\CC^2$ descends to an action of $\op{SL}_2(\RR)$ on $\mathbb H\coloneqq\{z\in\CC:\Im z>0\}$ by fractional linear transformations. Explicitly,
\[\begin{bmatrix}
	a & b \\ c & d
\end{bmatrix}z\coloneqq\frac{az+b}{cz+d}.\]
We would like some arithmetic input to this action, so we introduce some subgroups.
\begin{definition}[congruence subgroup]
	For a positive integer $N$, we define $\Gamma(N)$ as the kernel of the reduction map $\op{SL}_2(\ZZ)\to\op{SL}_2(\ZZ/N\ZZ)$. Explicitly,
	\[\Gamma(N)\coloneqq\left\{\begin{bmatrix}
		a & b \\ c & d
	\end{bmatrix}\in\op{SL}_2(\ZZ):\begin{bmatrix}
		a & b \\ c & d
	\end{bmatrix}\equiv\begin{bmatrix}
		1 & 0 \\ 0 & 1
	\end{bmatrix}\pmod N\right\}.\]
	A subgroup $\Gamma\subseteq\op{SL}_2(\ZZ)$ is a \textit{congruence subgroup} if and only if it contains $\Gamma(N)$ for some positive integer $N$.
\end{definition}
We will spend the rest of the section collecting some facts about $\op{SL}_2(\ZZ)$ and its action on $\HH$.
\begin{proposition}
	The group $\op{SL}_2(\ZZ)$ acts discontinuously on $\mathbb H$.
\end{proposition}
\begin{proof}
	For compact subsets $K_1,K_2\subseteq\HH$, we must show that
	\[S\coloneqq\{g\in\op{SL}_2(\ZZ):K_2\cap gK_1\ne\emp\}\]
	is a finite set. Well, note that $g\coloneqq\begin{bsmallmatrix}
		a & b \\ c & d
	\end{bsmallmatrix}$ has
	\[\Im g(z)=\frac y{\left|cz+d\right|^2}\]
	by a direct computation. Thus, the values of $c$ and $d$ are bounded in $S$. Because $\begin{bsmallmatrix}
		1 & b \\ & 1
	\end{bsmallmatrix}$ behaves a lateral shift (to the left or right by $\left|a\right|$), we see that there are only finitely many possible values of $b$. Lastly, $a$ is determined by $(b,c,d)$ because $ad-bc=1$, so we conclude that $S$ is finite.
\end{proof}
\begin{proposition}
	The action of $\op{SL}_2(\ZZ)$ on $\HH$ has a fundamental domain given by
	\[F\coloneqq\left\{z\in\HH:\left|\Re z\right|<\frac12,\left|z\right|>1\right\}.\]
	Namely, any class of $\op{SL}_2(\ZZ)\backslash\HH$ has a representative in $\ov F$, and $z_1,z_2\in F$ with $g(z_1)=z_2$ must have $z_1=z_2$ (and $g=\pm I_2$).
\end{proposition}
\begin{proof}
	This is essentially a matter of making the previous proof explicit. For the first claim, choose $z\in\HH$, and apply $\op{SL}_2(\ZZ)$ until $\Im z$ is maximized; then apply elements of the form $\begin{bsmallmatrix}
		1 & b \\ & 1
	\end{bsmallmatrix}$ until $\Re z\in[-1/2,1/2]$. For the second claim, one does some explicit algebra and casework on $z$ and $g$.
\end{proof}
\begin{remark}
	Any finite-index subgroup $\Gamma\subseteq\op{SL}_2(\ZZ)$ can also be given a fundamental domain by taking $\bigcup_{g\in\Gamma\backslash\op{SL}_2(\ZZ)}gF$, where the union is merely over a set of representatives for $\Gamma\backslash\op{SL}_2(\ZZ)$.
\end{remark}
\begin{proposition}
	Fix a congruence subgroup $\Gamma\subseteq\op{SL}_2(\ZZ)$. Then the quotient $\Gamma\backslash\HH$ can be compactified and then given the structure of a compact Riemann surface.
\end{proposition}
\begin{proof}
	Define $\HH^*\coloneqq\HH\sqcup\PP^1_\QQ$, where the points of $\PP^1_\QQ$ are called ``cusps.'' Note that $\Gamma$ acts on $\PP^1_\QQ$ separately and with only finitely many orbits (because $\Gamma\subseteq\op{SL}_2(\ZZ)$ has finite index). We will explain how $\Gamma\backslash\HH^*$ can be given the structure of a compact Riemann surface. Let $\ov\Gamma\subseteq\op{PSL}_2(\RR)$ be the image of $\Gamma$; there are three cases for $a\in\HH^*$
	\begin{itemize}
		\item If the stabilizer of $a$ in $\ov\Gamma$ is trivial, then the discontinuity of our action implies that this is the case in an open neighborhood of $a$. So we map $a$ to the fundamental domain and take a chart there.
		\item If the stabilizer of $a$ in $\ov\Gamma$ is nontrivial and $a\in\HH$, then we use the map $z\mapsto\frac{z-a}{z-\ov a}$ to send $a$ to the origin, and it sends everything else to the unit disk. Tracking through how fractional linear transformations behave, we see that the stabilizer must now be a finite collection of rotations about the origin, so we take roots to build our charts.
		\item If the stabilizer of $a$ in $\ov\Gamma$ is nontrivial and $a\in\PP^1_\QQ$, use $\op{SL}_2(\ZZ)$ to move $a$ to $\infty$, and a similar argument as the previous point can move everything to the unit disk again.
		\qedhere
	\end{itemize}
\end{proof}

\subsection{Modular Forms}
Here is our definition.
\begin{definition}[modular form]
	Fix an integer $k$ and finite-index subgroup $\Gamma\subseteq\op{SL}_2(\ZZ)$. Then a \textit{modular form} of weight $k$ and level $\Gamma$ is a holomorphic function $f$ on $\HH^*$ such that
	\[f\left(\frac{az+b}{cz+d}\right)=(cz+d)^kf(z)\]
	for any $\begin{bsmallmatrix}
		a & b \\ c & d
	\end{bsmallmatrix}\in\Gamma$. The vector space of such $f$ is denoted by $M_k(\Gamma)$. If $f$ vanishes on the cusps of $\HH^*$, we say that $f$ is a \textit{cusp form}, and 
\end{definition}
\begin{remark}
	Being holomorphic on $\HH^*$ is a somewhat tricky condition. Because $\Gamma\backslash\HH^*$ has already been given the structure of a compact Riemann surface, it is enough to show that $f$ has at worst removable singularities, so it is enough to show that $f$ is bounded approaching the cusps in $\PP^1_\QQ$. More explicitly, if $\Gamma\supseteq\Gamma(N)$, then $q\coloneqq e^{2\pi iz/N}$ is a local chart around $\infty\in\PP^1_\QQ$, so we want the Fourier expansion
	\[f(z)=\sum_{n\in\ZZ}a_nq^n\]
	to have $a_n=0$ for $n<0$.
\end{remark}
\begin{remark} \label{rem:odd-weight}
	Suppose $k$ is odd and $\Gamma=\op{SL}_2(\ZZ)$. Then $g=-I_2$ tells us that $f(z)=(-1)^kf(z)$, so $f=0$.
\end{remark}
\begin{remark}
	If $k=0$, then we are asking for holomorphic functions on $\Gamma\backslash\HH^*$, but this is a compact Riemann surface, so our modular forms of weight $0$ are constant.
\end{remark}
\begin{remark}
	More formally, we see that $M(\Gamma)$ is a graded ring, with grading given by the weight. The point is that the product of modular forms of weights $k$ and $\ell$ produces a modular form of weight $k+\ell$.
\end{remark}
We would like to classify modular forms for $\op{SL}_2(\ZZ)$.
\begin{proposition} \label{prop:mk-fin-dim}
	Fix a finite-index subgroup $\Gamma\subseteq\op{SL}_2(\ZZ)$. Then $M_k(\Gamma)$ is finite-dimensional.
\end{proposition}
\begin{proof}
	If $M_k(\Gamma)$ only has $0$, then we are done. Else, choose a nonzero element $f_0$. Then division by $f_0$ sends $f\in M_k(\Gamma)$ to meromorphic functions $f/f_0$ on $X\coloneqq\Gamma\backslash\HH^*$. Now, this collection of holomorphic functions $f/f_0$ on $X$ have prescribed poles at the zeroes of $f$, so an argument with Laurent expansions in local charts around these poles explains that the space of such holomorphic functions on $X$ is finite-dimensional.
\end{proof}
Thus, $M_k(\op{SL}_2(\ZZ))$ is relatively small. We now want to show that it is frequently nonempty when $k$ is even (see \Cref{rem:odd-weight}).
\begin{lemma}
	For even $k\ge4$, define
	\[E_k(z)\coloneqq\frac12\sum_{(m,n)\in\ZZ^2\setminus\{(0,0)\}}\frac1{(mz+n)^{k}}.\]
	Then $E_k\in M_k(\op{SL}_2(\ZZ))$.
\end{lemma}
\begin{proof}
	With $k\ge4$, one can check that $E_k$ is absolutely convergent, and it is weight $k$ essentially by construction. To check that $E_k$ is holomorphic at $\infty$, we compute its Fourier expansion. The Fourier transform of $f(u)\coloneqq(u-\tau)^{-1}$ is
	\[\mc Ff(v)=\begin{cases}
		2\pi i\op{Res}_{u=t}\left(e^{2\pi iuv}(u-\tau)^{-k}\right) & \text{if }v>0 \\
		0 & \text{if }v\le0,
	\end{cases}=\begin{cases}
		\frac{(2\pi i)^k}{(k-1)!}v^{k-1}e^{2\pi iv\tau} & \text{if }v>0, \\
		0 & \text{if }v\le0.
	\end{cases}\]
	Thus, the Poisson summation formula and a little rearrangement tells us that
	\[E_k(z)=\zeta(k)+\frac{(2\pi i)^k}{(k-1)!}\sum_{n=1}^\infty\sigma_{k-1}(n)q^n,\]
	where $\sigma_{k-1}(n)$ is the sum of the $(k-1)$st powers of the divisors of $n$.
\end{proof}
\begin{remark}
	A computation of $\zeta(k)$ (for even $k$) reveals that $G_k(z)\coloneqq\zeta(k)^{-1}E_k(z)$ has rational coefficients. For example, one can see that $\Delta\coloneqq G_4^3-G_6^2$ lives in $S_{12}(\op{SL}_2(\ZZ))$.
\end{remark}
\begin{lemma}
	There exists an element in $S_{12}(\op{SL}_2(\ZZ))$ which does not vanish on $\HH$.
\end{lemma}
\begin{proof}
	We recall the Jacobi triple product formula given by
	\[\sum_{n\in\ZZ}q^{n^2}x^n=\prod_{n=1}^\infty\left(1-q^{2n}\right)\left(1+q^{2n-1}x\right)\left(1+q^{2n-1}x^{-1}\right).\]
	Substituting $q\mapsto q^{3/2}$ and $x\mapsto-q^{-1/2}$ and rearranging, we see
	\[\eta(z)\coloneqq q^{1/24}\prod_{n=1}^\infty\left(1-q^n\right)=\sum_{n\in\ZZ}\chi(n)q^{n^2/24},\]
	where $\chi\pmod{12}$ is the primitive quadratic character. (Explicitly, $\chi(\pm)=1$ and $\chi(\pm5)=-1$.) Note $\eta(z)=\theta_\chi(-z/12)$.

	We claim that $\eta^{24}$ is the required function. The infinite product tells us that $\eta$ does not vanish on $\HH$, but $\eta$ vanishes at $\infty\in\HH^*$ (which is $q=0$). Thus, it remains to show that $\eta^{24}$ is modular with weight $12$. The infinite product explains that $\eta^{24}$ satisfies the modularity property for $\begin{bsmallmatrix}
		1 & 1 \\ & 1
	\end{bsmallmatrix}$, so it remains to check for $\begin{bsmallmatrix}
		& -1 \\ & 1
	\end{bsmallmatrix}$. Well, plugging $\theta_\chi$ into \Cref{prop:theta-functional-eq}, we see
	\[\sqrt{-iz}\eta(z)=\eta\left(-\frac1z\right),\]
	which completes the proof upon raising to the $24$th power.
\end{proof}
\begin{remark}
	The argument of \Cref{prop:mk-fin-dim} tells us that $S_{12}(\op{SL}_2(\ZZ))$ is actually one-dimensional. Thus, it is spanned by $\Delta$.
\end{remark}
And here is our classification result.
\begin{theorem}
	The ring $M(\op{SL}_2(\ZZ))$ is generated by $G_4$ and $G_6$. In particular,
	\[\dim M_{12a+2b}(\op{SL}_2(\ZZ))=\begin{cases}
		a+1 & \text{if }2b\in\{0,4,6,8,10\}, \\
		a & \text{if }2b=2.
	\end{cases}\]
\end{theorem}
\begin{proof}
	We abbreviate the group $\op{SL}_2(\ZZ)$ from our notation. Dimension arguments imply that it is enough to show the last computation. The argument of \Cref{prop:mk-fin-dim} implies that multiplication by $\Delta$ provides an isomorphism $M_{k}\to S_{k+12}$ for all $k$; additionally, because we have only one cusp, we see that either $M_k=S_k$ or $\dim M_k=\dim S_k+1$. Thus, $\dim M_{k+12}=1+\dim M_k$ always, so it remains to show the result for $k<12$.

	Examining what we've done so far, it remains to show $\dim M_k=1$ for even $k\in[4,10]$ and $\dim M_2=0$.
	\begin{itemize}
		\item Take $k\in\{4,6,8,10\}$. To show $\dim M_k=1$, we will show $\dim S_k=0$ (and then use $E_k$ to increase dimension). Well, suppose for contradiction that we have a nonzero element $f\in S_k$. On one hand, we see $E_{6(12-k)}(f/\Delta)^6$ is a modular form of weight $0$, so it is constant, so we may say $E_{6(12-k)}=\Delta^6/f^6$ by adjusting $f$ by a constant multiple. On the other hand, this means $E_{6(12-k)}$ fails to vanish on $\HH$, so $\Delta^{(12-k)/2}/E_{6(12-k)}$ is a modular form of weight $0$ with no poles but a zero at the cusp, which is impossible.
		\item Take $k=2$. Suppose for contradiction that we have a nonzero element $f\in M_2$. By adjusting $f$ by a constant multiple, the previous tells us we have $fE_4=E_6$. However, a computation shows $E_4\left(e^{2\pi i/3}\right)=0$, which would $\Delta$ has a zero in $\HH$, which we know is false.
		\qedhere
	\end{itemize}
\end{proof}
Our next goal is to make a discussion of $L$-functions.
\begin{definition}[$L$-function]
	For $f\in M_k(\op{SL}_2(\ZZ))$ with Fourier expansion $f(z)=\sum_{n=1}^\infty a_nq^n$, we define
	\[L(s,f)\coloneqq\sum_{n=1}^\infty\frac{a_n}{n^s}.\]
\end{definition}
We should probably check that this converges.
\begin{proposition}
	For $f\in S_k(\op{SL}_2(\ZZ))$ with Fourier expansion $f(z)=\sum_{n=1}^\infty a_nq^n$. Then $\left|a_n\right|=O\left(n^{k/2}\right)$.
\end{proposition}
\begin{proof}
	A direct computation shows that $\left|f(z)(\Im z)^{k/2}\right|$ is $\op{SL}_2(\ZZ)$-invariant; because $f$ is a cusp form, we see that $\left|f(z)(\Im z)^{k/2}\right|$ is bounded on $\HH$ by some constant $C$. Now, for any $y\in\RR$, we see
	\[\left|a_n\right|e^{-2\pi ny}=\left|\int_{\RR/\ZZ}f(x+iy)e^{-2\pi inx}\,dx\right|\le\int_0^1\left|f(x+iy)\right|\,dx\le Cy^{-k/2}.\]
	Choosing $y=1/n$ completes the proof.
\end{proof}
\begin{remark}
	In general, we know we can write $f=f_0+cE_k$ for cusp form $f_0$, so our computation of the Fourier expansion of $E_k$ reveals that 
\end{remark}
Thus, $L(s,f)$ converges for $\Re s$ sufficiently large. Here is our functional equation.
\begin{theorem}
	For $f\in M_k(\op{SL}_2(\ZZ))$, define
	\[\Lambda(s,f)\coloneqq(2\pi)^{-s}\Gamma(s)L(s,f).\]
	Then $\Lambda$ has a meromorphic continuation to $\CC$ and satisfies the functional equation
	\[\Lambda(s,f)=(-1)^{k/2}\Lambda(k-s,f).\]
\end{theorem}
\begin{proof}
	Summing the identity
	\[\int_{\RR^+}e^{-2\pi ny}y^s\,\frac{dy}y=(2\pi)^{-s}\Gamma(s)n^{-s}\]
	for $n\ge1$ shows that
	\[\Lambda(s,f)=\int_{\RR^+}f(iy)y^s\,\frac{dy}y.\]
	The result now follows because $f(iy)=(-1)^{k/2}y^{-k}=f(i/y)$ by the modularity of $f$.
\end{proof}

\section{June 14th}
Today we plan on covering \S1.4--1.6.

\subsection{Hecke Operators}
Following Bump, we begin by discussing Hecke operators for $\op{SL}_2(\ZZ)$ and then will discuss Hecke operators for different level in remarks later.
\begin{notation}
	Fix a positive integer $k$. Given holomorphic $f\colon\HH\to\CC$ and $\gamma\coloneqq\begin{bsmallmatrix}
		a & b \\ c & d
	\end{bsmallmatrix}\in\op{GL}_2(\RR)$ with $\deg\gamma>0$, we define $f|_\gamma\colon\HH\to\CC$ by
	\[(f|_\gamma)(z)\coloneqq(\det\gamma)^{k/2}(cz+d)^{-k}f(\gamma\cdot z).\]
\end{notation}
One can check that this creates a right action of $\op{GL}_2(\RR)^+$ on holomorphic functions $\HH\to\CC$.

One would like to know that this action sends modular forms to modular forms, but this has the nasty side effect of adjusting level. Nonetheless, one can check a congruence subgroup $\Gamma\subseteq\op{SL}_2(\ZZ)$ continues to make the conjugate $\gamma^{-1}\Gamma\gamma\cap\op{SL}_2(\ZZ)$ a congruence subgroup for any $\gamma\in\op{GL}_2(\QQ)^+$.\footnote{The corresponding level depends on $\alpha$ and $\Gamma$.} From here, one can indeed check that $f$ being a modular form for a congruence subgroup $\Gamma$ of weight $k$ makes $f|_\gamma$ continue to be a modular form for the congruence subgroup $\gamma^{-1}\Gamma\gamma\cap\op{SL}_2(\ZZ)$ of weight $k$.

We will spend most of the rest of this subsection in level $1$, so we set $\Gamma\coloneqq\op{SL}_2(\ZZ)$ until stated otherwise. Our construction of Hecke operators will rest on certain double coset computations, which we now carry out.
\begin{lemma} \label{lem:double-coset-quotient}
	Fix $\gamma\in\op{GL}_2(\QQ)^+$. Then
	\[\left|\frac{\Gamma\gamma\Gamma}{\Gamma}\right|=\left|\frac{\Gamma}{\gamma^{-1}\Gamma\gamma\cap\Gamma}\right|,\]
	which is finite.
\end{lemma}
\begin{proof}
	Note that the right-hand side is in fact finite because $\gamma^{-1}\Gamma\gamma\cap\Gamma$ is a congruence subgroup, so it suffices to compute
	\[\frac{\Gamma\gamma\Gamma}{\Gamma}\cong\frac{\Gamma\gamma\Gamma\gamma^{-1}}{\Gamma}\cong\frac{\gamma\Gamma\gamma^{-1}}{\Gamma\cap\gamma\Gamma\gamma^{-1}}\cong\frac{\Gamma}{\gamma^{-1}\Gamma\gamma\cap\Gamma},\]
	as required.
\end{proof}
\begin{definition}[Hecke operator]
	Fix a modular form $f$ of weight $k$ and level $\Gamma=\op{SL}_2(\ZZ)$. For each $\alpha\in\op{GL}_2(\QQ)^+$, we define the \textit{Hecke operator}
	\[f|T_\alpha\coloneqq\sum_{\Gamma\gamma\subseteq\Gamma\alpha\Gamma}f|_\gamma.\]
	Note $T_\alpha$ only depends on $\Gamma\alpha\gamma$, so we let $\mc R$ denote the free abelian group generated by $\{T_\alpha\}_{\Gamma\alpha\Gamma}$.
\end{definition}
Modularity of $f$ implies that the choice of representatives $\gamma$ for $\Gamma\gamma\subseteq\Gamma\alpha\Gamma$ does not remember. In fact, for any $\gamma'\in\Gamma$, we see that applying $f|T_\alpha|_{\gamma'}$ merely rearranges right cosets in the sum and thus just equals $f|_{\gamma'}|T_\alpha$, meaning $f|T_\alpha$ will continue to be a modular form of weight $k$.

As suggested by the letter $\mc R$, we would like to define a ring structure. Unsurprisingly, this will be by composition. A direct computation reveals that
\[f|T_\alpha|T_\beta=\sum_{\sigma\in\Gamma\backslash\op{GL}_2(\QQ)^+/\Gamma}m(\alpha,\beta,\sigma)f|T_\sigma,\]
where
\[m(\alpha,\beta,\sigma)\coloneqq\#\{(\Gamma\alpha',\Gamma\beta'):\sigma\in\Gamma\alpha'\beta'\}.\]
(Importantly, one must check that $m(\alpha,\beta,\sigma)$ only depends on $\Gamma\sigma\Gamma$, for example.) Thus, we may extrapolate a definition of $T_\alpha\cdot T_\beta$ from the right-hand side above. A direct computation shows that this multiplication is associative, so we get a (a priori non-commutative) ring; the identity is $T_{I_2}$. (This is not immediate from composition being commutative, sadly.)

We would like to show that $\mc R$ is commutative. Approximately speaking $\mc R$ is the convolution algebra on $\Gamma\backslash\op{GL}_2(\QQ)^+/\Gamma$, so this will be done by providing an anti-involution on the level of these double cosets. As such, we want to understand these double cosets more.
\begin{lemma} \label{lem:gl2-double-coset}
	We have
	\[\Gamma\backslash\op{GL}_2(\QQ)^+/\Gamma=\left\{\Gamma\begin{bmatrix}
		d_1 \\ & d_2
	\end{bmatrix}\Gamma:d_1,d_2\in\QQ,\frac{d_1}{d_2}\in\ZZ^+\right\}.\]
\end{lemma}
\begin{proof}
	For $\alpha\in\op{GL}_2(\QQ)^+$, we need to show that $\Gamma\alpha\Gamma$ has a unique representative in the required form. Existence follows by putting (some positive integer multiple of) $\alpha$ into Smith normal form. Uniqueness follows by a direct computation of what elements of $\Gamma\begin{bsmallmatrix}
		d_1 \\ & d_2
	\end{bsmallmatrix}\Gamma$ look like.
\end{proof}
\begin{proposition}
	The Hecke algebra $\mc R$ is commutative.
\end{proposition}
\begin{proof}
	We must show that $m(\alpha,\beta;\sigma)=m(\beta,\alpha;\sigma)$. It is enough to show that $m(\alpha,\beta;\sigma)=m(\beta,\alpha;\sigma^\intercal)$ because $T_{\sigma}=T_{\sigma^\intercal}$ by selecting the representative $\sigma$ to be diagonal (via the above lemma).

	It will be useful to have some explicit representatives for \Cref{lem:double-coset-quotient}. Begin with any choice of representatives $\{\alpha_i\}$, and then replace a given $\alpha_i$ with an element of $\Gamma\alpha_i\cap\alpha_i^\intercal\Gamma$ so that
	\[\Gamma\alpha\Gamma=\bigcup_i\Gamma\alpha_i=\bigcup_i\alpha_i\Gamma.\]
	We similarly set $\{\beta_j\}$ and $\{\sigma_k\}$ for representatives for \Cref{lem:double-coset-quotient}. Then
	\[\{(i,j):\sigma\in\Gamma\alpha_i\beta_j\Gamma\}=\sum_km(\alpha,\beta;\sigma_k)=\left|\frac{\Gamma\sigma\Gamma}\Gamma\right|m(\alpha,\beta;\sigma).\]
	We now take $\alpha\mapsto\alpha^\intercal$ and $\beta\mapsto\beta^\intercal$ on the left-hand side to see that $m(\alpha,\beta;\sigma)=m(\beta,\alpha;\sigma^\intercal)$.
\end{proof}
\begin{notation}
	Now that our Hecke operators are commutative, we will choose to write $T_\alpha$ on the left, writing $T_\alpha f$ for $f|T_\alpha$.
\end{notation}
We next show that these Hecke operators are self-adjoint. This requires an inner product.
\begin{definition}[Petersson inner product]
	Fix cusp forms $f$ and $g$ of weight $k$ and some level $\Gamma(N)$. Then we define
	\[\langle f,g\rangle\coloneqq\frac1{[\op{SL}_2(\ZZ):\Gamma(N)]}\int_{\Gamma(N)\backslash\HH}f(z)\ov{g(z)}\,y^k\frac{dx\,dy}{y^2}.\]
\end{definition}
One can check that $f$ and $g$ being modular forms implies that the integral is well-defined. Being a cusp form implies that the integral converges (in particular, it will vanish rapidly approaching any cusp of the compact space $\Gamma(N)\backslash\HH^*$).
\begin{proposition}
	Each operator $T_\alpha\in\mc R$ is self-adjoint with respect to the Petersson inner product.
\end{proposition}
\begin{proof}
	Fix cusp forms $f$ and $g$ and some $\alpha\in\op{GL}_2(\QQ)^+$. Light rearrangement verifies $\langle f|_\alpha,g\rangle=\langle f,g|_{\alpha^{-1}}\rangle$. For example, this implies that the computed inner product only depends on the ambient double coset $\Gamma\alpha\Gamma$. As such, we compute
	\[\langle T_\alpha f,g\rangle=\left|\frac{\Gamma\alpha\Gamma}{\Gamma}\right|\langle f|_\alpha,g\rangle,\]
	and we can now move the $\alpha$ over to $g$ and rearrange everything back into $\langle f,T_\alpha g\rangle$ after a little work.
\end{proof}
Thus, $\mc R$ becomes a commutative family of self-adjoint operators acting on the finite-dimensional vector space $S_k(\Gamma)$, so the operators in $\mc R$ are simultaneously diagonalizable by a basis of ``Hecke eigenforms.''

Our last goal is to show that $L(f,s)$ for Hecke eigenforms $f$ admit Euler products. For this, we will use a special subset of Hecke operators.
\begin{notation}
	For positive integer $n$, we define
	\[T_n\coloneqq\sum_{\substack{d_1d_2=n\\d_2\mid d_1}}T_{\op{diag}(d_1,d_2)}.\]
	Letting $\Delta_n\subseteq\ZZ^{2\times2}$ be the subset with determinant $n$, the proof of \Cref{lem:gl2-double-coset} implies
	\[T_nf=\sum_{\Gamma\delta\subseteq\Delta_n}f|_\delta.\]
\end{notation}
Let's compute some representatives.
\begin{lemma}
	For positive integer $n$, we have
	\[\Delta_n=\bigsqcup_{\substack{a,d>0\\ad=n\\0\le b<d}}\Gamma\begin{bmatrix}
		a & b \\ d
	\end{bmatrix}.\]
\end{lemma}
\begin{proof}
	The backward inclusion is clear. The union being disjoint is a direct computation. Lastly, the forward inclusion follows by picking up some element of $\Delta_n$ and doing row-reduction to adjust the bottom-left entry.
\end{proof}
We now compute the behavior of $T_n$.
\begin{lemma} \label{lem:compute-hecke-operator}
	For a cusp form $f$ of weight $k$ and level $\Gamma$ with Fourier expansion $f=\sum_{m\ge1}a_mq^m$, we have
	\[T_nf=\sum_{m\ge1}\Bigg(\sum_{\substack{ad=n\\a\mid m}}\left(\frac ad\right)^{k/2}da_{md/a}\Bigg)q^m.\]
\end{lemma}
\begin{proof}
	Direct expansion with the above lemma shows
	\[T_nf(z)=\sum_{\substack{ad=n\\0\le b<d}}\sum_{m\ge1}\left(\frac ad\right)^ka_me^{2\pi im(az/d)}e^{2\pi im(b/d)}.\]
	Now, we sum over $b$ and rearrange the sum into the desired result.
\end{proof}
\begin{lemma}
	Fix a nonzero cusp Hecke eigenform $f$ of weight $k$ where the operator $T_n$ has eigenvalue $n^{1-k/2}\lambda_n$ for some function $\lambda$. Give $f$ the Fourier expansion $f=\sum_{m\ge1}a_mq^m$.
	\begin{listalph}
		\item $a_1\ne1$.
		\item If $a_1=1$, then $\lambda_m=a_m$ for all $m\ge1$.
		\item If $a_1=1$, then the function $a_\bullet$ is multiplicative.
	\end{listalph}
\end{lemma}
\begin{proof}
	Using the previous lemma, we see that
	\begin{equation}
		n^{1-k/2}\lambda_na_m=\sum_{\substack{ad=n\\a\mid m}}\left(\frac ad\right)^{k/2}da_{md/a}. \label{eq:recursive-hecke}
	\end{equation}
	If $\gcd(m,n)=1$, then the sum must have $(a,d)=(1,n)$, so the sum collapses to $\lambda_na_m=a_{mn}$. The result follows.
\end{proof}
And here is our result.
\begin{theorem}
	Fix a nonzero cusp Hecke eigenform $f$ of weight $k$. Give $f$ the Fourier expansion $f=\sum_{m\ge1}a_mq^m$, scaled so that $a_1=1$. Then
	\[L(s,f)=\prod_p\frac1{1-a_pp^{-s}+p^{k-1-2s}}.\]
\end{theorem}
\begin{proof}
	The previous lemma yields
	\[L(s,f)=\prod_p\Bigg(\sum_{\nu=0}^\infty A\left(p^\nu\right)p^{-\nu s}\Bigg),\]
	so we want to compute this infinite sum. Well, \eqref{eq:recursive-hecke} provides the two-term recurrence
	\[a_{p^{\nu+1}}-a_pa_{p^\nu}+p^{k-1}a_{p^{\nu-1}}=0,\]
	from which we can evaluate the sum.
\end{proof}
Let us conclude by saying a little about Hecke operators attached to congruence subgroups $\Gamma(n)$. We require two important congruence subgroups.
\begin{definition}
	For positive integer $N$, we define
	\begin{align*}
		\Gamma_0(N) &\coloneqq \left\{\begin{bmatrix}
			a & b \\ c & d
		\end{bmatrix}\in\op{SL}_2(\ZZ):c\equiv0\pmod N\right\}, \\
		\Gamma_1(N) &\coloneqq \left\{\begin{bmatrix}
			a & b \\ c & d
		\end{bmatrix}\in\op{SL}_2(\ZZ):a,d\equiv1\pmod N,c\equiv0\pmod N\right\}.
	\end{align*}
	Note $\Gamma(N)\subseteq\Gamma_1(N)\subseteq\Gamma_0(N)$.
\end{definition}
It will be helpful to be able to twist a modular form by a Dirichlet character.
\begin{definition}
	For a weight $k$ and positive integer $N$, we define
	\[M_k(\Gamma_0(N),\chi)\coloneqq\left\{f\in M_k(\Gamma_0(N)):f|_\gamma=\chi(d)f\text{ for }\gamma=\begin{bmatrix}
		a & b \\ c & d
	\end{bmatrix}\in\Gamma_0(N)\right\}.\]
	In the sequel, we abbreviate $\chi(d)$ to $\chi(\gamma)$. We define $S_k(\Gamma_0(N),\chi)$ analogously.
\end{definition}
Now, one can show that our characters produce an orthogonal decomposition
\[S_k(\Gamma_1(N))=\bigoplus_{\chi\pmod N}S_k(\Gamma_0(N),\chi).\]
As such, it suffices to define Hecke operators on the spaces $M_k(\Gamma_0(N),\chi)$ and reassemble later. One does this essentially by doing double coset computations with $\Gamma_0(N)\backslash\op{GL}_2(\ZZ_N)^+/\op{GL}_2(\ZZ)$, where $\ZZ_N$ refers to the localization. The arguments of the above theory more or less goes through.

\subsection{Twisting}
We are going to discuss a few converse theorems. The proofs are rather technical, so (as usual) we will not include them in any nontrivial detail.

The moral of the story is that we produced a functional equation for our $L$-function by taking the Mellin transform of a functional equation of a modular form. Morally, one should be able to take the functional equation for the $L$-function and then take the inverse Mellin transform to recover the functional equation of a modular form. In particular, an $L$-function satisfying a suitable functional equation will then be forced to arise from a modular form!

Following Bump, we will sketch two results of this type.
\begin{theorem} \label{thm:hecke-converse}
	Fix a nonnegative integer $k$ and a sequence $\{a_m\}_{m\ge1}$ be a sequence of complex numbers of polynomial growth, and define
	\[L(s)\coloneqq\sum_{n\ge1}\frac{a_n}{n^s}.\]
	Assume the following.
	\begin{listalph}
		\item Analytic continuation: $\Lambda(s)\coloneqq(2\pi)^{-s}\Gamma(s)L(s,f)$ has an analytic continuation to all $s\in\CC$.
		\item Bounded: $\Lambda(s,f)$ is bounded in vertical strips $\{s:\sigma_1\le\Re s\le\sigma_2\}$.
		\item Functional equation: we have $\Lambda(s)=(-1)^{k/2}\Lambda(k-s)$.
	\end{listalph}
	Then $f\coloneqq\sum_{m\ge1}a_mq^m$ lives in $S_k(\op{SL}_2(\ZZ))$.
\end{theorem}
\begin{remark}
	By controlling the pole produced by a modular form which is not a cusp form, one can state a similar result valid for modular forms.
\end{remark}
As outlined above, we want two lemmas.
\begin{lemma} \label{lem:mellin-inversion}
	For continuous $\varphi\colon\RR^+\to\CC$, we note that the Mellin transform
	\[\mc Mf(s)\coloneqq\int_{\RR^+}\varphi(y)y^s\,\frac{dy}y\]
	is converges absolutely on some vertical strip $\{s:\sigma_1\le\Re s\le\sigma_2\}$. Then for $\sigma$ in this strip, we see
	\[\varphi(y)=\frac1{2\pi i}\int_{\Re s=\sigma}\varphi(s)y^{-s}\,ds.\]
\end{lemma}
\begin{proof}
	Bump proves this by relating the Mellin transform to the Fourier transform (which can be done via the isomorphism $\exp\colon\RR\to\RR^+$ of topological groups) and then appealing Fourier inversion. One can also prove this in the same way as the Fourier inversion formula.
\end{proof}
\begin{lemma}[Phragm\'en--Lindel\"of] \label{lem:pl-principle}
	Fix a function $f$ holomorphic on some strip $\{s:\sigma_1\le\Re s\le\sigma_2,\Im s>c\}$ and satisfying a growth condition $f(\sigma+it)=O\left(e^{t^\alpha}\right)$ (as $t\to\infty$) for some real $\alpha$. Then if $f(\sigma+it)=O\left(t^M\right)$ (as $t\to\infty$) for $\sigma\in\{\sigma_1,\sigma_2\}$, then the same bound holds uniformly for $\sigma\in[\sigma_1,\sigma_2]$.
\end{lemma}
\begin{proof}
	By replacing $f$ with $f(s)/s^M$, we may assume that $M=0$. Without loss of generality, we may take $t$ large so that the desired strip occupies a small sector of $\CC$. By shifting and dividing up $[\sigma_1,\sigma_2]$, we may assume that $\sigma_2>0$ is small and $\sigma_1=-\sigma_2$. The point is that our arguments are close to $\pi/2$, so we choose $m\equiv2\pmod4$ of moderate size so that $m\arg s\approx\pi$ for desired $s$. Now, for small $\varepsilon>0$, we consider
	\[g_\varepsilon(s)\coloneqq f(s)e^{\varepsilon s^m}.\]
	One can show that $g_\varepsilon(s)$ is bounded on a rectangle determined by the constraints of $f$, so one receives a bound on $f$ by taking $\varepsilon\to0^+$.
\end{proof}
We now prove \Cref{thm:hecke-converse}.
\begin{proof}[Proof of \Cref{thm:hecke-converse}]
	Define $f$ as in the conclusion, and we want to show that $f\in S_k(\op{SL}_2(\ZZ))$. (The polynomial growth condition on $a_\bullet$ is included so that $L(s)$ converges for $\Re s$ large.) Note $f$ has a Fourier expansion, so $f(z+1)=f(z)$ already; it is thus sufficient to check the functional equation for $\begin{bsmallmatrix}
		0 & -1 \\ 1 & 0
	\end{bsmallmatrix}$. By analytic continuation, we may check the functional equation for $iy\in i\RR^+$. Now, we recall that
	\[\int_{\RR^+}f(iy)y^s\,\frac{dy}y=\Lambda(s),\]
	so Mellin inversion (\Cref{lem:mellin-inversion}) yields
	\[f(iy)=\frac1{2\pi i}\int_{\Re s=\sigma}\Lambda(s)y^{-s}\,ds.\]
	Now, we use the functional equation to replace $s$ with $k-s$. Note $\Lambda$ exhibits rapid decay for $\Re s$ very large and very small, so \Cref{lem:pl-principle} tells us that we exhibit this rapid decay uniformly on any vertical strip $\{s:\sigma_1\le\Re s\le\sigma_2\}$. The point is that we don't have to worry about convergence issues, so we send $k-s\mapsto-s$, from the modularity of $f$ follows.
\end{proof}
\begin{remark}
	In fact, if $L(s)$ further admits an Euler product of the form
	\[L(s)=\prod_p\frac1{1-a_pp^{-s}+p^{k-1-2s}},\]
	then $f$ is a Hecke eigenform. The point is that the Euler product implies a particular recursion among the Fourier coefficients, from which one can use \Cref{lem:compute-hecke-operator} to show that we have a Hecke operator with the expected Hecke eigenvalues.
\end{remark}
Our next converse theorem, due to Weil, requires us to twist our modular forms by Dirichlet characters. In particular, we will deduce our converse theorem from twisted functional equations.
\begin{notation}
	For $f\in S_k(\Gamma_0(N),\psi)$ with Fourier expansion $f=\sum_{m\ge1}a_mq^m$ and Dirichlet character $\chi\pmod D$, we define
	\begin{align*}
		f_\chi(z) &\coloneqq \sum_{n=1}^\infty\chi(n)a_nq^n, \\
		L(s,f,\chi) &\coloneqq \sum_{n=1}^\infty\frac{\chi(n)a_n}{n^s}, \\
		\Lambda(s,f,\chi) &\coloneqq (2\pi)^{-s}\Gamma(s)L(s,f,\chi).
	\end{align*}
	Technically, one does not require $f$ to be a modular form.
\end{notation}
Here is our functional equation.
\begin{proposition}
	Fix $f,g\in S_k(\Gamma_0(N))$ and primitive Dirichlet equation $\chi\pmod D$ such that $\gcd(N,D)=1$. If $f=g|_{w_N}$ where $w_N\coloneqq\begin{bsmallmatrix}
		& -1 \\ N
	\end{bsmallmatrix}$, then
	\[\Lambda(s,f,\chi)=i^k\chi(N)\psi(D)\frac{\tau(\chi)^2}D\left(D^2N\right)^{-s+k/2}\cdot\Lambda(k-s,g,\ov\chi).\]
\end{proposition}
\begin{proof}
	Note that $w_N$ normalizes $\Gamma_0(N)$, so our hypothesis at least makes sense. A discrete Fourier transform shows
	\[f_\chi=\frac{\chi(-1)\tau(\chi)}D\sum_{m\in(\ZZ/D\ZZ)^\times}\ov\chi(m)f|_{\begin{bsmallmatrix}
		D & m \\ & D
	\end{bsmallmatrix}}.\]
	For example, one can use some rearrangement to shows that this implies
	\begin{align*}
		f_\chi|_{\begin{bsmallmatrix}
			& -1 \\ D^2N
		\end{bsmallmatrix}} &= \chi(N)\frac{\tau(\chi)}{D}\sum_{r\in(\ZZ/D\ZZ)^\times}\chi(r)g|_{\begin{bsmallmatrix}
			D & -r \\ -Nm & s
		\end{bsmallmatrix}\begin{bsmallmatrix}
			D & r \\ & D
		\end{bsmallmatrix}} \\
		&= \chi(N)\psi(D)\frac{\tau(\chi)^2}D\cdot g_{\ov\chi}.
	\end{align*}
	Now, in the usual way, plug in $iy$ into this equation and apply the Mellin transform to conclude.
\end{proof}
One would like a converse theorem from these functional equations.
\begin{theorem}[Weil]
	Fix a positive integer $N$ and Dirichlet character $\psi\pmod N$. Further, fix sequences of complex $\{a_m\}$ and $\{b_m\}$ exhibiting polynomial growth, and define the functions $f\coloneqq\sum_{m}a_mq^m$ and $g\coloneqq\sum_mb_mq^m$ so that we can define $L(s,f,\chi)$ and so on as usual. Lastly, fix a finite set of primes $S$ (including the prime divisors of $N$), and we assume the following for all Dirichlet characters $\chi$ with conductor $D$ or a prime not in $S$.
	\begin{listalph}
		\item Analytic continuation: $\Lambda(s,f,\chi)$ and $\Lambda(s,g,\ov\chi)$ has an analytic continuation to all $s\in\CC$.
		\item Bounded: $\Lambda(s,f,\chi)$ and $\Lambda(s,g,\ov\chi)$ are bounded in vertical strips.
		\item Functional equation: we have
		\[\Lambda(s,f,\chi)=i^k\chi(N)\psi(D)\frac{\tau(\chi)^2}D\left(D^2N\right)^{-s+k/2}\Lambda(s,g,\ov\chi).\]
	\end{listalph}
	Then $f\in S_k(\Gamma_0(N),\psi)$.
\end{theorem}
\begin{proof}
	Several pages of manipulation of $2\times2$ matrices. The primary difficulty is that $\Gamma_0(N)$ may potentially have lots of generators, so the same proof technique will not work verbatim. Nonetheless, redoing the proof of \Cref{thm:hecke-converse} does imply
	\[f_\chi|_{\begin{bsmallmatrix}
		& -1 \\ D^2N
	\end{bsmallmatrix}}=\chi(N)\psi(D)\frac{\tau(\chi)^2}D\cdot g_{\ov\chi}.\]
	One now does a lengthy computation to bootstrap this into the required result.
\end{proof}

% \subsection{The Rankin--Selberg Method}


\end{document}
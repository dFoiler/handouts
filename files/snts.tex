\documentclass{article}
\usepackage[utf8]{inputenc}

\newcommand{\nirpdftitle}{Student Number Theory Seminar}
\usepackage{import}
\inputfrom{../../notes}{nir}
\usepackage[backend=biber,
    style=alphabetic,
    sorting=ynt
]{biblatex}
\setcounter{tocdepth}{2}

\pagestyle{contentpage}

\title{Student Number Theory Seminar}
\author{Nir Elber}
\date{Fall 2024}
\usepackage{graphicx}

\begin{document}

\maketitle

\tableofcontents

\section{September 4th: Yuchan Lee}
Today we are talking a bit about orbital integrals. As some motivation, orbital integrals are the object of interest on the geometric side of the Arthur--Selberg trace formula. It is frequently desirable to compute these orbital integrals in order to use the trace formula to extract spectral information.

\subsection{}
Today, $F$ is a nonarchimedean local field. We let $\mf o$ denote its ring of integers (with uniformizer $\pi$), and we let $\kappa$ be its residue field (with cardinality $q$). We further fix a classical Lie group $G$ over $K$, with Lie algebra $\mf g$.

A regular semisimple element $\gamma\in\mf g(\mf o)$ gives rise to an orbit $\OO_\gamma$. We want to compute its volume, which is the stable orbital integral of $\gamma$. Over the past two decades, we have developed two methods for the computation: analytic and geometric, where the geometric method uses the theory of Bruhat--Tits buildings.

For some $\mf g$, we consider the map $\varphi_n\colon\mf g(\mf o)\to\mf o^n$ given by sending some $m$ to the coefficients of its characteristic polynomial. Notably, $\OO_\gamma$ is the fiber of $\chi_\gamma$ for some regular semisimple $\gamma$. Notably, $\varphi_n$ ugprades to a morphism of schemes, so one can hope to compare what happens on the geometric and special fibers. For example, one has the following result.
\begin{theorem}[Weil]
	For a smooth scheme $\mf X$ over $\mf o$, one has
	\[\int_{\mf X(\mf o)}\left|\omega\right|=\frac{\left|\mf X(\kappa)\right|}{q^n}.\]
\end{theorem}
However, one cannot hope for our map $\varphi_n$ to be smooth over $\mf o$. Thus, the strategy is to stratify our integral over $\varphi_n^{-1}(\chi_\gamma)(\mf o)$ and then smoothen over each stratum. The stratification depends on the theory of finitely generated $\mf o$-modules, and the smoothening is by some $\pi$-adic congruence condition.

Let's begin by discussing $\mf{gl}_n$. For simplicity, we will suppose that $\chi_\gamma(xx)$ is irreducible, and we let $L$ be a free $\mf o$-module of rank $n$. Now, we can see
\[\OO_\gamma=\{m\in\op{End}_{\mf o}(L):\chi_m(x)=\chi_\gamma(x)\}.\]
Letting $d$ be the valuation of $\det\gamma$, we also define $M\subseteq L$ to be the image of some $m$. The point is that we can look at the quotient $L/M$ and separate it according to the theory of finitely generated modules over $\mf o$ (which is a principal ideal domain). Splitting up the integral accordingly permits an explicit computation of the orbital integrals.

\end{document}
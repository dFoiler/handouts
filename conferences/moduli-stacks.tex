\documentclass{article}
\usepackage[utf8]{inputenc}

\newcommand{\nirpdftitle}{Student Arithmetic Geometry Seminar}
\usepackage{import}
\inputfrom{../../notes}{nir}
\usepackage[backend=biber,
    style=alphabetic,
    sorting=ynt
]{biblatex}
\setcounter{tocdepth}{2}

\pagestyle{contentpage}

\title{Student Arithmetic Geometry Seminar}
\author{Nir Elber}
\date{Spring 2024}
\usepackage{graphicx}

\begin{document}

\maketitle

\tableofcontents

\section{January 19: Martin Olsson}
Today, we will provide an introduction to what this seminar will be about.

\subsection{Motivating Paper}
We are motivated by Deligne--Mumford's paper ``The irreducibility of moduli of curves of given genus.'' The main theorem is as follows.
\begin{theorem}[Deligne--Mumford] \label{thm:dm}
	Fix a nonnegative integer $g\ge2$ and a field $k$. Then the moduli space $\mc M_g$ of curves of genus $g$ over $k$ is irreducible.
\end{theorem}
There is quite a bit which goes into this proof.
\begin{itemize}
	\item Curve theory.
	\item The formalism of moduli problems.
	\item The Semistable reduction theorem.
	\item Jacobian varieties.
	\item The notion of stable curves and the compactification $\ov{\mc M}_g$.
	\item Deformation theory.
	\item The theory of stacks and coarse moduli spaces.
	\item An analytic description to get from $\CC$.
\end{itemize}
From these ingredients, there are other directions to go in.
\begin{itemize}
	\item Finer geometry of $\mc M_g$.
	\item Moduli of curves with marked points $\mc M_{g,n}$.
	\item Higher-dimensional directions, such as abelian varieties.
\end{itemize}
Let's go ahead and provide a sketch.
\begin{proof}[Sketch of \Cref{thm:dm}]
	The result is known over $\CC$ via some analytic methods. We want to get to other fields. We begin with the following exercise.
	\begin{proposition} \label{prop:reduce-conn}
		Fix a complete discrete valuation ring $(R,\mf m)$ with fraction field $K$ and residue field $\kappa\coloneqq R/\mf m$. Fix a smooth proper morphism $f\colon X\to\Spec R$. If $X_{\ov K}$ is connected, then $X_{\ov\kappa}$ is also connected.
	\end{proposition}
	\begin{proof}[Sketch]
		One can assume that $\kappa$ is algebraically closed by some argument. Then the Formal functions theorem tells us that
		\[H^0(X,\OO_X)=\limit_nH^0(X_n,\OO_{X_n}),\]
		where $X_n\coloneqq X\times_V\Spec V/\mf m^n$. Now, suppose for the sake of contradiction that $X_\kappa=X^1_\kappa\sqcup X^2_\kappa$. Then by taking nilpotent thickenings, we see that $X_n=X^1_n\sqcup X^2_n$, so
		\[\limit_nH^0(X_n,\OO_{X_n})=\limit_nH^0(X_n^1,\OO_{X_n^1})\times\limit_nH^0(X_n^2,\OO_{X_n^2}).\]
		So we are receiving a product of rings $A_1\times A_2$, which are flat over $R$, so by viewing the global sections back in $H^0(X,\OO_X)$, which should be $\ov K$ upon algebraic closure, we will receive our contradiction.
	\end{proof}
	Now, the (coarse) moduli space $\mc M_g$ fails to be either smooth or proper, but some theory of stacks allows us to reduce to this case. Namely, the point is to find some compactification $\mc M_g\into\ov{\mc M}_g$ where $\ov{\mc M}_g$ is a smooth proper $\ZZ$-stack with $\mc M_g\subseteq\ov{\mc M}_g$ dense in each fiber. As such, we can imagine using \Cref{prop:reduce-conn} to pull back the result from $\ov{\mc M}_g$ to $\mc M_g$.
\end{proof}

\subsection{Thinking about the Moduli Spaces}
Let's describe what $\mc M_g$ is. By the functor of points description, we merely need to describe maps $S\to\mc M_g$, which we declare to be in natural bijection with genus-$g$ curves $\pi\colon C\to S$; i.e., $\pi$ is a proper, flat, and smooth morphism whose geometric fibers are genus-$g$ curves.
\begin{example}
	There is a family of curves over a field $k$ given by the equations
	\[y^2=(x-a_1)\cdots(x-a_n),\]
	where $a_1,\ldots,a_n$ are allowed to vary. Viewing the $a_\bullet$s as giving a point in affine space, we see that we are (approximately speaking) producing a rational map $\AA^n\to\mc M_g$, where perhaps we need to check that we have a curve of the correct genus.
\end{example}
Now, using a functor of points description, smoothness of $\mc M_g$ over $\Spec\ZZ$ is requiring the following (in the sense of formal smoothness): for any surjection $A'\onto A$ with kernel $J$ such that $J^2=0$, any morphism $\Spec A\to\mc M_g$ induces a unique dashed arrow.
% https://q.uiver.app/#q=WzAsNCxbMCwwLCJcXFNwZWMgQSJdLFswLDEsIlxcU3BlYyBBJyJdLFsxLDAsIlxcbWMgTV9nIl0sWzEsMSwiXFxTcGVjXFxaWiJdLFsxLDNdLFswLDFdLFswLDJdLFsyLDNdLFsxLDIsIiIsMSx7InN0eWxlIjp7ImJvZHkiOnsibmFtZSI6ImRhc2hlZCJ9fX1dXQ==&macro_url=https%3A%2F%2Fraw.githubusercontent.com%2FdFoiler%2Fnotes%2Fmaster%2Fnir.tex
\[\begin{tikzcd}
	{\Spec A} & {\mc M_g} \\
	{\Spec A'} & \Spec\ZZ
	\arrow[from=2-1, to=2-2]
	\arrow[from=1-1, to=2-1]
	\arrow[from=1-1, to=1-2]
	\arrow[from=1-2, to=2-2]
	\arrow[dashed, from=2-1, to=1-2]
\end{tikzcd}\]
As such, we can unwind the functor of points description of $\mc M_g$ to prove something like smoothness, which is quite remarkable.

Analogously, we can describe $\ov{\mc M}_g$ via the functor of points: maps $S\to\ov{\mc M}_g$ are ``stable curves'' $\pi\colon C\to S$, which will be a proper flat morphism whose geometric fibers by $\ov s\to S$ satisfy the following.
\begin{itemize}
	\item The geometric fibers are nodal curves, meaning that the completion at any closed point is $\kappa(\ov s)\bb x$ or $\kappa(\ov s)\bb{x,y}/(xy)$.
	\item Every rational component (namely, irreducible component whose normalization is $\PP^1$) has three distinguished points. (These three points are desirable, for example, to ensure that its automorphism group is trivial.)
\end{itemize}
Now, we can also check being proper via the functor of points description, using the valuative criterion. Namely, for a complete discrete valuation ring $R$ with fraction field $K$, we would like to know that any map $\Spec K\to\ov{\mc M}_g$ induces a unique dashed arrow.
% https://q.uiver.app/#q=WzAsNCxbMCwxLCJcXFNwZWMgUiJdLFsxLDEsIlxcU3BlY1xcWloiXSxbMSwwLCJcXG92e1xcbWMgTX1fZyJdLFswLDAsIlxcU3BlYyAgSyJdLFswLDFdLFszLDBdLFszLDJdLFsyLDFdLFswLDIsIiEiLDEseyJzdHlsZSI6eyJib2R5Ijp7Im5hbWUiOiJkYXNoZWQifX19XV0=&macro_url=https%3A%2F%2Fraw.githubusercontent.com%2FdFoiler%2Fnotes%2Fmaster%2Fnir.tex
\[\begin{tikzcd}
	{\Spec  K} & {\ov{\mc M}_g} \\
	{\Spec R} & \Spec\ZZ
	\arrow[from=2-1, to=2-2]
	\arrow[from=1-1, to=2-1]
	\arrow[from=1-1, to=1-2]
	\arrow[from=1-2, to=2-2]
	\arrow["{!}"{description}, dashed, from=2-1, to=1-2]
\end{tikzcd}\]
And again, we can directly turn this into a geometry problem of curves by tracking through the moduli interpretation, which is the Semistable reduction theorem.

\section{January 26th: Martin Olsson}

Today we discuss the setup for our moduli problems.

\subsection{The Yoneda Lemma}
Here is the statement.
\begin{theorem}[Yoneda] \label{thm:yo}
	Fix a category $\mc C$, and let $\mathrm{PSh}(\mc C)$ denote the category of presheaves on $\mc C$.
	\begin{listalph}
		\item For $X\in\mc C$, the functor $h_X\colon A\mapsto\op{Mor}(A,X)$ is a presheaf $h_X\colon\mc C\opp\to\mathrm{Set}$.
		\item There is a natural bijection
		\[\op{Mor}_{\mathrm{PSh}(\mc C)}(h_X,F)\to FX.\]
		\item The construction $h_\bullet$ forms a fully faithful embedding $h_\bullet\colon\mc C\to\mathrm{PSh}(\mc C)$.
	\end{listalph}
\end{theorem}
\begin{proof}
	We won't bother to show (a). For (b), the forward map sends $\eta\colon h_X\Rightarrow F$ to $\eta_X({\id_X})$, and one can check that this is a bijection. To show (c), we note that we need to show
	\[\op{Mor}_\mc C(X,Y)\simeq\op{Mor}_{\mathrm{Pre}(\mc C)}(h_X,h_Y),\]
	but this simply follows by taking $F=h_Y$ in (b).
\end{proof}
\begin{remark}
	Most of the time, we will take $\mc C=\mathrm{Sch}(S)$ for a fixed base scheme $S$.
\end{remark}
Anyway, we can now make the following definition.
\begin{definition}[representable]
	Fix a category $\mc C$. A functor $F\colon\mc C\opp\to\mathrm{Set}$ is \textit{representable} if and only if $F\simeq h_X$ for some object $X\in\mc C$. In the sequel, we may want to fix the isomorphism $F\simeq h_X$, which can be specified by an element $\xi\in FX$.
\end{definition}
Here are some examples.
\begin{example}
	Take $\mc C\coloneqq\mathrm{Sch}$, and consider the functor $F\colon\mc C\opp\to\mathrm{Set}$ defined by $FY\coloneqq\Gamma(Y,\OO_Y)^n$. We claim $F$ is represented by $\AA^n_\ZZ=\Spec\ZZ[x_1,\ldots,x_n]$. Indeed, we see
	\[\op{Mor}_{\mathrm{Sch}}(Y,\AA^n_\ZZ)\simeq\op{Hom}_{\mathrm{Ring}}(\ZZ[x_1,\ldots,x_n],\OO_Y(Y))\simeq\OO_Y(Y)^n,\]
	as desired. To specify this isomorphism, \Cref{thm:yo} tells us that it is enough to track through the identity map in $\op{Mor}(\AA^n_\ZZ,\AA^n_\ZZ)\simeq\Gamma(\AA^n_\ZZ,\OO_{\AA^n_\ZZ})^n$, which we can track through is $(x_1,\ldots,x_n)$. Of course, we could choose other isomorphisms, such as determined by the element $(x_n,\ldots,x_1)\in\Gamma(\AA^n_\ZZ,\OO_{\AA^n_\ZZ})^n$.
\end{example}
\begin{example}
	Take $\mc C\coloneqq\mathrm{Sch}$, and consider the functor $F\colon\mc C\opp\to\mathrm{Set}$ defined by
	\[FY\coloneqq\left\{(f_\bullet)\in\Gamma(Y,\OO_Y)^n:(f_1,\ldots,f_n)=\Gamma(Y,\OO_Y)\right\}.\]
	Then $F$ is represented by $\AA^n_\ZZ\setminus\{0\}$.
\end{example}
\begin{example} \label{ex:moduli-pn}
	Take $\mc C\coloneqq\mathrm{Sch}$, and consider the functor $F\colon\mc C\opp\to\mathrm{Set}$ defined by setting $FY$ to the collection of surjections $\OO_Y^{\oplus n}\onto\mc L$ up to isomorphism, where $\mc L/Y$ is a line bundle. Then $F$ is represented by $\PP^n_\ZZ$. The representing element in $F\left(\PP^n_\ZZ\right)$ is given by the surjection $(x_0,\ldots,x_n)\colon\OO_{\PP^n_\ZZ}\onto\OO_{\PP^n_\ZZ}(1)$.
\end{example}
\begin{remark}
	By identifying a line with the corresponding quotient space, we see that the above interpretation of $\PP^n_k$ agrees with the interpretation as ``lines in $k^{n+1}$.''
\end{remark}
\begin{example}
	Fix a field $k$ and homogeneous polynomial $f\in k[x_0,\ldots,x_n]$ of degree $N$, and consider $V(f)\subseteq\PP^n_k$. A map $Y\to V(f)$ will certainly produce a map $Y\to\PP^n_\ZZ$, so our answer should be a subfunctor of \Cref{ex:moduli-pn}. But then we want to determine which surjections $(s_0,\ldots,s_n)\colon\OO_Y^{\oplus n}\onto\mc L$ (which really consists of $n+1$ global sections of $\mc L$ globally generating), but then to live in $V(f)$, we request that $f(s_0,\ldots,s_n)=0$.
\end{example}
\begin{remark}
	Note $\AA^n_\ZZ\subseteq\PP^n_\ZZ$ (e.g. going to the standard affine open $x_0\ne0$). So we expect to have an inclusion $h_{\AA^n_\ZZ}\subseteq h_{\PP^n_\ZZ}$, and one can track through that it simply corresponds to the provided map $(s_0,\ldots,s_n)\colon\OO_Y^{\oplus(n+1)}\onto\mc L$ having $s_0$ be an isomorphism. In this case, after identifying $\OO_Y$ with $\mc L$ via $s_0$, we see that the other sections are indeed providing an element of $\Gamma(Y,\OO_Y)^n$.
\end{remark}
\begin{example}[Hilbert scheme]
	Fix a flat separated $X$-scheme $S$. Then the Hilbert functor assigns a $Y$-scheme $S$ to flat proper (locally of finite presentation) subschemes $Z\subseteq X\times_SY$. It turns out that this functor is representable if $X$ is quasi-projective, which is a result due to Grothendieck.
\end{example}

\subsection{Functors Not Representable}
Representable functors have the tendency to be sheaves. For the Zariski topology, here is the relevant definition.
\begin{definition}[Zariski sheaf]
	A functor $F\colon\mc C\opp\to\mathrm{Set}$ is a \textit{Zariski sheaf} if and only if we have the usual equalizer diagram
	\[FY\to\prod_{\alpha\in\kappa}FY_\alpha\twoheadleftarrow\prod_{\alpha,\beta\in\kappa}F(Y_\alpha\cap Y_\beta),\]
	where $\{Y_\alpha\}_{\alpha\in\kappa}$ is a Zariski open cover of $Y$.
\end{definition}
\begin{example}
	If $F\simeq h_X$ is representable, then $F$ is a Zariski sheaf because morphisms glue.
\end{example}
\begin{nex}
	Define the functor $F$ as taking a scheme $Y$ and taking the quotient of the set of $(n+1)$-tuples $(s_0,\ldots,s_n)\in\Gamma(Y,\OO_Y)$ by multiplication by $\Gamma(Y,\OO_Y^\times)$. This is not a sheaf, but if we do sheafification, we will recover \Cref{ex:moduli-pn}. Thus, this functor is not representable.
\end{nex}
As another remark, we have the following.
\begin{lemma} \label{lem:field-extension-rep}
	If $X$ is a scheme, and $L/K$ is a field extension, then the map $X(\Spec K)\to X(\Spec L)$ is injective.
\end{lemma}
\begin{proof}
	Indeed, a map $\Spec K\to X$ is simply a point $x\in X$ together with an inclusion $\kappa(x)\to K$, which by a similar description for $\Spec L\to X$ is uniquely determined by that map $\Spec L\to X$.
\end{proof}
\begin{example}
	Define the functor $F$ by taking a scheme $Y$ and returning elliptic curves over $Y$ (up to isomorphism). But there are distinct elliptic curves over $\QQ$ which become isomorphic over $\ov\QQ$. For example, $y^2=x^3+D$ and $y^2=x^3-D$ is one such example, which become isomorphic over $\QQ(i)$ where we can send $(x,y)\mapsto(-x,iy)$. Thus, $F$ is not representable by \Cref{lem:field-extension-rep}.
\end{example}
\begin{remark}
	One can argue similarly for curves of genus $g\ge2$.
\end{remark}

\section{February 16th: Martin Olsson}
I missed one week, and then one week was cancelled. I am a little too tired to take detailed notes.
\begin{remark}
	Recall that the Yoneda lemma asserts that representing a functor $F\colon\mathrm{Sch}_S\opp\to\mathrm{Set}$ amounts to specifying an object $X$ in addition to an element $\xi\in FX$ so that the induced map $h_X\Rightarrow F$ (given by $\xi$) is representable.
\end{remark}
\begin{example}
	The Hilbert functor $\op{Hilb}_{\PP^n}$ is represented by some scheme $H$. In particular, we are provided an isomorphism $\sigma\colon h_H\Rightarrow\op{Hilb}_{\PP^n}$, so passing $\id_H$ through implies we are being given a flat proper subscheme $\mc Z\subseteq\PP^n_H$ so that $\op{Hilb}_H(S)$ is merely given by $h_H(S)$, which consists of pullbacks of $\mc Z$ along the morphisms $f\colon S\to H$. The point is that we can really see the element of $\op{Hilb}_H(S)$ provided to us by some $f\in h_H(S)$ as $f^*\mc Z$.
\end{example}
\begin{example}
	The functor $\mc M_{0,n}$ takes a scheme $S$ and then asks for curves $f\colon C\to S$ (i.e., $f$ is proper and flat where all fibers are isomorphic to $\PP^1$), and we mark $n$ points (in fact, $S$-points) somewhere on $C$. The issue here is that we are considering these up to isomorphism, but it turns out that an automorphism of $C$ is basically determined by where it sends three points. When $S=\Spec k$, this is classical; in general, one needs to make some more global argument. So, for example, one finds that $\mc M_{0,3}$ is $\Spec\ZZ$ (where the universal curve is $\PP^1_\ZZ$ with the marked points $\{0,1,\infty\}$).
\end{example}
\begin{example}
	Continuing from the above example, we see $\mc M_{0,4}$ is $\PP^1_\ZZ\setminus\{0,1,\infty\}$ basically by trying to mark a fourth point of $\PP^1_\ZZ$ (outside $\{0,1,\infty\}$). Here are universal curve is given by
	\[\PP^1\times\left(\PP^1_\ZZ\setminus\{0,1,\infty\}\right).\]
	Then our first three marked points are $\{0\}\times\mc M_{0,4}$ and $\{1\}\times\mc M_{0,4}$ and $\{\infty\}\times\mc M_{0,4}$, and the last marked point is given by what we chose in $\mc M_{0,4}$ to begin with, so it is given by the diagonal embedding.
\end{example}
\begin{remark}
	One takes the compactification $\ov{\mc M}_{0,n}$ of $\mc M_{0,n}$ by taking nodal curves (without automorphisms) instead of just curves. Using this one can realize $\ov{\mc M}_{0,4}$ by $\PP^1$, and we can track through what the points $\{0,1,\infty\}$ mean. For example, our universal curve is $\PP^1\times\PP^1$ blown up at three points in $\{(0,0),(1,1),(\infty,\infty)\}$.
\end{remark}
\begin{remark}
	Continuing, it turns out that $\ov{\mc M}_{0,5}$ ought to be the universal curve over $\ov{\mc M}_{0,4}$. The point is that choosing $5$ points amounts to choosing $4$ points first (which is $\mc M_{0,4}$) and then the last point goes up a dimension. This continues inductively.
\end{remark}

\section{February 23rd: Charley Hutchison}
Today we are talking about deformation of curves. Work over a field $k$. For our purposes, a deformation of a morphism $X_0\to\Spec k$ is a pullback square of the form
% https://q.uiver.app/#q=WzAsNCxbMCwwLCJYXzAiXSxbMCwxLCJcXFNwZWMgayJdLFsxLDEsIlMiXSxbMSwwLCJYIl0sWzAsM10sWzMsMl0sWzEsMl0sWzAsMV1d&macro_url=https%3A%2F%2Fraw.githubusercontent.com%2FdFoiler%2Fnotes%2Fmaster%2Fnir.tex
\[\begin{tikzcd}
	{X_0} & X \\
	{\Spec k} & S
	\arrow[from=1-1, to=1-2]
	\arrow[from=1-2, to=2-2]
	\arrow[from=2-1, to=2-2]
	\arrow[from=1-1, to=2-1]
\end{tikzcd}\]
where $X\to S$ should be flat. For our purposes, we will mostly work with ``infinitesimal deformations'' where $S$ is an Artinian local ring such as $S=\Spec k[\varepsilon]/\left(\varepsilon^2\right)$; we'll write $k[\varepsilon]\coloneqq k[\varepsilon]/\left(\varepsilon^2\right)$.
\begin{example}
	Let $A\onto B$ be a closed embedding where $B=A/I$. Then getting a deformation by $\Spec k[\varepsilon]/\left(\varepsilon^2\right)$ is asking for an ideal $\widetilde I\subseteq A[\varepsilon]$ such that
	\[\frac{A[\varepsilon]}{\widetilde I}\otimes_{k[\varepsilon]}k\cong A/I.\]
	Namely, we have just turned our scheme-theoretic diagram into a ring-theoretic problem. For example, given an $A$-module map $\varphi\colon I\to B$, one can define
	\[I_\varphi\coloneqq\{x+\varepsilon y:x\in I,y\in A,\varphi(x)\equiv y\pmod I\},\]
	and one can see that $I_\varphi\subseteq A[\varepsilon]$ will do the trick. In fact, all possible deformations will take this form: one can take such an ideal $\widetilde I$ and then define a map $I\to B$ by lifting a given $x\in I$ to some $\widetilde x+\varepsilon\widetilde y$ and then outputting $\widetilde y$. (This is well-defined by flatness.)
\end{example}
The above example classifies our ``affine'' infinitesimal deformations of a closed embedding. Flatness is crucial to the discussion, though it is not totally obvious how. Namely, a $\op{Tor}^1$ computation tells us that a module $M$ over $k[\varepsilon]$ succeeds in being flat if and only if
\[0\to k\varepsilon\to k[\varepsilon]\to k\to0\]
succeeds in staying exact after applying $-\otimes_{k[\varepsilon]}M$. The point is that we are able to build a large diagram
% https://q.uiver.app/#q=WzAsMjEsWzAsMSwiMCJdLFsxLDEsIkkiXSxbMiwxLCJcXHdpZGV0aWxkZSBJIl0sWzMsMSwiSSJdLFswLDIsIjAiXSxbNCwxLCIwIl0sWzEsMiwiQSJdLFsyLDIsIkFbXFx2YXJlcHNpbG9uXSJdLFszLDIsIkEiXSxbNCwyLCIwIl0sWzEsMywiQiJdLFsyLDMsIkFbXFx2YXJlcHNpbG9uXS9cXHdpZGV0aWxkZSBJIl0sWzMsMywiQiJdLFs0LDMsIjAiXSxbMSw0LCIwIl0sWzIsNCwiMCJdLFszLDQsIjAiXSxbMCwzLCIwIl0sWzEsMCwiMCJdLFsyLDAsIjAiXSxbMywwLCIwIl0sWzAsMV0sWzEsMl0sWzIsM10sWzMsNV0sWzQsNl0sWzYsN10sWzcsOF0sWzgsOV0sWzE3LDEwXSxbMTAsMTFdLFsxMSwxMl0sWzEyLDEzXSxbMTgsMV0sWzEsNl0sWzYsMTBdLFsxMCwxNF0sWzE5LDJdLFsyLDddLFs3LDExXSxbMTEsMTVdLFsyMCwzXSxbMyw4XSxbOCwxMl0sWzEyLDE2XV0=&macro_url=https%3A%2F%2Fraw.githubusercontent.com%2FdFoiler%2Fnotes%2Fmaster%2Fnir.tex
\[\begin{tikzcd}
	& 0 & 0 & 0 \\
	0 & I & {\widetilde I} & I & 0 \\
	0 & A & {A[\varepsilon]} & A & 0 \\
	0 & B & {A[\varepsilon]/\widetilde I} & B & 0 \\
	& 0 & 0 & 0
	\arrow[from=2-1, to=2-2]
	\arrow[from=2-2, to=2-3]
	\arrow[from=2-3, to=2-4]
	\arrow[from=2-4, to=2-5]
	\arrow[from=3-1, to=3-2]
	\arrow[from=3-2, to=3-3]
	\arrow[from=3-3, to=3-4]
	\arrow[from=3-4, to=3-5]
	\arrow[from=4-1, to=4-2]
	\arrow[from=4-2, to=4-3]
	\arrow[from=4-3, to=4-4]
	\arrow[from=4-4, to=4-5]
	\arrow[from=1-2, to=2-2]
	\arrow[from=2-2, to=3-2]
	\arrow[from=3-2, to=4-2]
	\arrow[from=4-2, to=5-2]
	\arrow[from=1-3, to=2-3]
	\arrow[from=2-3, to=3-3]
	\arrow[from=3-3, to=4-3]
	\arrow[from=4-3, to=5-3]
	\arrow[from=1-4, to=2-4]
	\arrow[from=2-4, to=3-4]
	\arrow[from=3-4, to=4-4]
	\arrow[from=4-4, to=5-4]
\end{tikzcd}\]
and note that the Nine lemma tells us that the top row is exact if and only if the bottom row is exact, and these conditions correspond to flatness of $\widetilde I$ and $A[\varepsilon]/\widetilde I$ respectively.
\begin{example}
	Take $A\coloneqq k[x]$ and $I\coloneqq(f)$. Then we see that a deformation corresponds to an $A$-linear map $(f)\to k[x]/(f)$. For example, one can map $f\mapsto f'$.
\end{example}
Approximately speaking, one may restate this as follows: given closed subscheme $Z_0\subseteq X_0$ (everything is affine here) defined by the ideal sheaf $\mc I\subseteq\OO_{X_0}$, deformations living inside $X_0$ are parameterized by
\[\op{Hom}_{X_0}\left(\mc I/\mc I^2,\OO_{X_0}/\mc I\right),\]
which is just global sections of the normal bundle $\mc N_{Z_0/X_0}$. This is fairly compelling geometrically: an infinitesimal deformation of $Z_0$ is a global vector field ``normal'' to $Z_0$.
\begin{remark}
	Given a smooth separated scheme $X$, one can give $X$ an affine cover and then run the above classification construction. Now, there is some gluing data corresponding to taking local deformations to global ones, and this data satisfies some cocycle condition. As such, it will turn out that our deformations correspond to classes in $H^1(X,\mc T_{X/k})$.
\end{remark}

\section{March 1st: Reed Jacobs}
Today we're talking about Siegel modular varieties and the analytic theory of abelian varieties.

\subsection{Siegel Modular Varieties}
Fix a positive integer $g\ge1$. Our ``upper half-plane'' is given by
\[\HH^g\coloneqq\left\{\tau\in\CC^{g\times g}:\tau=\tau^\intercal,\Im\tau>0\right\},\]
where $\Im\tau>0$ means that the matrix $\Im\tau$ is positive-definite and in particular invertible. Notably, $\dim\HH^g=\frac12g(g+1)$ because we are looking at symmetric matrices.

To build our lattices, fix some positive integers $(e_1,\ldots,e_g)$ such that $e_i\mid e_{i+1}$ for each $i$, and we define
\[\Omega_\tau\coloneqq\begin{bmatrix}
	\tau \\
	E
\end{bmatrix},\]
where $E\coloneqq\op{diag}(e_1,\ldots,e_g)$. Then our lattice $\Lambda_\tau$ is the $\ZZ$-span of $\Omega_\tau$.

Notably, $\CC^g/\Lambda_\tau$ is a complex torus, and it has a Riemann form given by $H\colon(x,y)\mapsto x\Im(\tau)^{-1}\ov y^{-\intercal}$; thus, it is an abelian variety. What is important is that we are essentially getting all complex tori equipped with Riemann forms (i.e., abelian varieties). Approximately speaking, we will use the Riemann form to produce lots of holomorphic functions on our complex torus. To get us set up, we do note that any lattice $\Lambda\subseteq\CC^g$ with Riemann form $H$ can be given a basis such that
\[H_\RR=\begin{bmatrix}
	& E \\ -E
\end{bmatrix}\]
where $E$ is some diagonal matrix in the above form, and $\Lambda=E\ZZ^{2g}$. (This is some version of the Smith normal form.) Now, the point is that being a Riemann form enforces some algebraic conditions on our lattice. Another adjustment of bases actually puts us in the above form, completing our argument.

From here, the next natural question is when two $E$s as above may give the same abelian variety. Set $w\coloneqq\begin{bsmallmatrix}
	& E \\
	-E
\end{bsmallmatrix}$ as before, and one finds that $\op{Sp}(w)$ acts on $\HH^g$ by some analogue of fractional linear transformations. Explicitly,
\[\begin{bmatrix}
	A & B \\
	C & D
\end{bmatrix}\tau\coloneqq(A\tau+BE)(C\tau+DE)^{-1}E.\]
Namely, one can check that $C\tau+DE$ is forced to be invertible. As such, one finds that the moduli space of $g$-dimensional principally polarized abelian varieties is given by $\mc A_g=\op{Sp}_{2g}(\ZZ)\backslash\HH^g$.

In analogy to classical modular forms, which can be viewed as invariant differentials on $\op{SL}_2(\ZZ)\backslash\HH$, one can define Siegel modular forms to be invariant differentials on $\op{Sp}_{2g}(\ZZ)\backslash\HH^g$. There is some problem in making this generalization because passing to an arithmetic subgroup because the notion of weight is a little complicated, and somehow we are really trying to look at a line bundle on a stack, which has its own sort of complications.

\section{March 8th: Feiyang Lin}
Today we are discussing curves of genus $6$. As a warning, we are doing intersection theory with rational coefficients.

\subsection{Tautological Chow Rings}
Our goal is to show that the chow ring of $\mc M_6$ is ``tautological.''
\begin{theorem}
	The Chow ring $A^*(\mc M_6)$ of $\mc M_6$ is tautological.
\end{theorem}
Let's review what being tautological means. Let $\mc C_g$ be the universal curve on $\mc M_g$ so that we have a projection $\pi\colon\mc C_g\to\mc M_g$. By gluing together the canonical bundle, we get a canonical bundle $\omega_\pi\in\op{Pic}\mc C_g$, which produces a Chern class $\kappa\in c_1(\omega_\pi)$. This can then be pushed repeatedly to $\kappa_i\coloneqq\pi_*(\kappa^{i+1})$. Alternatively, we can just take $\mathbb E\coloneqq\pi_*\omega_\pi$ to be a vector bundle of rank $g$ on $\mc M_g$, and its Chern class is similarly $\lambda_i\coloneqq c_i(\mathbb E)$ on $A^i(\mc M_g)$.

With all of this information, we let $R^*(\mc M_g)$ be the $\QQ$-subalgebra of $A^*(\mc M_g)$ generated by the $\kappa_\bullet$ and $\lambda_\bullet$. Work of many people have shown that the following.
\begin{itemize}
	\item High degree: $R^i(\mc M_g)=0$ for $i\ge g-1$.
	\item ``Top'' degree: $R^{g-2}(\mc M_g)$ is generated by a single nonzero class $[\mc H_g]$.
	\item For $g\le5$, one actually has $A^*(\mc M_g)=R^*(\mc M_g)$ is isomorphic to $\QQ[\kappa_1]/\left(\kappa_1^{g-1}\right)$.
	\item Even in degree $6$, one finds that
	\[R^*(\mc M_6)\cong\frac{\QQ[\kappa_1,\kappa_2]}{\left(127\kappa_1^3-2304\kappa_1\kappa_2,113\kappa_1^4-3686\kappa_2^2\right)}.\]
\end{itemize}

\subsection{Building a Filtration}
Our goal is to show that $A^*(\mc M_g)=R^*(\mc M_6)$. The outline will be to build a filtration
\[Z_1\subseteq Z_2\subseteq\cdots\subseteq Z_n=\mc M_6.\]
Then given surjections $R^*(\mc M_6)\onto A^*(Z_{i+1}/Z_i)$ along with knowing that the complement of $Z_{i+1}\setminus Z_i$ lives in $R^*(\mc M_6)$, we will know that$A^*(\mc M_6)$ is contained in $R^*(\mc M_6)$, as needed. Indeed, this is by an inductive argument: one has a right-exact sequence
\[A^*(Z_i)\to A^*(Z_{i+1})\to A^*(Z_i\setminus Z_{i+1})\to0,\]
and we know that $R^*(\mc M_6)$ surjects on to the left and right terms of the exact sequence by construction, so it surjects onto the middle term.

So it remains to construct the magical filtration. We will use the following result.
\begin{theorem}[Vistoli]
	Let $X$ be a smooth DM stack with action by $G\in\{{\op{GL}_n},{\op{SL}_n}\}$. Then the quotient $q\colon X\to X/G$ gives rise to a surjection $A^*(X/G)\to A^*(X)$ (by pullback), and $A^*(X/G)$ is generated by any arbitrary set of lifts from $A^*(X)$.
\end{theorem}
We are not going to prove this. But here is our filtration.
\begin{theorem}[Penev--Vakil]
	The stack $\mc M_6$ is a disjoint union of the following.
	\begin{itemize}
		\item The ``Mukai--general'' curves of genus $6$.
		\item Bi-elliptic curves: double-covers of elliptic curves.
		\item Smooth planar quintic curves.
		\item Trigonal curves: triple-covers of $\PP^1$.
		\item Hyperelliptic curves.
	\end{itemize}
\end{theorem}
Wait, what is a Mukai--general curve? The point is to write down curves which are not among any of the other families. For example, not being trigonal nor hyperelliptic means that it has no $g^1_3$ or $g^2_5$; it turns out that this implies that we have a $g^1_4$. Brill--Noether theory now produces a line bundle $\mc L$ on $\mc C$ such that $h^0(\mc L)=2$; letting $\mc M\coloneqq K_C\otimes\mc L^\lor$ be the Serre dual, we see that $h^0(\mc M)=h^1(\mc L)=3$.

Quickly, we claim that we can find a unique vector bundle $\mc E$ of rank $2$ fitting in a nontrivial extension
\[0\to\mc L\to\mc E\to\mc M\to0\]
such that $h^0(\mc E)=5$ and is globally generated. The point is that one gets a map
\[H^0(\mc M^{\otimes2})=\op{Ext}^1(\mc M,\mc L)\to H^0(\mc M)^\lor\otimes H^1(\mc M)\]
which is dual to the multiplication map $H^0(\mc M)\otimes H^0(\mc M)\to H^0(\mc M^{\otimes2})$. Now, $\mc M$ is a $g^2_6$ and hence produces a map $C\to\PP^2$, allowing us to move everything over to $\PP^2$. Then one can go through the possibilities of what this map looks like. Note that $\mc E$ will end up being globally generated because both $\mc L$ and $\mc M$ are.

At this point in the talk, I lost the ability to take notes and had to pay attention.

\section{March 15th: Ruoxi Li}
I was not tall enough for this talk.

\section{March 22nd: Vivasvat Vatatmaja}
Today, we work over a base field $k$ everywhere, which we will not comment again on. We begin with a definition.
\begin{definition}
	Let $G$ will be an algebraic group, and let $X$ is a scheme (or possibly an algebraic space). Suppose $X$ has a $G$-action given by $j\colon G\times X\to X\times X$ via $(g,x)\mapsto(gx,x)$.
	\begin{itemize}
		\item The action is \textit{proper} if $j$ is proper.
		\item The action is \textit{free} if $j$ is a closed embedding.
		\item Considering the map $j^{-1}\Delta_X\to\Delta_X$, we see that the action has trivial stabilizer if and only if this map has trivial fibers.
	\end{itemize}
\end{definition}
Here is our main theorem for today.
\begin{theorem}
	The integral chow ring $A^*(\mc M_{1,1})$ of $\mc M_{1,1}$ is $\ZZ[t]/(12t)$.
\end{theorem}
In particular, we will need to discuss how to even make sense of the integral Chow ring. Here, $\mc M_{1,1}$ is the moduli space of curves of genus $1$ with a marked point, and we will show that it can be realized as a quotient stack $X/G$.

So we want to understand the Chow ring of a quotient stack $X/G$.
\begin{definition}
	Fix a group $G$ acting on an algebraic space $X$ and index $i$. Then choose a representation $V$ of $G$ such that $G$ acts freely on an open subset $U\subseteq V$ (and we take $U$ to be the largest such open subset), where the codimension of $V\setminus U$ is $\dim X-i$. Then we define $X_G\coloneqq(X\times U)/G$ and set $A_i^G(X)\coloneqq A_{i+\dim V-\dim G}(X_G)$.
\end{definition}
\begin{remark}
	One can show that $A_i^G(X)$ does not depend on the choices. This is the ``double fibration'' argument. Indeed, choose another representation $V'$ of $G$, and we let $U'$ again be the locus where $G$ acts freely. To compare the two, we note that $G$ acts diagonally on $V\oplus V'$, and its locus $W$ of free action contains $(U'\oplus V)\cup(V'\oplus U)$. Now, a codimension argument explains that
	\[A_{i+\dim V+\dim V'-\dim G}((X\times W)/G)=A_{i+\dim V+\dim V'-\dim G}((X\times(U\oplus V'))/G).\]
	Now, $(X\times(U\oplus V'))/G$ is a vector bundle on $(X\times U)/G$, so one finds that
	\[A_{i+\dim V+\dim V'-\dim G}((X\times(U\oplus V'))/G)\cong A_{i+\dim V-\dim G}((X\times U)/G),\]
	and a symmetric argument compares this to $V'$.
\end{remark}
\begin{remark}
	Na\"ively, one might want to take $G$-invariant subvarieties to define our Chow ring. One does see that $[Y]_G\in A^G_\bullet(X)$ by an explicit construction, and in fact, one can show that any $\alpha\in A^G_i(X)$ can be realized in some representation via $G$-invariant subvarieties.
\end{remark}
\begin{remark}
	One can show that we have flat pullback and proper pushforward and so on.
\end{remark}
\begin{example}
	Let $G\coloneqq\mathbb G_m$ act on a point $X\coloneqq\{*\}$. Here, $\mathbb G_m$ acts on any $d$-dimensional vector space $V$, and it acts freely on the set of nonzero values $U\subseteq V$. So $U/\mathbb G_m$ is $\PP^{d-1}$, so $A^G_i(X)=A_i\left(\PP^{d-1}\right)$ for $d$ large enough, for which classical computation provides $\ZZ t^i$, so $A^G(X)=\ZZ[t]$.
\end{example}
\begin{example}
	Let $G\coloneqq\op{GL}_n$ act on a point $X\coloneqq\{*\}$. Then we let $V$ be the vector space of $(n\times p)$ matrices where $p$ is very large, and we see that $G$ acts by matrix multiplication on $V$, and it acts freely on the open subset $U$ of matrices of maximal rank. Thus, $U/G$ is $\op{Gr}(n,p)$, and one finds that
	\[A^G_i(X)=\ZZ[c_1,\ldots,c_n]\]
	again by some classical computations.
\end{example}
We now take as our definition
\[\mc A^G_i(X/G)\coloneqq A^G_{i-g}(X).\]
At this point, we return to $\mc M_{1,1}$. The point is to use Weierstrass equations. Consider $\PP^9$ as the space of cubic forms in the variables $(x,y,z)$, and we let $X$ be the subspace of forms which are proportional to $y^2z-\left(x^3+e_1x^2z+e_2xz^2+e_3z^3\right)$ so that
\[G\coloneqq\left\{\begin{bmatrix}
	1 & & B \\
	& A \\
	& & A^{-2}
\end{bmatrix}:A\ne0\right\}\]
acts on $X$ via some change of coordinates representation hitting all Weierstrass equations. By restricting $X$ further to the open subset $U$ where $x^3+e_1x^2+e_2x+e_3$ has three distinct roots, we find that our curves are forced to be smooth. In particular, we have $\mc M_{1,1}\cong U/G$. We are now able to do a computation to prove our result.

\section{April 12th: Rose Lopez}
Today we will start discussing the Stable reduction theorem (which we will finish next week).

\subsection{The Statement}
Here is our goal.
\begin{theorem}
	The moduli stack $\ov{\mc M_{g,n}}$ of $n$-pointed stable curves of genus $g$ is proper.
\end{theorem}
The only way to access properness of a moduli problem is to use a valuative criterion, for which we have the following.
\begin{proposition} \label{prop:val-proper-stacks}
	Fix a reasonably nice morphism $f\colon\mf X\to\mf Y$. Then $f$ is proper if and only if it has finite type and any $2$-commutative diagram
	% https://q.uiver.app/#q=WzAsNCxbMCwwLCJcXFNwZWMgSyJdLFswLDEsIlxcU3BlYyBSIl0sWzEsMCwiXFxtZiBYIl0sWzEsMSwiXFxtZiBZIl0sWzIsMywiZiJdLFswLDEsIiIsMCx7InN0eWxlIjp7InRhaWwiOnsibmFtZSI6Imhvb2siLCJzaWRlIjoidG9wIn19fV0sWzAsMl0sWzEsM11d&macro_url=https%3A%2F%2Fraw.githubusercontent.com%2FdFoiler%2Fnotes%2Fmaster%2Fnir.tex
	\[\begin{tikzcd}
		{\Spec K} & {\mf X} \\
		{\Spec R} & {\mf Y}
		\arrow["f", from=1-2, to=2-2]
		\arrow[hook, from=1-1, to=2-1]
		\arrow[from=1-1, to=1-2]
		\arrow[from=2-1, to=2-2]
	\end{tikzcd}\]
	where $R$ is a discrete valuation ring and $K\coloneqq\op{Frac}R$ has an extension $R\to R'$ of discrete valuation rings so that $\op{Frac}R\to\op{Frac}R'$ has finite transcendence degree and with  a unique lift $\Spec R'\to\mf X$.
\end{proposition}
This is different from schemes because of the needed extension of discrete valuation rings. Anyway, this will basically serve as our definition of properness. As such, here is our theorem.
\begin{theorem}[Stable reduction] \label{thm:stable-reduction}
	Fix a discrete valuation ring $R$ with fraction field $K$, and set $\Delta\coloneqq\Spec R$ and $\Delta^*\coloneqq\Spec K$. Then suppose that $(\mc C^*,\sigma_1^*,\ldots,\sigma_n^*)$ is an $n$-pointed stable curve. Then there is an $n$-ponter family of stable curves $(\mc C',\sigma_1',\ldots,\sigma_n')$ which extends to $\mc C^*$ after extension by $K'$ (where $K'$ is the fraction field of some extension $R'$ of $R$).
\end{theorem}
After undoing some Yoneda business, we see that the above theorem plugs directly into the existence portion of \Cref{prop:val-proper-stacks}.

\subsection{Stable Curves}
Wait, what is a stable curve? Let's begin with a definition of curve.
\begin{definition}
	A \textit{curve} $C$ over a field $K$ is a $1$-dimensional scheme of finite type over $K$.
\end{definition}
If $C$ is proper, then we can define our genus as $g\coloneqq h^1(C,\OO_C)$. If $C$ is smooth, then we also get a dualizing sheaf $\omega_C$.

Now, to define stable, we recognize that we are in a moduli problem. For our (completed) moduli problem, we must allow for some controlled singularities.
\begin{definition}[node]
	Fix a curve $C$ over an algebraically closed field $K$. Then $p\in C$ is a \textit{node} if and only if
	\[\widehat{\OO_{C,p}}\cong k\bb{x,y}/(xy).\]
	We say that $C$ is \textit{nodal} if and only if $C$ is smooth except for some set of nodes.
\end{definition}
Approximately speaking, we are saying that the curve $C$ locally looks like a crossing. The normalization will be able to separate out the crossing into two pieces.
\begin{definition}[stable]
	An $n$-pointed curve $(C,p_1,\ldots,p_n)$ is \textit{prestable} if and only if $C$ is geometrically connected, nodal, and projective, where a genus-$1$ component has at least $1$ marked point, and any genus-$0$ component must have at least $3$ marked points. We say that it is \textit{stable} if genus-$0$ components have precisely $3$ marked points. We also require that the marked points $p_\bullet$ are distinct and smooth everywhere.
\end{definition}
There is also a weaker notion of semistable by weakening some inequalities appropriately.
\begin{remark}
	One can check by hand that stable implies that $2g-2+n>0$.
\end{remark}
\begin{remark}
	By working out with components, we see that a prestable curve is stable if and only if it has finite automorphism group, which is what is able to give us a DM stack. It turns out that this is equivalent to having $\omega_C(p_1+\cdots+p_n)$ be ample, which notably witnesses that $C$ is projective.
\end{remark}
Of course, one can now make these definitions work in families: a family of smooth proper curves of genus $g$ over $S$ is a morphism $\mc C\to S$ which is smooth and proper such that each geometric fiber is a connected curve of genus $g$. Then we add in $n$ marked points to each fiber (which corresponds to asking  for $n$ sections of the structure map $\mc C\to S$), and asking for stable, nodal, etc. can all be checked on the fibers.

\subsection{The Proof}
The point of \Cref{thm:stable-reduction} is to be able to fill in a gap in a family of stable curves over the closed point of a discrete valuation ring. Let's give a sketch of what's going on.
\begin{enumerate}
	\item We reduce to the smooth case.
	\item We find some flat extension to $\Spec R$. However, the produced extension maybe non-reduced or badly singular or similar.
	\item Lastly, we use birational geometry on surfaces to fix the constructed extension.
\end{enumerate}

\section{April 19th: Robin Huang}
I'll be honest; I'm too tired to take notes today.

\end{document}
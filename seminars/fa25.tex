\documentclass{article}
\usepackage[utf8]{inputenc}

\newcommand{\nirpdftitle}{Seminars: Fall 2025}
\usepackage{import}
\inputfrom{../../notes}{nir}
\usepackage[backend=biber,
    style=alphabetic,
    sorting=ynt
]{biblatex}
\setcounter{tocdepth}{2}

\pagestyle{contentpage}

\setlength{\headheight}{13.19003pt}
% (fancyhdr)	You might also make \topmargin smaller to compensate:
\addtolength{\topmargin}{-1.19003pt}

\title{Seminars}
\author{Nir Elber}
\date{Fall 2025}
\usepackage{graphicx}
\lhead{}

\begin{document}

\maketitle

\begin{abstract}
	This semester, I will just record all seminars I go to in an uncategorized manner. I will try to record the date, the speaker, and which seminar it was to maintain some semblance of organization.
\end{abstract}

\tableofcontents

\section{September 3: Arthur Representations and the Unitary Dual}
This talk was given by David Vogan at MIT for the Lie groups seminar.

\subsection{The Orbit Method}
Given a group $G$, we would like to understand its unitary representations, which more or less amounts to understanding the unitary dual of $G$. There is a long history of trying to solve this problem for various classes of groups $G$, and we will focus on reductive groups, roughly speaking because it is suitable for inductive arguments. In particular, we will focus on real reductive groups.

Our story begins with the ideas of Bertram Kostant. This is the method of coadjoint orbits, which explains where one should look for representations of Lie groups. The point is that unitary representations provide operators on a Hilbert space, as do quantum mechanical systems, so one might want to undo the quantization. It turns out then that a unitary representation un-quantizes to a Hamiltonian $G$-space, which is a symplectic manifold with $G$-action (and a moment map).
\begin{idea}
	Irreducible representations of $G$ correspond to a quantization of covers of coadjoint orbits.
\end{idea}
One must make precise what a quantization is, which is not known in general.

Anyway, the point is that want to construct our unitary representations geometrically (which amounts to quantizing coadjoint operators), and then we want to show that we have all them. For real reductive groups, there are not so many kinds of coadjoint orbits.
\begin{itemize}
	\item Hyperbolic (diagonalizable with real eigenvalues): quantization is real parabolic induction. Here, para\-bolic induction takes unitary representations to unitary representations.
	\item Elliptic (diagonalizable with non-real eigenvalues): quantization is cohomological parabolic induction. Here, we take unitary representations (of some Levi subgroups) to unitary representations under some positivity requirement on the codomain.
	\item Nilpotent: quantization is not totally understood, though something partial was suggested by Arthur and worked out completely by other authors.
\end{itemize}
In total, here is how one can construct unitary representations using the orbit method.
\begin{enumerate}
	\item Choose a ``$\theta$-stable'' Levi subgroup $L_\theta\subseteq L_\RR$, meaning that it is the centralizer of a compact torus.
	\item Fix some unipotent representation $\pi_\theta$ of $L_\theta$.
	\item Twist $\pi_\theta$ by a positive unitary character $\lambda$ of $L_\theta$.
	\item Take cohomological induction of $\pi_\theta(\lambda)$ to $\pi_\RR(\lambda)$ on $L_\RR$.
	\item Twist further by some unitary character $\nu$ of $L_\RR$.
	\item Take real induction up to $G$.
\end{enumerate}
There is even an explicit way to compute the infinitsemial character at the end, and it is explicit when $\pi_\theta$ is a ``special'' unipotent representation.
\begin{remark}
	One may want the infinitesimal character to be integral, but it is rarely so.
\end{remark}
\begin{example}
	We work some of this out for $\op{Sp}_{2n}(\RR)$.
\end{example}
\begin{proof}
	Decompose $n=n_r+\cdots+n_1+n_0$ into nonnegative integers, and we get a Levi subgroup which looks like
	\[L_\RR=\op{GL}(n_m,\RR)\times\cdots\times\op{GL}(n_1,\RR)\times\op{Sp}(2n_0,\RR).\]
	Having lots of $\op{GL}$s is fairly typical for Levi subgroups, which one can see by suitably taking subgraphs of the Dynkin diagram. One then choose a compact torus suitably and writes down some $L_\theta\subseteq L_\RR$. We then see that our orbit method asks for many unipotent representations of groups of smaller rank (mostly $\op{GL}$s or $\op U$s) and do some inductions and twisting by controlled characters.
\end{proof}
\begin{remark}
	The orbit method cannot provide a complete list of our representations, roughly speaking because of the hyperbolic step. The point is that we only allowed twisting by unitary characters, but sometimes you can twist by a non-unitary character and still end up with a unitary representation at the end due to frequent coincidences.
\end{remark}

\subsection{Arthur's Conjecture}
For the rest of the talk, we will be interested in the following conjecture.
\begin{conj}
	Suppose $G$ is a real reductive algebraic group and $\pi$ is a unitary representation of $G$ having integral infinitesimal character. Then $\pi$ is an Arthur representation, meaning that we get it from the orbit method.
\end{conj}
Much is known: there are no known counterexamples for classical $G$, nor are there any for $G_2$ or $E_6$. However, there are a few counterexamples for some exceptional groups: it fails for two representations of split $F_4$, at most six representations for split $E_7$, and at most twenty-seven representations for split $E_8$. (The ``at most'' phrase is present here because the definition of Arthur representation is difficult to calculate.)

\section{Sepember 8: The Affine Chabauty Method}
This talk was given by Marius Leonhardt at Boston University for the number theory seminar.

\subsection{The Statement}
We are interested in finding the integral solutions to curves, such as hyperelliptic curves. Fix a smooth projective curve $X$ over $\QQ$; after removing some finite number of ``cusps'' $D\subseteq X$ (which are just some closed points), we can form the affine curve $Y$; today, we will study the $S$-integral points on $Y$ for some finite set $S$ of primes.
\begin{remark}
	Technically, in order to make sense $S$-integral points, one should choose a regular model $\mc X$ of $X$ over $\ZZ_S$, choose a closure $\mc D$ of $D\subseteq X$ in $\mc X$, and set $\mc Y\coloneqq\mc X\setminus\mc D$. Then, we can speak coherently about $S$-integral points on $\mc Y$.
\end{remark}
We will want a few more invariants. Choose a basepoint $P_0\in Y(\QQ)$, and set $g$ to be the genus of $X$, $r$ to be the rank of the Jacobian, and $n$ to be the number of geometric points in $D$. Note we can write $n=n_1+2n_2$ for the totally real and (classes) of complex points.

We are interested in making the following classical result effective.
\begin{theorem}[Siegel]
	If $Y$ is hyperbolic, then $\mc Y(\ZZ_S)$ is finite.
\end{theorem}
Here is our main result.
\begin{theorem} \label{thm:affine-chab}
	Fix everything as above.
	\begin{listalph}
		\item If
		\[r+\#S+n_1+n_2-\#D<g+n-1,\]
		then $\mc Y(\ZZ_S)$ is contained in a finite computable subset of $\mc Y(\ZZ_p)$.
		\item If we also have $p>2g+n$, then $\#\mc Y(\ZZ_S)$ is bounded above by $\#\mc Y(\FF_p)+2g+n-2$ multiplied by the number of reduction types.
	\end{listalph}
\end{theorem}
\begin{remark}
	This result even works if the genus is $0$ or $1$!
\end{remark}
\begin{remark}
	The reduction type is a piece of combinatorial data $\Sigma=(\Sigma_\ell)_\ell$ partitioning $\mc Y(\ZZ_S)$; namely, we choose the component or cusp of $\mc X_{\FF_\ell}$ where the point reduces.
\end{remark}
Before stating our modifications, let's recall the usual Chabauty method. Consider the following diagram.
% https://q.uiver.app/#q=WzAsNSxbMCwwLCJYKFxcUVEpIl0sWzAsMSwiXFxvcHtKYWN9WChcXFFRKSJdLFsxLDEsIlxcb3B7SmFjfVgoXFxRUV9wKSJdLFsxLDAsIlgoXFxRUV9wKSJdLFsyLDEsIlxcbWF0aHJtIEheMChYX3tcXFFRX3B9LFxcT21lZ2FeMSleXFxsb3IiXSxbMCwzLCIiLDAseyJzdHlsZSI6eyJ0YWlsIjp7Im5hbWUiOiJob29rIiwic2lkZSI6InRvcCJ9fX1dLFsxLDIsIiIsMCx7InN0eWxlIjp7InRhaWwiOnsibmFtZSI6Imhvb2siLCJzaWRlIjoidG9wIn19fV0sWzMsNCwiXFxpbnQiXSxbMiw0LCJcXGxvZyIsMl0sWzMsMiwiXFxvcHtBSn0iLDJdLFswLDEsIlxcb3B7QUp9IiwyXV0=&macro_url=https%3A%2F%2Fraw.githubusercontent.com%2FdFoiler%2Fnotes%2Fmaster%2Fnir.tex
\[\begin{tikzcd}[cramped]
	{X(\QQ)} & {X(\QQ_p)} \\
	{\op{Jac}X(\QQ)} & {\op{Jac}X(\QQ_p)} & {\mathrm H^0(X_{\QQ_p},\Omega^1)^\lor}
	\arrow[hook, from=1-1, to=1-2]
	\arrow["{\op{AJ}}"', from=1-1, to=2-1]
	\arrow["{\op{AJ}}"', from=1-2, to=2-2]
	\arrow["\int", from=1-2, to=2-3]
	\arrow[hook, from=2-1, to=2-2]
	\arrow["\log"', from=2-2, to=2-3]
\end{tikzcd}\]
The Jacobian has rank $r$, and the last cohomology group has dimension $g$, so $r<g$ provides a nontrivial differential $\omega$ for which $X(\QQ)$ is contained in the set of $p$-adic points $A$ with $\int_{P_0}^A\omega=0$. This last condition is computable, which is the point of the method.

\subsection{How to Fix Chabauty}
Let's say something about the proof of \Cref{thm:affine-chab}.
\begin{itemize}
	\item Using logarithmic (instead of holomorphic) differentials, meaning that we allow some simple poles at $D$, then
	\[\dim\mathrm H^0(X_{\QQ_p},\Omega^1(D))=g+n-1,\]
	which exhibits the right-hand side of the inequality in (a).
	\item One should use generalized Jacobians to build a Jacobian $J_Y$ of $Y$. For example, one definition is given by $J_Y(K)$ to be the divisors of $Y_K$ of degree zero modulo principal divisors $\op{div}f$ where $f\in K(X)^\times$ has value $1$ on $D$. Notably, $J_Y$ is a semiabelian variety, so it fits into a diagram
	\[0\to T_D\to J_Y\to J\to0,\]
	where $T_D$ is some torus. This means that $J_Y(\QQ)$ has little chance of being finitely generated, so it is not enough to just write out the same Chabauty diagram as before.
	\item Instead, we look at $\ZZ_S$-points. Build the Chabauty diagram as follows.
	% https://q.uiver.app/#q=WzAsNyxbMCwwLCJcXG1jIFkoXFxaWl9TKSJdLFsxLDAsIlxcbWMgWShcXFpaX3ApIl0sWzAsMSwiWShcXFFRKSJdLFsxLDEsIlkoXFxRUV9wKSJdLFswLDIsIkpfWShcXFFRKSJdLFsxLDIsIkpfWShcXFFRX3ApIl0sWzIsMiwiXFxtYXRocm0gSF4wKFhfe1xcUVFfcH0sXFxPbWVnYV4xKEQpKV5cXGxvciJdLFszLDYsIlxcaW50Il0sWzAsMl0sWzAsMV0sWzEsM10sWzIsM10sWzIsNF0sWzMsNV0sWzUsNl0sWzQsNV1d&macro_url=https%3A%2F%2Fraw.githubusercontent.com%2FdFoiler%2Fnotes%2Fmaster%2Fnir.tex
	\[\begin{tikzcd}[cramped]
		{\mc Y(\ZZ_S)} & {\mc Y(\ZZ_p)} \\
		{Y(\QQ)} & {Y(\QQ_p)} \\
		{J_Y(\QQ)} & {J_Y(\QQ_p)} & {\mathrm H^0(X_{\QQ_p},\Omega^1(D))^\lor}
		\arrow[from=1-1, to=1-2]
		\arrow[from=1-1, to=2-1]
		\arrow[from=1-2, to=2-2]
		\arrow[from=2-1, to=2-2]
		\arrow[from=2-1, to=3-1]
		\arrow[from=2-2, to=3-2]
		\arrow["\int", from=2-2, to=3-3]
		\arrow[from=3-1, to=3-2]
		\arrow[from=3-2, to=3-3]
	\end{tikzcd}\]
	We need to show that the image of $\mc Y(\ZZ_S)$ in $J_Y(\QQ)$ is contained in a (small) finitely generated subgroup; in fact, we can get this rank down to
	\[r+\#S+n_1+n_2-\#D,\]
	from which (a) follows. From here, (b) follows by arguing as in the Chabauty--Coleman method.
	\item It remains to prove the bound on the rank, which is done using arithmetic intersection theory. Inspired by something with $p$-adic heights, we may take mod-$\ell$ intersections of points in $J_Y(\QQ)$ with reductions of the cusps, and one can compute that $\mc Y(\ZZ_S)$ has controlled image.
\end{itemize}

\section{September 9th: Singularities of Secant Varieties}
The (pre-)talk was given by Debaditya Raychaudhury at the Harvard--MIT algebraic geometry seminar, and it takes up the first two subsections; it was titled ``The Hodge Filtration on Local Cohomology.'' The rest is from the main talk.

\subsection{Properties of Local Cohomology}
Fix a subvariety $Z$ of a smooth variety $X$ over $\CC$. Then one has local cohomology sheaves $\mc H^\bullet_Z(-)$, which are understod as the derived functors of $\mathrm H^0_Z(-)$, which is the sheaf of sections supported on $Z$.
\begin{example}
	Locally, we can make everything affine, so say $X=\Spec R$ and $Z=\Spec R/I$. Then $H^0_I(R)$ consists of the $r\in R$ for which $I^kr=0$ for some $k$ large enough.
\end{example}
We would like to compute these sheaves.

One way is geometric. With $U\coloneqq X\setminus Z$, we let $j\colon U\into X$ be the inclusion. Then there is an exact sequence
\[0\to\OO_X\to j_*\OO_U\to\mathcal H^1(\OO_X)\to0\]
and $\mathrm R^{q-1}j_*\OO_U\cong\mathcal H^q_Z(\OO_X)$ for $q\ge2$.
\begin{example}
	Continuing with the affine situation, suppose further that $I=(f)$. Then we get an exact sequence
	\[0\to R\to R_f\to\mathrm H^1_{(f)}(R)\to0,\]
	so we can compute $\mathrm H^1_{(f)}(R)$ via some \v Cech complex. In general, with $I=(f_1,\ldots,f_s)$, one builds the \v Cech complex
	\[0\to R\to\bigoplus_{i_1}R_{f_{i_1}}\to\bigoplus_{i_1\le i_2}R_{f_{i_1}f_{i_2}}\to\cdots\]
	to compute the cohomology $\mathrm H^q_I(R)$; here, the differential maps are the usual ones for the \v Cech complex. For example, if $q>s$, then $\mathrm H^q_I(R)=0$ automatically!
\end{example}
Thus, we see that $\mathcal H^q_Z(\OO_X)$ vanishes for $q$ not so large: it vanishes as soon as $q$ is larger than the number of local defining equations. We are now allowed to make the following definition.
\begin{definition}
	We define the \textit{local cohomological dimension} $\op{lcd}(X,Z)$ to be the maximum $q$ such that $\mathcal H^q_Z(\OO_X)$ is nonzero.
\end{definition}
\begin{remark}
	It turns out that the minimal $q$ such that $\mathcal H^q_Z(\OO_X)$ is equal to the codimension of $Z\subseteq X$, which can be seen by a computation on affines.
\end{remark}
\begin{remark}
	As discussed above, if $Z\subseteq X$ is a local complete intersection, then $\op{lcd}(Z,X)$ will equal the codimension.
\end{remark}
Thus, we see that the local cohomological dimension does a reasonable job keeping track of singularities.
\begin{notation}
	We define $\op{lcdef}(Z)\coloneqq\op{lcd}(X,Z)-\op{codim}_X(Z)$
\end{notation}
\begin{remark}
	This quantity does not depend on $Z$! One way to see this is to show that $\op{lcdef}(Z)$ is the maximum $j$ for which
	\[\mathcal H^j(\QQ_Z^H[\dim Z])\ne0,\]
	which is more manifestly independent of $X$. Here, $\QQ_X^H$ is the Hodge module.
\end{remark}

\subsection{The Hodge Filtration}
It turns out that $\mathcal H^q_Z(\OO_X)$ has the structure of aa filtered $\mc D_X$-module, where $\mc D_X$ is defined as the (dual) differential ring, given on affines $U$ by
\[\mc D_X(U)\coloneqq\bigoplus_{\alpha\in\NN^n}\OO_X(U)\del^\alpha,\]
where $\del^\alpha=\del_1^{\alpha_1}\cdots\del_n^{\alpha_n}$; here $\del_i$ is dual to $dx_i$. Approximately speaking, this arises because $\mathcal H^q_Z(\OO_X)$ is the underling $\mc D$-module for the filtered module $\mathcal H^q(i_*i^!\QQ_X^H[z])$, where $i\colon Z\into X$ is the inclusion.

We would like to compute the Hodge filtration $F_\bullet\mathcal H^q_Z(\OO_X)$. Suppose we have a ``resolution of singularities'' $f\colon Y\to X$ fitting into a pullback square
% https://q.uiver.app/#q=WzAsNCxbMCwwLCJZXFxzZXRtaW51cyBFIl0sWzAsMSwiVSJdLFsxLDEsIlgiXSxbMSwwLCJZIl0sWzMsMiwiZiJdLFswLDNdLFsxLDJdLFswLDEsIiIsMix7ImxldmVsIjoyLCJzdHlsZSI6eyJoZWFkIjp7Im5hbWUiOiJub25lIn19fV1d&macro_url=https%3A%2F%2Fraw.githubusercontent.com%2FdFoiler%2Fnotes%2Fmaster%2Fnir.tex
\[\begin{tikzcd}[cramped]
	{Y\setminus E} & Y \\
	U & X
	\arrow[from=1-1, to=1-2]
	\arrow[equals, from=1-1, to=2-1]
	\arrow["f", from=1-2, to=2-2]
	\arrow[from=2-1, to=2-2]
\end{tikzcd}\]
where $Z\coloneqq f^{-1}(Z)$ is some simple normal crossings divisor. Namely, one uses the complex
\[0\to f^*\mathcal D_X\to\Omega^1_Y(\log E)\otimes f^*\mc D_X\to\cdots\to\omega_Y(E)\otimes f^*\mc D_X\to0.\]
Roughly speaking, one hits this with $F_{k-n}$ and computes the image of some cohomology of this complex.

Here is the sort of thing that one can prove.
\begin{theorem}
	Suppose $q\ge1$ and $\mathcal H^k_Z(\OO_X)=0$ for $k>q$. Then $F_\bullet\mc H^q_Z(\OO_X)$ is generated at level $\ell$ if and only if $\mathrm R^{q-1+i}f_*\Omega_Y^{n-i}(\log E)=0$.
\end{theorem}
We end this talk by defining one more invariant.
\begin{definition}
	We say that $c(Z)\ge k$ if and only if $F_p\mc H^q_Z(\OO_X)=0$ for all $p\le k$ and $q>\op{codim}_X(Z)$.
\end{definition}
Roughly speaking, this is expected to ``cohomologically'' generalize local complete intersections.
\begin{theorem}
	One has $c(Z)\ge k$ if and only if $\op{depth}\Omega_Z^p\ge\dim Z-p$ for all $p\le k$.
\end{theorem}
\begin{theorem}
	One has that $\dim Z-\op{lcdef}(Z)$ is the minimal value of $\op{depth}\Omega_Z^k+k$.
\end{theorem}

\subsection{The Secant Variety}
Given an ample line bundle $\mc L$ on a smooth variety $X$ of dimension $n$, we get an embedding $X\into\mathrm H^0(\mc L)$. Then we define the secant variety $\Sigma$ as the closure of the line spanned by any pair $x_1,x_2\in X$.
\begin{remark}
	With $\dim X=n$, we expect $\dim\Sigma=2n+1$: each point has $n$ degrees of freedom, and then the line adds another degree of freedom.
\end{remark}
\begin{example}
	If $X$ is $\PP^1$ embedded in $\PP^2$ (via $\OO(2)$), then $\Sigma=\PP^2$, which is smaller than expected.
\end{example}
\begin{example}
	If $X$ is $\PP^2$ embedded in $\PP^3$ (via $\OO(3)$), then $\dim\Sigma=3$.
\end{example}
The moral is that $\Sigma$ only achieves the expected dimension for sufficiently positive line bundles.
\begin{proposition}
	Suppose that $\mc L$ is $3$-very ample (which we won't define), then $\dim\Sigma=2n+1$, and the singular locus of $\Sigma$ is precisely $X$ unless $\Sigma$ is the whole space.
\end{proposition}
Thus, we will focus on our secant varieties arising from $3$-very ample line bundles.

With more positivity, we get more.
\begin{theorem}[Ullery--Chou--Song]
	If $\mc L=\mc K_X+(2n+2)\mc A+\mc B$ where $\mc A$ is very ample, and $\mc B$ is nef, then $\Sigma$ is normal, has DB singularities, is Cohen--Macaulay (CM), weakly rational, and thus has rational singularities if and only if $\mathrm H^i(\OO_X)=0$ for all $i>0$. 
\end{theorem}
To improve this result, one needs to know something about $\QQ_\Sigma^H[2n+1]$, where the $H$ adds emphasizes that it is a Hodge module.

\subsection{Mixed Hodge Modules}
Let's quickly say something about mixed Hodge modules. Suppose $Z$ is a smooth variety.
\begin{definition}
	A \textit{Hodge module $M$} is the data of a left $\mc D_Z$-module $M$ along with a Hodge filtration $F_\bullet M$ on $M$, satisfying some conditions, together with a $\QQ$-perverse sheaf $K$ and isomorphism $\alpha\colon K\otimes_\QQ\CC\to\mathrm{DR}_Z(M)$, where $\mathrm{DR}_Z(M)$ is the de Rham complex of $M$.
\end{definition}
Here, the de Rham complex is $M$ tensored with
\[\OO_Z\to\Omega^1_Z\to\cdots\to\omega_Z.\]
We remark that once we take graded pieces with respect to $F$, we get an object in $D^b(\mathrm{Coh}(Z))$.
\begin{remark}
	There is a derived category $D^b(\mathrm{MHM}(Z))$, even when $Z$ fails to be singular, and it admits the six functors. Thus, we may define
	\[\QQ_Z^H\coloneqq p^*\QQ_{\mathrm{pt}}^H,\]
	where $p\colon Z\to\mathrm{pt}$ is the constant map.
\end{remark}

\subsection{Main Result}
For our application, we need a log resolution of $X$. Let $X^{[2]}$ be the Hilbert scheme of lines on $X$. Then there is a tautological $\Phi\subseteq X^{[2]}\times X$ of pairs $(\xi,x)$ where $x\in\xi$. (Equivalently, $\Phi$ is the blow up of $X\times X$ along the diagonal.) Let $\theta\colon\Phi\to X^{[2]}$ and $q\colon\Phi\to X$ be the projections, so we may define $\mc E\coloneqq\theta_*q^*\mc L$. It turns out that $\mathrm H^0(\mc E)=\mathrm H^0(\mc L)$, and it turns out that $\PP\mc E$ projects onto $\Sigma$. We are now able to draw the following diagram.
% https://q.uiver.app/#q=WzAsNyxbMywxLCJcXFBQXFxtYXRocm0gSF4wKFxcbWMgTCkiXSxbMiwwLCJcXFBQXFxtYXRoY2FsIEUiXSxbMiwxLCJcXFNpZ21hIl0sWzEsMSwiWCJdLFsxLDAsIlxcUGhpIl0sWzAsMSwiXFx7eFxcfSJdLFswLDAsIlxcb3B7Qmx9X3tcXHt4XFx9fVgiXSxbMSwyLCJ0IiwwLHsic3R5bGUiOnsiaGVhZCI6eyJuYW1lIjoiZXBpIn19fV0sWzEsMF0sWzIsMCwiIiwwLHsic3R5bGUiOnsidGFpbCI6eyJuYW1lIjoiaG9vayIsInNpZGUiOiJ0b3AifX19XSxbMywyLCIiLDAseyJzdHlsZSI6eyJ0YWlsIjp7Im5hbWUiOiJob29rIiwic2lkZSI6InRvcCJ9fX1dLFs0LDEsIiIsMCx7InN0eWxlIjp7InRhaWwiOnsibmFtZSI6Imhvb2siLCJzaWRlIjoidG9wIn19fV0sWzQsMywicSJdLFs2LDQsIiIsMCx7InN0eWxlIjp7InRhaWwiOnsibmFtZSI6Imhvb2siLCJzaWRlIjoidG9wIn19fV0sWzUsMywiIiwwLHsic3R5bGUiOnsidGFpbCI6eyJuYW1lIjoiaG9vayIsInNpZGUiOiJ0b3AifX19XSxbNiw1XV0=&macro_url=https%3A%2F%2Fraw.githubusercontent.com%2FdFoiler%2Fnotes%2Fmaster%2Fnir.tex
\[\begin{tikzcd}[cramped]
	{\op{Bl}_{\{x\}}X} & \Phi & {\PP\mathcal E} \\
	{\{x\}} & X & \Sigma & {\PP\mathrm H^0(\mc L)}
	\arrow[hook, from=1-1, to=1-2]
	\arrow[from=1-1, to=2-1]
	\arrow[hook, from=1-2, to=1-3]
	\arrow["q", from=1-2, to=2-2]
	\arrow["t", two heads, from=1-3, to=2-3]
	\arrow[from=1-3, to=2-4]
	\arrow[hook, from=2-1, to=2-2]
	\arrow[hook, from=2-2, to=2-3]
	\arrow[hook, from=2-3, to=2-4]
\end{tikzcd}\]
The squares are all pullback squares.

Here is our main theorem. Recall that there is a complex $\underline\Omega_Z^p$ given by
\[\op{gr}_{-p}\mathrm{DR}(\QQ_Z^H[\dim Z])[p-\dim Z].\]
\begin{theorem}
	Suppose $\mc L$ is sufficiently positive.
	\begin{listalph}
		\item $\underline\Omega^p_\Sigma\cong\Omega_Z^{[p]}$ for all $0\le p\le k$ is equivalent to $\mathrm H^i(\OO_X)$ for all $1\le p\le k$. In particular, this implies that $\underline\Omega^p_\Sigma$ is a sheaf.
		\item $\mathcal Ext^i(\underline\Omega_\Sigma^{2n},\omega_\Sigma^\bullet[-2n-2])=0$ for all $i\ge1$ if and only if $X\cong\PP^1$.
		\item If $\mc L$ is $3$-very ample, and $\Sigma\ne\PP^n$, then
		\[\op{lcdef}(\Sigma)=\begin{cases}
			n-1 & \text{if }n\ge2\text{ and }\mathrm H^1(\OO_X)\ne0, \\
			n-2 & \text{if }n\ge2\text{ and }\mathrm H^1(\OO_X)=0, \\
			0 & \text{if }n=1.
		\end{cases}\]
	\end{listalph}
\end{theorem}
The speaker then did some intricate calculation to prove a partial result of (c), that $\op{lcdef}(\Sigma)\le n-1$.
\begin{remark}
	There are other theorems with more calculations of these invariants of secant varieties. For example, under the hypothesis of (c) above, the defect
	\[\sigma(Z)\coloneqq\dim_\QQ\frac{\op{WeilDiv}_\QQ(Z)}{\op{CarDiv}_\QQ(Z)}\]
	is finite if and only if $\mathrm H^1(\OO_X)=0$, and it is zero if and only if $X$ is $\PP^1$.
\end{remark}

\section{September 10: Unipotent Representations: Changing \texorpdfstring{$q$}{q} to \texorpdfstring{$-q$}{-q}}
This talk was given by George Lusztig at MIT for the Lie groups seminar.

\subsection{A Special Basis}
For today, $G$ is a split, connected reductive group over $k=\FF_q$. Each Weyl element $w\in W$ produces a subvariety $X_w$ of the flag variety $\mc B\times\mc B$ (where $\mc B$ is made of all the Borel subgroups) consisting of the pairs $(B,FB)$ for $w\in\OO_w$. It turns out that a unipotent representation of $G(k)$ appears in the cohomology of $X_w$ if and only if it appears in the Euler characteristic.

These unipotent representations have been classified as follows: as the Weyl group $W$ has its irreducible representations in canonical bijection with the conjugacy classes, the unipotent representations also have a classification according to these conjugacy classes. To explain how this bijection works, we give each $c$ a finite group $\Gamma_c$, then the unipotent representations are parameterized by pairs $(g,\rho)$ (up to conjugacy) where $g\in\Gamma_c$ and $\rho$ is an irreducible representation of $Z(g)$. We let $M(\Gamma_c)$ be this collection of pairs. For a pair $m$, we may write $\xi_m$ for the corresponding unipotent representation.

Here is the sort of thing that we are able to prove.
\begin{theorem}
	There is an ordered basis of $\CC[M(\Gamma_c)]$ where all the matrices relating the elements are upper-triangular with $1$s on the diagonal and otherwise positive entries; in fact, these entries are natural always except for a single $c$ in the case of $E_8$, in which case the entry is in $\ZZ[\zeta_5]$.
\end{theorem}
\begin{remark}
	It turns out that there is a canonical pairing on $\CC[M(\Gamma_c)]$ (which is akin to a nonabelian Fourier transform), and the basis of the theorem makes this pairing be represented by a matrix with nonnegative rational entries always except for one exception.
\end{remark}

\subsection{Negating Dimensions}
It turns out that the dimensions of the unipotent representations $\xi_m$ are polynomials in $q$, and we will write $D(\xi_m)$ for this dimension. We will assume that the long Weyl element is central. By tabulating these dimensions, one can show that there is an involution $(\cdot)'$ on the unipotent representations so that
\[D(\xi')(q)=\pm D(\xi)(-q).\]
However, this involution is not uniquely determined by this property.

Let's try to exhibit this involution. It turns out that $M(\Gamma_c)$ can be viewed as the set of irreducible objects in the tensor category of $\Gamma_c$-equivariant vector bundles. Then one can do something categorical.

\section{September 11: Statements of the Weil Conjectures}
This talk was given by Ari Krishna and Sophie Zhu at MIT for the STAGE seminar.

\subsection{Some History}
For today, $X$ will be a smooth proper variety over a finite field $\FF_q$. Let's give a statement of the Weil conjectures in the spirit of counting points.
\begin{conj}[Weil]
	Fix a finite field $\FF_q$.
	\begin{listalph}
		\item Fix a scheme $X$ of finite type over a field $\FF_q$. Then there are algebraic integers $\{\alpha_1,\ldots,\alpha_r\}$ and $\{\beta_1,\ldots,\beta_s\}$ such that
		\[\#X(\FF_{q^n})=\left(\alpha_1^n+\cdots+\alpha_r^n\right)-\left(\beta_1^n+\cdots+\beta_r^n\right)\]
		for all $n\ge0$.
		\item Rationality: suppose further that $X$ is proper of equidimension $d$. Then we can arrange these algebraic integers as
		\[\#X(\FF_{q^n})=\sum_{i=0}^{2d}(-1)^i\Bigg(\sum_{j=0}^{b_i}\alpha_{ij}^n\Bigg).\]
		\item Poincar\'e duality: with $X$ proper, the multi-sets $\{\alpha_{2d-i,j}:1\le j\le b_i\}$ and $\{q^d/\alpha_{ij}:1\le j\le b_i\}$ agree.
		\item Riemann hypothesis: with $X$ proper, $\left|\alpha_{ij}\right|=q^{i/2}$ for all $i$ and $j$.
		\item Betti numbers: with $X$ proper, if $X$ admits an integral model $\mc X$ over some subring $R\subseteq\CC$, then $b_i$ is the $i$th Betti number of $\mc X(\CC)$.
	\end{listalph}
\end{conj}
The history of these conjectures is long and fraught.
\begin{itemize}
	\item In the 1930s, Artin, Hasse, and Schmidt proved everything but the Riemann hypothesis for curves, and they proved the Riemann hypothesis for curves of genus at most $1$.
	\item In 1948, Weil proved the Weil conjectures for curves of any genus. This arose by combining two observations: first, counting $\#X(\FF_{q^n})$ should equal the number of fixed points of $F^n$, and second, these counts could be understood in terms of intersection theory with the graph of the Frobenius.
	\item In 1949, Weil proved the Riemann hypothesis for other varieties, namely certain Fermat varieties. At this point, the conjectures were finally stated.
	\item In the 1950s, Grothendieck and many others developed the theory of \'etale cohomology. By rather formal arguments, this proves everyting but the Riemann hypothesis.
	\item In 1974, Deligne finishes his first proof of the Weil conjectures.
	\item In 1980, Deligne strengthens his proof of the Weil conjectures.
\end{itemize}

\subsection{\texorpdfstring{$\zeta$}{zeta}-Functions}
The Weil conjectures admit an important reformulation in terms of $\zeta$-functions. Let's begin with the classical $\zeta$-function.
\begin{definition}
	The \textit{Riemann $\zeta$-function} $\zeta(s)$ is defined as the analytic continuation of the series
	\[\zeta(s)\coloneqq\sum_{n=1}^\infty\frac1{n^s}.\]
\end{definition}
The Riemann $\zeta$-function admits the following properties.
\begin{itemize}
	\item Euler product: one can write $\zeta(s)$ as a product
	\[\zeta(s)=\prod_{\text{prime }p}\frac1{1-p^{-s}}.\]
	\item Continuation: there is a meromorphic continuation to the plane, and it has only a simple pole at $s=1$.
	\item Functional equation: upon completing the $\zeta$-function as
	\[\xi(s)\coloneqq(s-1)\pi^{-s/2}\Gamma\left(\frac s2+1\right)\zeta(s),\]
	we have the functional equation $\xi(s)=\xi(1-s)$.
	\item Riemann hypothesis: it is expected that the only zeroes of $\zeta$ occur at the negative integer integers and along $\{s\in\CC:\Re s=1/2\}$.
\end{itemize}
This generalizes as follows.
\begin{definition}
	Fix a scheme $X$ of finite type over $\ZZ$. Then we define the \textit{arithmetic $\zeta$-function} $\zeta_X(s)$ as
	\[\zeta_X(s)\coloneqq\prod_{\text{closed }\mf p\in X}\frac1{1-\#\kappa(\mf p)^{-s}}.\]
\end{definition}
\begin{example}
	The Euler product implies that $\zeta(s)=\zeta_{\Spec\ZZ}(s)$.
\end{example}
In order to relate this to point-counts, we produce the following definition.
\begin{definition}
	Fix a scheme $X$ of finite type over $\FF_q$. Then we define
	\[Z_X(T)\coloneqq\exp\Bigg(\sum_{n\ge1}\#X(\FF_q)\frac{T^n}n\Bigg).\]
\end{definition}
\begin{remark}
	A direct calculation shows that $Z_X(q^{-s})=\zeta_X(s)$.
\end{remark}
We are now able to rewrite the Weil conjectures.
\begin{conj}[Weil]
	Fix a finite field $\FF_q$.
	\begin{listalph}
		\item Fix a scheme $X$ of finite type over $\FF_q$. There are algebraic integers $\{\alpha_1,\ldots,\alpha_r\}$ and $\{\beta_1,\ldots,\beta_s\}$ such that
		\[Z_X(T)\frac{(1-\beta_1T)\cdots(1-\beta_sT)}{(1-\alpha_1T)\cdots(1-\alpha_rT)}\]
		for some algebraic integers $\alpha_\bullet$s and $\beta_\bullet$s.
		\item Rationality: let $X$ be a smooth proper variety over $\FF_q$ of equidimension $d$. Then $Z_X$ admits a factorization as
		\[\frac{P_1(T)\cdots P_{2d-1}(T)}{P_0(T)\cdots P_{2d}(T)},\]
		where $P_i\in1+T\ZZ[T]$ for each $T$.
		\item Functional equation: with $X$ proper, we have $Z_X(1/q^dT)=\pm q^{d\chi/2}T^\chi Z_X(T)$, where $\pm$ is some sign, and $\chi$ is the Euler characteristic.
		\item Riemann hypothesis: with $X$ proper, we have $\left|\alpha_{ij}\right|=q^{i/2}$ for all $i$.
		\item Betti numbers: with $X$ proper, if $X$ admits an integral model $\mc X$ over some subring $R\subseteq\CC$, then $\deg P_i$ is the $i$th Betti number of $\mc X(\CC)$.
	\end{listalph}
\end{conj}
These statements are shown to be equivalent by expanding out the definition of $Z_X$ and taking logarithms.

\subsection{Proof for Curves}
We prove many of the Weil conjectures for curves. By keeping track of completions, we may as well assume that $X$ is smooth and proper. Let's start with rationality.
\begin{proposition}
	Fix a smooth proper curve $X$ over $\FF_q$. Then $Z_X(T)$ is a rational function of $T$.
\end{proposition}
\begin{proof}
	The point is to write $Z_X(T)$ out in terms of divisors, which will allow us to use Riemann--Roch. Recall $Z_X(T)$ is the product
	\[Z_X(T)=\prod_{\text{closed }p\in X}\left(1-T^{\deg p}\right)^{-1},\]
	which then expands out into the sum
	\[Z_X(T)=\sum_{\substack{D\in\op{Div}X\\D\ge0}}T^{\deg D},\]
	where $D\ge0$ means that $D$ is effective. There are now two cases: if $\deg D\le 2g-2$, we will handle this separately. Otherwise, when $\deg D\ge2g-2$, then Riemann--Roch implies that the number of effective divisors with this degree is $\left(q^{d-g+1}-1\right)/(q-1)$. (Namely, Riemann--Roch allows one to compute the dimension of the space of effective divisors with given degree; this is a finite vector space over $\FF_q$, so we can now compute its size!) This finishes the proof upon rewriting this out as a geometric series.
\end{proof}
\begin{remark}
	By inputting more effort, one can use this proof to prove the functional equation. If one is careful, then one can achieve an expansion
	\[Z_X(T)=\frac{f(T)}{(1-T)(1-qT)}\]
	for some polynomial $f(T)$ of degree $2g$ with integral coefficients. Note that this includes the Betti numbers conjecture!
\end{remark}
We now turn to the Riemann hypothesis, which of course is the hard part. This will depend on the following size bound.
\begin{theorem}[Hasse--Weil]
	Fix a smooth proper curve $C$ over a finite field $\FF_q$. Then
	\[\left|\#C(\FF_q)-(q+1)\right|\le2g\sqrt q.\]
\end{theorem}
Let's explain why this produces the Riemann hypothesis. Because we already have an expression
\[Z_X(T)=\frac{f(T)}{(1-T)(1-qT)},\]
we may factor $f(T)=\prod_{i=1}^{2g}(1-\alpha_iT)$, and we note that we are trying to show $\left|\alpha_i\right|=\sqrt q$ for all $i$. By the functional equation, it is enough to show merely that $\left|\alpha_i\right|\le\sqrt q$ for all $i$. Now, by definition of $Z_X(T)$, we see that
\[\sum_{n\ge1}\#C(\FF_{q^n})T^n=\frac d{dT}\log Z_X(T),\]
which can be computed directly to be
\[\sum_{n\ge1}\#C(\FF_{q^n})T^n=\sum_{i=1}^{2g}\Bigg(\frac{-\alpha_i}{1-\alpha_iT}+\frac1{1-T}+\frac1{1-qT}\Bigg),\]
which after expanding out the geometric series becomes
\[\sum_{n\ge1}\#C(\FF_{q^n})T^n=\sum_{n\ge1}\left(q^n+1-\sum_{i=1}^{2g}\alpha_i^n\right).\]
Thus, the Hasse--Weil bound shows that
\[\left|\sum_{i=1}^{2g}\alpha_i^n\right|\le2g\sqrt{q^n}.\]
Now, if $\left|\alpha_i\right|>\sqrt q$ for any $i$, then we can send $i\to\infty$ to achieve a contradiction because the left-hand side is too large.

\subsection{Intersection Theory on a Surface}
We will want to know something about intersection theory on a surface. We're in a talk, so we're allowed to just state the result we want.
\begin{theorem}
	Fix a smooth projective surface $X$ over an algebraically closed field $k$. Then there is a unique integral symmetric bilinear pairing $(\cdot,\cdot)$ on $\op{Div}X$ such that any two transverse curves $C,C'\subseteq X$ have
	\[(C,C')=\#(C\cap C').\]
\end{theorem}
There are many ways to the pairing $(C,C')$. The most geometric is to show that one can always wiggle one of the curves to make the intersection transverse.

For our bound, we need the following geometric input.
\begin{theorem}[Hodge index] \label{thm:hodge-index}
	Fix a smooth projective surface $X$ over an algebraically closed field $k$. Further, fix an ample line bundle $H$ in $\op{Div}X$. If we are given a divisor $D$ on $X$ which is not linearly equivalent to $0$ while $D\cdot H=0$, then $D\cdot D<0$.
\end{theorem}
\begin{proof}
	We will prove this in steps.
	\begin{enumerate}
		\item Suppose instead that $D\cdot H>0$ and $D^2>0$. Then we claim $mD$ is linearly equivalent to an effective divisor for sufficiently large $m$. Well, beacuse $D\cdot H>0$, $(K_X-mD)\cdot H<0$ for $m$ sufficiently large, so $K_X-mD$ cannot be effective. Thus, $\mathrm H^0(K_X-mD)=0$, so $\mathrm H^2(mD)=0$ by Serre duality. However, by Riemann--Roch for surfaces, one has
		\[h^0(mD)=h^1(mD)+\frac12mD\cdot(mD-K_X)+\chi(\OO_X),\]
		which becomes positive for $m$ large enough.

		\item Now, suppose for the sake of contradiction that $D^2>0$. Then we can take $H'\coloneqq D+nH$ to be ample for $n$ large enough, from which we find $D\cdot H'=D^2>0$, so the lemma implies that $mD$ is effective for $m$ large enough, which contradicts having $D\cdot H=0$.

		\item Lastly, suppose for the sake of contradiction that $D^2=0$. Because $D\cdot H=0$, we can find an effective divisor $E$ such that $D\cdot E\ne0$ while $E\cdot H=0$. Now, consider $D'\coloneqq nD+E$. One can calculate $(D')^2>0$ while $D'\cdot H=0$, so we reduce to the previous step.
		\qedhere
	\end{enumerate}
\end{proof}
To apply this, we will want to understand ample divisors.
\begin{theorem} \label{thm:get-ample}
	Fix a divisor $D$ on $X$. Then $D$ is ample if and only if $D^2>0$ and $D\cdot C>0$ for all irreducible curves $C$ on $X$.
\end{theorem}
Here is how this is applied.
\begin{theorem} \label{thm:apply-hodge-index}
	Let $X=C\times C'$ where $C$ and $C'$ are smooth projective curves. Set $\ell\coloneqq C\times\mathrm{pt}$ and $m\coloneqq\{\mathrm{pt}\}\times C'$. Then for any divisor $D$, we have
	\[D^2\le2(D\cdot\ell)(D\cdot m).\]
\end{theorem}
\begin{proof}
	As a lemma, we claim that if $H$ is ample, then
	\[\left(D^2\right)\cdot\left(H^2\right)\le(D\cdot H)^2.\]
	For this, one uses the Hodge index theorem on $E\coloneqq\left(H^2\right)D-(H\cdot D)H$, from which one can calculate $E^2<0$. Thus, as long as $D\ne0$, we get $\left(D^2\right)\left(H^2\right)-(D\cdot H)^2<0$; in all cases, we get the inequality.

	Now, by \Cref{thm:get-ample}, the divisor $H\coloneqq\ell+m$ is ample. Applying the above argument with $D'$ defined as
	\[D'=\left(H^2\right)\left(E^2\right)D-\left(E^2\right)(D\cdot H)H-\left(H^2\right)(D\cdot E)E\]
	where $E\coloneqq\ell-m$.
\end{proof}
We are now ready to prove the Hasse--Weil bound. We will do intersection theory on the surface $X\coloneqq C\times C$. Let $\Delta\subseteq X$ be the diagonal, and let $\Gamma\subseteq X$ be the graph. Then $\#C(\FF_q)=(\Delta\cdot\Gamma)$, which is what we want to bound. Here are our steps.
\begin{enumerate}
	\item We claim $\Delta^2=(2-2g)$. By the adjunction formula (note $\Delta\cong C$), we see
	\[2g-2=\Delta^2+\Delta\cdot K_X.\]
	However, one can expand out $K_X$ as $\mathrm{pr}_1^*C+\mathrm{pr}_2^*C$, which each have intersection number $2g-2$ with $\Delta$ by using the adjunction formula, so the result follows. 
	\item We claim $\Gamma^2=q(2-2g)$. By the adjunction formula (note $\Gamma\cong C$), we see
	\[2g-2=\Gamma^2+\Gamma\cdot K_X.\]
	After doing the same expansion of $K_X$, one calculates that $\Gamma\cdot\mathrm{pr}_1^*K_C=q(2g-2)$ and $\Gamma\cdot\mathrm{pr}_2^*K_C=2g-2$ by using the adjunction formula.
	\item We now apply \Cref{thm:apply-hodge-index} to $X=C\times C$. Take large integers $r$ and $s$, and set $D\coloneqq r\Gamma+s\Delta$. Then $D\cdot\ell=rq+s$ and $D\cdot m=r+s$. From \Cref{thm:apply-hodge-index}, one calculates that
	\[\left|N-(q+1)\right|\le g\left(\frac{rg}s+\frac sr\right),\]
	so the result follows by sending $\frac rs\to\frac1{\sqrt q}$.
\end{enumerate}

\section{September 15: Rational Points on \texorpdfstring{$X_0(N)^*$}{X0(N)+} for \texorpdfstring{$N$}{N} Non-Squarefree}
This talk was given by Sachi Haschimoto at Boston University of the Boston University number theory seminar. We are discussing joint work with Timo Keller and Samuel Le Fourn.

\subsection{Overview}
The motivation for our results arises from the following theorem of Mazur.
\begin{theorem} \label{thm:mazur}
	Fix a prime $p$, and let $E$ be an elliptic curve over $\QQ$ with (potential) complex multiplication. If $E$ admits a rational isogeny of degree $p$, then
	\[p\in\{2,3,5,7,13,37\}.\]
\end{theorem}
This is proved by classifying the points on certain modular curves.
\begin{definition}
	Fix a positive integer $N$. Then there is an affine curve $Y_0(N)$ defined over $\QQ$ such that $Y_0(N)(K)$ is in bijection with the $\overline K$-isomorphism classes of pairs $(E,C_N)$, where $E$ is an elliptic curve over $K$, and $C_N\subseteq E(K)$ is a cyclic subgroup of order $N$. We let $X_0(N)$ be the completion, and we set $J_0(N)\coloneqq\op{Jac}X_0(N)$.
\end{definition}
Thus, we see that \Cref{thm:mazur} is equivalent to saying that $Y_0(p)(\QQ)$ only has CM points except when $p$ is among the listed exceptions: simply take any such elliptic curve $E$ with rational isogeny $\varphi$ of degree $p$, and we can produce the pair $(E,\ker\varphi)\in Y_0(p)(\QQ)$. Equivalently, we may say that $X_0(p)(\QQ)$ only has CM points or cusps.
\begin{definition}[trivial]
	A modular curve $X_0(N)$ is \textit{trivial} if and only if $X_0(N)(\QQ)$ only has cusps and CM points.
\end{definition}
\begin{remark}
	The exceptional $p$ have $X_0(p)\cong\PP^1_\QQ$ (and therefore must have many points) except for $p=37$. When $p=37$, it turns out that $X_0(37)$ is genus $2$ and hence hyperelliptic, and one finds that the hyperelliptic involution produces the non-CM, non-cuspidal points.
\end{remark}
For this talk, we are interested in certain quotients of $X_0(N)$ rather than $X_0(N)$ itself.
\begin{definition}[Atkin--Lehner involution]
	Fix a positive integer $N$ and a divisor $Q\mid N$ with $\gcd(Q,N/Q)=1$. Then we define the \textit{Atkin--Lehner involution} $w_Q\colon X_0(N)\to X_0(N)$ by
	\[w_Q((E,C_N))\coloneqq\left(\frac E{C_N[Q]},\frac{C_N+E[Q]}{C_N[Q]}\right)\]
	One can check that this definition is smooth and therefore extends from $Y_0(N)$ to $X_0(N)$. One can also check that $w_Q^2=\id$.
\end{definition}
\begin{remark}
	We see that $w_Q$ also extends to $J_0(N)$ by pullback, and it still has $w_Q^2=\id$. To relate to modular forms, we note that $S_2(\Gamma_0(N))$ is the cotangent space of $0\in J_0(N)$, from which it follows that $w_Q$ also acts on $S_2(\Gamma_0(N))$. But $S_2(\Gamma_0(N))$ is just some finite-dimensional complex vector space, so we end up with some linear algebra.
\end{remark}
\begin{notation}
	We define $X_0(N)^*$ as the quotient of $X_0(N)$ by all the Atkin--Lehner involutions.
\end{notation}
\begin{remark}
	It turns out that rational points of $X_0(N)^*$ (minus cusps) correspond to elliptic curves which are isogenous to all their Galois conjugates, where the isogenies have degree dividing by $N$.
\end{remark}
Here is what is currently known, which is due to many people.
\begin{theorem}
	Fix a positive integer $N$. If $N$ is a prime-power, and the genus of $X_0(N)^*$ is positive, then all rational points of $X_0(N)^*(\QQ)$ are trivial, except when $N=5^3$, in which case there is only one interesting point.
\end{theorem}
\begin{remark}
	The hardest part of the theorem covers the cases $N\in\{125,169\}$, where the rank of the Jacobian equals the genus, so one has to work harder to make something like the Chabauty--Coleman method work.
\end{remark}
Here is what is expected
\begin{conj}[Elkies]
	For $N$ large enough, all rational points on $X_0(N)^*(\QQ)$ are either CM points or cusps.
\end{conj}

\subsection{Formal Immersion Method}
We will be interested in the following morphisms.
\begin{definition}[formal immersion]
	Fix locally Noetherian schemes $X$ and $Y$. A map $f\colon X\to Y$ is a \textit{formal immersion} at some point $x\in X$ if and only if it induces an isomorphism of residue fields and is injective on tangent spaces.
\end{definition}
Here is how these are used.
\begin{theorem}[Mazur] \label{thm:use-formal-immersion}
	Fix a morphism $f\colon X_0(N)\to A$ of schemes over $\ZZ$, where $A$ is an abelian scheme over $\ZZ$ of rank $0$. Suppose that $f$ is a formal immersion at $\infty$, where $f(\infty)=0$. Then any $(E,C_N)\in X_0(N)(\QQ)$ has $E$ with potentially good reduction at all primes $p>2$.
\end{theorem}
The point is that we have upgraded rationality of $E$ to some integrality. Indeed, an equivalent statement is that $j(E)\in\ZZ[1/2]$.
\begin{proof}[Sketch]
	Suppose not. Then $E$ becomes a cusp at some prime $p$, so by using Atkin--Lehner involutions, one can move $E$ to become $\infty$ at $p$. Now, $f((E,C_N))\in A(\QQ)$ is torsion because $A(\QQ)$ is torsion, so because torsion reduces$\pmod p$ injectively, we see that $f((E,C_N))\equiv f(\infty)\equiv0\pmod p$ and therefore $f(x)=f(\infty)=0$. But then our tangent spaces will fail to be injective.
\end{proof}
\begin{remark}
	If $N$ is squarefree, then the simple factors of $J_0(N)^*$ over $\QQ$ are all expected to have positive rank (by the Birch--Swinnerton-Dyer conjecture), so one does not expect \Cref{thm:use-formal-immersion} to be particularly helpful.
\end{remark}
Here is what we are able to prove.
\begin{theorem}
	Fix $N\ge1$ which is not $99$ or $147$ but is not squarefree and not a prime-power. For any $P\in Y_0(N)^*(\QQ)$, choose a lift $(E,C_N)$ to $Y_0(N)(K)$ for some number field $K$. Then
	\[\left(8\cdot3\cdot25\cdot49\cdot31\right)^Nj(E)\in\OO_K.\]
	In fact, if $p^2\mid N$ for some $p\notin\{2,3,5,7,13\}$, then $j(E)\in\OO_K$.
\end{theorem}
\begin{remark}
	The $N\in\{99,147\}$ cases have positive rank, so we do not have hope for an integrality statement.
\end{remark}
\begin{remark}
	The exponent $N$ is more or less sharp. The $31$ is never expected to appear.
\end{remark}
Our method proof, following Mazur, uses the following.
\begin{theorem}[Mazur]
	If $p\notin\{2,3,5,7,13\}$, then $J_0(p)^-\coloneqq J_0(p)/(1+w_p)$ has a rank zero quotient.
\end{theorem}
The exceptions arise because $\dim J_0(p)=0$, so there is no way to find such a quotient. These rank zero abelian varieties are good enough for our purposes most of the time, but we have also proven the following.
\begin{theorem}
	Fix distinct primes $p$ and $q$ with $p\in\{2,3,5,7,13\}$ and $q>23$. Then $J_0(pq)^{-p,+q}\coloneqq J_0(pq)/(1+w_p,1-w_q)$ has a rank $0$ quotient.
\end{theorem}
This is proven using some analytic techniques showing that some $L$-functions vanish. Now, the point of the proof of the main theorem is to do some reduction of levels to construct formal immersions to abelian varieties of rank $0$.

% Let's explain how we can prove our results. The moral is to reduce to squarefree levels.

\section{September 16: Higher Siegel--Weil Formula for Unitary Groups}
This talk was given by Mikayel Mkrtchyan at MIT for the MIT number theory seminar.

\subsection{The Kudla Program}
Let's start by recalling the classical formulation of the Siegel--Weil formula. An approximate statement is that an Eisenstein series $E(g,s_0)$ is approximately equal to a period $\int_{[H]}\theta(g,h)\,dh$ for some $\theta$-function $\theta$. This was proved for a dual pair $(G,H)$ of reductive groups over a global field $F$ in the case where $(G,H)=(\op U(2n),\op U(n))$.

The Kudla program geometrizes this into a statement that $E(g,s_0)$ should be equal to the degree of some $0$-cycle on a Shimura variety. There is also an arithmetic Siegel--Weil formula, which describes the derivative of the Eisenstein series as the degree of a $0$-cycle on the integral model(!) of a Shimura variety.

For this talk, we will be working over function fields. As such, we go ahead and fix a morphism $X'\to X$ of smooth curves over a finite field $k\coloneqq\FF_q$, and we will assume that $X'\to X$ is an \'etale double cover, where $\sigma\colon X'\to X'$ is the automorphism. We also set $F\coloneqq k(X)$ and $F'\coloneqq k(X')$.

With the above motivation in mind, we are now ready to set some notation. Let's describe the automorphic side of our Siegel--Weil formula.
\begin{definition}[Siegel--Eisenstein series]
	Fix the group $G\coloneqq\op U(n,n)$ over a global field, and let $P\subseteq G$ be a parabolic subgroup. Consider the induction
	\[\op{Ind}_{P(\AA)}^{G(\AA)}\left(\chi\left|\det\right|^{s+n/2}\right),\]
	where $\chi$ is some auxiliary character, and $s\in\CC$. To produce an automorphic form, we may choose some $\varphi(g,s)$ in this induction satisfying $\varphi(1,s)=1$, and then we define our Eisenstein series as
	\[E(g,s)\coloneqq\sum_{\gamma\in P(F)\backslash G(F)}\varphi(\gamma g,s).\]
	We may normalize this to $\widetilde E(g,s)$ via some silent process.
\end{definition}
\begin{remark}
	The Eisenstein series $E(g,s)$ admits a Fourier decomposition, indexed by Hermitian $n\times n$ matrices $T\in\op{Herm}_n(F)$. These can alternatively be parameterized by pairs $(\mc E,a)$ where $\mc E$ is a vector bundle of rank $n$, and $a\colon\mc E\to\sigma^*\mc E^\lor$ is some map. %In particular, $E_T(g,s)$ takes some pair $(\mc E,a)$ of a rank-$n$ vector bundle $\mc E$ on $X'$
\end{remark}
We now turn to the geometric side. Instead of Shimura varieties, we are able to use shtukas.
\begin{definition}
	A \textit{$\op U(n)$-bundle} on $X'$ is a pair $(\mc F,h)$ of a rank-$n$ vector bundle on $X'$ and some isomorphism $h\colon\mc F\to\sigma^*\mc F$. We let $\op{Bun}_{\op U(n)}$ denote the relevant moduli space.
\end{definition}
\begin{definition}
	We define the \textit{Hecke modification} $\op{HK}^1_{\op U(n)}$ as the collection of tuples $(x,\mc F_0,\mc F_1,f_{1/2},\iota_0,\iota_1)$ where $x\in X'$, $\mc F_0$ and $\mc F_1$ are $\op U(n)$-bundles, and $\iota_0\colon\mc F_{1/2}\to\mc F_0$ and $\iota_1\colon\mc F_{1/2}\to\mc F_1$ are some maps respeecting the unitary structure.
\end{definition}
\begin{remark}
	It turns out that the canonical map ${\op{HK}^1_{\op U(n)}}\to X'\times\op{Bun}_{\op U(n)}$ onto the first two coordinates is a $\PP^{n-1}$-bundle.
\end{remark}
\begin{definition}
	We define $\op{Sht}^r_{\op U(n)}$ as the moduli space of chains of $r$ Hecke modifications
	\[\mc F_0\to\mc F_1\to\cdots\to\mc F_r,\]
	where $\mc F_r$ is isomorphic to the Frobenius twist of $\mc F_0$. This is a Deligne--Mumford stack.
\end{definition}
\begin{remark}
	It turns out that $\dim\op{Sht}^r_{\op U(n)}=rn$ for even $r$ and vanishes for odd $r$. (One should modify the definition slightly to get interesting statements when $r$ is odd.)
\end{remark}
With our analogue of Shimura varieties in hand, we are able to define our special cycles.
\begin{definition}
	Fix a pair $(\mc E,a)$ of a vector bundle $\mc E$ of rank $m$ and a map $a\colon\mc E\to\sigma^*\mc E^\lor$. Then we define the special cycle $Z^r_{\mc E}(a)\to\op{Sht}^r_{\op U(n)}$ to parameterize the data of an $r$-shtuka
	\[\mc F_0\to\mc F_1\to\cdots\to\mc F_r\]
	equipped with commuting maps $\mc E\to\mc F_0$ which commute with our Frobenius twists and pullbacks by $\sigma$.
\end{definition}
\begin{remark}
	The map $Z^r_{\mc E}(a)\to\op{Sht}^r_{\op U(n)}$ is finite, so we are more or less producing a cycle.
\end{remark}
\begin{remark}
	If $a$ in the pair $(\mc E,a)$ is an isomorphism, then $\mc Z^r_{\mc E}(a)\cong\op{Sht}^r_{\op U(n-\op{rank}\mc E)}$.
\end{remark}
\begin{remark}
	The expected dimension of $Z^r_{\mc E}(a)$ is $r(n-\op{rank}\mc E)$.
\end{remark}
We are now raedy to state our main theorem.
\begin{theorem}
	Fix the following data.
	\begin{itemize}
		\item A pair $(\widetilde{\mc E},\widetilde a)$, where $\op{rank}\widetilde E=n$ with $\op{rank}\im\widetilde a=n-1$. Then $Z^r_{\widetilde{\mc E}}(\widetilde a)$ is proper, and the degree of the corresponding cycle equals
		\[\del_{s=0}^r\left(\widetilde E_{(\widetilde E,\widetilde a)(s)}\right)\]
		up to some explicit constant.
	\end{itemize}
\end{theorem}

\section{September 18th: The \'Etale Site}
This talk was given by Yutong Chen for the STAGE seminar at MIT.

\subsection{\'Etale Morphisms}
We will be interested in \'etale morphisms today. Intuitively, they are supposed to be the algebro-geometric version of a covering space in topology. Here is the easiest definition.
\begin{definition}[\'etale]
	A morphism $f\colon X\to S$ of schemes is \textit{\'etale} if and only if it is locally of finite presentation, flat, and unramified.
\end{definition}
While locally of finite presentation and flatness are fairly common notions, we should define what it means for a morphism to be unramified. We will define this in steps.
\begin{definition}[unramified]
	Fix an extension $A\subseteq B$ of discrete valuation rings with uniformizers $\pi_A$ and $\pi_B$, respectively. Then $A\subseteq B$ is \textit{unramified} if and only if $(\pi_B)=\pi_A\cdot B$ and the extension of residue fields is separable.
\end{definition}
\begin{definition}[unramified]
	Fix a map $f\colon A\subseteq B$ of local rings. Then $f$ is \textit{unramified} if and only if $f(\mf m_A)=\mf m_B$ and the field extension
	\[A/\mf m_A\to B/\mf m_B\]
	is separable.
\end{definition}
\begin{definition}[unramified]
	Fix a morphism $f\colon X\to S$ of schemes. Then $f$ is \textit{unramified} if and only if the local maps
	\[\OO_{S,f(x)}\to\OO_{X,x}\]
	are unramified for all $x\in X$.
\end{definition}
\begin{example}
	Open and closed immersions are unramified.
\end{example}
\begin{nex}
	Consider the squaring map $\AA^1_k\to\AA^1_k$ given by the ring map $k[t]\to k[t^2]$ defined by $t\mapsto t^2$. Then this map is not ramified at $0$. Indeed, this map is locally given by
	\[k[t^2]_{\left(t^2\right)}\to k[t]_{(t)},\]
	but the maximal ideal fails to go to the maximal ideal.
\end{nex}
There are many ways to think about \'etale morphisms.
\begin{definition}[\'etale]
	A morphism $f\colon X\to S$ is \textit{\'etale} if and only if it is smooth of relative dimension $0$.
\end{definition}
Here is one version of smoothness which is fairly hands-on.
\begin{definition}[smooth]
	Fix a morphism $f\colon X\to S$. Given $x\in X$, we say that $f$ is \textit{smooth} at $x$ if and only if the morphism locally looks like
	\[\Spec\frac{A[t_1,\ldots,t_n]}{(g_{r+1},\ldots,g_n)}\to\Spec A\]
	and the corresponding Jacobian matrix has full rank $n-r$. We may also say that $f$ is smooth of \textit{relative dimension $r$} in this situation.
\end{definition}
Of course, there are also many ways to define smoothness. Here is another useful criterion.
\begin{proposition}
	Fix a flat morphism $f\colon X\to S$ of irreducible varieties over a field $k$, and set $r\coloneqq\dim X-\dim S$. Then $f$ is smooth of relative dimension $r$ if and only if $\Omega_{X/S}$ is locally free of rank $r$.
\end{proposition}
Here are a few more ways to work with the yoga of \'etale morphisms.
\begin{proposition} \label{prop:basic-etale}
	Fix a ring $A$, an extension $B=A[t]/(p)$ where $p\in A[t]$ is monic, and a localization $C=B\left[q^{-1}\right]$ for some $q$. If $p'(t)\in C^\times$, then the natural map $\Spec C\to\Spec A$.
\end{proposition}
We will not prove this (all of these proofs are horribly annoying), but we will content ourselves with an example.
\begin{example}
	Fix $A\coloneqq k[x]$ and $B\coloneqq k[x,y]/\left(y^2-x(x-1)(x+1)\right)$. Then $\Spec B\to\Spec A$ is basically the projection from an elliptic curve to the affine line, so we expect to have some ramification at $(0,0)$, $(1,0)$, and $(-1,0)$. Accordingly, if we localize out by $x^3-x$, then we see that the map $\Spec C\to\Spec A$ is successfully \'etale, which can be checked because the derivative of $p(y)=y^2-\left(x^3-x\right)$ is in $C^\times$.
\end{example}
\begin{remark}
	It turns out that all \'etale morphisms can locally be factored like \Cref{prop:basic-etale}.
\end{remark}
\begin{proposition}
	Fix a smooth morphism $f\colon X\to S$ of relative dimension $r$ at a point $x\in X$. Further, fix some local functions $g_1,\ldots,g_r\in\OO_{X,x}$. Then the following are equivalent.
	\begin{listroman}
		\item The elements $dg_1,\ldots,dg_r$ form a local basis for $\Omega_{X/S}\otimes k(x)$.
		\item The elements $g_1,\ldots,g_r$ extend to an open neighborhood $U$ of $x$ such that $(g_1,\ldots,g_r)\colon U\to\AA^r_S$ is \'etale.
	\end{listroman}
\end{proposition}
\begin{remark}
	Property (i) is relatively easy to satisfy, so we know that such functions surely exist.
\end{remark}
\begin{remark}
	The point of (ii) is that $f$ now factors as
	\[X\supseteq U\to\AA_S^r\to S,\]
	where the map $U\to\AA_S^r$ is \'etale. Thus, smooth morphisms are ``just'' projections up to an \'etale map.
\end{remark}

\subsection{The Fundamental Group}
Continuing with our intuition that \'etale morphisms are covering spaces, we now try to define a fundamental group. It is difficult to make sense of paths in algebraic geometry, so instead we will use covering spaces. Here is the construction that we will try to generalize.
\begin{example}
	For a nice topological space $X$ (e.g., a manifold) with a basepoint $x\in X$, then there is a natural ``fiber'' functor
	\[\op{Fib}_x\colon\op{Cover}(X)\to\mathrm{Set}\]
	from the category of covering spaces of $X$ to sets given by sending $p\colon Y\to X$ to the fiber $p^{-1}(\{x\})$. By a path-lifting argument, one shows that
	\[\pi_1(X,x)=\op{Aut}({\op{Fib}_x}).\]
	(In particular, path-lifting desribes an action of $\pi_1(X,x)$ on all fibers in a compatible way.) We remark that this allows us to upgrade the fiber functor into an equivalence
	\[\op{Fib}_x\colon\op{Cover}(X)\to\mathrm{Set}(\pi_1(X,x)).\]
\end{example}
\begin{remark}
	Topology is aided by the existence of a universal cover. For example, one has a universal cover of $S^1$ given by $\RR\onto S^1$, but this covering space fails to be finite; similarly, the universal cover of $\CC^\times$ is the exponential map $\exp\colon\CC\onto\CC^\times$, which is not algebraic. Algebra is going to have some trouble producing coverings which are not finite (or algebraic), so we will have to content ourselves with some finite quotients.
\end{remark}
Accordingly, we find that we are contenting ourselves to work with finite covering spaces, which amounts to working with finite \'etale covers.
\begin{definition}[\'etale fundamental group]
	Fix a scheme $X$ and a geometric point $\ov x\into X$, and consider the corresponding category $\op{Fin\acute Et}(X)$ of finite \'etale covers of $X$. Then we define the \textit{\'etale fundamental group} $\pi_1(X,\ov x)$ to be the automorphism group of the fiber functor
	\[\op{Fib}_x\colon\op{Fin\acute Et}(X)\to\mathrm{Set}\]
	given by sending the cover $p\colon Y\to X$ to the covering to the fiber $Y\times_p\ov x$.
\end{definition}
\begin{remark}
	As in the topological case, one finds that $\mathrm{Fib}_x$ upgrades to an equivalence
	\[\op{Fib}_x\colon\op{Fin\acute Et}(X)\to\mathrm{Set}(\pi_1(X,\ov x)).\]
\end{remark}
As a sanity check, we note the following comparison theorem.
\begin{theorem}
	Fix an irreducible variety $X$ over $\CC$. Then $Y\mapsto Y(\CC)$ upgrades to an equivalence of categories between the finite \'etale covers of $X$ and the finite covers of $X(\CC)$.
\end{theorem}
\begin{example}
	Consider $X=\CC\left[x,x^{-1}\right]$ so that $X(\CC)=\CC^\times$. Then we see that $\pi_1^{\mathrm{\acute et}}(X,\ov 1)$ will be $\widehat\ZZ$ because it is the colimit of the automorphism groups of the finite covers of $\CC^\times$.
\end{example}
But now that we can do algebraic geometry, we can add in some arithmetic information.
\begin{example}
	Consider the point $X=\Spec k$ and an algebraic closure $\ov x=\Spec\ov k$. Then a finite \'etale cover $Y\to X$ will be a finite disjoint union of points. To describe our category, we are allowed to work with just the connected covers of $X$, which amounts to making $Y$ a point, so we may write $Y=\Spec L$. In order for the map $Y\to X$ to be an \'etale cover, it is equivalent to ask for the induced field extension $k\subseteq L$ to be finite and separable. The fiber of such an $L$ is given by
	\[(Y\times\ov x)(\ov k)=\Spec(L\otimes\ov k)(\ov k)=\op{Hom}_k(L,\ov k).\]
	Thus, $\mathrm{F\acute Et}(X)$ amounts to the category of finite separable extensions of $k$, and it is not hard to see that the automorphism group is simply $\op{Gal}(\ov k/k)$.
\end{example}

\subsection{Grothendieck Topologies}
The point of a Grothendieck topology is to recognize that what makes a topology important is not its open sets but instead the notion of covers. Thus, to specify a Grothendieck topology, we will try to specify the covers and make do with that.
\begin{definition}[Grothendieck topology]
	Fix a category $\mc C$ closed under finite products. A \textit{Grothendieck topology} on $\mc C$ is a collection of families $\mc T$ of the form $\{f_i\colon U_i\to U\}_i$ and satisfying the following.
	\begin{listalph}
		\item Isomorphisms: the family $\mc T$ contains all isomorphisms.
		\item Refinement: given a covering $\{U_i\to U\}_i$ in $\mc T$ and some coverings $\{V_{ij}\to U_i\}_j$, then the composite $\{V_{ij}\to U_i\to U\}_{i,j}$ continues to be in $\mc T$.
		\item Pullback: given a covering $\{U_i\to U\}_i$ in $\mc T$ and some object $V$ with a map $V\to U$, then the pullback $\{U_i\times_U V\to V\}_i$ is in $\mc T$.
	\end{listalph}
	In this situation, the pair $(\mc C,\mc T)$ is a site.
\end{definition}
Here is the motivating example.
\begin{example}[Zariski site]
	If $X$ is a topological space, then we can let $\mc C$ be the category of open sets in $X$ with morphisms given by inclusion. We can endow $\mc C$ with the structure of a Grothendieck topology by letting the covers simply be the open covers. If $X$ is a scheme, then this site is called the Zariski site.
\end{example}
Here is the site for today.
\begin{definition}[small \'etale site]
	Fix a scheme $X$, and consider the category $\op{\acute Et}(X)$ of all \'etale covers of $X$. Then we endow $\op{\acute Et}(X)$ with the structure of a Grothendieck topology by saying that a collection of morphisms $\{U_i\to U\}_i$ is a covering if and only if $\bigsqcup_i U_i\to U$ is surjective. This is called the \textit{(small) \'etale site} and is denoted $X_{\mathrm{\acute et}}$.
\end{definition}
\begin{remark}
	It turns out that a morphism of \'etale covers of $X$ is automatically \'etale. This can be proven using the usual techniques of cancellation.
\end{remark}
\begin{remark}
	By replacing the word \'etale with other adjectives, we also have an fppf site and fpqc site. We note that the Zariski site has the same definition where \'etale is replaced with open embeddings.
\end{remark}
As usual, once we have an object, we want some morphisms.
\begin{definition}[continuous]
	A \textit{continuous} map $F\colon(\mc C',\mc T')\to(\mc C,\mc T)$ is the data of a functor $F\colon\mc C\to\mc C'$ satisfying the following.
	\begin{listalph}
		\item For any covering $\{U_i\to U\}_i$ in $\mc T$, we require that $\{FU_i\to FU\}_i$ to be in $\mc T'$.
		\item Given a covering $\{U_i\to U\}_i$ in $\mc T$ and a map $V\to U$, then we require that $F(V\times_UU_i)\to FV\times_{FU}FU_i$ to be an isomorphism.
	\end{listalph}
\end{definition}
\begin{remark}
	If $f\colon X'\to X$ is a continuous map of topological spaces, then taking the pre-image indueces a functor of the categories of open sets, and one can see directly that taking the pre-image produces a continuous map of the Grothendieck topologies.
\end{remark}
\begin{remark}
	For any scheme $X$, there is a continuous map between the \'etale site
	\[X_{\mathrm{fpqc}}\to X_{\mathrm{fppf}}\to X_{\mathrm{\acute et}}\to X_{\mathrm{Zar}}.\]
\end{remark}
The point of having a notion of topology is that it lets us do sheaf theory.
\begin{definition}
	Fix a Grothendieck topology on a category $\mc C$. Then a presheaf $\mc F\colon\mc C\opp\to\mathrm{Ab}$ is a \textit{sheaf} if and only if the usual exact sequence
	\[\mc F(U)\to\prod_i\mc F(U_i)\to\prod_{i,j}\mc F(U_i\times_UU_j)\]
	is exact for all covers $\{U_i\to U\}_i$.
\end{definition}
\begin{example}
	A sheaf on the Zariski site is the usual notion of sheaf in scheme theory.
\end{example}
\begin{remark}
	Because open embeddings are already \'etale, fppf, and fpqc, we see that a sheaf on any of these sites must be a Zariksi sheaf as well.
\end{remark}
\begin{remark}
	Because the sites we care about are closed under arbitrary coproduct, it is enough to check it on coverings which look like $U'\to U$, though of course one cannot require either $U'$ or $U$ to be connected.
\end{remark}
We have yet to construct any sheaves! Here is the usual way to do so.
\begin{definition}
	Fix a scheme $X$. For any Zariski quasicoherent sheaf $\mc F$ on $X$, we define the \'etale pre\-sheaf $\mc F^{\mathrm{\acute et}}$ on $X_{\mathrm{\acute et}}$ by sending the cover $p\colon U\to X$ to
	\[\mc F_{\mathrm{\acute et}}(U)\coloneqq\op{Hom}(p^*\OO_X,p^*\mc F).\]
\end{definition}
\begin{remark}
	It turns out that this construction produces a sheaf. Something similar works for the fppf sites and fpqc sites. Let's explain this for the fpqc site. Indeed, fix a fpqc morphism $p\colon S'\to S$, so we set $S'\coloneqq S'\times_SS'$ with projection $q\colon S''\to S$, and we need to check that the usual sequence
	\[\mc F_{\mathrm{fpqc}}(S)\to\mc F_{\mathrm{fpqc}}(S')\to\mc F_{\mathrm{fpqc}}(S'')\]
	is exact. Accordingly, we see that we may as well replace $\mc F$ with the pullback to $S$ (so that $X=S$), and we have left to check that
	\[\op{Hom}(\OO_S,\mc F)\to\op{Hom}(\OO_{S'},p^*\mc F)\to\op{Hom}(\OO_{S''},q^*\mc F)\]
	is exact. Exactness now follows from some notion of descent.
\end{remark}
The last remark we should make about sheaves on a site is that we can do sheafification.
\begin{definition}[sheafification]
	Fix a site $\mc C$. Then there is a left adjoint to the forgetful functor $\mathrm{Sh}(\mc C)\to\mathrm{PSh}(\mc C)$, which we call sheafification.
\end{definition}

\section{September 22nd: Uniform Boundedness over Function Fields}
This talk was given by Jit Wu Yap at Boston University for the Boston University number theory seminar.

\subsection{The Main Theorems}
For today, we will work over a function field $K\coloneqq\CC(B)$, where $B$ is a smooth projective curve over $\CC$. We are interested in abelian varieties with semistable reduction.
\begin{definition}[semistable reduction]
	Fix an abelian variety $A$ over $K$. We say that $A$ has \textit{semistable reduction} if and only if there is a semiabelian scheme $G$ over $B$ such that $G_K\cong A$.
\end{definition}
Here are two theorems.
\begin{theorem} \label{thm:bounded-torsion-ff}
	Fix an integer $g$. Then there is an integer $N$ only depending on $g$ and the genus of $B$ such that all $g$-dimensional abelian varieties $A$ over $K$ of semistable reduction has
	\[\op{ord}(x)\le N\]
	for all $x\in A(K)_{\mathrm{tors}}$.
\end{theorem}
\begin{remark}
	Intuitively, this is a result on the boundedness of torsion, analogous to Maur's theorem for $g=1$ over $K=\QQ$. There is a long history of such results. In the 1990s, results were achieved for $g=1$ over a number field. For $g\ge2$, Silverberg showed the result when $A$ has complex multiplication. Cadoret--Tamagawa's results were used as an input to Bakkar--Tsiermann showing this for $g\ge2$ when $A$ has real multiplication in 2018.
\end{remark}
\begin{theorem} \label{thm:lang-silverman-ff}
	Fix an integer $g$. Then there is a positive constant $c$ depending only on $g$ and the genus of $B$ such that all $g$-dimensional abelian varieties $A$ of semistable reduction, then the N\'eron--Tate height of any $x\in A(K)$ for which $\overline{\ZZ x}=A$ is bounded below by $ch_{\mathrm{Fal}}(A)$.
\end{theorem}
The Faltings height and the Weil height machine work just fine for function fields over $\CC$. For example, one can define the Faltings height as the usual Weil height of the point $A\in\mc A_{g,3}^*$ against some canonically defined ample line bundle.
\begin{remark}
	This is referred to as a ``Lang--Silverman'' result because it was conjectured by them. In the case of $g=1$, this was shown by Hindry--Silverman in 1988.
\end{remark}
\begin{remark}
	The methods are largely arithmetic. The fact that we are working over $\CC$ is used only once in the proof: we use Faltings's Arakelov inequality, which asserts that there is a positive constant $c$ depending only on $g=\dim A$ for which
	\[h_{\mathrm{Fal}}(A)\le c\left(\left|S\right|+g(B)+1\right),\]
	where $S$ is the set of places of bad reduction of $A$. Accordingly, if this inequality is true over number fields $K$ (which is known as a higher-dimensional Szpiro conjecture), then \Cref{thm:bounded-torsion-ff,thm:lang-silverman-ff} hold. However, Szpiro's conjecture is known to be hard: just in $g=1$, it is known to imply the $abc$ conjecture.
\end{remark}

\subsection{Some Ideas}
Here is the main idea for the results.
\begin{enumerate}
	\item If $x\in A(K)$, then the Arakelov inequality is able to place constraints on $x\in A(K_v)$ for many places $v$.

	\item However, $K$-points of small N\'eron--Tate height will equidistribute. Here is a formal statement for elliptic curves: given ascending collections $F_n\subseteq A(\ov K)$ of Galois-invariant points, then it is known that
	\[\frac1{\#F_n}\sum_{x\in F_n}\delta_x\to\mu,\]
	for some suitably defined Haar measure $\mu$. Thus, we cannot expeect to have too many points in $A(K)$ of small N\'eron--Tate height.
\end{enumerate}
Here are a couple of tools.
\begin{enumerate}
	\item Over $\CC$, there is a notion of ``transfinite diameter'' of a sequence of points $\{x_1,\ldots,x_n\}$ defined as
	\[\frac1{n^2}\sum_{i\ne j}-\log\left|x_i-x_j\right|.\]
	This turns out to measure how close a given set of points are to a large-degree hypersurface.

	\item It turns out that abelian varieties over local fields satisfies ``degeneration by ultrafilters.'' Roughly speaking, given a countable collection $\{K_n\}_n$ of complete, algebraically closed field, nonarchimedean fields, and $g$-dimensional principally polarized abelian varieties $A_n$ over $K_n$, then there is a ``limit'' $A_\omega$ over some complete, algebraically closed, nonarchimedean field $K_\omega$ for any ultrafilter $\omega$ on $\NN$. Roughly speaking, one fixes a compactification $\ov X$ of $X\coloneqq\mc A_{g,3}$. Then define $\lambda_{\del X}(A)$ to be $-\log$ of the distance of $A$ to $\del\ov X$. Then we may define
	\[A^\varepsilon\coloneqq\Bigg\{(x_n)_n\in\prod_{n=1}^\infty K_n:\left|x_n\right|^{1/\lambda_{\del X}(A_n)}\text{ is bounded}\Bigg\}.\]
	It now turns out that there is a morphism $\Spec A^\varepsilon\to\mc A_{g,3}$ making the diagram
	% https://q.uiver.app/#q=WzAsMyxbMCwwLCJcXFNwZWMgQV5cXHZhcmVwc2lsb24iXSxbMCwxLCJcXFNwZWMgS19uIl0sWzEsMCwiXFxtYyBBX3tnLDN9Il0sWzEsMiwiQV9uIiwyXSxbMCwxLCJcXG9we3ByfV9uIiwyXSxbMCwyXV0=&macro_url=https%3A%2F%2Fraw.githubusercontent.com%2FdFoiler%2Fnotes%2Fmaster%2Fnir.tex
	\[\begin{tikzcd}[cramped]
		{\Spec A^\varepsilon} & {\mc A_{g,3}} \\
		{\Spec K_n}
		\arrow[from=1-1, to=1-2]
		\arrow["{\op{pr}_n}"', from=1-1, to=2-1]
		\arrow["{A_n}"', from=2-1, to=1-2]
	\end{tikzcd}\]
	commute. Of course, one can certainly induce a map to $\ov X$, so the difficulty is showing that we do not end up in the boundary in the limit.
\end{enumerate}

\section{September 23: Unlikely Intersections}
This talk was given by Xinyu Fang as a pre-talk for the Harvard--MIT algebraic geometry seminar.

\subsection{The Ax--Schanuel Theorem}
Today, we are going to discuss the following result, which is a version of unlikely intersections for $\exp$. Define $\pi\colon\CC^n\to\left(\CC^\times\right)^n$ by
\[\pi(z_1,\ldots,z_n)\coloneqq(\exp(2\pi iz_1),\ldots,\exp(2\pi iz_n)).\]
We are interested in when $\pi$ sends algebraic subvarieties to algebraic subvarieties.
\begin{definition}[bialgebraic]
	A subvariety $L\subseteq\CC^n$ is \textit{bialgebraic} if and only if $\pi(L)$ continues to be algebraic. 
\end{definition}
\begin{example}
	If $L\subseteq\CC^n$ is a linear subspace cut out by some (rational) equations of the form
	\[\sum_ia_iz_i=c,\]
	then $\pi(L)$ continues to be an algebraic subvariety now cut out by
	\[\prod_iw_i^{a_i}=1,\]
	where $w_i$ is $\exp(2\pi iz_i)$.
\end{example}
\begin{theorem}
	Every bialgebraic subvariety over $\CC$ is a linear subspace.
\end{theorem}
Here is a slightly easier corollary.
\begin{theorem}[Ax--Lindermann--Weierstrass]
    Let $V\subseteq\left(\CC^\times\right)^n$ be an algebraic subvariety. Then any maximal algebraic subvariety $W\subseteq\pi^{-1}(V)$ is bialgebraic.
\end{theorem}
Here is our theorem.
\begin{theorem}[weak Ax--Schanuel]
	Fix algebraic subvarities $V_1\subseteq\CC^n$ and $V_2\subseteq(\CC^\times)^n$ such that the analytic component $U$ of the intersection $V_1\cap V_2$ admits unexpected codimension, meaning
	\[\codim U<\codim V_1+\codim V_2.\]
	Then $U$ is contained in a proper bialgebraic subvariety.
\end{theorem}
Intuitively, we are saying that having large intersection is explained by having large linear subspaces.

We will be interested in applications to Shimura varieties. These are some fancy quotients $\Gamma\backslash\Omega$ where $\Omega$ is some complex manifold, and $\Gamma\subseteq\Omega$ is a discrete subgroup. Having such a quotient gives us some uniformization map $\pi\colon\Omega\onto Y$.
\begin{example}[Modular curve]
	The group $\mathrm{SL}_2(\ZZ)$ acts on the upper-half plane $\HH$, and the quotient is named $Y(1)$.
\end{example}
\begin{example}[PEL type]
	The symplectic group $\mathrm{Sp}_{2g}(\ZZ)$ acts on
	\[\HH_g\coloneqq\{M\in M_g(\CC):Z^\intercal=Z\text{ and }\im Z>0\}\]
	via similar fractional linear transformations
	\[\begin{bmatrix} A & B \\ C & D \end{bmatrix}\cdot Z\coloneqq(AZ+B)(CZ+D)^{-1}.\]
	The quotient is $\mc A_g$, which turns out to be a moduli space of principally polarized $g$-dimensional abelian varieties.
\end{example}
Here is our version of the Ax--Schanuel theorem.
\begin{theorem}
	Let $\pi\colon\Omega\to Y$ be the uniformization map of a Shimura variety $Y$. Fix algebraic subvarieties $V_1\subseteq\Omega$ and $V_2\subseteq Y$ such that the analytic component $U$ of the intersection $V_1\cap V_2$ admits unexpected codimension, meaning
	\[\codim_\Omega U<\codim_\Omega V_1+\codim_YV_2.\]
	Then $U$ is contained in a proper weakly special subvariety of $\Omega$.
\end{theorem}
Here, weakly special means a translate of a special subvareity, where a special subvariety is roughly speaking one which is itself a Shimura variety.
\begin{example}
	The special subvarieties of $Y(1)^n$ are given by $Y(1)^m\times\{0\}$ up to rearranging the coordinates.
\end{example}

\section{September 30th: Homological Sieve and Manin's Conjecture}
This talk was given by Sho Tanimoto. 

\subsection{Manin's Conjecture}
Define height functions $H\colon\PP^n(\QQ)\to\RR_{>0}$ in the usual way; for a projective variety, one gets a height function
\begin{definition}[Fano]
	A smooth projetive variety $X$ is \textit{Fano} if and only if $-K_X$ is an ample divisor. A two-dimensional Fano variety is a \textit{del Pezzo} variety.
\end{definition}
\begin{example}
	Cubic hypersurfaces in $\PP^n$ for $n\ge3$ are Fano.
\end{example}
\begin{example}
	Given a del Pezzo surface $S$ over an algebraically closed field, set $e\coloneqq(-K_S)^2$. If $e=9$, then $S\cong\PP^2$; if $e=3$, then $S$ is a smooth cubic surface in $\PP^3$.
\end{example}
\begin{conj}[Manin]
	Fix a smooth variety $X$ over a number field $k$ such that $X(k)$ is not then. Then there is a thin set $Z\subseteq X(k)$ such that
	\[N(X(k)\setminus Z,-K_X,T)\sim cT(\log T)^{\rho-1},\]
	where $c$ is some explicit constant and $\rho$ is the Picard rank.
\end{conj}
Here, a thin subset is a finite union of images of maps $f_i\colon Y_i\to X$ which are generically finite but not birational. For example, we are allowed to remove some affie spaces.
\begin{remark}
	There is a lot of history here. For example, del Pezzo surfaces in large degree are toric, so the conjecture is known. In smaller degrees, some is known but nothing in degree smaller than $4$.
\end{remark}
We are interested in working over global function fields. By valuative criteria, we find that we are basically interested in studying the space of morphisms $\PP^1\to X$, which can in turn by classified by nef $\alpha$.

\section{October 1st: \texorpdfstring{$L$}{ L}-Functions and the Homological Theta Lift}
This talk was given by Jialiang Zou at MIT for the Lie groups seminar.

\subsection{Basics on Theta Lifting}
Fix a $p$-adic field $F$. We set $G_m\coloneqq\op{GL}_m(F)$ and $H_n\coloneqq\op{GL}_n(F)$, and we let $S(M_{mn})$ be the Schwarz functions on the matrices. We now let $\omega_{mn}$, frequentiyl denoted $\omega$, be the representation of $G_m\times H_n$ acting on $S(M_{mn})$ in the usual way, with some twists by characters to make the representation unitary. For example, Tate's thesis takes $m=n=1$.

Let $\pi$ be an irreducible representation of $H_n$, and we let $\omega[\pi]$ be the maximal $\pi$-isotypic quotient of $\omega$, which lets us write
\[\omega[\pi]=\pi\boxtimes\Theta(\pi),\]
where $\Theta(\pi)$ is now some representation of $G_m$, which is known as the big theta lift of $\pi$. This construction is almost functorial: there is a functor $\Theta^\lor\colon\op{Rep}(H_n)\to\op{Rep}(G_m)$, which is also $\op{Hom}_{H_n}(\omega,\pi)$ (where we are silently taking the smooth vectors). Here is some of what is known about this theta lift.
\begin{theorem}[Minguez]
	If $\Theta(\pi)\ne0$, then it admits a unique irreducible quotient $\theta(\pi)$, and
	\[\dim_\CC\op{Hom}_{G_m\times H_n}(\omega,\pi\boxtimes\theta(\pi))=1.\]
	Furthermore, when $m\ge n$, if $\pi$ is a Langlands quotient of some product $\tau_1\times\cdots\times\tau_r$ of essentially discrete series representations, then $\theta(\pi)$ is some explicit Langlands quotient of the contragradients $\tau_\bullet^\lor$.
\end{theorem}
\begin{example}
	If $m=n$, then $\theta(\pi)=\pi^\lor$.
\end{example}

\subsection{Two Approaches}
We are interested in the structure of $\Theta$, which we continue to think about as $\op{Hom}_{H_n}(\omega,-)$. Algebraically or homologically, we have the following.
\begin{theorem}[Adams--Prasad--Savin]
	For all $i\ge0$, the representation $\op{Ext}^i_{H_n}(\omega,\pi)$ is of finite length, and it vanishes for $i\ge\op{rank}H_n$. Furthermore, the virtual ``Euler characteristic'' representation
	\[\sum_i(-1)^i\mathrm{Ext}^i_{H_n}(\omega,\pi)\]
	is $0$ if $m<n$ and is $\pi\times1_{m-n}$ if $m\ge n$.
\end{theorem}
\begin{remark}
	One can stratify $M_{mn}$ by the rank of the matrices, building subrepresentations $J_k$ of the rank-$k$ matrices. It turns out that the Euler characteristic is additive for this stratification.
\end{remark}
\begin{remark}
	When $m\ge n$, one expects that $\op{Ext}^i_{H_n}(\omega,\pi)=0$ for all $i\ge1$, which will let us compute $\Theta(\pi)$ as the dual of the Euler characteristic.
\end{remark}
Let's describe an analytic approach due to Fang--Sun--Xue. Working in the case $m=n$, the stratification has an open dense subrepresentation $\omega^\circ\coloneqq S(\op{GL}_n(F))$, also known as $\pi\boxtimes\pi^\lor$ where $\pi$ is the regular representation. Then we have a short exact sequence
\[0\to\omega^\circ\to\omega\to\omega/\omega^\circ\to0,\]
and the long exact sequence produces a map
\[\op{Hom}_{G_n\times H_m}(\omega,\pi\boxtimes\pi^\lor)\to\op{Hom}_{G_n\times H_m}(\omega^\circ,\pi\boxtimes\pi^\lor),\]
and the problem turns into wanting to lift some eigendistributions on $\omega^\circ$ to $\omega$.
\begin{idea}
	Obstructions to lifting is given by poles of $L$-functions.
\end{idea}
In particular, we have a Godemont--Jacquet $L$-function defined by
\[Z(s,\varphi,f)\coloneqq\int_{\op{GL}_n(F)}\varphi(g)f(g)\left|\deg g\right|^{s+(n-1)/2}\,dg,\]
where $\varphi$ is a Schwartz function and $f$ is some matrix coefficient. The greatest common divisor of these $L$-functions is $L(s,\pi)$, so its poles can be seen to measure $Z$'s ability to extend.
\begin{theorem}
	If $L(s,\pi)$ is holomorphic at $s=1/2$, then
	\[\op{Hom}_{G_n\times H_n}(\omega/\omega^\circ,\pi^\lor\boxtimes\pi)=0\]
	and
	\[\op{Hom}_{H_n}(\omega/\omega^\circ,\pi)=0.\]
\end{theorem}
This more or less follows from the exact sequence and Howe duality. The point is that the zeta integral provides an explicit lift.
\begin{corollary}
	If $L(s,\pi)$ or $L(s,\pi^\lor)$ is holomorphic at $s=1/2$, then
	\[\Theta(\pi)=\pi^\lor.\]
\end{corollary}
\begin{proof}[Idea]
	Let's argue in the case $L(s,\pi)$. The point is that the map
	\[\op{Hom}_{H_n}(\omega,\pi)\to\op{Hom}_{H_n}(\omega^\circ,\pi)\]
	is an isomorphism, and the second one is $\pi^\lor$.
\end{proof}

\subsection{Relating the Approaches}
The moral is that having $\op{Ext}^i_{H_n}(\omega,\pi)=0$ for $i\ge1$ or $L(s,\pi)$ holomorphic at $s=1/2$ will imply $\Theta(\pi)=\pi^\lor$. One may hope that there is some relation between the $L$-functions and the homological lifting. This motivates the following result.
\begin{theorem}
	Assume $m\ge n$ and $\pi$ is an irreducible representation of $H_n$. If either $L(s,\pi)$ or $L(s,\pi^\lor)$ are holomorphic at $s=\frac{1+m-n}2$, then $\op{Ext}^i_{H_n}(\omega,\pi)=0$ for $i\ge1$, and
	\[\Theta(\pi)=\begin{cases}
		1_{m-n}\times\pi^\lor & \text{if }L(s,\pi)\text{ is holomorphic at }s=(1+m-n)/2, \\
		\pi^\lor\times1_{m-n} & \text{if }L(s,\pi^\lor)\text{ is holomorphic at }s=(1+m-n)/2.
	\end{cases}\]
\end{theorem}
Both claims are proved by a more or less explicit calculation. For example, one can write $\pi$ as a Langlands quotient and then just calculate. Then $L(s,\pi)$ being holomorphic at a given point can be seen on the level of this decomposition, and $\op{Ext}$ vanishing can be seen via a calculation with the long exact sequence, comparing $\pi$ with a Jacquet lift.

\section{October 2: The Lefschetz Trace Formula}
This talk was given by Arav Karighattam for the STAGE seminar at MIT.

\subsection{The Tools}
We are going to use $\ell$-adic cohomology to prove all the Riemann conjectures, with the expception of the Riemann hypothesis. We willl recall the statements as we get to them.

Let's recall our Chow groups.
\begin{definition}[Chow group]
	Fix a variety $X$ over a field $k$ of equidimension $d$. Then we define the \textit{Chow group} $\op{CH}^\bullet(X)$ as the graded ring, where $\op{CH}^i(X)$ contains the codimension-$i$ cycles (up to rational equivalence). The product $[A]\cdot[B]$ is given by the intersection $[A\cup B]$, which makes sense when $A$ and $B$ are generically transverse, meaning that a generic point $x$ in $A\cap B$ has $T_xA+T_xB=T_xX$.
\end{definition}
As usual, we will not bother to show that this product makes sense, which requires some notion of the Moving lemma or a different approach.

Our main tool will be $\ell$-adic cohomology.
\begin{definition}[$\ell$-adic cohomology]
	Fix a variety $X$ over a field $k$. For a prime $\ell$ distinct from $\op{char}k$, we define \textit{$\ell$-adic cohomology} as
	\[\mathrm H^i_\ell(X)\coloneqq\left(\lim\mathrm H^i(X_{k^{\mathrm{sep}}};\underline{\ZZ/\ell^\bullet\ZZ})\right)\otimes_\ZZ\QQ.\]
	This cohomology groups assemble into a graded commutative ring $\mathrm H^\bullet_\ell(X)$, where the product is given by the cup product.
\end{definition}
\begin{remark}
	As usual, the cup product can be defined on the level of \v{C}ech cocycles.
\end{remark}
It turns out that $\mathrm H^\bullet_\ell$ assembles into a Weil cohomology theory with coefficients in $\QQ_\ell$. Let's quickly review what we are given.
\begin{itemize}
	\item The cohomology groups $\mathrm H^\bullet_\ell$ are supported in degrees $[0,2\dim X]$.
	\item There is a K\"unneth formula
	\[\mathrm H^\bullet_\ell(X\times Y)\cong\mathrm H^\bullet_\ell(X)\otimes\mathrm H^\bullet_\ell(Y)\]
	induced by the projections.
	\item If $X$ has equidimension $d$, then the cup product produces a perfect pairing
	\[\mathrm H^i_\ell(X)\times\mathrm H^i_\ell(X)(d)\to\mathrm H^{2d}_\ell(X)(d)\to\QQ_\ell,\]
	where the last map is a trace map.
	\item There is a cycle class map $\op{cl}_X\colon\op{CH}^i(X)\to\mathrm H^{2i}(X)(i)$.
\end{itemize}
These data are subject to many compatibilities.

\subsection{The Lefschetz Trace Formula}
Here is our theorem.
\begin{theorem} \label{thm:lefschetz}
	Fix a regular endomorphism $\varphi\colon X\to X$ of a smooth projective variety $X$ of equidimension $d$ over a field $k$. Then
	\[(\Gamma_\varphi\cdot\Delta_X)=\sum_{r=0}^{2d}(-1)^r\tr\left(\varphi^*;\mathrm H^r_\ell(X)\right).\]
	Here, $\Gamma_\varphi$ is the graph of $\varphi$, $\Delta_X$ is the diagonal, so $(\Gamma_\varphi\cdot\Delta_X)$ should be thought of as the number of fixed points of $\varphi$ (counted with the correct multiplicities).
\end{theorem}
\begin{proof}
	This is purely formal from the construction of a Weil cohomology theory. It turns out that the intersection number $(\Gamma_\varphi\cdot\Delta_X)$ agrees with the scalar
	\[\op{cl}_{X\times X}(\Gamma_\varphi\cdot\Delta)\in\mathrm H^{4d}(X\times X)(2d),\]
	where the target is identified with $\QQ_\ell$ via Poincar\'e duality. By a coherence property, we see that we want to evaluate $\op{cl}_{X\times X}(\Gamma_\varphi)\cup\op{cl}_{X\times X}(\Delta)$.

	Let's explain how to compute $\op{cl}_{X\times X}(\Gamma_\varphi)$, and then one can compute $\op{cl}_{X\times X}(\Delta)$ by setting $\varphi=\id_X$. Well, for each degree $r$, fix a basis $\{e_{i,r}\}$ of $\mathrm H^r_\ell(X)$, which then has a dual basis $\{f_{i,2d-r}\}$ of $\mathrm H^{2d-r}_\ell(X)(d)$. We will take $e_{i,r}\cup f_{2d-i,r}=1$ as our sign convention. The K\"unneth formula explains that $\mathrm H^\bullet(X\times X)$ can be identified with $\mathrm H^\bullet(X)\otimes\mathrm H^\bullet(X)$, so we get to write
	\[\op{cl}_{X\times X}(\Gamma_\varphi)=\sum_{i,r}a_{i,r}\boxtimes f_{i,2d-r}\]
	for some coefficients $a_{i,r}$ which we would like to solve for. To do so, we note that
	\[\op{cl}_{X\times X}(\Gamma_\varphi)\cup(1\boxtimes e_{j,r})=a_{j,r}\boxtimes e_{2d},\]
	where there graded commutative signs cancel out after expanding out $\boxtimes$ as a cup product. Thus,
	\[\op{pr}_{1*}\left(\op{cl}_{X\times X}(\Gamma_\varphi)\cup(1\boxtimes e_{j,r})\right)=a_{j,r}.\]
	We can now collapse the left-hand side. Note $\Gamma_\varphi=({\id_X},\varphi)_*1_X$, so we can rewrite this as
	\[a_{j,r}=\op{pr}_{1*}\left(({\id_X},\varphi)_*1_X\cup\op{pr}_2^*e_{j,r}\right).\]
	By the projection formula, this collapses to $\varphi^*e_{j,r}$, so
	\[\op{cl}_{X\times X}(\Gamma_\varphi)=\sum_{i,r}\varphi^*e_{j,r}\boxtimes f_{i,2d-r}.\]
	Plugging in $\varphi=\id_X$, we see similarly that
	\[\op{cl}_{X\times X}(\Delta_X)=\sum_{i,r}e_{j,r}\boxtimes f_{i,2d-r}.\]
	This is also
	\[\op{cl}_{X\times X}(\Delta_X)=\sum_{i,r}(-1)^rf_{i,2d-r}\boxtimes e_{j,r},\]
	so $\op{cl}_{X\times X}(\Gamma_\varphi)\cup\op{cl}_{X\times X}(\Delta)$ is
	\[\sum_{j,r}(-1)^r\left(\varphi^*e_{j,r}\cup f_{j,2d-r}\right)\boxtimes e_{2d}\]
	even after keeping track of signs in the graded commutativity. Now, the sum over $j$ of the piece in parantheses is exactly the trace of $\varphi^*$ acting on a given basis of $\mathrm H^i_\ell(X)$, so we conclude.
\end{proof}

\subsection{Some Weil Conjectures}
Here is the main input to our proofs.
\begin{proposition} \label{prop:frob-by-lefschetz}
	Fix a smooth projective variety $X$ of equidimension $d$ over $\FF_q$. Then
	\[\left|X(\FF_q)\right|=\sum_{r=0}^{2d}(-1)^r\tr\left(\mathrm{Frob}_q^*;\mathrm H^r_\ell(X)\right).\]
\end{proposition}
\begin{proof}
	Let $\varphi$ be the Frobenius. By \Cref{thm:lefschetz}, we only need to show that $(\Gamma_\varphi\cdot\Delta_X)$ is in fact $\left|X(\FF_q)\right|$. Certainly the fixed points of the Frobenius acting on $X(\ov\FF_q)$ is precisely $X(\FF_q)$, so it remains to see that our intersection is actually transverse. This is true because all tangent spaces of $\Gamma_\varphi$ are horizontal (the derivative of $x^q$ vanishes in $\FF_q$) while tangent spaces of $\Delta_X$ are diagonal.
\end{proof}
\begin{corollary}
	Fix a smooth projective variety $X$ of equidimension $d$ over $\FF_q$. Then
	\[Z(X,T)=\exp\Bigg(\sum_{n=1}^\infty X(\FF_{q^n})\frac{T^n}n\Bigg)=\prod_{r=0}^{2d}\det\left(1-\mathrm{Frob}_q^*T;\mathrm H^i(X)\right)^{(-1)^{i+1}}.\]
\end{corollary}
\begin{proof}
	We start with a linear algebraic fact. In general, if $\varphi\colon V\to V$ is a linear opeator over a field $k$, then we claim
	\[-\log\det(1-\varphi T;V)\stackrel?=\sum_{n=1}^\infty\tr(\varphi^n;V)\frac{T^n}n.\]
	Combining this linear algebraic fact with \Cref{prop:frob-by-lefschetz} completes the argument.

	To see the claim, note that both sides of the identities are additive in the pair $(V,\varphi)$, and the case of scalars $c$ acting on a one-dimensional space amounts to the Taylor expansion of $\log$ as $-\log(1-\lambda T)=\sum_{n\ge1}\lambda^nT^n/n$. To complete the proof, we note that the identity is insensitive to changing the base field, so we may base-change to the algebraic closure, diagonalize $\varphi$, and reduce to the case of scalars.
\end{proof}
We will now have achieved rationality as soon as we can show that the polynomials
\[P_i(T)\coloneqq\det\left(1-\mathrm{Frob}_q^*T;\mathrm H^i(X)\right)\]
live in $1+T\ZZ[T]$. Certainly it is in $1+T\QQ_\ell[T]$, and the alternating product of these polynomials is $Z$ and therefore is in $1+T\ZZ[[T]]$, so one can make an argument that the $P_i$s must be in $1+T\ZZ[T]$.

It remains to prove the functional equation. This follows from Poincar\'e duality. Indeed, $\mathrm{Frob}_*\mathrm{Frob}^*=q^d$ on cohomology, so
\begin{align*}
	Z\left(X,q^{-d}T^{-1}\right) &= \prod_{r=0}^{2d}\det\left(1-q^{-d}\mathrm{Frob}^*T^{-1};\mathrm H^i(X)\right)^{(-1)^{r+1}} \\
	&= \prod_{r=0}^{2d}\det\left(1-q^{-d}\mathrm{Frob}_*T^{-1};\mathrm H^{2d-i}(X)(d)\right)^{(-1)^{r+1}}.
\end{align*}
This can be unwound into the functional equation.

\section{October 6: The Visibility of Elements of the Tate--Shafaverich Group}
This talk was given by Jerson Caro for the Boston Unviersity number theory seminar. It is joint work with Barinder Banwait and Shiva Chidambaram.

\subsection{Visualization}
Fix an elliptic curve $E$ over $\QQ$, and choose an integer $n\ge2$. Taking Galois cohomology of the Kummer exact sequence
\[0\to E[n]\to E\stackrel n\to E\to0\]
produces the short exact sequence
\[0\to\frac{E(\QQ)}{nE(\QQ)}\to\mathrm H^1(\QQ;E[n])\to\mathrm H^1(\QQ;E)[n]\to0.\]
Comparing with the localizations, we produce a short exact sequence
\[0\to\frac{E(\QQ)}{nE(\QQ)}\to\op{Sel}_n(E/\QQ)\to\Sha(E/\QQ)[n]\to0,\]
where $\op{Sel}_n(E/\QQ)$ is the kernel of the map $\mathrm H^1(\QQ;E[n])\to\prod_v\mathrm H^1(\QQ_v;E)[n]$, and $\Sha(E/\QQ)$ is the kernel of the map $\mathrm H^1(\QQ;E)\to\prod_v\mathrm H^1(\QQ_v;E)$.

One can give $\mathrm H^1(\QQ;E)$ a geometric realization as torsors.
\begin{definition}[torsor]
	A \textit{torsor} of $E$ is a genus $1$ curve $C$ equipped with a simply transitive action of $E$. An isomorphism of torsors is an isomorphism of the underlying curves commuting with the action of $E$.
\end{definition}
\begin{remark}
	It turns out that torsors are parameterized by $\mathrm H^1(\QQ;E)$. In short, a torsor $C$ is a twist of $E$: any geometric point $p_0\in C(\ov\QQ)$ defines an isomorphism $E\to C$ given by $P\mapsto P+p_0$. Twists are parameterized by $\mathrm H^1(\QQ;\op{Aut}E)$, so it is a matter of checking that the torsors correspond to $1$-cocycles of $E\subseteq\op{Aut}E$.
\end{remark}
For example, we see that $\Sha(E/\QQ)$ contains ``everywhere locally trivial'' torsors.

By considering the addition structure on torsors, one can show the following.
\begin{theorem}[Cassels]
	If $C$ is an $E$-torsor representing an element of $\Sha(E/\QQ)[n]$, then $C$ admits a divisor over $\QQ$ of degree $n$.
\end{theorem}
Here is another way to visualize, due to Mazur: fix an embedding $\varphi\colon E\to A$, where $A$ is an abelian variety. Given a torsor $C$, if the corresponding class $\xi$ vanishes under the pushforward $\varphi_*\colon\mathrm H^1(E;\QQ)\to\mathrm H^1(A;\QQ)$, then $C$ embeds into $A$. Indeed, the short exact sequence
\[0\to E\to A\to B\to0\]
of group schemes produces the exact sequence
\[A(\QQ)\to B(\QQ)\to\mathrm H^1(E;\QQ)\to\mathrm H^1(\QQ;A).\]
We are given that $\xi$ vanishes in $\mathrm H^1(\QQ:A)$, so it comes from some $\beta\in B(\QQ)$, and one can check that the fiber in $A$ is isomorphic to $C$. This motivates the following definition.
\begin{definition}
	Fix an abelian variety $A$ containing $E$. Then we define the \textit{visibility} as
	\[\op{Vis}_A\mathrm H^1(\QQ;E)\coloneqq\ker\left(\mathrm H^1(\QQ;E)\to\mathrm H^1(\QQ;A)\right).\]
	Similarly, we define $\op{Vis}_A\Sha(E/\QQ)$ to be $\op{Vis}_A\mathrm H^1(\QQ;E)\cap\Sha(E/\QQ)$.
\end{definition}
\begin{remark}
	Note $\op{Vis}_A\mathrm H^1(\QQ;E)$ is finite because it fits into the short exact sequence
	\[0\to\frac{A(\QQ)}{E(\QQ)}\to B(\QQ)\to\op{Vis}_A\mathrm H^1(\QQ;E)\to0,\]
	and one can compare ranks of the two groups on the left.
\end{remark}
It turns out that every element $\xi$ of $\Sha(E/\QQ)$ can be visualized. Here are two constructions.
\begin{itemize}
	\item Let $C$ be the corresponding torsor, and choose a field $K$ with $C(K)\ne\emp$. (One can show that $K$ should take degree $n$ when $\xi$ has order $n$.) Then $A=\op{Res}_{K/\QQ}E$ will work.
	\item It turns out that $\xi\in\Sha(E/\QQ)[n]$ produces an Azumaya algebra $\mc A$ of rank $n^2$ over $\QQ(E)$. Upon choosing some field $L\subseteq\mc A$ of degree $n$ over $\QQ(E)$, one can take $A$ to be the Jacobian of the curve corresponding to $L$.
\end{itemize}

\subsection{Minimal Visualization}
We are interested in the ``smallest'' abelian variety visualizing an element $\xi\in\Sha(E/\QQ)$.
\begin{definition}[visibility dimension]
	For $\xi\in\Sha(E/\QQ)$, we define the \textit{visibility dimension} as the minimal dimension of an abelian variety visualizing $\xi$.
\end{definition}
Here is some of what is known.
\begin{itemize}
	\item Mazur showed that elements in $\Sha(E/\QQ)[3]$ have visibility dimension $2$, and Bruin--Dahmen gave explicit constructions by Jacobians of genus $2$ curves.
	\item Kenke showed that elements in $\Sha(E/\QQ)[2]$ have visibility dimension $2$, and Bruin gave an explicit construction by Jacobians of genus $2$ curves.
	\item Fisher gave examples of elements in $\Sha(E/\QQ)[6]$ and $\Sha(E/\QQ)[7]$ that are not visualized in abelian surfaces. He also showed that elements of $\Sha(E/\QQ)[4]$ cannot be visualized in principally polarized abelian varieties, such as Jacobians.
\end{itemize}
For abelian surfaces, here is one way to make progress.
\begin{definition}[congruent]
	Two elliptic curves $E$ and $F$ are \textit{$n$-congruent} if and only if there is an isomorphism $E[n]\cong F[n]$ of Galois modules.
\end{definition}
\begin{proposition}
	Given $\xi\in\Sha(E/\QQ)[n]$, if there is an abelian surface $A$ visualizing $\xi$, then there is a curve $F$ which is $n$-congruent to $E$ for which
	\[A=\frac{E\times F}\Delta,\]
	where $\Delta$ is the diagonal embedding of the $n$-torsion.
\end{proposition}
The idea is that visualization provides a short exact sequence
\[0\to E\to A\to F\to0\]
of abelian varieties, so there is an isogeny $E\times F\to A$. It is then a matter of studying the kernel.

Another way to make sense of minimality is categorically.
\begin{definition}[visibility category]
	We define the \textit{$m$-visibility category} $\mc V(E/\QQ,m)$ whose objects are abelian varieties visualizing every element in $\Sha(E/\QQ)[m]$, and the morphisms are given by inclusions (commuting with the embedding from the torsor).
\end{definition}
\begin{remark}
	It turns out that minimal objects in $\mc V(E/\QQ,m)$ can have different dimensions; examples have been found as early as $m=5$.
\end{remark}
The main result is that $n\in\{2,3\}$ have canonical minimal elements.

\section{October 7: Near Coincidences and Nilpotent Division Fields}
This talk was given by Harris Daniels for the MIT number theory seminar.

\subsection{Constructibility}
Our story begins historically.
\begin{theorem}[Gauss--Wantzel]
	A regular $n$-gon is constructible if and only if $\varphi(n)$ is a power of $2$.
\end{theorem}
In other words, $n$ is a power of $2$ times a semiprime product of Fermat primes (which are one more than a power of $2$). The key input is the following.
\begin{theorem}
	A complex number $\alpha\in\CC$ is constructible if and only if $\alpha$ belongs to a number field $K$ for which $\op{Gal}(K/\QQ)$ is a $2$-group.
\end{theorem}
Thus, we are asking when $\QQ(\zeta_n)$ has Galois group which is a $2$-group.

Here is our first main result.
\begin{theorem} \label{thm:elliptic-constructible}
	Let $E$ be the elliptic curve $y^2=x^3-x$ and $n\ge2$. Then $\QQ(E[n])$ is constructible if and only if $\varphi(n)$ is a power of $2$.
\end{theorem}
\begin{proof}
	We show the directions independently.
	\begin{itemize}
		\item If $\QQ(E[n])$ is constructible, then the Weil pairing forces $\QQ(\zeta_n)\subseteq\QQ(E[n])$, which then forces $\zeta_n$ to be constructible.
		\item If $\varphi(n)$ is constructible, then $n$ is the product of a power of $2$ and some Fermat primes, so we can do casework. In this case, one can explicitly show that $\QQ\left(E[2^{\nu+1}]\right)/\QQ(E[2^\nu])$ is a power of $2$. Additionally, when $p$ is a Fermat prime, one finds that the image of
		\[\op{Gal}(\QQ(E[p],i)/\QQ(i))\to\op{GL}_2(\FF_p)\]
		is diagonal by the theory of complex multiplication, and this diagonal subgroup has size equal to a power of $2$ because $p$ is a Fermat prime.
		\qedhere
	\end{itemize}
\end{proof}
Our motivation comes from the following result.
\begin{theorem}
	Fix an elliptic curve $E$. If $\QQ(E[n])/\QQ$ is abelian, then $n\le6$.
\end{theorem}
We are interested in generalizing such a result where abelian is replaced by nilpotent. One way to access this condition is that nilpotent groups are the products of their Sylow subgroups, which is an instance of \Cref{thm:elliptic-constructible}.

\subsection{Nilpotency}
We are going to rely on the following ``standard'' conjecture.
\begin{conj}
	If $p>11$, then one has a reasonable control over rational points on modular curves of the non-split Cartan subgroup.
\end{conj}
\begin{theorem}
	If $E$ does not have complex multiplication with $\QQ(E[n]/\QQ)$ is nilpotent, then either $n<25$ or is a power of $2$.
\end{theorem}
In the base case $E[p]$, one appeals to some classification results about subgroups of $\op{PGL}_2$.

In the $E[p^2]$ case, one finds that $\QQ\left(E[p^2]\right)=\QQ(E[p],\zeta_{p^2})$, which is not expected to happen. Here is the sort of result one knows about these things.
\begin{theorem}
	Fix an elliptic curve $E$ over $\QQ$, and let $p$ be a prime and $n\in\ZZ^+$. If $\QQ\left(E[p^{n+1}]\right)=\QQ(E[p^n])$, then $(p,n)=(2,1)$.
\end{theorem}
One shows this basically by comparing when $p$-power roots of unity can be found in $E[p^\bullet]$. Of course, in our application, we should be allowed to add roots of unity.
\begin{theorem}
	If $\QQ\left(E[p^{n+1}]\right)=\QQ(E[p^n],\zeta_{p^{n+1}})$, then $p\in\{2,3\}$ and $n=1$.
\end{theorem}
This seems to also come down to some group theory.

\section{October 8: A new approach to Hecke Correspondences}
This talk was given by Keerthi Madapusi for the Harvard number theory seminar.

\subsection{Rapoport--Zink Spaces}
For today, $G$ is a reductive group over $\ZZ_p$, and $\mu\colon\mathbb G_{m,\OO}\to G_\OO$ is miniscule cocharacter defined over some unramified extension $\OO$ of $\ZZ_p$. We also let $\check\ZZ_p$ be the maximal unramified ring of integers over $\ZZ_p$, whose fraction field is $\check\QQ_p$.
\begin{definition}[Deligne--Lusztig variety]
	Fix some $b\in G(\check\QQ_p)$. Then we define the \textit{affine Deligne--Lusztig variety} as
	\[X_\mu b(\overline\FF_p)\coloneqq\left\{g\in G(\check\QQ_p)/G(\check\ZZ_p):g^{-1}b\sigma(g)\in G(\check\ZZ_p)\mu(p)G(\check\ZZ_p)\right\}.\]
\end{definition}
Here is our main theorem.
\begin{theorem}
	There is a formally smooth formal scheme $RZ^{G,\mu}_b$ over $\check\ZZ_p$ such that
	\[RZ^{G,\mu}_b(\overline\FF_p)\cong X_{\sigma(\mu)}b(\overline\FF_p).\]
\end{theorem}
\begin{remark}
	This is special to the reductive case: if $G$ fails to be reductive, formal smoothness can fail.
\end{remark}
\begin{remark}
	There are global applications to integral models of Shimura varieties.
\end{remark}
The $RZ$ stands for Rapoport--Zink. Accordingly, let's give some historical remarks.
\begin{example}
	With $G=\op{GL}_n$, one can let $\mu_d$ be the cocharacter
	\[z\mapsto\op{diag}(\underbrace{z,\ldots,z}_d,\underbrace{1,\ldots,1}_{n-d}).\]
	It turns out that $X_\mu b(\overline\FF_p)\ne\emp$ if and only if there is a $p$-divisible group $\mc H$ over $\overline\FF_p$ of dimension $d$ and height $n$ such that the Dieudonn\'e module $\mathbb D(\mc H)[1/p]$---which is some $n$-dimensional vector space over $\check\QQ_p$ equipped with a $\sigma$-linear automorphism $F$---has $F=b\sigma$. When nonempty, we define $RZ_r^{G,\mu}$ to be the formal moduli space of pairs $(\mc G,\xi)$ where $\mc G$ is a $p$-divisible group of height $n$, and $\xi\colon\mc G\to\mc H$ is a quasi-isogeny.
\end{example}
\begin{remark}
	Rapoport and Zink show that $RZ_r^{G,\mu}$ is a formal scheme. (Note that it is much easier to show that it is an ind-formal scheme.) One can use the above case to handle $G$ of ``EL type,'' meaning that $(G,\mu)$ embeds into some $({\op{GL}_n},\mu_d)$, and $G$ is cut out by some endomorphisms.
\end{remark}
\begin{remark}
	Wansu Kim and Howard--Pappas handled more general $G$, still requiring some embedding $(G,\mu)\into({\op{GL}_n},\mu_d)$.
\end{remark}
\begin{remark}
	Later, Scholze--Weinstein constructed the diamond associated to $RZ^{G,\mu}_b$ and the shtukas with one leg, which more or less explain as a formal model of our moduli space on perfectoid inputs.
\end{remark}

\subsection{\texorpdfstring{$p$}{p}-Divisible Groups}
The main ``philosophical'' input is a linear algebraic characterization of $p$-divisible groups over rather general bases.
\begin{theorem}
	Fix a $p$-complete ring $R$. Then there is a canonical equivalence and $p$-divisible groups over $R$ and vector bundles over $R^{\mathrm{syn}}$ with Hodge--Tate weights in $\{0,1\}$. Here, $R^{\mathrm{syn}}$ is the ``syntonification'' of $R$.
\end{theorem}
This makes the syntonification appear as a natural construction. Namely, one expects $RZ^{G,\mu}_b$ to be moduli of certain isogenies to given objects which look like these vector bundles on $R^{\mathrm{syn}}$.
\begin{definition}
	We define $\mathrm{BT}^{G,\mu}$ to be the functor taking $R$ to the $G$-bundles on $R^{\mathrm{syn}}$ bounded by $\mu$.
\end{definition}
\begin{theorem}
	The moduli functor $\mathrm{BT}^{G,\mu}$ is a smooth formal Artin stack over $\OO$.
\end{theorem}

\section{October 9: Constructible Sheaves, Base Change, and \texorpdfstring{$L$}{ L}-functions}
This talk was given by Mikayel Mkrtchyan for the STAGE seminar at MIT.

\subsection{Constructible Sheaves}
For today, all schemes will be finite type and separated over a field. Unless otherwise stated, $X_0$ and $\mc F_0$ will be objects defined over a finite field $\FF_q$, and $X$ and $\mc F$ will be their base-changes to the algebraic closure $\ov\FF_q$. This convention also holds for other similar letters. While we're here, we fix our characteristic to be $p>0$, and we choose a prime $\ell\ne p$. We work with the \'etale topology throughout.

Let's begin by stating some foundational results.
\begin{theorem}
	Fix a smooth proper morphism $f\colon X\to Y$ of qcqs schemes. For any $\ell$-adic local system $\mc L$, the pushforward $\mathrm R^if_*\mc L$ is a local system on $Y$.
\end{theorem}
\begin{remark}
	The topological intuition is that a proper submersion $f\colon X\to Y$ of real manifolds is a locally trivial fibration, which is known as Ehresmann's lemma.
\end{remark}
We may be interested in upgrading this, removing properness or smoothness. One no longer expects to get local systems from higher pushforwards: instead, we get constructible sheaves.
\begin{definition}
	Fix a sheaf $\mc F$ on $X$ with finite stalks coprime to $p$. Then $\mc F$ is \textit{constructible} if and only if there is a Zariski locally closed disjoint union $X=\bigsqcup_iX_i$ such that $\mc F|_{X_i}$ is a local system for each $i$.
\end{definition}
\begin{example}
	If $i\colon Z\to X$ is a closed subset, then $i_*\underline{\ZZ/\ell\ZZ}$ is constructible.
\end{example}
\begin{remark}
	We can also choose a stratification $\bigsqcup_iX_i$ to be some \'etale locally closed disjoint union. The point is that one can check if a sheaf is a local system after \'etale base change.
\end{remark}
Here are some indications that we have given a good definition.
\begin{theorem}[finitude]
	Fix a morphism $f\colon X\to Y$ of qcqs schemes. If $\mc F$ is constructible on $X$, then $\mathrm R^if_*\mc F$ is constructible.
\end{theorem}
\begin{theorem}
	Fix an open subset $j\colon U\to X$, and set $i\colon Z\to X$ to be the complement. Then the data of a constructible sheaf on $X$ is equivalent to the data of a triple $(\mc F_Z,\mc F_U,f)$, where $\mc F_Z$ is a constructible sheaf on $Z$, and $\mc F_U$ is a constructible sheaf on $U$, and $f$ is a map $\mc F_Z\to i^*j_*\mc F_U$.
\end{theorem}
\begin{proof}[Sketch]
	It is not hard to build the triple from $\mc F$ by restricting to $Z$ and $U$. Given a triple $(\mc F_Z,\mc F_U,f)$, surely we know what the stalks are, and $f$ tells us how to glue.
\end{proof}
\begin{example}
	Fix a smooth curve $X$ over $\CC$, and choose a finite subset $Z\subseteq X$ of ``cusps,'' and let $U\coloneqq X\setminus Z$ be its kernel. Then a constructible sheaf on $X$ has equivalent data to a local system $\mc L$ on $U$ (which is equivalent to the data of a representation $\pi_1(U)\to\op{GL}(L)$ for some given vector space $L$), a finite group $V_z$ at each $z\in Z$, and the last map $f\colon\mc F_Z\to i^*j_*\mc F_U$ amounts to a map $V_z\to L^{I_z}$. To see this last map, we see that $i^*j_*\mc L$ is the colimit of $\mc L(U\setminus z)$ where $U$ is an open neighborhood of $z$, but these sections turn out to be given by $L^{I_z}$ via the Riemann--Hilbert correspondence. In particular, the maps $V_z\to L^{I_z}$ vanishing corresponds to adding skyscraper sheaves.
\end{example}
\begin{example}
	Similarly, let $X_0$ be a smooth curve over $\FF_q$, let $U_0\subseteq X_0$ be a nonempty open subset, and set $Z_0\coloneqq X_0\setminus U_0$. Then the data of a constructible sheaf on $X_0$ is equivalent to the data of a local system $\mc L$ on $U_0$ (which is the data of a representation of $\pi_1^{\mathrm{\acute et}}(U)$ on some $L$), a representation of $\op{Gal}(\ov{k(z)}/k(z))$ on some $V_z$ for each $z\in Z_0$, and a map $V_z\to L^{I_z}$ for each $z$. Here, $I_z$ is the inertia subgroup which is the kernel of $\pi_1^{\mathrm{\acute et}}(U,z)\to\op{Gal}(\ov{k(z)}/k(z))$.
\end{example}
We are now allowed to make the following central definition.
\begin{definition}[$\ell$-adic sheaf]
	An \textit{$\ell$-adic sheaf} is a compatible system of constructible sheaves of constructible sheaves of $(\ZZ/\ell^\bullet\ZZ)$-modules. Here, the compatibility requires that the locally closed stratifications stabilize for higher powers of $\ell$. We think about this as by taking inverse limits over the compatible systems.
\end{definition}
\begin{remark}
	The stalks of an $\ell$-adic sheaf are $\ZZ_\ell$-modules. We will frequently (and silently) make these $\QQ_\ell$-vector spaces.
\end{remark}

\subsection{Proper Base Change}
Suppose we have a commutative square as follows.
% https://q.uiver.app/#q=WzAsNCxbMSwwLCJYIl0sWzEsMSwiWSJdLFswLDEsIlknIl0sWzAsMCwiWCciXSxbMCwxLCJmIl0sWzMsMCwiZyciXSxbMywyLCJmJyIsMl0sWzIsMSwiZyJdXQ==&macro_url=https%3A%2F%2Fraw.githubusercontent.com%2FdFoiler%2Fnotes%2Fmaster%2Fnir.tex
\[\begin{tikzcd}[cramped]
	{X'} & X \\
	{Y'} & Y
	\arrow["{g'}", from=1-1, to=1-2]
	\arrow["{f'}"', from=1-1, to=2-1]
	\arrow["f", from=1-2, to=2-2]
	\arrow["g", from=2-1, to=2-2]
\end{tikzcd}\]
Given an $\ell$-adic sheaf $\mc F$ on $X$, then there is a base-change morphism
\[g^*f_*\mc F\to f'_*(g')^*\mc F\]
defned by using various adjunctions: note
\begin{align*}
	\op{Hom}(g^*f_*\mc F,f'_*(g')^*\mc F) &= \op{Hom}(f_*\mc F,g_*f'_*(g')^*\mc F) \\
	&= \op{Hom}(f_*\mc F,f_*g'_*(g')^*\mc F) \\
	&\supseteq f_*\op{Hom}(\mc F,g_*(g')^*\mc F),
\end{align*}
and the last set has a canonical adjunction map. Using something about $\delta$-functors, one can upgrade our given map to a base change map
\[g^*\mathrm R^if_*\mc F\to\mathrm R^if'_*(g')^*\mc F.\]
These maps are in general not isomorphisms.
\begin{example}
	Consider the pullback square
	% https://q.uiver.app/#q=WzAsNCxbMCwwLCJcXGVtcCJdLFsxLDAsIlxcbWF0aGJiIEdfbSJdLFsxLDEsIlxcQUFeMSJdLFswLDEsIjAiXSxbMCwxLCJnJyJdLFswLDMsImYnIiwyXSxbMywyLCJnIl0sWzEsMiwiZiJdXQ==&macro_url=https%3A%2F%2Fraw.githubusercontent.com%2FdFoiler%2Fnotes%2Fmaster%2Fnir.tex
	\[\begin{tikzcd}[cramped]
		\emp & {\mathbb G_m} \\
		0 & {\AA^1}
		\arrow["{g'}", from=1-1, to=1-2]
		\arrow["{f'}"', from=1-1, to=2-1]
		\arrow["f", from=1-2, to=2-2]
		\arrow["g", from=2-1, to=2-2]
	\end{tikzcd}\]
	of intersections. Then one can compute that $g^*f_*\underline{\QQ_\ell}=\QQ_\ell$, but $f'_*(g')^*\underline{\QQ_\ell}$ vanishes.
\end{example}
However, in good situations, the result holds.
\begin{theorem}[Proper base change] \label{thm:proper-base-change}
	If the square
	% https://q.uiver.app/#q=WzAsNCxbMSwwLCJYIl0sWzEsMSwiWSJdLFswLDEsIlknIl0sWzAsMCwiWCciXSxbMCwxLCJmIl0sWzMsMCwiZyciXSxbMywyLCJmJyIsMl0sWzIsMSwiZyJdXQ==&macro_url=https%3A%2F%2Fraw.githubusercontent.com%2FdFoiler%2Fnotes%2Fmaster%2Fnir.tex
	\[\begin{tikzcd}[cramped]
		{X'} & X \\
		{Y'} & Y
		\arrow["{g'}", from=1-1, to=1-2]
		\arrow["{f'}"', from=1-1, to=2-1]
		\arrow["f", from=1-2, to=2-2]
		\arrow["g", from=2-1, to=2-2]
	\end{tikzcd}\]
	is a pullback with $f$ proper, then the map $g^*\mathrm R^if_*\mc F\to\mathrm R^if'_*g^*\mc F$ is an isomorphism.
\end{theorem}
One application is that we can make sense of cohomology with compact supports.
\begin{definition}
	Fix a Zariski open embedding $j\colon U\to X$ and a sheaf $\mc F$ on $U$. Then there is a functor $j_!\colon\op{Sh}(U)\to\op{Sh}(X)$ which is left adjoint to $j^*$. It is given by the sheafification of the presheaf sending some \'etale open $V\to X$ to
	\[\begin{cases}
		f(V) & \text{if }V\text{ factors through }U, \\
		0 & \text{else}.
	\end{cases}\]
\end{definition}
\begin{remark}
	One can see that $j_!$ is exact: indeed, its stalks are identity in $U$ and vanish outside $U$.
\end{remark}
\begin{remark}
	For a sheaf $\mc F$ on $X$, we let $j\colon U\to X$ be an open embedding and let $i\colon Z\to X$ be the complement. Then we have a short exact sequence
	\[0\to j_!j^*\mc F\to\mc F\to i_*i^*\mc F\to0.\]
	One can check exactness on stalks.
\end{remark}
\begin{definition}
	Fix a morphism $f\colon X\to Y$ of separated schemes over a field $k$. Then we define the functor $f_!\colon\op{Sh}(X)\to\op{Sh}(Y)$ as follows. By a theorem of Nagata, $f$ factors through some compactification $j\colon X\to\ov X$ where the induced map $\overline f\colon\ov X\to Y$ is proper. Then we define
	\[\mathrm R^if_!\mc F\coloneqq\mathrm R^i\overline f_*(j_!\mc F).\]
	If $X$ is separated over a field $k$, then we can define $\mathrm H^i_c(X;\mc F)$ as $\mathrm R^ip_!\mc F$ where $p\colon X\to\Spec k$ is the structure morphism.
\end{definition}
\begin{remark}
	One can use \Cref{thm:proper-base-change} (and the Leray spectral sequence) to show that this definition is independent of $j$.
\end{remark}
\begin{remark}
	The functor $\mathrm R^if_!$ sends constructible sheaves to constructible sheaves.
\end{remark}
\begin{remark}
	Given any Cartesian square
	% https://q.uiver.app/#q=WzAsNCxbMSwwLCJYIl0sWzEsMSwiWSJdLFswLDEsIlknIl0sWzAsMCwiWCciXSxbMCwxLCJmIl0sWzMsMCwiZyciXSxbMywyLCJmJyIsMl0sWzIsMSwiZyJdXQ==&macro_url=https%3A%2F%2Fraw.githubusercontent.com%2FdFoiler%2Fnotes%2Fmaster%2Fnir.tex
	\[\begin{tikzcd}[cramped]
		{X'} & X \\
		{Y'} & Y
		\arrow["{g'}", from=1-1, to=1-2]
		\arrow["{f'}"', from=1-1, to=2-1]
		\arrow["f", from=1-2, to=2-2]
		\arrow["g", from=2-1, to=2-2]
	\end{tikzcd}\]
	then the base change maps $g^*\mathrm Rf_!\to \mathrm R^if'_!(g')^*$ will always be an isomorphism.
\end{remark}
\begin{remark}
	The functors $\mathrm R^if_!$ are not the derived functors of $f_!$.
\end{remark}

\subsection{\texorpdfstring{$L$}{ L}-Functions}
We now settle into our conventions, where $X_0$ is a variety over $k\coloneqq\FF_q$, and $\mc F_0$ is a sheaf on $X_0$. Then one finds that there are Galois actions on $\mathrm H^i(X;\mc F)$ and $\mathrm H^i_c(X;\mc F)$ as follows: for $\sigma\in\op{Gal}(\ov k/k)$, then we have a morphism $(\sigma\times{\id_{X_0}})\colon X\to X$, so we induce a map
\[\mathrm H^i(X;\mc F)\to\mathrm H^i(X;(\sigma\times{\id_{X_0}})^*\mc F)=\mathrm H^i(X;\mc F),\]
where the last identification holds because $\mc F$ started its life over $k$.
\begin{notation}
	We let $\mathrm{Frob}_{\mathrm{arith}}\in\op{Gal}(\ov k/k)$ be the arithmetic Frobenius $x\mapsto x^{\left|k\right|}$, and we let $\mathrm{Frob}$ be the geometric Frobenius.
\end{notation}
The geometric Frobenius is convenient because it provides the correct morphism on points.

For calculations later, it will be helpful to have $L$-functions of general sheaves.
\begin{definition}
	Fix a constructible $\ell$-adic sheaf $\mc F_0$ on $X_0$ over a finite field $k\coloneqq\FF_q$. Then we define the zeta function
	\[Z(X_0;\mc F_0,T)\coloneqq\prod_{\text{closed }x\in X_0}\frac1{\det\left(1-\mathrm{Frob}T^{\deg x};\mc F_x\right)}.\]
\end{definition}
\begin{remark}
	A short calculation shows that
	\[Z(X_0,\mc F_0;T)=\exp\Bigg(\sum_{m\ge1}\Bigg(\sum_{x\in X_0(\mathbb F_{q^m})}\tr(\mathrm{Frob}_x;\mc F_x)\Bigg)\frac{T^m}m\Bigg).\]
	Here, $\mathrm{Frob}_x\in\op{Gal}(\ov{k(x)}/k(x))$ is the geometric Frobenius, and it acts on the stalk of $\mc F_x$ by viewing the stalk as the pullback to the point.
\end{remark}
\begin{example}
	One can check that $\mc F_0=\underline{\ZZ_\ell}$ recovers the usual zeta function.
\end{example}
One still has a rationality result.
\begin{theorem}[Grothendieck--Lefschetz trace formula]
	Fix a constructible $\ell$-adic sheaf $\mc F_0$ on $X_0$ over a finite field $k\coloneqq\FF_q$. Then
	\[Z(X_0;\mc F_0,T)=\prod_{i=0}^{2\dim X}\det\left(1-\mathrm{Frob}T;\mathrm H^i_c(X;\mc F)\right)^{(-1)^{i+1}}.\]
\end{theorem}
\begin{remark}
	One can view this as a ``global expression'' for the ``locally defined'' Euler product.
\end{remark}
The result becomes more memorable if we pass through the sheaf-function dictionary.
\begin{definition}
	Fix a constructible $\ell$-adic sheaf $\mc F_0$ on $X_0$ over a finite field $k\coloneqq\FF_q$. Then we define the function $G_{\mc F_0}\colon X_0(\FF_{q^m})\to\overline\QQ_\ell$ by
	\[G_{\mc F_0}(x)\coloneqq\tr(\mathrm{Frob}_x;\mc F_x).\]
\end{definition}
\begin{remark}
	Note $G\colon\mathrm{Sh}(X)\to\mathrm{Fun}(X(\FF_{q^m}),\QQ_\ell)$ factors through the Grothendieck group $K_0(\mathrm{Sh}(X))$.
\end{remark}
This gives a relative version of the trace formula.
\begin{theorem}[Grothendieck--Lefschetz trace formula]
	Fix a morphism $f_0\colon X_0\to Y_0$ and a sheaf $\mc F_0$. Then the diagram
	% https://q.uiver.app/#q=WzAsOCxbMCwwLCJLXzAoXFxtYXRocm17U2h9KFhfMCkpIl0sWzAsMSwiS18wKFxcbWF0aHJte1NofShZXzApKSJdLFsxLDAsIlxcbWF0aHJte0Z1bn0oWF8wKFxcRkZfe3FebX0pLFxcUVFfXFxlbGwpIl0sWzEsMSwiXFxtYXRocm17RnVufShZXzAoXFxGRl97cV5tfSksXFxRUV9cXGVsbCkiXSxbMiwwLCJcXG1jIEYiXSxbMiwxLCJcXGRpc3BsYXlzdHlsZVxcc3VtX2koLTEpXmlcXG1hdGhybSBSXmlmXyFcXG1jIEYiXSxbMywwLCJnIl0sWzMsMSwieVxcbWFwc3RvXFxzdW1fe2coeCk9eX1nKHgpIl0sWzAsMSwiXFxtYXRocm0gUmZfISIsMl0sWzIsMywiXFxpbnQiXSxbMCwyLCJHIl0sWzEsMywiRyJdLFs0LDUsIiIsMCx7InN0eWxlIjp7InRhaWwiOnsibmFtZSI6Im1hcHMgdG8ifX19XSxbNiw3LCIiLDAseyJzdHlsZSI6eyJ0YWlsIjp7Im5hbWUiOiJtYXBzIHRvIn19fV0sWzQsNiwiIiwxLHsic3R5bGUiOnsidGFpbCI6eyJuYW1lIjoibWFwcyB0byJ9fX1dLFs1LDcsIiIsMSx7InN0eWxlIjp7InRhaWwiOnsibmFtZSI6Im1hcHMgdG8ifX19XV0=&macro_url=https%3A%2F%2Fraw.githubusercontent.com%2FdFoiler%2Fnotes%2Fmaster%2Fnir.tex
	\[\begin{tikzcd}[cramped]
		{K_0(\mathrm{Sh}(X_0))} & {\mathrm{Fun}(X_0(\FF_{q^m}),\QQ_\ell)} & {\mc F} & g \\
		{K_0(\mathrm{Sh}(Y_0))} & {\mathrm{Fun}(Y_0(\FF_{q^m}),\QQ_\ell)} & {\displaystyle\sum_i(-1)^i\mathrm R^if_!\mc F} & {y\mapsto\sum_{g(x)=y}g(x)}
		\arrow["G", from=1-1, to=1-2]
		\arrow["{\mathrm Rf_!}"', from=1-1, to=2-1]
		\arrow["\int", from=1-2, to=2-2]
		\arrow[maps to, from=1-3, to=1-4]
		\arrow[maps to, from=1-3, to=2-3]
		\arrow[maps to, from=1-4, to=2-4]
		\arrow["G", from=2-1, to=2-2]
		\arrow[maps to, from=2-3, to=2-4]
	\end{tikzcd}\]
	commutes.
\end{theorem}

\section{October 10: The \texorpdfstring{$p$}{ p}-Adic Abel--Jacobi Map}
This talk was given by Yashi for the Automorphic student seminar at Johns Hopkins University.

\subsection{Construction of the Map}
Fix an unramified extension $F$ of $\QQ_p$. Choose a variety $X_r$ over $F$ admitting a smooth model $\mc X_r$ over $\OO_F$. Then the Abel--Jacobi map will be a map
\[\op{AJ}_F^{\mathrm{\acute et}}\colon\op{CH}^{r+1}(X_r)_{0,\QQ}(F)\to\mathrm H^{2r+1}\left(F,\varepsilon_{X_r}\mathrm H_{\mathrm{\acute et}}^1(\overline X_r;\QQ_p(r+1))\right).\]
Let's explain some of this notation. In the application, $X_r$ will be the Kuga--Sato variety $W_r\times A^r$, where $A$ is some elliptic curve with complex multiplication by an order $R$, and $W_r$ is the desingularization of $\mc E^r$, where $\mc E$ is the universal elliptic curve sitting over a curve $C\coloneqq Y_1(N)$. As such $\dim X_r=2r+1$. Then $\varepsilon_{X_r}=\varepsilon_{W_r}\varepsilon_{A_r}$, where $\varepsilon_{W_r}\in\QQ[\op{Aut}W_r]$ and $\varepsilon_{A_r}\in\QQ[\op{Aut}A^r]$. Roughly speaking, they ensure that we land in $\op{Sym}^r$ part of our cohomology. Slightly more precisely, $\varepsilon_{W_r}$ is the product of the averaging operators
\[\frac1{N^r}\sum_{a\in(\ZZ/N\ZZ)^r}t_a\qquad\text{and}\qquad\frac1{2^rr!}\sum_{\theta\in\{\pm1\}^r\ltimes S_r}\op{sgn}(\theta)\theta,\]
and $\varepsilon_{A^r}$ is the same as the right averaging operator. Roughly speaking, we are projecting onto the sign component.

We are going to care about the image of our map on some cycles of the form $\Delta_\varphi=\mc E_{X_r}\gamma_\varphi$, where $\gamma_\varphi$ is the $r$-fold graph of some endomorphism $\varphi$ on a chosen $(A',\varphi)\in C$. Denote this point by $p$. (Sorry!) Let's now define our map. By construction, there is a projection $\pi_r\colon X_r\to C$. Given $p\in C$, we define $X_p\subseteq X_r$ as the fiber and $X_r^b$ as the complement. The Gysin sequence of $X_r=X_p\sqcup X_r^b$ reads as
\[\mathrm H^{2r-1}(\overline X_p;\QQ_p)(r)\to\mathrm H^{2r+1}(\overline X_r;\QQ_p)(r+1)\to\mathrm H^{2r+1}(\overline X_r^b;\QQ_p)\to\mathrm H^{2r}(\overline X_p;\QQ_p)(r)_0\to0\]
after appropriate twisting, and the last term is a kernel. Applying $\varepsilon_{X_r}$ turns this into
\[0\to\varepsilon_{X_r}\mathrm H^{2r+1}(\ov X;\QQ_p)(r+1)\to\varepsilon_{X_r}\mathrm H^{2r+1}(\ov X_r^b;\QQ_p)(r+1)\to\varepsilon_{X_r}\mathrm H^{2r}(\overline X_p;\QQ_p)(r)\to0.\]
Now, we will send our $\Delta_\varphi\in\op{CH}^{r+1}(X_r)_{0,\QQ}(F)$ to the extension $V_\Delta$ fitting into the diagram
% https://q.uiver.app/#q=WzAsMTAsWzAsMCwiMCJdLFswLDEsIjAiXSxbMSwxLCJcXHZhcmVwc2lsb25fWFxcbWF0aHJtIEheezJyKzF9KFxcb3YgWF9yO1xcUVFfcCkocisxKSJdLFsyLDEsIlxcdmFyZXBzaWxvbl9YXFxtYXRocm0gSF57MnIrMX0oXFxvdiBYX3JeYjtcXFFRX3ApKHIrMSkiXSxbMywxLCJcXHZhcmVwc2lsb25fWFxcbWF0aHJtIEheezJyfShcXG92IFhfcDtcXFFRX3ApKHIpIl0sWzQsMSwiMCJdLFs0LDAsIjAiXSxbMywwLCJcXFFRX3AiXSxbMiwwLCJWX1xcRGVsdGEiXSxbMSwwLCJcXHZhcmVwc2lsb25fWFxcbWF0aHJtIEheezJyKzF9KFxcb3YgWF9yO1xcUVFfcCkocisxKSJdLFs4LDddLFs3LDRdLFs3LDZdLFs0LDVdLFszLDRdLFs4LDMsIiIsMSx7InN0eWxlIjp7ImJvZHkiOnsibmFtZSI6ImRhc2hlZCJ9fX1dLFsyLDNdLFsxLDJdLFswLDldLFs5LDhdLFs5LDIsIiIsMSx7ImxldmVsIjoyLCJzdHlsZSI6eyJoZWFkIjp7Im5hbWUiOiJub25lIn19fV1d&macro_url=https%3A%2F%2Fraw.githubusercontent.com%2FdFoiler%2Fnotes%2Fmaster%2Fnir.tex
\[\begin{tikzcd}[cramped]
	0 & {\varepsilon_X\mathrm H^{2r+1}(\ov X_r;\QQ_p)(r+1)} & {V_\Delta} & {\QQ_p} & 0 \\
	0 & {\varepsilon_X\mathrm H^{2r+1}(\ov X_r;\QQ_p)(r+1)} & {\varepsilon_X\mathrm H^{2r+1}(\ov X_r^b;\QQ_p)(r+1)} & {\varepsilon_X\mathrm H^{2r}(\ov X_p;\QQ_p)(r)} & 0
	\arrow[from=1-1, to=1-2]
	\arrow[from=1-2, to=1-3]
	\arrow[equals, from=1-2, to=2-2]
	\arrow[from=1-3, to=1-4]
	\arrow[dashed, from=1-3, to=2-3]
	\arrow[from=1-4, to=1-5]
	\arrow[from=1-4, to=2-4]
	\arrow[from=2-1, to=2-2]
	\arrow[from=2-2, to=2-3]
	\arrow[from=2-3, to=2-4]
	\arrow[from=2-4, to=2-5]
\end{tikzcd}\]
where the right map sends $1$ to the cycle class $\op{cl}_p([\Delta_\varphi])\in\varepsilon_X\mathrm H^{2r}(\ov X_p;\QQ_p)(r)$.
\begin{remark}
	In general, there is a general definition of $\op{AJ}$ as the edge map of a certain Leray spectral sequence.
\end{remark}

\subsection{Calculation of the Map}
It turns out that $\op{AJ}$ actually lands in
\[\op{Ext}^1_{\mathrm{cris}}\left(F,\varepsilon_{X_r}\mathrm H^{2r+1}(\ov X_r;\QQ_p(r+1))\right),\]
where the subscript means that we are computing $\op{Ext}$ in the abelian category of crystalline representations. Indeed, everything in the construction has been some \'etale cohomology group of a scheme smooth over $\OO_F$, which produces a crystalline representation. Then we can pass through $D_{\mathrm{cris}}(V)\coloneqq(V\otimes B_{\mathrm{cris}})^{G_F}$, which brings us to the category of filtered Frobenius modules. Crystalline cohomology becomes de~Rham cohomology, which allows us to calculate the right-hand side becomes cusp forms.

\section{October 14: A Slope Conjecture for Hypergeometric Motives}
This talk was given by David Roberts at MIT for the MIT number theory seminar.

\subsection{Slopes}
Fix positive integers $a\le b$, and define $M\coloneqq a+b$. Then we define the polynomial
\[f(a,b;t;x)\coloneqq a^am^m-t(mx-b)^a.\]
The discriminant in the variable $x$ is $\pm a^\bullet b^\bullet t^{m-1}(t-1)$, so we can think about this as having singularities at $t\in\{0,1,\infty\}$; let $c$ be the $2$-adic valuation of the discriminant. (The monodromy of this polynomial and in particular the behavior of these singularities determine this polynomial $f$.) There is a way to change variables to make this an actual trinomial, but we will not bother. As $t$ varies, we may define
\[K(a,b)_t\coloneqq\frac{\QQ_p[x]}{(f(a,b;t;x))}.\]
This is a separable algebra over $\QQ_p$.
\begin{example}
	One finds
	\[K(1,15)_t=\frac{\QQ_2[x]}{\left(x^{16}-16x+15t\right)}.\]
	As $t\to0$, this looks like $\QQ_2[x]/\left(x^{15}-16t\right)x$. As $t\to\infty$, this looks like $\QQ_2[x]/\left(x^{16}+15t\right)$. (These can be seen using some Newton polygon arguments.) For values of $\left|t\right|_2$ in between, there are complicated patterns.
\end{example}
\begin{remark}
	There is an upper numbering filtration of $\op{Gal}(\ov\QQ_p/\QQ_p)$. This induces a filtration on the $p$-adic algebra $K(a,b)_t$, which turns into some sum of slopes producing $c$. Slope $0$ means a part with no ramification, slope $1$ means a part with tame ramification, and higher slope means a part with wild ramification. Then $c$ is the sum of all the slopes, so we define $\hat c$ to be the sum of the slopes where we replaced slopes which are $0$s to slopes which are $1$s.
\end{remark}
It turns out that there is a lot of structure contained in the image of the $(\nu_2(t),c)$s, which has reently been proved for $\nu_2(t)$ odd. Our main result is a refined conjecture to the slopes.
\begin{conj}
	There is a piecewise-linear function $\sigma_i$ such that the augmented slope $\hat s_i$ is bounded above by $\sigma_i$, and equality holds when $p\nmid\nu_p(t)$.
\end{conj}

\section{October 15: Intersection Number of Rankin--Selberg Cycles}
This talk was given by Zeyu Wang at MIT for the MIT Lie groups seminar.

\subsection{Classial Rankin--Selberg}
For today, $G$ is a split reductive group over $\FF_q$, and $C$ is a smooth projective geometrically connected curve. We let $G^\lor$ be the Langlands dual group (over $\ov\QQ$), and we choose a prime $\ell$ distinct from the characteristic $p$ of $\FF_q$. Lastly, we let $\op{Bun}_G$ denote the stack of $G$-bundles on $C$. We may stratify $\op{Bun}_G$ as
\[\op{Bun}_G=\bigsqcup_{d\in\pi_1(G)}\op{Bun}_G^d,\]
and we let $\op{Fun}_c(\op{Bun}_G(\FF_q))$ be the collection of $\ov\QQ$-valued functions with finite support on each $\op{Bun}_G^d(\FF_q)$. The usual bundles argument shows that
\[\op{Bun}_G(\FF_q)=\prod_{\text{closed }x\in C}G(\OO_x){\mathbin\bigg\backslash}\prod_{\text{closed }x\in C}G(K_x){\mathbin\bigg/}G(K(C)).\]
Note $\op{Fun}_c(\op{Bun}_G(\FF_q))$ has an action by the Hecke algebra
\[\mc H_G\coloneqq\bigotimes'_x\op{Fun}_c(G(\OO_x)\backslash G(K_x)/G(\OO_x)).\]
We are interested in decomposing the action of the Hecke algebra on these functions.
\begin{definition}
	Let $\sigma$ be a $G^\lor$-local system on $C$. Say $f\in\op{Fun}_c(\op{Bun}_G(\FF_q))$ is an eigenfunction with eigenvalue $\sigma$ if and only if
	\[h\cdot f=h(\sigma(\op{Frob}_x))\cdot f\]
	for all $h$ in the Hecke algebra.
\end{definition}
For our setting, we will look at $G\coloneqq\op{GL}_n(\FF_q)\times\op{GL}_{n-1}(\FF_q)$. Then our test function $f$ can take the form $f_n\otimes f_{n-1}$, and it should have eigenvalue $\sigma=(\sigma_n,\sigma_{n-1})$, and we may as well assume that $\sigma_n$ and $\sigma_{n-1}$ are irreducible local systems. Now, given such a $\sigma$, we have an $L$-function
\[L(s,\sigma)\coloneqq\det\left(1-q^{-s}\mathrm{Frob};\mathrm H^1(C_{\ov\FF_q};\sigma)\right),\]
which is some polynomial in $q$. A little linear algebra can also identify this with $\tr\left(\mathrm{Frob};\land^\bullet\mathrm H^1(\sigma)(s)\right)$, where the $s$ at the end is a Tate twist.

The ``Rankin--Selberg'' in the title means that we will be interested in periods against the subgroup $H\coloneqq\op{GL}_{n-1}$, embedded in $G$ as $h\mapsto(\op{diag}(h,1),h)$. Here is the classical story.
\begin{theorem}[Jacquet--Piatetski-Shapiro--Shalika]
	The integral
	\[I(f)\coloneqq\int_{\op{Bun}_H(\FF_q)}f\]
	is a finite sum, and it equals the special value $L(1/2,\sigma)$ up to some explicit normalization factors. Furthermore, if $f'$ has eigenvalue $\sigma^*$ (which is the dual), then
	\[\frac{I(f)I(f')}{\langle f,f'\rangle}=q^{\dim\op{Bun}_H}\cdot\frac{\widetilde L(1/2,\sigma\oplus\sigma^*)}{\op{Res}_{s=1}\widetilde L(s,\op{Ad}\sigma_n)\cdot\op{Res}_{s=1}\widetilde L(s,\op{Ad}\sigma_{n-1})},\]
	where $\langle-,-\rangle$ is some canonical inner product, and $\widetilde L$ is the norrmalized $L$-function.
\end{theorem}
To geometrize, we replace functions with sheaves. For example, an eigenfunction $f\in\op{Bun}_G(\FF_q)$ become eigensheaves $\mathbb L_\sigma$. Taking integrals $\int_{\op{Bun}_H(\FF_q)}f$ corresponds to taking compactly supported cohomology after pullback $\Gamma_c(\pi^*\mathbb L_\sigma)$. Then the $L$-function should correspond to $\land^\bullet\mathrm H^1(\sigma)(1/2)$ under the sheaf-function correspondence.
\begin{theorem}[Lysenko]
	Up to a Whittaker functional, there is an isomorphism
	\[\Gamma_c(\pi^*\mathbb L_\sigma)\cong\land^\bullet\mathrm H^1(\sigma)(1/2).\]
\end{theorem}
\begin{remark}
	One can recover the original identification of $I$ with $L(1/2,\sigma)$ by taking trace of the Frobenius action.
\end{remark}

\subsection{Higher Rankin--Selberg}
We now pass to shtukas. We pass from $\op{Bun}_G(\FF_q)$ to the moduli stack $\op{Sht}_{G,\mu}$ of shtukas over $C^r$, where $\mu$ is some dominant coweight of $G$. Roughly speaking, $\op{Sht}_{G,\mu}$ is a sequeence of $r$ total $G$-bundles $\{\mc E_1,\ldots,\mc E_r\}$ along with some rational maps $\mc E_i\to\mc E_{i+1}$ with at most a pole at some $c_i\in C$ and bounded by the $\mu_i$. (There should also be some morphism $\mathrm{Frob}^*\mc E_r\to\mc E_1$.) Note that setting $r=0$ recovers $\op{Bun}_G(\FF_q)$ from $\mathrm{Sht}_{G,\mu}$. The replacement of $\op{Fun}_c(\op{Bun}_G(\FF_q))$ is $\mathrm H^*_c(\op{Sht}_{G,\mu})$, which turns out to still carry an action by the Hecke algebra.

Consider the coweights $\mu_+=\op{diag}(1,0,\ldots,0)$ and $\mu_-=\op{diag}(0,\ldots,0,-1)$, and define $\mu$ as some sequence $(\mu_{\varepsilon_1},\ldots,\mu_{\varepsilon_r})$ where $\mu_i\in\{\pm1\}$ for each $i$. Then there is a map
\[\pi\colon\op{Sht}_{H,\mu_\varepsilon}^d\to\op{Sht}_{G,\mu_\varepsilon}^d
\]
given by $(\mc E_i)\mapsto(\mc E_i\oplus0,\mc E_i)$. (Here, the $d$ denotes the degree of $\mc E_r$.) It also turns out that $\op{Sht}_{G,\mu}^d$ contains a special cycle $Z_\mu^d$ which is simply the pushforward of $\op{Sht}^d_{H,\mu}$.
\begin{theorem}
	There is a pairing $\langle-,-\rangle$ which relates $\langle Z_{r,\sigma},Z_{r,\sigma}\rangle$ to a special value.
\end{theorem}
The proof proceeds by taking trace of Hecke composed with Frobenius action.

\section{October 16: Katz's Proof of the Riemann Hypothesis for Curves}
This talk was given by Jane Shi at MIT for the STAGE seminar.

\subsection{Review}
Katz's idea is to ``spread out'' the Riemann hypothesis over a family of curves.  More precisely, suppose that we already have the Riemann hypothesis for a given curve $C_0$ over $\FF_q$. Then given a family of curves $f\colon X\to U$ where one fiber is $C_0$, then we will be able to prove the Riemann hypothesis for the full family.

Let's begin by reviewing some $\ell$-adic cohomology. Fix a scheme $X$ of finite type over $\FF_q$. Then we have a zeta function
\[Z(X,T)\coloneqq\exp\Bigg(\sum_{n\ge1}\#X(\FF_{q^n})\frac{T^n}n\Bigg).\]
We already know rationality, which tells us that if $X$ is smooth proper of equidimension $d$, then it can be written as an alternating product. Here is our goal.
\begin{theorem} \label{thm:rh-for-curve}
	Fix a smooth projective irreducible curve $X_0$ over $\FF_q$ of genus $g$. Then
	\[Z(X,T)=\frac{P_1(T)}{(1-T)(1-qT)},\]
	where $P_1(T)$ is a polynomial of degree $2g$, where all roots have absolute value $q^{1/2}$.
\end{theorem}
\begin{remark}
	Previously, we showed that we can take $P_1(T)$ to be
	\[\det\left(1-T\mathrm{Frob};\mathrm H^1(X_{\ov{\FF_q}};\QQ_\ell)\right).\]
\end{remark}
Let's be more precise about our Frobenius polynomials.
\begin{itemize}
	\item There is an arithmetic Frobenius $\sigma_q$, which is the generator of $\op{Gal}(\ov\FF_q/\FF_q)$.
	\item The inverse of the arithmetic Frobenius $\sigma_q$ is the geometric Frobenius $F_q$.
	\item For our variety $X$ defined over $\FF_q$, the base-change $X_{\ov\FF_q}$ admits a Frobenius morphism acting on the $\Spec\ov\FF_q$ factor.
	\item Given a closed point $x\colon\Spec k\to X$ of a variety $X$ over $k$, there is an induced map
	\[\pi_1(\Spec k)\to\pi_1(X)\]
	of \'etale fundamental groups. Accordingly, if $k=\FF_q$, then the left-hand group is $\op{Gal}(\ov\FF_q/\FF_q)$, so we can choose a (geometric) Frobenius element generating it (well-defined up to conjugacy). We will call this element $\mathrm{Frob}_x$.
\end{itemize}
\begin{remark}
	Given an $\ell$-adic local system $\mc F$, we produce a representation
	\[\pi_1(X)\to\op{GL}(\mc F_{\ov x})\]
	of the fiber. The target is some finite-dimensional $\QQ_\ell$-vector space, so we are allowed to consider the action of our Frobneius $\mathrm{Frob}_x$ on this vector space.
\end{remark}
\begin{remark}
	In the sequel, we will have reason to work with non-constant coefficients. Namely, we will be working with a family $f\colon X\to U$ of curves, so the sheaf $\mathrm R^if_*\QQ_\ell$ will be interesting to us. In particular, by the Proper base change theorem, we know that
	\[\left(\mathrm R^if_*\QQ_\ell\right)_s=\mathrm H^i(X_{s,\ov\FF_q};\QQ_\ell)\]
	for any point $s\into X$.
\end{remark}
Let's apply the previous remark. For our family $f\colon X\to U$, we see that
\begin{align*}
	L(\mathrm R^if_*\QQ_\ell;T) &= \prod_{\text{closed }p\in U}\det\left(1-T^{\deg p}\mathrm{Frob}_p;(\mathrm R^if_*\QQ_\ell)_s\right)^{-1} \\
	&= \prod_{\text{closed }p\in U}\det\left(1-T^{\deg p}F_q;\mathrm H^i_{\mathrm{\acute et}}(X_{p,\ov\FF_q};\QQ_\ell)\right)^{-1}.
\end{align*}
By the trace formula (and Poincar\'e duality), if $U$ is an affine curve, one can alternatively write
\[L(\mc F;T)=\frac{\det\left(1-TF_q;\mathrm H^1_c(U;\mc F)\right)}{\det\left(1-TF_q;\mathrm H^2_c(U;\mc F)\right)},\]
but by Poincar\'e duality, this last demoninator is simply
\[\det\left(1-qTF_q;\mc F_{\pi_1^{\mathrm{geo}}}\right),\]
where we are silently keeping track of some Tate twist (which turns the $TF_q$ into $qTF_q$). The subscript $(-)_{\pi_1^{\mathrm{geo}}}$ refers to co-invariants.

\subsection{Redution to a Single Curve}
For our reduction, we will require the notion of purity. Throughout, the base $U$ is a smooth affine geometrically connected curve.
\begin{definition}
	Fix an embedding $\iota\colon\ov\QQ_\ell\into\CC$. For an $\ell$-adic local system $\mc F$ on $U$, we say that $\mc F$ is \textit{$\iota$-pure of weight $w$} if and only if
	\[\left|\alpha\right|=\op Np^{w/2}\]
	for all closd points $p\in U$ and all eigenvalues $\alpha$ of the action of $\mathrm{Frob}_p$ on $\mc F_p$. We say that $\mc F$ is \textit{$\iota$-real} if and only if the characteristic polynomial
	\[\iota\det\left(1-T\mathrm{Frob}_p;\mc F_p\right)\]
	has real coefficients for all closed points $p\in U$.
\end{definition}
\begin{remark}
	For a family $f\colon X\to U$ of curves, our local system $\mc F=\mathrm R^if_*\QQ_\ell$ is $\iota$-real (for any $\iota$) by Lefschetz trace formula arguments.
\end{remark}
Here is our main result.
\begin{theorem} \label{thm:spread-out-bound}
	Fix an $\iota$-real $\ell$-adic local system $\mc F$ on a smooth affine geometrically connected curve $U$. Suppose that every eigenvalue $\beta$ of $F_q$ acting on $\left(\mc F^{\otimes 2k}\right)_{\pi_1^{\mathrm{geo}}}$ (for every $p$ and $k$) has $\left|\beta\right|\le1$. Then for every closed point $p$, every eigenvalue $\alpha$ of $\mathrm{Frob}_p$ acting on $\mc F_p$ has eigenvalue $\left|\alpha\right|\le1$.
\end{theorem}
\begin{proof}
	We compare Euler factors of the $L$-function $L_{2k}$ of $\mc F^{\otimes2k}$ as $k$ gets large. The Euler factor ``at $p$'' is
	\[L_{p,2k}\coloneqq\det\left(1-T^{\deg p}\mathrm{Frob}_p;\mc F^{\otimes2k}\right)^{-1}=\exp\Bigg(\sum_{n\ge1}\tr\left(\mathrm{Frob}_p^n;\mc F\right)^{2k}\frac{T^{n\deg p}}n\Bigg).\]
	By looking at the roots and using the fact that $\mc F$ is $\iota$-real, we see that this Euler factor has nonnegative real coefficients, meaning it lives in $1+T\RR_{\ge0}[[T]]$. Thus, by multiplying our Euler factors together, we see that the power series $L_{p,2k}$ is bounded above by $L_{2k}$ term-wise.

	Now, the radius of convergence of the full $L$-function $L$ can be read off of a calculation with $\mc F_{\pi_1^{\mathrm{geo}}}$, which is where we may apply the hypothesis. In particular, the radius of convergence is bounded above by $1$, so we find that the eigenvalues of $\mathrm{Frob}_p$ acting on $\mc F^{\otimes2k}$ has absolute value bounded above by $q^{\deg p}$. Sending $k\to\infty$ completes the proof.
\end{proof}
\begin{corollary}
	Fix an $\iota$-real $\ell$-adic local system $\mc F$ on a smooth affine geometrically connected curve $U$. If there is a closed point $p$ on $U$ suc hthat the eigenvalues of $\mathrm{Frob}_p$ on $\mc F_p$ have eigenvalue bounded by $1$, then th same is true for all eigenvalues.
\end{corollary}
\begin{proof}[Sketch]
	The key idea is that $F_q^d$ acts by $\mathrm{Frob}_p$ on $\left(\mc F^{\otimes{2k}}\right)_{\pi_1^{\mathrm{geo}}}$. (Roughly speaking, Frobenius can be transported to different points because the action will only differ by something in $\pi_1^{\mathrm{geo}}$.) The result then follows from the theorem.
\end{proof}
In light of the corollary, it remains to prove the Riemann hypothesis for a well-behaved curve and living in some rather general families. %(Bounding by $1$ can be turned into the required equality by doing some twisting.)
\begin{lemma}
	Fix a genus $g\ge1$, and choose two smooth projective geometrically connected curves $C_0$ and $C_1$ over $\FF_q$. After a field extension, there exists a smooth affine geometically connected curve $U$ and a family $f\colon C\to U$ so that $C_0$ and $C_1$ are some fibers of $f$.
\end{lemma}
\begin{proof}[Sketch]
	For genus $1$, choose some $N\ge4$, and one can use the moduli space $Y(N)$ of a pair of an elliptic curve along with a point of order $N$. For large $N$, one finds that $U$ is a curve, and we can use the universal elliptic curve on $Y(N)$.

	Now, for genus $g>1$, we recall (from Deligne and Mumford) that there is a moduli space $H_g^\circ$ classifying genus $g$ quasiprojectve smooth geometrically connected curves. We can get the required $U$ by choosing a generic curve which goes through every $\FF_q$-point.
\end{proof}
Let's now explain how the Riemann hypothesis gets transferred between curves.
\begin{proof}[Proof of \Cref{thm:rh-for-curve} from a single curve]
	Suppose we have the Riemann hypothesis for a single curve $C_0$ of genus $g$, and we want to move it to any other curve $C_1$ of genus $g$. Then we get an upper bound on the eigenvalues of the Frobenius on the full family $U$, which comes down to the required upper bound $\left|\alpha\right|<q^{1/2}$ of eigenvalues $\alpha$ for $C_1$. To complete the proof, we know that $\alpha\mapsto q/\alpha$ should be an involution of the roots by the functional equation, so the equality is forced!
\end{proof}

\subsection{Computations on a Family of Curves}
We will compute with the Fermat curves. Choose some $d$ coprime to $q$, and we define the smooth projective Fermat curve
\[F_d\colon X^d+Y^d=Z^d.\]
The Weil conjectures in this case are due to Weil.
\begin{theorem}[Weil]
	The Riemann hypothesis holds for $F_d$ if $\gcd(d,q)=1$.
\end{theorem}
\begin{proof}[Sketch]
	This is an explicit calculation with some character sums. Given two characters $\chi_1,\chi_2\colon\FF_q^\times\to\CC^\times$, we define their Jacobi sum as
	\[J(\chi_1,\chi_2)\coloneqq\sum_{a\in\FF_q}\chi_1(a)\chi_2(1-a).\]
	If both $\chi_1$ and $\chi_2$ are nontrivial, then one can calculate $\left|J(\chi_1,\chi_2)\right|^2=\sqrt q$.

	Now, to calculate $\#F_d(\FF_{q^n})$, this amounts to calculating the solutions to $x^m+y^m=1$ and adding some points at infinity. Counting solutions to $x^m+y^m=1$ turns into a sum over Jacobi sums, which can then be compared to
	\[\#F_d(\FF_{q^n})=1+q^n-\sum_\alpha\alpha^n,\]
	where the sum is over the eigenvalues $\alpha$ of the Frobenius.
\end{proof}
Now, the genus of $F_d$ is $\binom{d-1}2$, so we have many remaining genera, for which we will take quotients.
\begin{remark}
	Note that a non-constant map $C\to C'$ will have the Riemann hypothesis transfer from $C$ to $C'$. Indeed, the Frobenius eigenvalues of $C'$ are a subset of the Frobenius eigenvalues of $C$. For example, one can see this by splitting $\op{Jac}C=\op{Jac}C'\oplus B$ for some other abelian variety $B$. Then, upon taking Tate modules, we find that the Frobenius eigenvalues of $C$ are precisely the Frobenius eigenvalues of $C'$ along with the Frobenius eigenvalues of $B$.
\end{remark}
\begin{lemma}
	For any genus $g\ge1$, there is a Fermat curve $F_d$ (with $\gcd(d,q)=1$) and a curve $C$ of genus $g$ along with a quotient map $F_d\to C$.
\end{lemma}
\begin{proof}
	If $p\ne2$, then use the hyperelliptic curves $y^2=x^d+1$, where we choose $d\in\{2g+1,2g+2\}$ to avoid $p$. If $p=2$, we can use the curves $y^2-y+x^{2g+1}$. A short calculation shows that these quotients recover all genera.
\end{proof}

\subsection{Persistence of Purity}
We close the seminar by stating the following result.
\begin{theorem}
	Fix an $\iota$-real $\ell$-adic local system $\mc F$ on some smooth affine geometrically connected curve $U$. Suppose that there is a closed point $p$ for which every Frobenius eigenvalue of $\mathrm{Frob}_p$ acting on $\mc F_p$ has absolute value equal to $1$. Then this is true for all closed points on $\mc F$.
\end{theorem}
We will say more about this result next week. It upgrades \Cref{thm:spread-out-bound}.

\section{October 21: The Arithmetic of Thin Orbits}
This talk was given by Katherine Strange at Boston College for the BC--MIT number theory seminar.

\subsection{Quadratic Families}
We are working with continued fractions today. We remark that one can encode these in terms of M\"obius transformations, viewing $\op{SL}_2(\ZZ)$ acting on $\PP^1(\QQ)$. Here is the sort of thing we are interested in.
\begin{conj}[Zaremba]
	Fix the alphabet $A\coloneqq\{1,2,3,4,5\}$, and consider the group
	\[\Gamma_A\coloneqq\left\langle\begin{bmatrix}
		a & 1 \\ 1 & 0
	\end{bmatrix}\right\rangle.\]
	Then
	\[Z_A\coloneqq\left\{a\in\ZZ:\text{there is }p\in\NN\text{ such that }(p,q)\in\Gamma_A\cdot(0,1)\right\}\]
	is all $\ZZ$.
\end{conj}
As another sort of dynamics question, given three curvatures $(c_1,c_2,c_3)$, there are exactly two curvatures $c_4$ and $c_4'$ whose corresponding circle can be tangent to the other three. Thus, there is a canonical involution on the quadruple $(c_1,c_2,c_3,c_4)$ replacing $c_4$ with $c_4'$. This is some subgroup of $\op O_Q(\ZZ)$ for some quadratic form $Q$.
\begin{definition}
	A subgroup $A$ of an algebraic group is \textit{thin} if and only if it is infinite index and Zariski dense.
\end{definition}
These questions can be seen visually. For example, the Appolonian circles can be placed into a ``superpacking.''
\begin{remark}
	The ``obvious'' local-to-global conjecture is false: there are packings missing quadratic or quartic families of curvatures. The point is to compute the Kronecker symbol on the tangencies of the circles, so you can see that a given Kronecker symbol will be constant on the total packing.
\end{remark}
\begin{theorem}
	The thin semigroup
	\[\Phi_1\coloneqq\left\langle\begin{bmatrix}
		1 & 1 \\ 0 & 1
	\end{bmatrix},\begin{bmatrix}
		1 & 0 \\ 4 & 1
	\end{bmatrix}\right\rangle\]
	has $\Phi_1\cdot(2,3)$ missing denominatorics which are squares.
\end{theorem}

\section{October 21: Multiple Polylogarithms}
This talk was given by Daniiel Rudenko at Boston College for the BC--MIT number theory seminar.

\subsection{Algebraic \texorpdfstring{$K$}{ K}-Theory}
We are interested in the Goncharov program.
\begin{definition}[algebraic $K$-theory]
	Fix a field $F$. Then we define the \textit{algebraic $K$-groups} $\{K_n(F)\}_{n\ge0}$ as the $n$th homotopy group of
	\[B\Omega\bigsqcup_{n\ge1}\op{BGL}_n(F),\]
	where $B\Omega$ refers to some group completion operation from algebraic topology.
\end{definition}
\begin{example}
	One has $K_0(F)=\ZZ$ (related to the dimension of a vector space), and $K_1(F)=F^\times$ (related to the determinant).
\end{example}
\begin{example}
	In characteristic not $2$, one can calculate $K_2(F)\cong\land^2F^\times/\langle x\otimes(1-x)\rangle$, which is known as ``Milnor $K$-theory.'' For example, the Hilbert symbol factors through $K_2(F)$.
\end{example}
We may be interested in calculating $K_n(F)$ for $n\ge0$. There is also a description for $n=3$, which we will give later. Here is something of interest to number theorists.
\begin{theorem}[Borel]
	Fix a number field $F$. Then $K_{2n}(F)=0$, and $\op{rank}K_{2n-1}(F)$ is the number of complex embeddings of $K$ if $2\mid n$, and it is the number of real embeddings plus the numer of complex embeddings if $2\nmid n$.
\end{theorem}
\begin{remark}
	Borel also showed that there is a regulator $\op{reg}\colon K_{2n-1}(F)\to\RR^{d_n}$, embedding as a lattice, and its covolume turns out to be a mysterious rational number times an easy transcendental number times $\zeta_{F}(n)$.
\end{remark}
\begin{theorem}[Suslin]
	Fix a field $F$. Then there is a natural grading on $K_n(F)_\QQ$ given by
	\[K_n^{(p)}(F)\coloneqq\frac{\land^nF^\times}{\langle x\otimes(1-x)\otimes\cdots\rangle}.\]
	Then there is an exact sequence
	\[0\to K_3^{(2)}(F)\to B_2(F)\to\land^2F^\times\to K_2^{(2)}F\to0,\]
	where $B_2(F)$ is the quotient of $\ZZ[F^\times]$ by the kernel of the map sending $[x]\mapsto(x\land(1-x))$.
\end{theorem}
\begin{remark}
	It turns out that the regulator on $K_3^{(2)}(\CC)$ factors through $B_2(\CC)$. The corresponding regulator $B_2(\CC)\to\RR$ turns out to be the dilogarithm.
\end{remark}
This suggests that we might be able to ``compute'' $K$-theory by putting it into some kind of resolution and then be able to work with elements satisfying some properties.

\subsection{Mixed Tate Motives}
Zagier has some conjectures for number fields, based on some calculations. Beilinson and Deligne have explained why these conjectures might be true. Our story begins with motives.
\begin{conj}[Beilinson--Deligne]
	Fix a field $F$. Then there is an abelian category $\mathrm{MM}(F)$ of mixed Tate motives with the following properties.
	\begin{listalph}
		\item There are Tate twists $\QQ(n)$ for $n\in\ZZ$.
		\item One has $\op{Ext}^1(\QQ(0),\QQ(n))\cong K_{2n-i}^{(i)}(F)$.
	\end{listalph}
\end{conj}
\begin{remark}
	This conjecture is known for number fields $F$, coming from the (known) triangulated category of Voevodsky motives.
\end{remark}
As usual, motives should be thought of as something like a usual cohomology theory for varieties. For example, this category is Tannakian, so there is a certain Lie (co)algebra $\mc L(F)$ parameterizing the representations. It then turns out tht $K_{2n-i}^{(i)}(F)$ is $\op{Extt}^i_{\mathrm{MM}}(\QQ(0),\QQ(n))$, which can then be computed as the $i$th cohomology group of the complex
\[\mc L\to\land^2\mc L\to\land^3\to\mc L\to\cdots,\]
thereby realizing algebraic $K$-theory in the complex. The Goncharov program now hopes to be able to compute the regulator using generalizations of the dilogarithm, which he named $\op{Li}_{n_1,\ldots,n_k}\in\mc L_n(F)$ (where $n_1+\cdots+n_k=n$) as some periods.
\begin{conj}[Universality] \label{conj:universality}
	The functions $\op{Li}_{n_1,\ldots,n_k}$ span $\mc L_n(F)$.
\end{conj}
Intuitively, this tells us that very general periods can be found as linear combinations of these generalized polylogarithms. Here is our main result.
\begin{theorem}
	Fix a number field $F$. Then \Cref{conj:universality} holds, and there are explicit relations.
\end{theorem}
\begin{example}
	One can calcualte that $\mc L_3(F)=B_3(F)$.
\end{example}

\subsection{Algebraic Topology}
Our approach is homotopical in nature. One can view
\[\underline{\op{GL}}\coloneqq\bigsqcup_{n\ge1}B\op{GL}_n(F)\]
as an $\mathbb E_\infty$-algebra. It is known that there is $\mathbb E_\infty$-homology, so one can ask to try to compute the homology $\mathrm H_{n,d}(\underline{\op{GL}})$.
\begin{theorem}[Kupers--Galatino--Randal-Williams]
	One has $\mathrm H_{n,d}=0$ for $d\le 2n-2$ and $n\ge1$.
\end{theorem}
\begin{conj} \label{conj:compute-e-infinity-hom}
	Fix a field $F$. Then
	\[\bigoplus\mathrm H_{n,2d-1}(\underline{\op{GL}})\]
	is the Lie coalgebra $\mc L$ of the mixed Tate motives.
\end{conj}
Our main theorem then follows from the following.
\begin{theorem}
	\Cref{conj:compute-e-infinity-hom} holds for number fields $F$.
\end{theorem}
\begin{remark}
	The method involved uses the Steinberg module of a vector space. In particular, one uses that it satisfies a certain ``Koszul'' property. Then one shows that
	\[\mathrm H_{n,2n-1}(\underline{\op{GL}})=\mathrm H_1(\mathrm{GL}_n,(\mathrm{St}_n\otimes\mathrm{St}_n)/{\sim}),\]
	and one finds generators which look like polylogarithms. All the main theorems follow.
\end{remark}

\section{October 22: Canonical Representations of Surface Groups}
This talk was given by Aaron Landesman at Harvard for the Dynamics, Geometry and Moduli Spaces seminar. This is joint work with Josh Lam and Daniel Litt.

\subsection{Canonical Representations}
Today, we are interested in $\Sigma_{g,n}$, which is a genus-$g$ surface with $n$ punctures. For example, we may be interested in the fundamental group, which we can understand throug representations $\rho\colon\pi_1(\Sigma)\to\op{GL}_r(\CC)$. We may also be interested in the mapping class group $\mathrm{Mod}_{g,n}\coloneqq\op{Homeo}^+(\Sigma)$, which has an action on representations $\pi_1(\Sigma)\to\op{GL}_r(\CC)$ (even considered up to conjugation).
\begin{definition}
	A \textit{canonical representaion} $\rho\colon\pi_1(\Sigma)\to\op{GL}_r(\CC)$ is one whose conjugacy class has finite orbit under the action by $\mathrm{Mod}_{g,n}$.
\end{definition}
These representations turn out to be quite important, but there are less interesting constructions.
\begin{example}
	If $\rho\colon\pi_1(\Sigma)\to\op{GL}_r(\CC)$ has finite image $H$, then $\rho$ is canonical: $\pi_1(\Sigma)$ is finitely generated, so there are only finitely many maps $\pi_1(\Sigma)\to H$ anyway.
\end{example}
We will be interested in canonical representations with infinite image.
\begin{remark}
	From the perspective of algebraic geometry, it turns out that $\mathrm{Mod}_{g,n}=\pi_1(\mc M_{g,n})$. Loosely speaking, a canonical representation on $\Sigma_{g,n}$ is then a local system on the universal curve over $\mc M_{g,n}$, but we have to take some pullback to a dominant cover of $\mc M_{g,n}$. For example, this perspecitive explains how $\pi_1(\mc M_{g,n})$ acts on $\pi_1(\Sigma)$: there is a fibration $\Sigma\to\mc C_{g,n}\to\mc M_{g,n}$, which more or less induces the exact sequence
	\[1\to\pi_1(\Sigma)\to\pi_1(\mc M_{g,n+1})\to\pi_1(\mc M_{g,n})\to1.\]
\end{remark}
\begin{example}
	If $(g,r)=(0,1)$, then all canonical representations have finite image.
\end{example}
\begin{example}
	If $(g,r,n)=(0,2,3)$, then $\mathrm{Mod}_{0,3}=S_3$ (the automorphisms of $\PP^1$ preserving the three punctures can only permute the three points), which is notably finite. Thus, every representation is canonical.
\end{example}
\begin{example}[Tykhyy]
	If $(g,r,n)=(0,2,4)$, then canonical representations roughly correspond to algebraic solutions to some differential equation. These have all been classified, but it took something like fifty to one hundred papers. Here, $\pi_1(\Sigma_{0,4})=\pi_1(\CC\setminus\{0,1,2\})$ is the free group on three elements, so a representation to $\op{GL}_2(\CC)$ is given by three matrices. Then the action by $\mathrm{Mod}_{0,4}$ is more or less given by the braid group on three generators, generated by the maps
	\[\sigma_1\colon(A,B,C)\mapsto\left(B,B^{-1}AB,C\right)\qquad\text{and}\qquad\sigma_2\colon(A,B,C)\mapsto\left(A,C,C^{-1}BC\right).\]
	Thus, the canonical representations turn into some explicit linear algebra.
\end{example}
\begin{example}
	Consider $(g,n)=(0,4)$. We will use the fact that every representation with $n=3$ is canonical. In particular, put three of our punctures at $\{0,1,\infty\}$, and we can specify a local system on $\Sigma_{0,4}$ as having arbitrary monodromy $A$ at $0$, monodromy $M=\op{diag}(1,-1)$ at $1$, and the forced monodromy $B$ at $\infty$. Now, we have a double cover $\Sigma_{0,4}\to\Sigma{0,3}$ sending $0$ and $1$ to $0$, ramified at the point over $1\in\Sigma_{0,3}$, and sending $\lambda$ and $\infty$ to $\infty$. Then one can check that the corresponding composite is canonical. In particular, one calculates that the mondromy at the point of ramification of the double cover is $M^2=1$. Letting $A$ vary, we receive a $2$-dimensional family of canonical representations. There are other infinite-dimensional families are similar.
\end{example}
\begin{conj}[Tykhyy] \label{conj:tyk}
	Fix $(g,r)=(0,2)$. If $\rho$ is a canonical reprsentation with Zariski dense image, and no loop around a puncture has scalar monodromy, then $n\le6$ and $\rho$ is explicitly classified.
\end{conj}

\subsection{Higher Classification Results}
Here is our main theorem.
\begin{theorem}[Lam--L--Litt]
	If $\rho$ has at least one loop around a puncture with infinite image, then \Cref{conj:tyk} is true.
	% Fix $(g,r)=(0,2)$ with $n\ge3$. Fix a canonical reprsentation $\rho$ with Zariski dense image, and no loop around a puncture has scalar monodromy. If one of the loops around a puncture has infinite image, then $n\le6$ and is given by Tykhyy.
\end{theorem}
The complementary case is also known.
\begin{theorem}[Bronstrin--Moret]
	If $\rho$ has all loops around a puncture with finite image, then \Cref{conj:tyk} is true.
\end{theorem}
Let's begin discussing $g>0$.
\begin{example}[Kodaira--Parshings trick]
	If $r$ is large with respect to $g$ (e.g., $r\ge 6^{2g}$ for $g\ge2$), then there are lots of these canonical representations. For example, suppose we want to construct a canonical representation of $\pi_1(\Sigma_{g,1})\to\op{GL}_r(\CC)$. We need to choose something without choices, so let $D_p\to\Sigma$ be the triple cover branched only at $p\in\Sigma$; note that there are only finitely many of these because they correspond to representations $\pi_1(\Sigma\setminus\{p\})\to S_3$. Then $p\mapsto\prod_{D_p}\mathrm H^1(D_p;\CC)$ provides us with a canonical representation, and one can check that the image is infinite. (Roughly speaking, the image is infinite because moving in a loop with $p$ must change the covering $D_p$ because there are only finitely many maps between curves of genus at least $2$.)
\end{example}
\begin{example}
	If $r=2$ and $\ge2$, it turns out that all canonical representations have finite image.
\end{example}
\begin{theorem}[L--Litt]
	If $r<\sqrt {g+1}$, then any canonical representation has finite image.
\end{theorem}
\begin{remark}
	It is conjectured that irreducible canonical representations have $r$ growing exponentially with $g$.
\end{remark}
\begin{corollary}
	Fix $r<\sqrt g$. Consider the action on tuples $(A_1,A_2,\ldots,A_{2g})$ of $r\times r$-matrices, where we may permute the entries of the tuple, invert the first entry, or replace $A_1$ by $A_1A_2$. If the tuple $(A_1,\ldots,A_{2g})$ has finite orbit up to conjugation, then they generate a finite group.
\end{corollary}
\begin{proof}
	This follows by considering the action of the outer automorphism group of the free group (which contains the mapping class group).
\end{proof}

\section{October 23: The Riemann Hypothesis for Hypersurfaces}
This talk was given by Leonid Gorodetskii at MIT for the STAGE seminar.

\subsection{Spreading Out for Hypersurfaces}
Today, we are giving Katz's proof of the Riemann hypothesis for hypersurfaces of $\PP^{n+1}$. The general idea is to spread out from a single example using moduli spaces. Let $f\colon X\to U$ be a smooth proper family over $\FF_q$. Then $\mathrm R^if_*\QQ_\ell$ defines some local system on $U$, and one has
\[\det\left(1-T\mathrm{Frob}_u;\mathrm R^if_*\QQ_\ell\right)=\det\left(1-T\mathrm{Frob}_{q^{\deg u}};\mathrm H^i(X_u;\QQ_\ell)\right),\]
where $u$ is a closed point of $U$. We called this polynomial $P_i(X_u,T)$. With $U$ a curve, we already know that $P_0$ and $P_2$ live in $1+\ZZ[T]$, so the known rationality results imply that $P_1(X_u,T)$ is in $1+\ZZ[T]$ as well.
\begin{corollary}
	Fix everything as above. Then $\mathrm R^if_*\QQ_\ell$ is real.
\end{corollary}
\begin{lemma}
	Fix everything as above. Then the following are equivalent.
	\begin{listroman}
		\item The Riemann hypothesis holds for $X_u$.
		\item $\mathrm H^i(X_u;\QQ_\ell)$ is pure of weight $i$.
		\item $\mathrm R^if_*\QQ_\ell$ is pure of weight $i$ at $u$.
	\end{listroman}
\end{lemma}
\begin{proof}
	Unwind the definitions of purity.
\end{proof}
Our key spreading our result is the following.
\begin{theorem}[Persistence of purity] \label{thm:persistence-purity}
	Fix a smooth affine geometrically connected curve $U$ over $\FF_q$, and let $\mc F$ be a real $\ell$-adic local system on $U$. Then the following are equivalent.
	\begin{listroman}
		\item The $\ell$-adic local system $\mc F$ is pure of weight $w\in\frac12\ZZ$ at some point $u$.
		\item The $\ell$-adic local system $\mc F$ is pure of weight $w\in\frac12\ZZ$ at all points $u$.
	\end{listroman}
\end{theorem}
\begin{proof}[Sketch]
	By twisting, we may assume that $w=0$. More precisely, we use the Tate twists $\QQ_\ell(i)$, which are the $\ell$-adic local systems in which the action by $\mathrm{Frob}_q$ in the \'etale fundamental group is given by $q^{-i}$. For example, we see that $\QQ_\ell(i)$ is pure of weight $-2i$.\footnote{Technically, we may need some half-integer twists, which requires us to consider field extensions of $\QQ_\ell$. This causes no problems as soon as we suitably generalize our definition of $\ell$-adic system.} The result now follows from \Cref{thm:spread-out-bound} and some careful duality.
\end{proof}
In order to work with hypersurfaces, we need to know something about their cohomology. Here is the cohomology of projective space.
\begin{theorem}
	Fix some $n\ge0$. Then
	\[\mathrm H^i(\PP^n;\QQ_\ell)=\begin{cases}
		0 & \text{if }i\text{ is odd}, \\
		\QQ_\ell(-i/2) & \text{if }i\text{ is even}.
	\end{cases}\]
	In particular, the Riemann hypothesis holds.
\end{theorem}
\begin{proof}[Sketch]
	Use the Gysin sequence on the decomposition $\PP^n=\AA^n\sqcup\PP^{n-1}$, and induct on $n$. Note that the statement has no content for $n=0$, and we also already know it for $n=1$. Approximately speaking, the given even-dimensional classes arise because the whole cohomology ring is generated by the hyperplane class in the image of the cycle class map $\op{CH}^1(\PP^n)\to\mathrm H^2(\PP^n;\QQ_\ell)(1)$.
\end{proof}
\begin{remark}
	To check that this makes sense, we note that it gives $Z(\PP^n;T)$ is
	\[\prod_{i=0}^n\frac1{\det\left(1-T\mathrm{Frob}_q;\QQ_\ell(-i)\right)}=\prod_{i=0}^n\frac1{1-q^iT},\]
	which can then be expanded by hand.
\end{remark}
Now, by the weak Lefschetz theorem and Poincar\'e duality, we are able to say something about the cohomology of hypersurfaces as well.
\begin{theorem}
	Fix a smooth hypersurface $X\subseteq\PP^{n+1}$.
	\begin{listalph}
		\item For $i\in\{0,1,\ldots,2n\}\setminus\{n\}$, we have $\mathrm H^i(X;\QQ_\ell)=\mathrm H^i\left(\PP^{n+1};\QQ_\ell\right)$.
		\item For $i=n$, we have $\mathrm H^n(X;\QQ_\ell)\supseteq\mathrm H^n\left(\PP^{n+1};\QQ_\ell\right)$.
	\end{listalph}
\end{theorem}
\begin{proof}
	Omitted.
\end{proof}
The hard Lefschetz theorem (which we do not currently know for \'etale cohomology at our point in the theory) motivates the following defintion.
\begin{definition}
	Fix a hypersurface $X\subseteq\PP^{n+1}$. Then
	\[\op{Prim}^n(X)=\begin{cases}
		\mathrm H^n(X;\QQ_\ell) & \text{if }n\text{ is odd}, \\
		\mathrm H^n(X;\QQ_\ell)/\mc L^{n/2} & \text{if }n\text{ is even},
	\end{cases}\]
	where $\mc L$ is some class coming from a hyperplane class.
\end{definition}
\begin{lemma} \label{lem:check-primitive-rh}
	Fix a hypersurface $X\subseteq\PP^{n+1}$ over $\FF_q$. Then the Riemann hypothesis holds for $X$ if and only if $\op{Prim}^n(X)$ is pure of weight $n$.
\end{lemma}
\begin{proof}
	Define
	\[P(T)\coloneqq\det\left(1-T\mathrm{Frob}_q;\mathrm{Prim}^n(X)\right).\]
	Tracking through the definitions, we see that $Z(X;T)$ is $P(T)Z(\PP^n;T)$ if $n$ is odd and is $P(T)^{-1}Z(\PP^n;T)$ if $n$ is even. The result now follows because we already have the Riemann hypothesis for $\PP^n$.
\end{proof}
Thus, here is our spreading out result.
\begin{theorem}
	Suppose there is a smooth hypersurface $X_0\subseteq\PP^{n+1}$ of degree $d$ over $\FF_p$ satisfying the Riemann hypothesis. Then the Riemann hypothesis holds for all smooth hypersurfaces $X\subseteq\PP^{n+1}$ of degree $d$ over $\FF_q$ for any power $q$ of $p$.
\end{theorem}
\begin{proof}
	Quickly, note that we may immediately upgrade the Riemann hypothesis from $X_0$ to $X_0\otimes\FF_q$ by computing some eigenvalues. Thus, we may assume that $X_0$ is defined over $\FF_q$.

	Now, choose another smooth hypersurface $X_1\subseteq\PP^{n+1}$ of degree $d$ over $\FF_q$. Smoothness allows us to say that $X_0$ is cut out by a single equation $F_0$, and $X_1$ is smooth over $F_1$. Then the equation
	\[tF_1(x)+(1-t)F_0(X)\]
	defines a family of hypersurfaces in $\PP^{n+1}$ over $\AA^1$. After removing finitely many points $t\in\AA^1$, we may assume that this is a smooth family of smooth hypersurfaces over some affine curve $U\subseteq\AA^1$. The result now follows from \Cref{thm:persistence-purity} applied to $\mc F\coloneqq\mathrm R^if_*\ov\QQ_\ell(n/2)$. (The algebraic closure is desired here in order to take a half-twist.)
\end{proof}

\subsection{A Single Example}
We are now reduced to proving the Riemann hypothesis for a single hypersurface.
\begin{lemma}
	Fix a smooth hypersurface $X\subseteq\PP^{n+1}$ of degree $d$ over $\FF_p$. Then the Riemann hypothesis is true for $X$ if and only if
	\[\#X(\FF_q)=\#\PP^n(\FF_q)+O_X\left(q^{n/2}\right)\]
	for all powers $q$ of $p$.
\end{lemma}
\begin{proof}
	By \Cref{lem:check-primitive-rh}, we only have to check the eigenvalues in the primitive part, and by Poincar\'e duality, we are allowed to only check that the eigenvalues $\alpha$ satisfy $\left|\alpha\right|\le q^{n/2}$. Because we can expand out the size of $\#X(\FF_q)$ as sum of some powers of the eigenvalues, it is enough to consider the sum of all these eigenvalues (otherwise we get some domination), and the result follows.
\end{proof}
Let's now begin with our calculation. Here are the hypersurfaces used by Katz.
\begin{itemize}
	\item If $p\nmid d$, then we can take the Fermat hypersurface cut out by the equation $\sum_{i=1}^{n+2}x_i^d=0$. This calculation is due to Weil.
	\item If $p\mid d$ and $d\ge3$, then we can use the Gabber hypersurface $x_1^d+\sum_{i=1}^{n+1}x_ix_{i+1}^{d-1}=0$.
	\item If $p=d=2$ and $n$ is odd, then it turns out that $\mathrm{Prim}^n(X)=0$, so there is nothing to check.
	\item Lastly, if $p=d=2$ and $n=2m$ is even, then we can use $\sum_{i=1}^{m+1}x_ix_{m+1+i}=0$.
\end{itemize}
We will only do the calculation in the Fermat case because the other ones are more of the same. We proceed in steps. Set $N\coloneqq n+2$ for brevity.
\begin{enumerate}
	\item Because $X\subseteq\PP^{n+1}$, it should be cut out by a single polynomial $F(x)=0$. Accordingly, let $X^{\mathrm{aff}}\subseteq\AA^{N}$ be cut out by this same polynomial $F$, and we see that
	\[\#X^{\mathrm{aff}}(\FF_q)=1+(q-1)\#X(\FF_q).\]
	Thus, it is now enough to show that
	\[\#X^{\mathrm{aff}}(\FF_q)=q^{n+1}+O_{d,n}\left(q^{(n+2)/2}\right).\]
	\item Define $V^*(\FF_q)\subseteq X^{\mathrm{aff}}(\FF_q)\cap\mathbb G_m(\FF_q)^{N}$ to have the nonzero solutions. We claim that it is enough to achieve
	\[\#V^*(\FF_q)\stackrel?=\frac1q(q-1)^{N}+O_{d,n}\left(q^{N/2}\right).\]
	This is a matter of stratifying $X^{\mathrm{aff}}$. Indeed, for a subset $S\subseteq\{1,2,\ldots,N\}$, let $V_S^*$ be the solutions in $X^{\mathrm{aff}}(\FF_q)$ whose nonzero entries are exactly in $S$. Then by choosing what our nonzero entries should be, we calculate
	\begin{align*}
		\#X^{\mathrm{aff}}(\FF_q) &= \sum_{S\subseteq\{1,\ldots,N\}}\#V_S^*(\FF_q) \\
		&\stackrel*= \sum_{S\subseteq\{1,\ldots,N\}}\left(\frac1q(q-1)^{\#S}+O_d\left(q^{\#S/2}\right)\right) \\
		&= \frac1q\sum_{S\subseteq\{1,\ldots,N\}}(q-1)^{\#S}+O\left(q^{N/2}\right),
	\end{align*}
	where the last error term holds because the smaller error terms get smaller exponentially (even though there are an exponential number of them). Notably, we have applied the hypothesis at $\stackrel*=$. The result now follows by noticing that the last sum collapses to $((q-1)+1)^{N}$ by the binomial theorem.
\end{enumerate}
Before continuing with the calculation, we recall some facts about characters.
\begin{definition}[character]
	Fix a finite abelian group $G$. Then a \textit{character} is a homomorphism $G\to\CC^\times$. We let $G^\lor$ denote the group of characters.
\end{definition}
\begin{remark}
	Using the classification of finite abelian gruops, one can check that $G$ and $G^\lor$ have the same size. There is also a non-canonical isomorphism between $G$ and $G^\lor$.
\end{remark}
\begin{remark}
	The restriction maps induce a natural isomorphism $(G\times H)^\lor\to G^\lor\times H^\lor$. The inverse is given by sending the pair $(\chi_G,\chi_H)$ to the character $\chi_G\chi_H$.
\end{remark}
\begin{remark}
	Note that there are the dual identities
	\[\frac1{\#G}\sum_{g\in G}\chi(g)=\begin{cases}
		1 & \text{if }\chi=1, \\
		0 & \text{else},
	\end{cases}\qquad\text{and}\qquad\frac1{\#G}\sum_{\chi\in G^\lor}\chi(g)=\begin{cases}
		1 & \text{if }g=1, \\
		0 & \text{else}.
	\end{cases}\]
\end{remark}
\begin{example}
	Because $\FF_q^\times$ is cyclic, it is fairly easy to write down its multiplicative characters.
\end{example}
\begin{example}
	For any $a\in\FF_q$, there is an additive character $\psi_a\colon\FF_q\to\CC$ given by
	\[\psi_a(t)\coloneqq\exp\left(\frac{2\pi i}p\cdot\tr_{\FF_q/\FF_p}(ax)\right).\]
	One can check that these characters are distinct, so these give all the characters.
\end{example}
Our bounds will come from knowledge of Gauss sums.
\begin{definition}[Gauss sum]
	Fix an additive character $\psi_a$ and a multiplicative character $\chi$ of $\FF_q$. Then we define
	\[g(\chi;\psi_a)\coloneqq\sum_{t\in\FF_q^\times}\chi(t)\psi_a(t).\]
\end{definition}
\begin{remark}
	These are in some sense analogous to the function
	\[\Gamma(z)=\int_{\RR^+}t^ze^{-t}\,\frac{dt}t.\]
	Roughly speaking, we are integrading an additive and multiplicative character together over a multiplicative group.
\end{remark}
\begin{remark}
	Provided $a\ne0$, one can calculate
	\[\left|g(\chi;\psi_a)\right|^2=\begin{cases}
		q & \text{if }\chi\ne1, \\
		1 & \text{if }\chi=1.
	\end{cases}\]
	This sort of fact is used in many proofs of quadratic reciprocity, where one frequently takes $\chi$ to be the Legendre symbol (and receives a ``quadratic Gauss sum.'')
\end{remark}
We now continue with our calculation.
\begin{enumerate}[resume]
	\item To ease our calcuation, we let $\varphi\colon\mathbb G_m^N\to\mathbb G_m^N$ be the $d$th power map, and we let $\sigma\colon\mathbb G_m^N\to\AA^1$ be the summing (i.e., trace) map. As such, $V^*$ is the zero locus of $\sigma\circ\varphi$. Accordingly, note that we have an exact sequence
	\[1\to\ker\varphi\to\mathbb G_m^N\stackrel\varphi\to\mathbb G_m^N\coker\varphi\to1,\]
	so for example we see that $\#\ker\varphi(\FF_q)$ is $\#\mu_d^N(\FF_q)\le d^N$ does not depend on $q$. We now see that
	\[\#V^*(\FF_q)=\#\ker\varphi(\FF_q)\cdot\#\left\{t\in\mathbb G_m^N(\FF_q):t\in\im\varphi(\FF_q)\text{ and }\sigma(t)=0\right\}.\]
	\item We use our character theory. By the orthogonality relations, we know that
	\[1_{\sum t_i=0}=\frac1q\sum_{a\in\FF_q}\psi_a(t_1+\cdots+t_N)\qquad\text{and}\qquad1_{t\in\im\varphi}=\frac1{\#\coker\varphi(\FF_q)}\sum_{\chi\in\coker\varphi(\FF_q)^\lor}\chi(t).\]
	Thus, by cancelling out the kernel and cokernel, we see that
	\begin{align*}
		\#V^*(\FF_q) &= \frac1q\sum_{a\in\FF_q}\sum_{\chi\in\coker\varphi(\FF_q)^\lor}\sum_{t\in\mathbb G_m^N(\FF_q)}\chi(t)\psi_a(t_1+\cdots+t_n).
	\end{align*}
	For example, we see that the $a=0$ succeeds at being nonzero only when $\chi$ is trivial, where we receive $\frac1q(q-1)^N$.
	\item It remains to handle the values $a\ne0$, for which we use Gauss sums. Note that a character $\chi$ on $\coker\varphi$ can be lifted to $\mathbb G_m^N$ and therefore can be factored into $N$ characters $\chi_1\cdots\chi_N$. Thus, we may factor
	\[\sum_{t\in\mathbb G_m^N(\FF_q)}\chi(t)\psi_a(t)=\prod_{i=1}^N\underbrace{\sum_{t_i\in\FF_q^\times}\chi_i(t_i)\psi_a(t_i)}_{g(\chi_i;\psi_a)},\]
	so we see that the entire product has absolute value bounded by $q^{N/2}$ because $\left|g(\chi;\psi_a)\right|\le q^{N/2}$.
	\item We conclude. Plugging in the previous two steps yields
	\[\left|\#V^*(\FF_q)-\frac1(q-1)^N\right|\le\frac1q\underbrace{(q-1)}_a\cdot\underbrace{\#\coker\varphi(\FF_q)}_\chi\cdot q^{N/2}.\]
	The term $\frac1q(q-1)$ dies, and the size of $\coker\varphi(\FF_q)$ is the size of $\ker\varphi(\FF_q)$ is bounded independently of $q$. The total error term comes out to $q^{N/2}$, so we are done!
\end{enumerate}

\section{October 30: Deligne's Proof of Weil I and the Main Lemma}
This talk was given by Mohit Hulse at MIT for the STAGE seminar. The term ``main lemma'' is due to Milne; Deligne only names a consequence as ``the fundamental estimate.''

\subsection{The \'Etale Fundamental Group}
For today, we fix a connected scheme $X$ of finite type over a field $k$.
\begin{definition}
	Fix a scheme $X$ of finite type over a field $k$. For a geometric point $\ov x\into X$, we define the fiber functor $\omega_{\ov x}\colon\mathrm{F\acute Et}(X)\to\mathrm{Sets}$ by
	\[\omega_x(Y)\coloneqq Y_{\ov x}.\]
	We define $\pi_1^{\mathrm{\acute et}}(X,\ov x)$ as the automorphism group of $\omega_x$.
\end{definition}
\begin{remark}
	Once $\pi_1^{\mathrm{\acute et}}(X,\ov x)$ has been defined, we may upgrade $\omega_{\ov x}$ to a functor
	\[\omega_x\colon\mathrm{F\acute Et}(X)\to\mathrm{Sets}(\pi_1^{\mathrm{\acute et}}(X,\ov x)),\]
	and this latter functor turns out to be an equivalence.
\end{remark}
The theory of the \'etale fundamental group proves the following ``pro-representatbility'' result.
\begin{theorem}
	Fix a scheme $X$ of finite type over a field $k$, and choose a geometric point $\ov x\into X$. There is a cofiltered sequence $\{X_i\}$ of finite \'etale covers of $X$ such that
	\[\omega_{\ov x}=\colim_i\op{Hom}_X(X_i,-).\]
	In fact, one can choose the covers $X_i\to X$ to be Galois.
\end{theorem}
\begin{corollary}
	Fix a scheme $X$ of finite type over a field $k$, and choose a geometric point $\ov x\into X$. There is a cofiltered sequence $\{X_i\}$ of finite \'etale covers of $X$ such that
	\[\pi_1(X,\overline x)=\lim_i\op{Aut}_X(X_i).\]
\end{corollary}
\begin{example}
	If $X$ is the point $\Spec k$, then one can choose the $X_i$ to be finite Galois extensions of $k$, so we find that $\pi_1^{\mathrm{\acute et}}(X,\ov x)=\op{Gal}(k^{\mathrm{sep}}/k)$.
\end{example}
\begin{example}
	If $X$ is a smooth projective variety over $\CC$, then we see that $\pi_1^{\mathrm{\acute et}}(X,\ov x)$ is the profinite completion of $\pi_1(X,x)$. For example, $X=\mathbb G_m$ admits covers $\mathbb G_m\to\mathbb G_m$ by $x\mapsto x^n$ (where $\op{char}k\nmid n$), allowing us to compute $\pi_1(X,\ov x)=\widehat{\ZZ}$ when $\op{char}k=0$. This sort of process works for general Riemann surfaces because the finite \'etale covers of a Riemann surface all come from varieties.
\end{example}
The reason we care about the \'etale fundamental group is that it will allow us to understand local systems.
\begin{notation}
	Fix a scheme $X$ of finite type over a field $k$. Then $\op{Loc}(X_{\mathrm{\acute et}},\mathrm{FinSet})$ consists of the locally constant \'etale sheaves on $X$ valued in finite sets. In other words, there is some \'etale covering $\{U_i\}$ of $X$ so that the sheaf is constant when restricted to any of the given $U_i$.
\end{notation}
\begin{theorem}
	Fix a scheme $X$ of finite type over a field $k$, and choose a geometric point $\ov x\into X$. Then the fiber functor
	\[\op{Loc}(X_{\mathrm{\acute et}},\mathrm{FinSet})\to\mathrm{FinSet}\left(\pi_1^{\mathrm{\acute et}}(X,\ov x)\right)\]
	given by $\mc F\mapsto\mc F_{\ov x}$ is an equivalence, and any sheaf on the left-hand side is representable.
\end{theorem}
\begin{proof}[Sketch]
	The point is to choose a cover trivializing the sheaf, and then one can prove representability over $X$ explicitly by finding descent datum.
\end{proof}
Of course, we would like a way to extend this to $\ell$-adic sheaves, which we do as follows.
\begin{notation}
	Fix a scheme $X$ of finite type over a field $k$. Then $\op{Loc}(X_{\mathrm{\acute et}},\QQ_\ell)$ consists of the locally constant $\ell$-adic \'etale sheaves on $X$, meaning that they are valued in finite-dimensional $\QQ_\ell$-vector spaces.
\end{notation}
\begin{theorem}
	Fix a scheme $X$ of finite type over a field $k$, and choose a geometric point $\ov x\into X$. Then the fiber functor
	\[\op{Loc}(X_{\mathrm{\acute et}},\QQ_\ell)\to\mathrm{Rep}_{\QQ_\ell}\left(\pi_1^{\mathrm{\acute et}}(X,\ov x)\right)\]
	given by $\mc F\mapsto\mc F_{\ov x}$ is an equivalence.
\end{theorem}
\begin{proof}
	Unwind to the previous theorem.
\end{proof}
\begin{remark}
	If the corresponding representation on the right-hand side does not factor through a finite quotient of $\pi_1^{\mathrm{\acute et}}(X,\ov x)$, then one does not expect to be able to find a single cover trivializing the entire $\ell$-adic local system.
\end{remark}
Of course, in this seminar, we are interested in computing cohomology, so we pick up a few resutls to do so.
\begin{lemma} \label{lem:h0-by-etale}
	Fix a scheme $X$ of finite type over a field $k$, and choose a geometric point $\ov x\into X$. Then for any locally constant $\ell$-adic sheaf $\mc F$, we have
	\[\mathrm H^0(X;\mc F)\cong\left(\mc F_{\ov x}\right)^{\pi_1^{\mathrm{\acute et}}(X,\ov x)}.\]
\end{lemma}
\begin{proof}[Sketch]
	There is nothing to do if $\mc F$ is constant. If $\mc F$ is non-constant, then we can locally pass to a Galois \'etale cover $Y\to X$ where it is constant, and then we can compute $\mathrm H^0(X;\mc F)$ via \v{C}ech cohomology to prove the result.
\end{proof}

\subsection{The Main Lemma}
Recall the following definition.
\begin{notation}
	Fix a scheme $X$ of finite type over $\FF_q$, and let $\mc F$ be an $\ell$-adic local system. Then we define
	\[Z(\mc F_0;T)\coloneqq\prod_{\text{closed }x\in X}\frac1{\det\left(1-F_{x_0}T^{\deg x_0};\mc F_0\right)}.\]
\end{notation}
Earlier, we proved the following formula.
\begin{theorem}[Lefschetz trace formula]
	Fix a scheme $X$ of finite type over $k\coloneqq\FF_q$, and let $\mc F$ be an $\ell$-adic local system. Then
	\[Z(\mc F_0;T)=\prod_{i\ge0}\det\left(1-\mathrm{Frob}_qT;\mathrm H^i_c(X_{\ov k};\mc F)\right)^{(-1)^{i+1}}.\]
\end{theorem}
\begin{example}
	If $X$ is an affine curve, then $\mathrm H^0_c(X_{\ov k};-)$ vanishes, so we only have to worry about $\mathrm H^1_c(X_{\ov k};-)$ and $\mathrm H^2_c(X;-)$. By Poincar\'e duality, we can recover $\mathrm H^2_c(X_{\ov k};\mc F)$ as $\mathrm H^0(X_{\ov k};\mc F)$ and compute via \Cref{lem:h0-by-etale}.
\end{example}
For the main lemma, we recover the following definition.
\begin{definition}[weight]
	Fix an $\ell$-adic local system $\mc F$ on a scheme $X$ of finite type over $\FF_q$. Then $\mc F$ is of \textit{weight $\beta$} if and only if, for each closd point $x\in X$, the eigenvalues of the Frobenius $F_x$ acting on $\mc F_{\ov x}$ are algebraic numbers all of whose Galois conjugates have absolute value $q^{\beta/2}$.
\end{definition}
% Now, choose an affine curve $U$, which we embed into $\PP^1$, and we set $S\coloneqq\PP^1\setminus U$.
\begin{theorem}[Main lemma] \label{lem:main-deligne}
	Fix an $\ell$-adic local system $\mc F$ on an affine curve $U$ of finite type over $\FF_q$. Choose an integer $\beta$, and assume the following.
	\begin{listalph}
		\item Symplectic: there is a perfect alternating pairing $\psi\colon\mc F\otimes\mc F\to\QQ_\ell(-\beta)$.
		\item Big monodromy: the image of $\pi_1(U_{\ov\FF_q},\ov u)$ in $\op{GL}(\mc F_{\ov u})$ is open in $\op{Sp}(\mc F_{\ov u};\psi)$ in the $\ell$-adic topology for some geometric point $\ov u$.
		\item Rationality: the characteristic polynomials of $F_{\ov u}$ acting on $\mc F_{\ov u}$ are rational for all geometric points.
	\end{listalph}
	Then $\mc F$ has weight $\beta$.
\end{theorem}
\begin{example} \label{ex:get-rh}
	Fix a family $\pi\colon Y\to U$ of smooth projective hypersurfaces in $\PP^{d+1}$, where $d$ is odd. Then we will apply \Cref{lem:main-deligne} with $\mc F\coloneqq\mathrm R^d\pi_*\QQ_\ell$.
	\begin{listalph}
		\item Poincar\'e duality provides a pairing $\mathrm R^d\pi_*\QQ_\ell\times\mathrm R^d\pi_*\QQ_\ell\to\mathrm R^{2d}\pi_*\QQ_\ell=\QQ_\ell(-d)$. This is symplectic because $d$ is odd.
		\item Big monodromy turns out to be hard to check (and of course, it is not always true: one can take a constant family).
		\item Rationality follows from known cases of the Weil conjectures: the cohomology of the fiber of $\mathrm R^d\pi_*\QQ_\ell$ can be computed via proper base change to be $\mathrm H^d(Y_u;\QQ_\ell)$, which is known to be rational because the full zeta function is rational (and this is the only interesting cohomology group!).
	\end{listalph}
	The point is that having big monodromy proves the Riemann hypothesis.
\end{example}
\begin{proof}[Proof of \Cref{lem:main-deligne}]
	We proceed in steps.
	\begin{enumerate}
		\item We can use hypotheses (a) and (b) to produce an isomorphism
		\[\mathrm H^2_c(U_{\ov k},\mc F^{\otimes2k})\to\QQ_\ell(-k\beta-1)^{\oplus N}\]
		for some integer $N$. Indeed, we chain together the isomorphisms
		\begin{align*}
			\mathrm H^2_c(U_{\ov k};\mc F^{\otimes2k}) &= \mathrm H^0(U_{\ov k};\mc F^{\lor\otimes2k})^{\lor}(-1) \\
			&= \left(\mc F^{\lor\otimes2k}_{\ov u}\right)^{\pi_1^{\mathrm{\acute et}}(U;\ov u),\lor}(-1),
			%  \\
			% &= \left(\mc F^{\lor\otimes2k}_{\ov u}\right)_{\pi_1^{\mathrm{\acute et}}(U,\ov u)}(-1),
		\end{align*}
		where we have used Poinar\'e duality in the first line. We may identify $\mc F$ with its dual by (a), so we may ignore the dual. By the big monodromy result, we will reduce ourselves to understanding
		\[\op{Hom}_{\op{Sp}(\mc F_u)}(\mc F^{\otimes2k},\QQ_\ell)\cong\QQ_\ell(k\beta)^{\oplus N},\]
		for some $N$, which completes the proof after tracking through the various twists. (The isomorphism here follows from some representation theory: more or less, one can explicitly construct functionals on $\mc F_u^{\otimes2k}$ to be of the form $x_1\otimes\cdots\otimes x_{2k}\mapsto\psi(x_1,x_2)\cdots\psi(x_{2k-1},x_{2k})$.) Thus, we see that the main point is to check that $\pi_1$-invariants are $G$-invariants, where $G\coloneqq\op{Sp}(\mc F_u)$. Let $H$ be the Zariski closure of the image of $\pi_1$ in $G$; then $\pi_1$-invariants are $H$-invariants because fixing a vector is an algebraic equation. However, $H\subseteq G$ contains an $\ell$-adic open subgroup, so $\dim H=\dim G$, so $H=G$ follows because $G$ is connected.

		\item We apply Rankin's trick. Our two versions of $Z(\mc F;T)$ show that
		\[\prod_{\text{closed }u\into U}\frac1{\det\left(1-F_uT^{\deg u};\mc F^{\otimes2k}\right)}=\frac{\det\left(1-\mathrm{Frob}_qT;\mathrm H^1_c(U;\mc F^{\otimes2k})\right)}{\left(1-q^{k\beta+1}\right)^N},\]
		where the denominator in the right-hand side is computed from the first step. Now, the left-hand side is assumed to live in $\QQ((t))$ (which we can see in fact needs to have positive coefficients), so the right-hand side also lives in $\QQ((t))$. Thus, the radius of convergence of the left-hand side is at most the radius of convergence of each factor in the product. However, the radius of convergence of the right-hand side is just $1/q^{k\beta+1}$, so we conclude that the eigenvalues $\alpha$ of the determinants on the left-hand side must have
		\[\frac1{\left|\alpha\right|^{2k/\deg u}}\ge\frac1{q^{k\beta+1}}.\]
		Sending $k\to\infty$ (which is Rankin's trick!) implies that $\left|\alpha\right|\le q_{u_0}^{\beta/2}$. Because $\mc F_u$ is symplectic, we can run the same argument for the dual, so $\mc F_u$ also has the eigenvalue $q^\beta/\alpha$. The result follows from comparing the two resulting inequalities.
		\qedhere
	\end{enumerate}
\end{proof}
\begin{remark}
	The above proof basically features no algebraic geometry.
\end{remark}

\subsection{Applications}
We now use \Cref{lem:main-deligne} to derive some estimates.
\begin{lemma}
	Embed an affine curve $U$ over $\FF_q$ as $j\colon U\into\PP^1$, and let $S$ be the complement. Further, fix an $\ell$-adic local system $\mc F$ on $U$. Then Poincar\'e duality induces a perfect pairing
	\[\mathrm H^1\left(\PP^1;j_*\mc F^\lor\right)\otimes\mathrm H^1\left(\PP^1;j_*\mc F\right)\to\QQ_\ell(-1).\]
\end{lemma}
\begin{proof}
	There is a pairing already for $\mathrm Rj_*\mc F^\lor$ and $j_!\mc F$, but the differences between $j_!\mc F$ and $j_*\mc F$ turns out to be cancelled out.
\end{proof}
\begin{corollary}
	Embed an affine curve $U$ over $\FF_q$ as $j\colon U\into\PP^1$, and let $S$ be the complement.
	\begin{listalph}
		\item Let $\alpha$ be an eigenvalue of $\mathrm H^1\left(\PP^1;j_!\mc F\right)$. Then
		\[\left|\alpha\right|\le q^{\frac{\beta+1}2+\frac12}.\]
		\item Let $\alpha$ be an eigenvalue of $\mathrm H^1\left(\PP^1;j_*\mc F\right)$. Then
		\[q^{\frac{\beta+1}2-\frac12}\le\left|\alpha\right|\le q^{\frac{\beta+1}2+\frac12}.\]
	\end{listalph}
\end{corollary}
\begin{proof}
	For (a), we start by noting $\mathrm H^1(\PP^1;j_!\mc F)=\mathrm H^1(U,\mc F)$. The same sort of calculation as in the first step of \Cref{lem:main-deligne} shows that our (inverse) zeta function is
	\[\prod_{\text{closed }u\in U}\det\left(1-F_{u_0}T^{\deg u};\mc F_{\ov u}\right)=\frac1{\det\left(1-\mathrm{Frob}_qT;\mathrm H^1_c(U;\mc F)\right)}.\]
	Now, the left-hand product coverges absolutely if and only if the sum of the individual eignevalues converges absolutely. The moral is that
	\[\sum_{\alpha}q^{\beta\deg u/2}\left|t\right|^{\deg u}<\infty,\]
	where the sum is taken over all eigenvalues $\alpha$ of all closed points $u\in U$. Now, one can bound the number of closed points of $U\subseteq\AA^1$ and bound the geometric series to achieve (a).

	For (b), one uses the exact sequence
	\[0\to j_!\mc F\to j_*\mc F\to i_*i^*j_*\mc F\to0,\]
	where $i\colon S\into\PP^1$ is the inclusion. The right-hand term is supported on a finite set, so its $\mathrm H^1$ vanishes, so we achieve a surjection
	\[\mathrm H^1(\PP^1;j_!\mc F)\onto\mathrm H^1(\PP^1;j_*\mc F).\]
	We now get use the bound in (a), and the other bound follows from duality.
\end{proof}

\section{November 3: An Overivew of Lawrence--Venkatesh}
This talk was given by Frank for the Kisin seminar at Harvard.

\subsection{How to Prove Mordell}
The main application of the present method is another proof of the following theorem.
\begin{theorem}[Faltings, Lawrence--Venkatesh] \label{thm:mordell}
	Let $C$ be a smooth projective curve over a number field $K$. If the genus of $C$ is at least $2$, then $C(K)$ is finite.
\end{theorem}
The original proof by Faltings first shows the following.
\begin{theorem}[Tate's conjecture]
	Fix a number field $K$. Then the Tate module functor $T_\ell$ from abelian varieties over $K$ to Galois representations of $K$ is fully faithful.
\end{theorem}
This then implies the following.
\begin{theorem}[Shafaverich conjecture] \label{thm:sha}
	Fix a number field $K$ and a fixed finite set $S$ of places. Then there are only finitely many isomorphism classes of abelian varieties of dimension $g$ and with good reduction outside $S$.
\end{theorem}
\begin{remark}
	By Torelli's theorem, one can descend from isomorphism classes of abelian varieties to isomorphism classes of curves.
\end{remark}
\begin{proof}[Proof that \Cref{thm:sha} implies \Cref{thm:mordell}]
	Choose some \'etale cover $\varphi\colon C'\to C$ of degree more than $2$. Each $x\in C(\ov K)$ then induces the following diagram.
	% https://q.uiver.app/#q=WzAsNSxbMiwwLCJDIl0sWzEsMCwiQyciXSxbMCwwLCJDX3giXSxbMSwxLCJcXG9we0phY31DIl0sWzAsMSwiXFxvcHtKYWN9QyJdLFsxLDAsIlxcdmFycGhpIl0sWzQsMywiWzJdIl0sWzIsNCwiIiwwLHsic3R5bGUiOnsiYm9keSI6eyJuYW1lIjoiZGFzaGVkIn19fV0sWzEsMywiIiwyLHsic3R5bGUiOnsiYm9keSI6eyJuYW1lIjoiZGFzaGVkIn19fV0sWzIsMV0sWzIsMywiIiwxLHsic3R5bGUiOnsibmFtZSI6ImNvcm5lciJ9fV1d&macro_url=https%3A%2F%2Fraw.githubusercontent.com%2FdFoiler%2Fnotes%2Fmaster%2Fnir.tex
	\[\begin{tikzcd}[cramped]
		{C_x} & {C'} & C \\
		{J'} & {J'}
		\arrow[from=1-1, to=1-2]
		\arrow[dashed, from=1-1, to=2-1]
		\arrow["\lrcorner"{anchor=center, pos=0.125}, draw=none, from=1-1, to=2-2]
		\arrow["\varphi", from=1-2, to=1-3]
		\arrow[dashed, from=1-2, to=2-2]
		\arrow["{[2]}", from=2-1, to=2-2]
	\end{tikzcd}\]
	Here, $J'$ is a generalized Jacobian where we only quotient by rational functions with divisor supported in $\varphi^{-1}(\{x\})$; this $J'$ turns out to be an extension of the Jacobian by a torus. (The vertical arrows are rational, not defined over the fiber $\varphi^{-1}(\{x\})$.) For degree reasons, the cover $C_x\to C$ is only ramified at $x$.

	Now, if $x\in C(K)$, then Riemann--Hurwitz controls the genus of $C_x$, and we can control the field of definition of $C_x$ by the choice of $\varphi$. Additionally, we can control where $C_x$ has good reduction: it may only have bad reduction where $C$ and $C'$ have bad reduction, at $2$, or where $\varphi$ fails to be \'etale. Thus, there are only finitely many isomorphism classes of such $C_x$ by \Cref{thm:sha}, and the result follows because there are not many maps $C_x\to C$ (by de~Franchis's theorem).
\end{proof}
\begin{remark}
	This argument is not due to Faltings: it was already understood in the function field case. This is known as Parshin's trick.
\end{remark}
% \begin{remark}
% 	The original proof by Faltings proves this by showing Tate's conjecture for in degree $1$. Explicitly, Faltings shows that $\mathrm H^1_{\mathrm{\acute et}}(-;\ZZ_\ell)$ is fully faithful from the category of abelian varieties to the category of Galois representations over a number field. This is then used to show that there are only finitely isomorphism many classes of 
% \end{remark}
Morally, we are moving around the diagram
% https://q.uiver.app/#q=WzAsNCxbMCwwLCJDKEspIl0sWzAsMSwiQyhcXG92IEspIl0sWzEsMSwiXFx7Q194XFx0byBDXFx0ZXh0eyB3aXRoIGNvbnRyb2xsZWQgZ2VudXN9XFx9Il0sWzEsMCwiXFx7Q194XFx0ZXh0eyB3aXRoIGNvbnRyb2xsZWQgcmVkdWN0aW9ufVxcfSJdLFswLDNdLFswLDEsIiIsMix7InN0eWxlIjp7InRhaWwiOnsibmFtZSI6Imhvb2siLCJzaWRlIjoidG9wIn19fV0sWzMsMiwiIiwwLHsic3R5bGUiOnsidGFpbCI6eyJuYW1lIjoiaG9vayIsInNpZGUiOiJ0b3AifX19XSxbMSwyXV0=&macro_url=https%3A%2F%2Fraw.githubusercontent.com%2FdFoiler%2Fnotes%2Fmaster%2Fnir.tex
\[\begin{tikzcd}[cramped]
	{C(K)} & {\{C_x\text{ with controlled reduction}\}} \\
	{C(\ov K)} & {\{C_x\to C\text{ with controlled genus}\}}
	\arrow[from=1-1, to=1-2]
	\arrow[hook, from=1-1, to=2-1]
	\arrow[hook, from=1-2, to=2-2]
	\arrow[from=2-1, to=2-2]
\end{tikzcd}\]
where the point is that the vertical arrows are inclusions, and the bottom arrow has finite fibers by de Franchis's theorem. Note that we only had to apply Shafaverich's conjecture to the family of curves of the form $\{C_x\}$, so one might expect to be able to prove Mordell's conjecture using something weaker about families $\mc X\to C'$ for a given cover $C'\to C$.

\subsection{The Period Map}
We are now prepared to say what Lawrence--Venkatesh is about. Given a family of varieties, we would like to understand the cohomology of the fibers, where ``cohomology'' can refer to any of \'etale, de Rham, cystalline, and so on. We are going to understand such cohomology via a $p$-adic period map.

Throughout, we fix a number field $K$ and a finite set $S$ of places, and we let $\OO_{K,S}$ be the ring of $S$-integers. Now, fix a smooth projective map $Y\to X$ of smooth varieties over $K$ which spreads out to $\mc Y\to\mc X$ over $\OO_{K,S}$. We will furthermore suppose that the relative de Rham $\mc H^q$ and the pushforwards $\mathrm R^\bullet\pi_*\Omega^p_{\mc Y/\mc X}$ are all locally free on $X$ (which can be achieved by possibly enlarging $S$). In this situation, thre is a Gauss--Manin connection
\[\nabla\colon\mc H^q\to\mc H^q\otimes\Omega^1_{\mc X/\OO_{K,S}}.\]
We now fix an archimedean place $\iota\colon K\into\CC$ and a place $v$ above an odd prime $p$ (avoiding $S$) so that $v$ is unramified.

Furthermore, we fix some $x_0\in\mc X(\OO_{K,S})$, so it has a reduction $\ov x_0\in\mc X(\FF_v)$. Accordingly, we may choose a system of parameters $\{z_1,\ldots,z_m\}$ for $\OO_{\mc X,\ov x_0}$. Upon choosing a basis $\{e_i\}$ of $\mc H^q$, we may expand out the connection as
\[\nabla(e_i)=\sum_jA_{ji}e_j,\]
where the coefficients $A_{ji}$ are integral. It follows that we can solve the system of differential equations
\[\nabla\Bigg(\sum_if_ie_i\Bigg)=0\]
where the $f_\bullet$s live in $K[[z_1,\ldots,z_m]]$. By the aforementioned integrality, the $f_\bullet$s acquire a positive radius of convergence.

Now, if $x\equiv x_0\pmod v$, one gets an isomorphism
\[\mathrm H^q_{\mathrm{dR}}(\mc Y_{x_0}/K_v)\to\mathrm H^q_{\mathrm{dR}}(\mc Y_x/K_v),\]
and this isomorphism is compatible with the comparison isomorphism from de Rham to crystalline cohomology. (The above map arises by plugging in the $f_\bullet$s, which vanish under $\nabla$.) Thus, we can give de Rham cohomology a Frobenius action compatible with the Gauss--Manin connection. On the other hand, the Hodge filtration on de Rham cohomology will vary as $x$ varies over $x_0\pmod v$.

The moral is that we have a map
\[\Phi_v\{x\in X(K_v):x\equiv x_0\pmod v\}\to\mc L_v,\]
where $\mc L_v$ consists of triples $\left(\mathrm H^q_{\mathrm{dR}}(\mc Y_x/K_v),F_{\mathrm{Hodge}},\varphi\right)$, where $F_{\mathrm{Hodge}}$ is the Hodge filtration, and $\varphi$ is the Frobenius action. By the Gauss--Manin connection, we see that we should view $\mc L_v$ as parameterizing triples of a vector space, a filtration, and a Frobenius.
\begin{remark}
	Everything present also works complex analytically. This produces a map
	\[\Phi_\CC\colon\widetilde X\to\mc L_\CC,\]
	where $\widetilde X$ is the universal cover, and $\mc L_\CC$ parameterizes possible Hodge filtrations on the de Rham cohomology of the fiber.
\end{remark}
\begin{lemma}
	The dimension of the Zariski closure of the image of $\Phi_v$ is at least the dimension of $\Gamma\cdot h_0$, where $h_0$ is the image of (some lift of) $x_0$ under $\Phi_\CC$, and $\Gamma$ is the Zariski closure of the image of $\pi_1(X,x_0)$ in $\mathrm H^q_{\mathrm B}(Y)$.
\end{lemma}
The moral is that complex analytic control of monodromy provides some $v$-adic control of monodromy.

\subsection{Introducing Galois Representations}
The space $\mc L_v$ turns out to parameterize Galois representations. Let's explain how to pass from crystalline cohomology to Galois representations.
\begin{theorem}
	Fix a smooth proper variety $Z$ over a $p$-adic field $E$ with good reduction. Then there is an isomorphism
	\[\left(\mathrm H^*_{\mathrm{\acute et}}(Z_{\ov E};\QQ_p)\otimes_{\QQ_p}B_{\mathrm{cris}}\right)^{\op{Gal}(\ov E/E)}\cong\mathrm H^n_{\mathrm{crys}}(\mc Z/\OO_v)[1/p].\]
	In the sequel, we will denote the functor on the left-hand side by $D_{\mathrm{crys}}$.
\end{theorem}
\begin{definition}
	When $Z$ has good reduction, we may say that the Galois representation
	\[\rho\colon\op{Gal}(\ov E/E)\to\op{Aut}\left(\mathrm H^q_{\mathrm{\acute et}}(Z_{\ov E};\QQ_p\right)\]
	is crystalline, meaning that $D_{\mathrm{crys}}$ preserves dimension.
\end{definition}
\begin{theorem}[Fontaine]
	The functor $D_{\mathrm{crys}}$ provides a fully faithful functor from crystalline representations to filtered Frobenius modules.
\end{theorem}
\begin{example}
	It turns out that $D_{\mathrm{crys}}$ sends $\mathrm H^q_{\mathrm{\acute et}}(\mc Y_x/\QQ_p)$ to exactly the triple $\Phi_v(x)$. Furthermore, the above theorem tells us that $\Phi_v$ is injective.
\end{example}
Everything so far is local. Here is our global input.
\begin{proposition} \label{prop:l-v-small}
	Fix notation as above. Suppose
	\[\dim Z(\phi_v^{[K_v:\QQ_p]})<\dim_\CC\Gamma\cdot h_0'.\]
	Then
	\[\{x\in\mc X(\OO_{K,S}):x\equiv x_0\pmod v,\rho_x\text{ is semisimple}\}\]
	lies in a proper $K_v$-analytic subvariety of the residue disk of $X(K_v)$.
\end{proposition}
\begin{proof}
	The smallness will arise from the following finiteness result.
	\begin{lemma}[Faltings]
		Fix integers $w,d\ge0$. There are only finitely many semisimple representations $\rho\colon\op{Gal}(\ov K/K)\to\op{GL}_d(\QQ_p)$ (up to conjugation) satisfying the following.
		\begin{listalph}
			\item $\rho$ is unramified outside $S$.
			\item For all $\mf p$ in $S$, the characteristic polynomial of $\mathrm{Frob}_{\mf p}$ has all roots algebraic and with complex absolute value $\#\kappa(\mf p)^{w/2}$.
			\item For all $\mf p$ in $S$, the characteristic polynomial of $\mathrm{Frob}_{\mf p}$ has integer coefficients.
		\end{listalph}
	\end{lemma}
	The proof of this lemma is technically elementary, but it is quite tricky.

	Note $\rho_x$ satisfies all three of the above conditions, so we may fix a choice of $\rho$ for $\rho_x\cong\rho$. Now, because $D_{\mathrm{crys}}$ is fully faithful, we know that $\rho_x$ and $\rho$ agreeing on $\op{Gal}(\ov K/K)$ is equivalent to $\Phi_v(x)$ being equivalent to $D_{\mathrm{crys}}(\rho)$. It follows that $\Phi_v(x)$ lives in some orbit of $Z(\phi_v)$ in $\mc L_v$.

	The proof then follows by comparing dimensions: one finds that the dimension of the $Z(\phi_v)$-orbit is strictly less than the dimension of the Zariski closure of $\Phi_v(\Omega_v)$. It follows that $\Phi_v(\Omega_v)$ cannot be contained in any orbit by $Z(\phi_v)$. This unwinds into the desired statement.
\end{proof}
As before, one can view the above argument as chasing around the following diagram
% https://q.uiver.app/#q=WzAsNCxbMCwwLCJcXG1jIFgoXFxPT197SyxTfSkiXSxbMCwxLCJcXE9tZWdhX3YiXSxbMSwwLCJcXHtaKFxccGhpX3YpXFx0ZXh0eyBvcmJpdHN9XFx9Il0sWzEsMSwiXFx7XFx0ZXh0e2ZpbHRyYXRpb25zIG9uIH1cXG1hdGhybSBIXnFcXH0iXSxbMCwxLCIiLDAseyJzdHlsZSI6eyJ0YWlsIjp7Im5hbWUiOiJob29rIiwic2lkZSI6InRvcCJ9fX1dLFsyLDMsIiIsMCx7InN0eWxlIjp7InRhaWwiOnsibmFtZSI6Imhvb2siLCJzaWRlIjoidG9wIn19fV0sWzAsMl0sWzEsM11d&macro_url=https%3A%2F%2Fraw.githubusercontent.com%2FdFoiler%2Fnotes%2Fmaster%2Fnir.tex
\[\begin{tikzcd}[cramped]
	{\mc X(\OO_{K,S})} & {\{Z(\phi_v)\text{ orbits}\}} \\
	{\Omega_v} & {\{\text{filtrations on }\mathrm H^q\}}
	\arrow[from=1-1, to=1-2]
	\arrow[hook, from=1-1, to=2-1]
	\arrow[hook, from=1-2, to=2-2]
	\arrow[from=2-1, to=2-2]
\end{tikzcd}\]
with vertical inclusions, and the bottom inclusion has positive codimension.

Here are some problems with applying \Cref{prop:l-v-small}.
\begin{itemize}
	\item One may have wanted being in an analytic subvariety to an algebraic subvariety. For example, if the base is in high dimensions, then one may want to show finiteness by showing Zariski non-density in order to inductively apply \Cref{prop:l-v-small}. Luckily, if the codimension is high enough, one can achieve Zariski non-density by some results of Bakker and Tsimermann.
	\item One only gains control of $x$ where $\rho_x$ is semisimple. To fix this, it frequently suffices to look at a locus where two representations have the same semisimplification.
	\item It is difficult to verify the big monodromy hypothesis inductively to achieve finiteness, as described in the first point. 
\end{itemize}
\begin{remark}
	Professor Kisin thinks that the proof of big monodromy in Lawrence--Venkatesh is ``sus.''
\end{remark}
Next time, we (excluding me) will discus the following theorem.
\begin{theorem}[Lawrence--Venkatesh]
	The Shafaverich conjecture holds for hypersurfaces of abelian varieties of dimension at least $4$.
\end{theorem}
What is surprising here is that Lawrence--Venkatesh require a big monodromy result.

\section{November 4: The Gross--Siebert Program}
This talk was given by Yonghwan at MIT as a pre-talk for the Harvard--MIT algebraic geometry seminar.

\subsection{Mirror Symmetry}
Mirror symmetry is interested in the following result.
\begin{theorem}[Closed string mirror symmetry]
	Fix a smooth Calabi--Yau variety $X$, and let $X^\lor$ be the mirror Calabi--Yau. Then there is a mirror map
	\[\left(\mathrm H^*(X^\lor;\land^\bullet T_{X^\lor}),\nabla^{\mathrm{GM}}\right)\cong\left(\op{QH}^*(X),\nabla^q\right).\]
	Further, there is a $J$-function whih is a canonical flat section of the quantum $D$-module, and there is an $I$-funtion which is a canonical flat sectino of the Gauss--Manin system.
\end{theorem}
Here, $\op{QH}^*(X)$ is the quantum cohomology. Its underlying vector space is $\mathrm H^*(X;\land)$, where $\land$ is the Norkiov ring $\QQ[[q^A]]$, where the basis is taken over $A$ in some cohomology classes. The Gromov--Witten invariants then fit into this picture as the coefficients of the multiplication.
\begin{example}
	The mirror of an elliptic curve $E$ is itself. In this case, the quantum cohomology is just regular cohomology.
\end{example}
\begin{example}
	The mirror of a K3 surface $X$ is itself. In this case, the quantum cohomology is again regular cohomology.
\end{example}
There is also a story of mirror symmetry beyond Calabi--Yau surfaces. For Fano surfaces in general, the mirror is a pair of a quasiprojective variety $X^\lor$ and a regular function $W$ on $X^\lor$.
\begin{example}
	For $X=\PP^n$, the mirror turns out to be $\left(\CC^\times\right)^n$ and $W$ is
	\[W(z_1,\ldots,z_n)\coloneqq z_1+\cdots+z_n+\frac{u^{-1}}{z_1\cdots z_n}.\]
\end{example}
The statement of mirror symmetry in this case also says something about quantum cohomology of $X$ in terms of some total cohomology of $X^\lor$ of a deformed de Rham cohomology. The Gross--Siebert program aims to compute some mirror varieties in algebraic geometry.

\section{November 4: Automorphism Groups of Riemann Surfaces}
This talk was given by Jennifer Paulhus at MIT for the MIT number theory seminar.

\subsection{Riemann Surfaces}
Let $X$ be a Riemann surface of genus $g$. Then $G\coloneqq\op{Aut}X$ is finite, and $X/G$ is another Riemann surface, say of genus $h$. The map $X\onto X/G$ has degree $d\coloneqq\#G$, so we expect fibers have to size $d$, but there may be ramification. This covering has some monodromy group; let $\{m_1,\ldots,m_r\}$ be the orders of the monodromy. We are interested in what sort of tuples $[h;m_1,\ldots,m_r]$ can occur.
\begin{remark}
	Everything in the story descends to discuss curves over number fields.
\end{remark}
There is already a classification of the possible groups $G$ which can appear.
\begin{theorem}[Riemann existence]
	A finite group $G$ acts on some Riemann surface $X$ of genus $g>1$ if and only if there are elements $\{a_1,b_1,\ldots,a_h,b_h\}\cup\{c_1,\ldots,c_r\}$ generating $G$ and for which
	\[\prod_{i=1}^r[a_i,b_i]\cdot\prod_{i=1}^rc_i=1\]
	and the Riemann--Hurwitz formula is satisfied with $h$ being the genus of $X/G$.
\end{theorem}
\begin{remark}
	One can see that the $c_\bullet$s are the monodromy elements.
\end{remark}
\begin{remark}
	Here is one way to view this theorem: $X$ has universal cover given by the upper-half plane $\mc H$, so $X/G$ becomes the orbit space of some Fuchsian group.
\end{remark}
Here is the sort of question we may be interested in. Given a group $G$, we can look for the possible signatures $[h;m_1,\ldots,m_r]$, and then afterward we have to look for which signatures can be realized.
\begin{example}
	The signature $[0;3,3,9]$ satisfies the Riemann--Hurwitz formula for the cyclic group $C_9$, but it never happens.
\end{example}
\begin{example}
	The signature $[0,2,\ldots,2]$ satisfies the Riemann--Hurwitz formula for $\op{SL}(2,q)$, but there is only one element of order $2$ in this group.
\end{example}
\begin{definition}
	A group $G$ acts with \textit{almost all possible signatures} if and only if there are only finitely many signatures satisfying the Riemann--Hurwitz formula whose signature cannot be realized in a Riemann surface $X$.
\end{definition}
\begin{example}
	If the commutator subgroup $[G:G]$ fails to contain an element of all possible orders, then the signatures $[h;n_i]$ for any $h>0$ is satisfies Riemann--Hurwitz, but it cannot be realized.
\end{example}
\begin{example}
	If $G$ can be generated by elements of order $n_i$ for any possible order $n_i$, then we can again find infinitely many signatures.
\end{example}
It turns out that the conditions given in the two examples are sharp.
\begin{example}
	It turns out that any finite nonabelian simple group acts with almost all possible signatures. For example, the work of many group theorists showed that every element of $G$ is a commutator.
\end{example}
One may be interested in obtaining the finite list of missed signatures for finite nonabelian simple groups. This turns out to be difficult. For signature $[h;m_1,\ldots,m_r]$ for $h\ge2$, all signatures are doable; one input is that any finite nonabelian simple group is generated by two elements. It turns out that $h=0$ is very hard, so let's determine what we can do for $h=1$
\begin{theorem}
	For $n>6$, every potential signature $[h;m_1,\ldots,m_r]$ for $h\ge1$ is an actual signature for the group $A_n$.
\end{theorem}

\section{November 5: Local Constancy of Automorphic Sheaves}
This talk was given by Joakim Faergeman for the Lie Groups seminar at MIT.

\subsection{Geometric Langlands}
Fix a smooth projective curve $X$ over $\CC$, and let $G$ be a complex connected reductive group. Here is one version of the geometric Langlands correspondence.
\begin{theorem}[de Rham geometric Langlands]
	Let $\op{Bun}_G(X)$ be the moduli stack of $G$-bundles on $X$, and consider the (derived) category $\mc D(\op{Bun}_G(X))$ of $D$-modules; this is the automorphic side. Further, let $\op{LS}_{\check G}(X)$ be the moduli stack of $\check G$-bundles on $X$ with a connection; this is the spectral side. Then
	\[\mc D(\op{Bun}_G(X))\cong\op{QCoh}(\op{LS}_{\check G}(X)).\]
\end{theorem}
As usual, there are technical problems here; for example, the right-hand side turns out to be too small.

Ben-Zvi and Nadler have introduced a Betti version of geometric Langlands. The goal is to replace vector bundles with connections with topological $\check G$-local systems; this is then merely the data of a group homomorphism $\pi_1(X)\to\check G$.
\begin{remark}
	One can make morphisms $\pi_1(X)\to\check G$ more explicit: $\pi_1(X)$ can be presented with generators $\{a_1,b_1,\ldots,a_g,b_g\}$ (where $g$ is the genus of $X$) satisfying $\prod_i[a_i,b_i]=1$.
\end{remark}
But now one must modify the automorphic side.
\begin{definition}
	Fix a smoooth algebraic stack $\mc Y$ over $\CC$, and let $\op{Shv}(\mc Y)$ be the category of complex sheaves on $\mc Y(\CC)$. For some closed conical subset $\Lambda\subseteq T^*\mc Y$, we let $\op{Shv}_\Lambda(\mc Y)$ be the collectio of sheaves whose singular support is contained in $\Lambda$, where the singular support measures where (and in which direction!) $\mc F$ fails to be locally constant.
\end{definition}
\begin{example}
	With $\Lambda=0$, we see that $\op{Shv}_\Lambda(\mc Y)$ has all topological local systems on $\mc Y$.
\end{example}
\begin{example}
	We will take $\mc Y=\op{Bun}_G(X)$. Then $T^*\op{Bun}_G(X)$ consists of pairs $(\mc P_G,\varphi)$ of a $G$-bundle $\mc P_G$ and a ``Higgs field'' $\varphi$; a Higgs field is just some global section of $\mc Y^*_{\mc P_G}\otimes\Omega^1_X$. Then we will take $\Lambda$ to be the nilpotent cone $\op{Nilp}(X)$ of $X$ inside $T^*\mathrm{Bun}_G(X)$, meaning that the Higgs field $\varphi$ is nilpotent.
\end{example}
\begin{theorem}[Betti geometric Langlands] \label{thm:betti-gglc}
	Let $\op{LS}_{\check G}^{\mathrm{top}}$ be the moduli of group maps $\pi_1(X)\to\check G$ up to $\check G$-conjugation. Then
	\[\op{Shv}_{\op{Nilp}(X)}(\op{Bun}_G(X))\cong\op{QCoh}\left(\op{LS}_{\check G}^{\mathrm{top}}(X)\right).\]
\end{theorem}

\subsection{The Main Result}
Note that the right-hand side in \Cref{thm:betti-gglc} only depends on the homtopy type of $X$ (i.e., the genus), so the left-hand side should be the same. This is our main theorem today.
\begin{theorem}
	The category $\op{Shv}_{\op{Nilp}(X)}(\op{Bun}_G(X))$ depends only on the topological space of $X$.
\end{theorem}
Of course, we are interested in proving this without using the proof of Betti geometric Langlands.
\begin{example}
	Consider $G=\op{SL}_2$. Then $\op{Bun}_G(X)$ consists of rank-$2$ vector bundles $\mc E$ for which $\det\mc E\cong\OO_X$. Now, the nilpotent cone of $X$ stratifies according to degree into some subspaces $\op{Nilp}^d(X)$, which consists of short exact sequences
	\[0\to\mc L\to\mc E\to\mc L^\lor\to0\]
	where $\varphi$ is some nonzero function $\mc L^\lor\to\mc L\otimes\Omega^1_X$. Thus, the geometry of $\op{Nilp}(X)$ will depend on $\Gamma\left(X,\mc L^2\otimes\Omega^1_X\right)$. Riemann--Roch explains what this vector spaces is when $\deg\mc L<0$ or $\deg\mc L>2g-2$, but in general, Brill--Noether theory explains that $\op{Nilp}(X)$ will depend on the complex structure of $X$.
\end{example}
Here is the setup for the argument: we will work with a family $X\to S$ of smooth plane curves. Then the stack $\op{Bun}_G(X/S)$ lies over $S$ with fiber $\op{Bun}_G(X_s)$ over $s\in S$. It is a result of Nadler and Yun that one can find some $\op{Nilp}^U\subseteq T^*\op{Bun}_G(X/S)$ whose fiber over $s\in S$ is $\op{Nilp}(X_s)$, and then our goal becomes to show that the projection maps
\[\op{Shv}_{\op{Nilp}^U}(\op{Bun}_G(X/S))\to\op{Shv}_{\op{Nilp}(X_s)}(\op{Bun}_G(X_s))\]
is an equivalence when the base $S$ is contractible. A key in the proof is to have an irreducible component $\op{Nilp}^{\mathrm{Kos}}$ which ``controls'' $\op{Shv}_{\op{Nilp}}(\op{Bun}_G(X))$ which behaves well in families.

\section{November 6: Lefschetz Principles}
This talk was given by Jack Miller at MIT for the STAGE seminar.

\subsection{Fibrations, Diffeomorphically}
We are (still) trying to show the Riemann hypothesis part of the Weil conjectures. As usual, the veracity of the Riemann hypothesis is insensitive to base-change of the base finite field. After some reductions, it turns out that it will be enough to handle the middle-dimension cohomology of a smooth projective even-dimensional varieties. As such, we will let $n+1$ be the dimension of some varieties to be chosen later, where $n=2m+1$ is odd.

As in our previous cases of the Weil conjectures, we are on the hunt for many fibrations $X\to\PP^1$. Such fibrations are the algebro-geometric version of finding some smooth submersions $M\to\RR$. In the real analytic case, this amounts to the following result.
\begin{theorem}[Ehresmann]
	If $f\colon M\to N$ is a smooth submersion of closed manifolds, then $f$ is a locally trivial fibration.
\end{theorem}
\begin{corollary}
	Fix a closed manifold $M$. Every smooth map $M\to\RR$ admits a critical point.
\end{corollary}
\begin{remark}
	Of course, is not hard to prove the corollary directly: a smooth map $M\to\RR$ must admit a maximum, which is critical.
\end{remark}
\begin{corollary}
	Fix a closed manifold $M$. If $M\to S^1$ is a submersion, then $M$ is diffeomorphic to a mapping torus.
\end{corollary}
Thus, admitting a submersion to a curve places strong requirements on $M$.

Here is the algebro-geometric version of this.
\begin{theorem}[Ehresmann]
	If $f\colon Y\to S$ is smooth proper, and $\mc F$ is a locally constant constructible \'etale sheaf on $Y$, then the higher pushforwards $\mathrm R^pf_*\mc F$ are locally constant constructible.
\end{theorem}
However, the real analytic version tells us that we are going to need to allow some singularities for our fibrations. Let's explain the sort of singularities one finds in the diffeomorphic setting.
\begin{theorem}[Morse's lemma]
	Fix a $d$-dimensional closed manifold $M$. Then there is an open subset $U\subseteq\op{Fun}(M,\RR)$ (using the $C^2$ topology) satisfying the following for each function in $U$.
	\begin{itemize}
		\item There are finitely many critical points, and at most one in each fiber.
		\item The critical points $p$ are non-degenerate, and each bad fiber $f^{-1}(\{p\})$ looks like a conic of the form
		\[0=x_1^2+\cdots+x_i^2-x_{i+1}^2-\cdots-x_d^2.\]
		In other words, the Hessian is non-degenerate.
	\end{itemize}
\end{theorem}
\begin{remark}
	Picard and Lefschetz proved a variant of this for complex manifolds.
\end{remark}

\subsection{Lefschetz Pencils}
Let's now try to find our fibrations in the algebro-geometric setting.
\begin{definition}
	Fix a projective variety $X$, and let $\mc L$ be a very ample line bundle on $X$ inducing $X\into\PP\Gamma(X,\mc L)$. Then a \textit{pencil} $D$ of $(X,\mc L)$ is a $1$-parameter line of hyperplanes of $\PP\Gamma(X,\mc L)^\lor$; in other words, $D$ is a map $\PP^1\to\PP\Gamma(X,\mc L)^\lor$.
\end{definition}
\begin{definition}
	Fix a projective variety $X$, and let $\mc L$ be a very ample line bundle on $X$. The \textit{axis} of a pencil $D\colon\PP^1\to\PP\Gamma(X,\mc L)^\lor$ is
	\[A(D)\coloneqq\bigcap_tD_t.\]
\end{definition}
\begin{remark}
	In fact, $A=D_0\cap D_\infty$
\end{remark}
\begin{definition}
	Fix a projective variety $X$, and let $\mc L$ be a very ample line bundle on $X$. Then a pencil $D\colon\PP^1\to\PP\Gamma(X,\mc L)^\lor$ is \textit{Lefschetz} if and only if it satisfies the following.
	\begin{listalph}
		\item Transverse: $A$ intersects $D$ transversely.
		\item Smoothness: the intersectios $X\cap D_t$ are nonsingular for all but finitely many $t$. Let $S\subseteq\PP^1$ be the singular locus.
		\item For each $t\in S$, the singularities $P$ in $X_t$ are single ordinary double points, meaning that
		\[\widehat\OO_{X,p}=\frac{k[[x_1,\ldots,x_d]]}{Q(x_1,\ldots,x_{d+1})},\]
		where $d=\dim X$ and $Q$ is a non-degenerate quadratic.
	\end{listalph}
\end{definition}
This definition is a little involved, so we should make sure that such things exist.
\begin{theorem}
	Fix a projective variety $X$ over an algebraically closed field $k$ of characteristic $0$. For any very ample sheaf $\mc L$, there is a Lefschetz pencil.
\end{theorem}
\begin{remark}
	If $\op{char}k>0$, then there is a Lefschetz pencil for some tensor power of $\mc L$.
\end{remark}
\begin{proof}[Sketch]
	Suppose $X\ne\PP\Gamma(X,\mc L)$, and we go ahead and assume that $X$ is geometrically connected. As usual, set $d\coloneqq\dim X$ and $N\coloneqq\dim\Gamma(X,\mc L)$. We are going to use incidence correspondence
	\[\Phi\coloneqq\{(x,H):x\in H\text{ and }T_xX\subseteq H\}.\]
	To explain this, note that the ``enemy'' is basically where a hyperplane is tangent to $X$; otherwise, the intersection is automatically transverse! The condition $T_xX\subseteq H$ is equivalent to the tangency because $H$ has codimension $1$. Note that $\Phi$ is a correspondence fitting in the diagram
	% https://q.uiver.app/#q=WzAsMyxbMSwwLCJcXFBoaSJdLFswLDEsIlgiXSxbMiwxLCJcXFBQXFxHYW1tYShYLFxcbWMgTCleXFxsb3IiXSxbMCwxLCJcXG9we3ByfV8xIiwyXSxbMCwyLCJcXG9we3ByfV8yIl1d&macro_url=https%3A%2F%2Fraw.githubusercontent.com%2FdFoiler%2Fnotes%2Fmaster%2Fnir.tex
	\[\begin{tikzcd}[cramped]
		& \Phi \\
		X && {\PP\Gamma(X,\mc L)^\lor}
		\arrow["{\op{pr}_1}"', from=1-2, to=2-1]
		\arrow["{\op{pr}_2}", from=1-2, to=2-3]
	\end{tikzcd}\]
	so we let $X^\lor$ denote the image of $X$ in $\PP\Gamma(X,\mc L)^\lor$. (We call $X^\lor$ the dual variety, though it notably also depends on $\mc L$.) Here are some properties of this correspondence.
	\begin{itemize}
		\item The projection $\op{pr}_1\colon\Phi\to X$ is a projective bundle. Indeed, at each $x\in X$, the fiber above $x$ simply consists of the hyperplanes containing $T_xX\subseteq T_x\PP\Gamma(X,\mc L)$. As such, each fiber is isomorphic to $\PP^{N-d-1}$, which we can see by working out some equations of hyperplanes. It follows that $\Phi$ is irreducible, projective, and has dimension $N-1$.
		\item The projection $\op{pr}_2\colon\Phi\to X^\lor$ has fibers consisting of the singular locus in $H\cap X$. Indeed, the fiber above $H$ is exactly the pairs $(x,H)$ where $H$ is tangent to $X$ at $x$. Approximately speaking, the ramification behavior of $\op{pr}_2$ controls the singularities we see.
	\end{itemize}
	Let's now sketch the rest of the proof. Bertini's theorem lets us find a single valid $H$ to start our Lefschetz pencil, so we just need to show that we can continue it to a pencil. Well, one simply controls the singularities of $X$ (they have positive codimension at least $2$), and then one argues that lines exist generically.
\end{proof}
Blowing up produces fibrations.
\begin{definition}[Lefschetz fibration]
	Fix a Lefschetz pencil $D$ for $(X,\mc L)$. Then the \textit{Lefschetz fibration} is the incidence correspondence $X^*$ sitting in the following diagram.
	% https://q.uiver.app/#q=WzAsMyxbMSwwLCJYXioiXSxbMCwxLCJYIl0sWzIsMSwiRCJdLFswLDFdLFswLDJdXQ==&macro_url=https%3A%2F%2Fraw.githubusercontent.com%2FdFoiler%2Fnotes%2Fmaster%2Fnir.tex
	\[\begin{tikzcd}[cramped]
		& {X^*} \\
		X && D
		\arrow[from=1-2, to=2-1]
		\arrow[from=1-2, to=2-3]
	\end{tikzcd}\]
\end{definition}
\begin{remark}
	It turns out that $X^*$ is the blow up of $X$ along $X\cap A(D)$.
\end{remark}
\begin{remark}
	It also turns out that the map $X^*\to\PP^1$ is proper, flat, and it admits a section.
\end{remark}

\subsection{Symplectic Monodromy}
Let's explain how we will get to use the Main lemma. Fix a nice variety $X\subseteq\PP^N$ over $\FF_q$, which we return to being $n$-dimensional, where $n=2m+1$. Then one can find a Lefschetz pencil over some extension of $\FF_q$: first find it over the algebraic closure, and then descend everything.
\begin{remark}
	It turns out that Lefschetz pencils also exist without doing the extensions. This is due to Poonen, Nguyen, and Gunther.
\end{remark}
Now, let $S\subseteq\PP^1$ be the collection of singular values, and we let $U\subseteq\PP^1\setminus S$ denote the complement. It turns out that the associated map $\pi\colon X^*\to\PP^1$ is now smooth and proper over $U$, so the higher pushforward $\mc V\coloneqq\mathrm R^n\pi_*\QQ_\ell$ is an $\ell$-adic local system over $U$ by the Proper base change theorem.

Thus, we are granted some representation
\[\pi_1(U)\to\op{GL}(V),\]
where $V$ is some fiber of $\mc V$. The tameness of the singularities of $X^*\to\PP^1$ turns out to provide tameness of the representation. To be more precise, we say something about inertia.
\begin{definition}
	For some ramified $s\in S$, let $\mathbb D_s$ be the formal disk $\Spec\widehat{\OO}_{\PP^1,s}$ so that the puncture $\mathbb D_s^\circ$ is its fraction field. Then the \textit{inertia subgroup} is the image of the map
	\[\pi_1(\mathbb D^\circ)\to\pi_1(U).\]
\end{definition}
\begin{remark}
	The map is only defined up to conjugation because it has suppressed moving some basepoints around.
\end{remark}
\begin{example}
	We work over $\CC$ for psychological reasons. Consider the Legendre family $\mc E\to\PP^1$ of elliptic curves given by
	\[\mc E_\lambda\colon Y^2Z=X(X-Z)(X-\lambda Z).\]
	This admits singularities at $\{0,1,\infty\}$, where we have nodal singularities. (There is someting variable change one has to do to produce a definition at $\lambda=\infty$.) Thus, we see that this is a Lefschetz pencil! So we set $U\coloneqq\PP^1\setminus\{0,1,\infty\}$, and we get a representation
	\[\pi_1(U)\to\op{GL}_2\left(\mathrm H^1(E_\eta;\QQ_\ell)\right),\]
	where $\eta$ is the generic point. It turns out that the monodromy loops $\gamma_0$ and $\gamma_1$ around $0$ and $1$ go to $\begin{bsmallmatrix}
		1 & 2 \\ 0 & 1
	\end{bsmallmatrix}$ and $\begin{bsmallmatrix}
		1 & 0 \\ 2 & 1
	\end{bsmallmatrix}$. There is apparently some geometric argument for this.
\end{example}
\begin{remark}
	In general, for such tame representations, it turns out that each element of the monodromy fixes a codimension $1$ subspace, and it modifies the remaining one-dimensional subspace in a controlled way; for example, it turns out that we should output a matrix of determinant $1$. This is (definitionally) a transvection. Eventually, one can hope to use these transvections to prove a big monodromy result.
\end{remark}
\begin{remark}
	In the specialization
	\[\mathrm H_1(\mc E_\eta;\QQ_\ell)\to\mathrm H_1(\mc E_s;\QQ_\ell),\]
	there is a cyclic which vanishes. Appropriately, this may be called a vanishing cycle.
\end{remark}
Thus, we see that we will be interested in some specializations. For example, letting $j\colon U\into\PP^1$ denote the inclusion, we may be interested in when the canonical map $\mc F\to j_*j^*\mc F$ is an isomorphism. It turns out that this is equivalent to the injectivity of the cospecialization map
\[\mc F_s\to\mc F_\eta\]
and has image in $\mc F_\eta^{I_s^{\mathrm{tame}}}$.
\begin{definition}[vanishing cycle]
	Fix everything as above and some $s\in S$. Then we define the space of \textit{vanishing cycles} to be the kernel of
	\[E_s\coloneqq\left(\mathrm H^n(X_{\ov\eta};\QQ_\ell(n))^\lor\to\mathrm H^n(X_s;\QQ_\ell(n))^\lor\right),\]
	where this is the dual of the cospecialization map.
\end{definition}
\begin{remark}
	It turns out that $E_s$ is one-dimensional and hence isomorphic to some Tate twist $\QQ_\ell(m)$, so we may choose a generator $\delta_s(-m)\in E_s(-m)$.
\end{remark}
\begin{remark}
	It turns out that there is an exact sequence
	\[0\to\mathrm H^n(X_s;\QQ_\ell)\to\mathrm H^n(X_\eta;\QQ_\ell)\stackrel{\delta_s}\to\QQ_\ell(m-n)\to0.\]
\end{remark}
The point of all this is that we are able to compute some transvections.
\begin{theorem}[Picard--Lefschetz]
	Fix everything as above, and choose $\sigma_s\in I_s$. For each $x\in\mathrm H^n(X_\eta;\QQ_\ell)$, we have
	\[\sigma_s(x)=x\pm t(\sigma_s)(x\cup\delta_s)\delta_s.\]
	Here, $t\colon I_s\to\ZZ_\ell(1)$ is the winding number; alternatively, it is the natural projection once $I_s$ is identified with $\widehat{\ZZ}(1)^{(p)}$, where the superscript means we are taking prime-to-$p$ roots of unity.
\end{theorem}
\begin{remark}
	One can verify that the Tate twists work out.
\end{remark}
\begin{remark}
	The sign $\pm$ is $(-1)^{(n+1)(n+2)/2}$. In particular, it only depends on $n\pmod4$.
\end{remark}
Next time, we will show that these transvections to prove a big monodromy result.

\section{November 7: Metaplectic Whittaker Functions and Quantum Groups}
This talk was given by Valentin Buciumas at Johns Hopkins University for the Number theory seminar.

\subsection{The Statements}
For today, $F$ is a nonarchimedean local field, $\OO\subseteq F$ is the ring of integers, $\pi$ is a uniformizer, and $q$ is the cardinality of the residue field. Let $G$ be a split reductive group over $F$, and let $K$ be a maximal compact subgroup.
\begin{theorem}[Satake--Langlands]
	The spherical Hecke algebra $\mc H(G,K)\coloneqq C_c^\infty(K\backslash G/K)$ is isomorphic to $\CC[\Lambda^\lor]^W$, which is isomorphic to $K_0(\mathrm{Rep}G^\lor)$.
\end{theorem}
The point is that unramified smooth representations are indexed by semisimple conjugacy classes of $G^\lor$. Here is a generalization to spherical varieties.
\begin{theorem}[Sakellaridis]
	Let $X$ be a spherical variety $H\backslash G$ satisfying some technical conditions. Then the Hecke algebra $\mc H(G,K)$ acts on $C_c^\infty(X/K)$, which is isomorphic to some $\CC[\delta_X^{1/2}\Lambda_X^\lor]^{W_X}$, which is isomorphic to some $\op{Rep}G_X^\lor$.
\end{theorem}
\begin{example}
	With $X=(\mathrm{GL}_n\times\mathrm{GL}_n)\backslash\mathrm{GL}_{2n}$ has $G_X=\mathrm{Sp}_{2n}$.
\end{example}
\begin{example}
	With $X=\mathrm{Sp}_{2n}\backslash\mathrm{GL}_{2n}$ has $G_X^\lor=\op{GL}_n$.
\end{example}
We are interested in generalizations to quantum groups. Suppose we want to pass from $G$ to a metaplectic cover $\widetilde G$. It is known that there is an isomorphism
\[\mc H(\widetilde G,K)\cong\op{Rep}G^\lor_{Q}\]
for some reductive group $G_Q$. It is then conjectured that one can descend this to
\[\left(\op{Ind}_U^{\widetilde G}\psi\right)^K=\mc W_\psi(\widetilde G,K)\cong\op{Rep}U_\xi\mf g^\lor.\]
Now, it remains to make sense of these terms and explain why one might conjecture this. There has been a lot of progress on these conjectures in the last few years.

\subsection{Metaplectic Groups}
Here is our definition.
\begin{definition}
	Fix a positive integer $n$, and choose symmetric bilinear form $B$ on $\Lambda^\lor$ which is $W$-invar\-iant (equivalently, one can use $Q$). Suppose that $q\equiv1\pmod n$ so that there is an embedding $\mu_n\subseteq F$. Then there is a covering
	\[1\to\mu_n\to\widetilde G\to G\to1\]
	which is known as the metaplectic group. In brief, for torus $T\subseteq G$, we have the commutation relation $[a^\lambda,b^\mu]=(a,b)^{B(\lambda,\mu)}$, where $(-,-)$ is the Hilbert symbol on $F$.
\end{definition}
\begin{remark}
	If $G$ is simple and simply connected, then there is essentially a unique primitive $n$-cover $\widetilde G$. In fact, there is further a splitting of $U$ and $K$ to $\widetilde G$.
\end{remark}
In the sequel, we will assume the existence of splittings $U\to\widetilde G$ and $K\to\widetilde G$.
\begin{notation}
	We define the Hecke algebra $\mc H(\widetilde G,K)$ to consist of smooth compactly supported functions $f$ on $\widetilde G$ for which
	\[f(\varepsilon k_1gk_2)=\varepsilon f(g)\]
	for any root of unity $\varepsilon$.
\end{notation}
\begin{notation}
	We define the Whittaker space as follows. Let $(\Lambda^\lor,\Phi^\lor,\Lambda,\Phi)$ be the root datum of $G$. Set
	\[\widetilde\Lambda^\lor\coloneqq\{\lambda\in\Lambda^\lor:B(\lambda,\mu)\equiv0\pmod n\text{ for }\mu\in\Lambda^\lor\}\]
	and $\widetilde\alpha_i=n_i\alpha_i$ where $n_i$ is the smallest such that $n_iW(\alpha_i^\lor)\equiv0\pmod n$. Define $\widetilde\Lambda$ and $\widetilde\alpha$ analogously, and we let $G_{(Q,n)}$
\end{notation}
\begin{example}
	Suppose $G$ is simply connected, and choose $Q$ so that $Q(\alpha_i)=1$ for each simple root $\alpha_i$. Then
	\[G^\lor_{(Q,n)}=\begin{cases}
		G^\lor & \text{if }n\text{is odd}, \\
		G & \text{if }n\text{ is even}.
	\end{cases}\]
	One can see this explicitly in some cases; for example, type $B$ and $C$ may swap as long roots become short ones.
\end{example}
\begin{theorem}
	There are isomorphisms
	\[\mc H(\widetilde G,K)\cong\CC[\widetilde\Lambda^\lor]^W\cong K_0\op{Rep}G^\lor_{(Q,n)}.\]
\end{theorem}
It is an observation of Lurie that the correct object to put on the other side of $\mc W_\psi(\widetilde G,K)$ is a quantum group.

\subsection{Quantum Groups}
By introducing some $q$s to the Serre relations, one can quantize $U\mf g$ to $U_q\mf g$, which produces a quantum group. Roughly speaking, the representation theory of $U\mf g$ is an abelian symmetric monoidal category, but $U_q\mf g$ is an abelian braided monoidal category.

Once we have $q$ in hand, there are a few ways to specialize $\xi$. The standard plugging in yields $U_\xi^{\mathrm{KC}}\mf g$. If we take some divided powers (replacing $e_i$ with $e_i/[k]_q!$ but none for the $f_i$s), then we get $U_\xi^{\mathrm{Lus}}\mf g$. It turns out that the representation theory of $U_\xi^{\mathrm{Lus}}\mf g$ looks like the representation theory of $G(k)$ in characteristic $p$ when $k$ has characteristic $p$.

For example, $\op{Rep}U_\xi^{\mathrm{Lus}}\mf g$ is category with some highest weight theory, and one can do Kazhdan--Lusztig combinatorics. Let's explain how to build some representations. Start with the representation theory of $G(\CC)$, and we can build some irreducible representations $V_\lambda$. Then $V_\lambda$ is in fact defined integrally, so one can extend it to $U_q\mf g$ and then specialize it to $U_\xi^{\mathrm{Lus}}\mf g$. The representation won't be irreducible after the specialization, but there is still some combinatorics with filtrations that one can do.

Returning to our metaplectic groups, there is quantum Frobenius
\[q\mathrm{Fr}\colon\op{Rep}U\mf g_{(Q,n)}^\lor\to\op{Rep}U_\xi\mf g^\lor,\]
which we will not discuss. Let's take stock.
\begin{itemize}
	\item Note $\mc H(\widetilde G,K)$ has a canonical basis $h_K$ given by the $\mu_nKgK$. Similarly, $\mc W_\psi(\widetilde G,K)$ has a canonical basis given $p$-adically.
	\item Note $K_0\op{Rep}G_{(Q,n)}^\lor$ has a canonical basis given by the $[V_\lambda]$s. Similarly, $K_q\op{Rep}U_\xi\mf g^\lor$ admits a basis from the $\lambda$s described above.
\end{itemize}
Our main theorem says that there is a way to go between the various bases listed in the ``similarly'' sentences. There is some precise formula involving the quantum Frobenius, some $\op{Ext}$-groups, and there are some strange Gauss sums.

\section{November 12: Counting Solutions to Generalized Fermat Equations}
This talk was given by Santiago Arango-Pi\~nerosat at Harvard for the Harvard number theory seminar.

\subsection{The Statement}
To start off, supposed that we are interested in counting rational numbers $Q$ such that the numerator is an $a$th power, the numerator of $a-1$ is a $b$th power, and the denominator is a $c$th power.

To make this precise, we consider $Q$ as a point in $\PP^1(\QQ)$, which is represented by a coprime pair $(s,t)\in\ZZ^2$ up to sign. The numerator is $s$, the denominator is $t$, and the numerator of $Q-1$ is $s-t$; let $\Omega(a,b,c)$ be the set of projective points of interest. The height $\op{ht}Q$ is simple $\left|s\right|+\left|t\right|$, and we will be interested in computing
\[\frac{N(\Omega(a,b,c);h)}{N(\PP^1(\QQ);h)},\]
where the $h$ means we are counting up to $h$. Assuming inddependence, we expect this fraction to be
\[\frac{h^{1+1/a}}{h^2}+\frac{h^{1+1/b}}{h^2}+\frac{h^{1+1/c}}{h^2}\sim\frac{h^\chi}{h^2},\]
where
\[\chi=\frac1a+\frac1b+\frac1c-1.\]
Notably, if $\chi<0$, one may expect to have finitely many solutions; when $\chi>0$, we expect $N(\Omega(a,b,c);h)$ to be proportional to $h^\chi$.
Here is our result.
\begin{theorem}
	Suppose that $\chi>0$. Then there is an explicit constant $\kappa$ for which $N(\Omega(a,b,c);h)$ is on the order of $\kappa h^\chi$.
\end{theorem}
Let's describe some previous work. Thus far, we may be interested in counting solutions to $x^a\pm y^b\pm z^c$, but we may more generally be interested in counting integral solutions to
\[F(x,y,z)=Ax^a+By^b+Cz^c.\]
Let $\Omega(F)$ be the corresponding set, and let $U$ be the scheme cut out by this equation (over $\ZZ$).
\begin{theorem}[Darmon--Granville]
	If $\chi<0$, then $\Omega(F)$ is finite.
\end{theorem}
\begin{theorem}[Beukers]
	The set $\Omega(F)$ can be parameterized by finitely many polynomials. Thus, if $\chi>0$, then there are infinitely many solutions.
\end{theorem}
\begin{theorem}[Ponnen--Schaefer--Stoll]
	There is an explicit classificatin of the solutions to $x^2+y^3+z^7=0$.
\end{theorem}
In the example $x^2+y^3+z^7=0$, one finds that $\mathbb G_m$ acts on $U$ by $\lambda\cdot(x,y,z)=(\lambda^{21}x,\lambda^{14}y,\lambda^6z)$. This is not quite a $\mathbb G_m$-torsor, but one can still take a quotient stack and study it that way.

To generalize this, consider the subgroup $H\subseteq\mathbb G_m^3$ cut out by the equatino $\lambda_0^a=\lambda_1^b=\lambda_\infty^c$. Then away from a finite set of primes $S$, one has
\[[U/H]_R\cong\PP^1(a,b,c)_R,\]
where $\PP^1(a,b,c)$ is some explicit stack which is just $\PP^1$ except it has $\mu_a$-action at $0$, $\mu_b$-action at $1$, and $\mu_c$-action on $c$. Then the method of Poonen--Schaever--Stoll is able to prove our main result.
\begin{theorem}
	Fix $F$ as above for which $\chi>0$ and $U(\ZZ)\ne0$. Then there is an explicitly computable constant $\kappa(F)>0$ such that
	\[N(\Omega(F);h)=\kappa(F)h^\chi+O_{\varepsilon,F}\left(h^{\chi/2+\varepsilon}\right).\]
\end{theorem}
The proof is done in three steps.
\begin{enumerate}
	\item Covering: one can find $\varphi\colon\PP^1_\QQ\to\PP^1_\QQ$ which is geometrically Galois with Galois group givnen by an explicit triangle group, Beyli of signature $(a,b,c)$ (meaning that it is ramified only at $\{0,1,\infty\}$ with the given ramifications). For example, for $(2,3,3)$, one can use modular curves.
	\item Twisting: one can spread $\varphi$ out to integral model over some $R=\ZZ_S$. It then turns out that $\PP^1(a,b,c)_R$ is $\left[\PP^1_R/\op{Aut}\Phi\right]$, which will let us present the points explicitly. In particular, $\Omega(F)$ embeds in
	\[\bigsqcup_{\tau\in\mathrm H^1(\QQ,\op{Aut}\Phi_{\ov\QQ})}\varphi_\tau(\PP^1_\tau(\QQ)),\]
	where $\varphi_\tau$ is some twist.
	\item Sieving: one can explicitly compute with $\mathrm H^1$ in order to enumerate the disjoint union. In particular, the problem reduces to counting the images of some rational maps, which reduces to the following.
\end{enumerate}
\begin{lemma}
	Let $\varphi\colon\PP^1_\QQ\to\PP^1_\QQ$ be a geometrically Galos map of degree $d$, then there exists an explicitly computable constant $\kappa(\varphi)>0$ such that
	\[N(\varphi(\PP^1(\QQ));h)=\kappa(\varphi)h^{2/d}+O_{\varphi,\varepsilon}(h^{1/d+\varepsilon}).\]
\end{lemma}
\begin{proof}[Sketch]
	This is basically done by counting points in some bounded region in the target, and then one uses the geometrically Galois hypothesis in order to count fibers.
\end{proof}

\section{November 12: The Hodge Conjecture}
This talk was given by Pierre Deligne at Harvard for the Millenium Prize Problems lectures.

\subsection{The Conjecture}
The talk began by introducing complex projective varieties, in steps. We started with the real line, completed to $\PP^1(\RR)$, added variables to get to $\PP^n(\RR)$, and then we passed to complex numbers. Then we introduced (Betti) homology via cycles. Cohomology was introduced using the de~Rham complex.

Hodge showed that there is an action by $S^1$ on the de Rham cohomology, which gives rise to the Hodge filtration. In particular, being in bidegree $(p,p)$ corresponds. Hodge conjectured that the algebraic cycles is the intersection of $\mathrm H^i_{\mathrm B}(X;\ZZ)$ and being in bidegree $(p,p)$. This turns out to be wrong, but it is still conjectured to be true up to tensoring by $\QQ$.

Here are some difficulties.
\begin{itemize}
	\item Proving the conjecture requires constructing cycles. For example, cutting by hyperplanes can only prove the conjecture in codimension $1$. Alternatively, one can try to decompose a given reducible cycle into maybe more interesting irreducible ones. For example, the projective surface $wz=xy$ contains the curve
	\[\left(s^3,s^2t,st^2,t^3\right).\]
	The decomposition here turns out to be complicated.
	\item Here is one way to disprove: given a Hodge cycle, one can look at the corresponding class in a different cohomology theory (via comparison theorems) such as \'etale, and we can ask if the corresponding cycle satisfies various criteria that algebraic cycles satisfy such as the Tate conjecture. However, this turns out to not work in most situations, and it is difficult to construct Hodge cycles anyway.
\end{itemize}
In short, one cares because the Hodge conjecture would produce a well-founded theory of motives. Note that Yves Andre has built a good theory of motives in characteristic zero by passing to motivated cycles, but there is some hope to do this in positive characteristic using abelian varieties.

\section{November 13: The Riemann Hypothesis}
This talk was given by Xinyu Zhou at MIT for the STAGE seminar.

\subsection{Reduction to the Blowup}
Today, we will use the technique of Lefschetz pencils in order to prove the Riemann hypothesis part of the Weil conjectures. In short, we would like to show that the action of the Frobenius on $\mathrm H^r(X;\QQ_\ell)$ has eigenvalues $\alpha$ with magnitude $q^{r/2}$. Note that we are allowed to extend the base field to prove this result.

By embedding $X$ diagonally into $X\times X$ and using the K\"unneth formula to expand out the cohomology in middle dimension $d$, we see that it is enough to prove the Riemann hypothesis for $\mathrm H^d(X\times X;\QQ_\ell)$. Thus, we may assume that $\dim X=2m+2$, and we will prove the result for $\mathrm H^{m+1}(X;\QQ_\ell)$. Additionally, by taking powers using a tensor-power trick, it is enough to merely check that each Frobenius eigenvalue $\alpha$ satisfies
\[q^{n/2}<\left|\alpha\right|<q^{n/2+1},\]
were $n+1=\dim X$.

We begin with a rather abstract result on weights.
\begin{definition}
	We say that an operator $F$ on a vector space $V$ satisfies $W_n$ if and only if all eigenvalues $\alpha$ of $F$ have
	\[q^{n/2}<\left|\alpha\right|<q^{n/2+1}.\]
\end{definition}
\begin{lemma} \label{lem:up-and-down-rh}
	Fix an operator $F$ on a vector space $V$.
	\begin{listalph}
		\item If $V$ satisfies $W_n$, and $W\subseteq V$ is some $F$-stable subspace, then both $W$ and $V/W$ satisfy $W_n$.
		\item If there is an $F$-stable filtration
		\[V\supseteq V_1\supseteq\cdots,\]
		and each $V_i/V_{i+1}$ satisfies $W_n$, then $V$ satisfies $W_n$.
	\end{listalph}
\end{lemma}
\begin{proof}
	This is a linear algebra exercise. Namely, (a) follows by suitably upper-triangularizing $F$ using $W$, and (b) follows similarly.
\end{proof}
We now set up some notation around Lefschetz pencils. Fix a very ample line bundle $\mc L$ on some $X$ inducing an embedding $X\into\PP\Gamma(X,\mc L)$, and we know that there is a Lefschetz pencil $D\colon\PP^1\to\PP\Gamma(X,\mc L)^\lor$. The axis will be denoted $A=D_0\cap D_\infty$, which is smooth, and we let $X^*$ be the blow up of $X$ along the axis. Thus, there is a surjection $\varphi\colon X^*\to X$ providing the blow-up, and there is a projection $\pi\colon X^*\to\PP^1$ for which the fiber over $t\in\PP^1$ is $X_t\coloneqq X\cap D_t$.
\begin{lemma}
	Fix everything as above. It suffices to prove the Riemann hypothesis for $X^*$.
\end{lemma}
\begin{proof}
	Let $N_{X/(A\cap X)}$ be the normal bundle of $A\cap X$ in $X$. It is a property of the Lefschetz pencil that
	\[\varphi^{-1}(A\cap X)=\PP N_{X/(A\cap X)}.\]
	For example, with $A\cap X$ of codimension in $2$, we see that the normal bundle has rank $2$.

	Thus, some theory of Chern classes provides a decomposition
	\[\mathrm H^*\left(\varphi^{-1}(A\cap X);\QQ_\ell\right)=\mathrm H^*(A\cap X;\QQ_\ell)\oplus\mathrm H^{*-2}(A\cap X;\QQ_\ell)(-1).\]
	In short, one can choose a class $\xi\in\mathrm H^2(\PP^n_X;\QQ_\ell)(1)$ corresponding to the line bundle $\OO(1)$, so we get a Lefschetz decomposition
	\[\mathrm H^*(\PP^n_X;\QQ_\ell)=\bigoplus_{i=0}^n\mathrm H^{*-2i}(X;\QQ_\ell)(-i)\xi^i.\]
	Indeed, this decomposition can be proven by reducing to the case where $X$ is affine, and then it follows by careful calculations of the cohomology of projective space (as one does for fields). Something similar holds for $\PP\mc E$ even when $\mc E$ is no longer a trivial bundle, which is the requested decomposition.

	For example, taking degree $0$ shows that there is an isomorphism $\QQ_\ell\to\varphi_*\QQ_\ell$. Additionally, one finds that $\mathrm R^\bullet\varphi_*\QQ_\ell$ is supported on $A\cap X$ in higher degrees (because $\varphi$ is an isomorphism away from $A\cap X$, so the fibers produce some trivial cohomology by proper base change). Further, the higher direct images $\mathrm R^\bullet\varphi_*\QQ_\ell$ vanishes outside degrees $0$ and $2$ (because the cohomology of the fiber at some $x\in A\cap X$ is cohomology of $\PP^1$, which is supported in degrees $0$ and $2$). Thus, the Leray spectral sequence
	\[E_2^{pq}=\mathrm H^p(X;\mathrm R^q\varphi_*\QQ_\ell)\Rightarrow\mathrm H^{p+q}(X^*;\QQ_\ell)\]
	degenerates,\footnote{It seems nontrivial to show that $d_3=0$, but it is true.} so we get a decomposition
	\[\mathrm H^*(X^*;\QQ_\ell)=\mathrm H^*(X;\QQ_\ell)\oplus\mathrm H^{*-2}(A\cap X;\QQ_\ell)(-1)\]
	from the spectral sequence. Thus, we obtain a splitting $\mathrm H^*(X;\QQ_\ell)\subseteq\mathrm H^*(X^*;\QQ_\ell)$, and the result follows by \Cref{lem:up-and-down-rh}.
\end{proof}

\subsection{The Three Groups}
We are now reduced to $X^*$, which we may understand by understanding the projection $\pi\colon X^*\to\PP^1$. Once again, we have a spectral sequence
\[E_2^{pq}=\mathrm H^p\left(\PP^1;\mathrm R^q\pi_*\QQ_\ell\right)\Rightarrow\mathrm H^{p+q}(X^*;\QQ_\ell).\]
Note that $\PP^1$ has cohomology supported in degrees $\{0,1,2\}$, so we only have $p\in\{0,1,2\}$. Additionally, we are only interested in $\mathrm H^{n+1}(X^*;\QQ_\ell)$, so we see that we are only interested in the groups
\[\mathrm H^0\left(\PP^1;\mathrm R^{n+1}\pi_*\QQ_\ell\right),\quad\mathrm H^1\left(\PP^1;\mathrm R^n\pi_*\QQ_\ell\right),\quad\text{and}\quad\mathrm H^2\left(\PP^1;\mathrm R^{n-1}\pi_*\QQ_\ell\right).\]
We would like to show all three groups satisfy $W_n$, so $W_n$ is satisfied by any subquotient by \Cref{lem:up-and-down-rh}, and the result will follow for $\mathrm H^{n+1}(X^*;\QQ_\ell)$.
\begin{remark}
	Weil II tells us that each of the sheaves $\mathrm R^\bullet\pi_*\QQ_\ell$ have an expected weight. Thus, we see that the rest of the proof amounts to proving some special cases of Weil II. This explains why it is tricky: Weil II is genuinely hard!
\end{remark}
Each of the these groups will be handled separately. Let's quickly handle the left and right groups.
\begin{itemize}
	\item To handle $\mathrm H^2\left(\PP^1;\mathrm R^{n-1}\pi_*\QQ_\ell\right)$, we recall that our vanishing cycles live in $\mathrm H^n(X_t;\QQ_\ell)=\left(\mathrm R^n\pi_*\QQ_\ell\right)_t$, where $t\in\PP^1$, so we receive a specialization map
	\[\mathrm H^*(X_t;\QQ_\ell)\to\mathrm H^*(X_\eta^*;\QQ_\ell)\]
	which is an isomorphism. This amounts to saying that all the fibers of $\mathrm R^{n-1}\pi_*\QQ_\ell$ is constant on $\PP^1$, so we can compute its cohomology as
	\[\mathrm H^2\left(\PP^1;\mathrm R^{n-1}\pi_*\QQ_\ell\right)=\left(\mathrm R^{n-1}\pi_*\QQ_\ell\right)_t(-1)=\mathrm H^{n-1}(X_t;\QQ_\ell)(-1).\]
	Now, let $Y\subseteq X_t$ be some smooth hyperplane section so that $X_t\setminus Y$ is affine, so $\mathrm H^{n-1}(X_t\setminus Y;\QQ_\ell)=\mathrm H^{n+1}(X_t\setminus Y;\QQ_\ell)=0$ by Poincar\'e duality, so excision tells us that we have an injection
	\[\mathrm H^{n-1}_c(X_t\setminus Y;\QQ_\ell)\to\mathrm H^{n-1}(X_t;\QQ_\ell)\to\mathrm H^{n-1}(Y;\QQ_\ell).\]
	We are thus reduced to proving the statement to $Y$, which has smaller dimension than $X$, so we may induct down.
	\item We omit details for the calculation for $\mathrm H^0\left(\PP^1;\mathrm R^{n+1}\pi_*\QQ_\ell\right)$ because it is done with similar tricks. In short, one again finds that $\mathrm R^{n+1}\pi_*\QQ_\ell$ is constant, and then the Gysin sequence provides a surjection
	\[\mathrm H^{n-1}(Y;\QQ_\ell)(-1)\to\mathrm H^{n+1}(X_t;\QQ_\ell),\]
	so we are done by an induction.
\end{itemize}
We now move on to the (hardest) middle cohomology group $\mathrm H^1\left(\PP^1;\mathrm R^n\pi_*\QQ_\ell\right)$. Let $S\subseteq\PP^1$ be the locus of singular fibers, and let $U$ be the complement of $S$, and we distinguish some basepoint $u\in U$. For brevity, set $V\coloneqq(\mathrm R^n\pi_*\QQ_\ell)_u$, and we let $E\subseteq V$ be the vanishing cycles. The cup product provides a symplectic pairing $V\times V\to\QQ_\ell(-n)$, so we may let $E^\perp\subseteq V$ be the orthogonal complement.
\begin{remark}
	The Hard Lefschetz theorem would imply that $E\cap E^\perp=0$. However, the first proof of the Hard Lefschetz theorem was strictly harder than the proof of the Riemann hypothesis.
\end{remark}
Without knowing that $E\cap E^\perp$ is trivial, we can still consider the filtration
\[V\supseteq E\supseteq E\cap E^\perp\supseteq0.\]
To start off, we note that any generator $\sigma_s$ in the inertia subgroup at $s$ of $\pi_1(U)$ has
\[\sigma_s(x)=x\pm t(\sigma_s)(x\cup\delta_s)\delta_s,\]
so one can check that $\pi_1(U)$ acts trivially on $V/E$ and $E\cap E^\perp$. As such, we may extend the above filtration into
\[\mc V\supseteq\mc E\supseteq\mc E\cap\mc E^\perp=0\]
of sheaves on $U$, where $\mc V=\mathrm R^n\pi_*\QQ_\ell|_U$. The aforementioned trivial action by $\pi_1$ implies that $\mc V/\mc E$ and $\mc E\cap\mc E^\perp$ are both constant sheaves. Pushing forward along $j\colon U\to\PP^1$, we ge a filtration
\[\mathrm R^n\pi_*\QQ_\ell\supseteq j_*\mc E\supseteq j_*(\mc E\cap\mc E^\perp)=0.\]
The main lemma is applied to the non-constant quotient.
\begin{lemma}
	For each $x\in U$, the action of $F_x$ on $\mc E/(\mc E\cap E^\perp)$ is rational. In other words, the characteristic polynomial has rational coefficients.
\end{lemma}
\begin{proof}
	Omitted.
\end{proof}
From here, one applies the Main lemma to $E/(E\cap E^\perp)$. For example, the cup-product gives us our symplectic pairing, and it is a theorem of Kazhdan--Margulis that the monodromy group has open image, so we complete.

It remains to handle the constant sheaves. There are two cases: either $E/(E\cap E^\perp)\ne0$ or $E\subseteq E^\perp$. We will focus on the first case because it is easier. It turns out that this means that there are no vanishing cycles in $E\cap E^\perp$ because having any vanishing cycles in $E\cap E^\perp$ implies that all of them are in there by the Picard--Lefschetz formula. Observe that there is an exact sequence
\[0\to j_*\mc E\to j_*\mc V\to j_*(\mc V/\mc E)\to0\]
of sheaves on $\PP^1$. Observe $j_*(\mc V/\mc E)$ is constant, so the map $\mathrm H^1\left(\PP^1;j_*\mc E\right)\to\mathrm H^1\left(\PP^1;\mathrm R^n\pi_*\QQ_\ell\right)$, is surjective, so we are reduced to handling $\mathrm H^1\left(\PP^1;j_*\mc E\right)$. For this, we note that we have a short exact sequence
\[0\to j_*(\mc E\cap\mc E^\perp)\to j_*\mc E\to j_*(\mc E/\mc E\cap\mc E^\perp)\to0.\]
Again, the left sheaf is constant, so we receive an injection
\[\mathrm H^1\left(\PP^1;j_*\mc E\right)\to\mathrm H^1\left(\PP^1;j_*(\mc E/\mc E\cap\mc E^\perp)\right).\]
Thus, we reduce to the quotient covered by the Main lemma.

\section{November 17: The \texorpdfstring{$p$}{p}-adic Simpson Correspondence}
This talk was given by Oakley at Harvard for the Mark Kisin seminar.

For today, $C$ is a complete, algebraically closed nonarchimedean field over $\QQ_p$, we let $B^+_{\mathrm{dR}}/t^2$ be the usual ring, which is a non-split extension
\[0\to C(1)\to B^+_{\mathrm{dR}}/t^2\to C\to0.\]
We alo let $X/C$ be a smooth proper rigid space, and we let $\widetilde\Omega^1_X$ be $\Omega^1_{X/C}(-1)$.

\subsection{Complex Nonabelian Hodge Theory}
Let's start by recalling the story for complex manifolds. Let $(X,\omega)$ be a compact K\"ahler manifold so that there is a Hodge decomposition
\[\mathrm H^n(X;\CC)=\bigoplus_{p+q=n}\mathrm H^p(X;\Omega_X^q).\]
We are interested in generalizing this to non-constant local systems.
\begin{definition}[Higgs bundle]
	Fix a compact K\"ahler manifold $X$. Then a \textit{Higgs bundle} $(E,\theta)$ on $X$ is a holomorphic vector bundle $E$ on $X$ together with an $\mathcal O_X$-linear map $\theta\colon X\to E\otimes\Omega^1_X$ satisfying $\theta\land\theta=0$.
\end{definition}
\begin{remark}
	One can reinterpret $\theta$ as a map $(\Omega^1_X)^\lor\to\op{End}(E)$. Then commutativity condition amounts to requiring that $\theta$ factors through $\op{Sym}$.
\end{remark}
\begin{theorem}[Simpson, Corlette, Dondaldson]
	Fix a smooth projective variety $X$ over $\CC$, and let $[\omega]$ be a hyperplane class (coming from a projective embedding). Then there is an equivalence between the category of local systems $\op{Loc}_\CC(X)$ and the category of semistable Higgs bundles $(E,\theta)$ for which $c_i(E)_\theta=0$ for all Chern classes $c_i$.
\end{theorem}
We will ignore semistability and the Chern classes for now because we will not mention such things in the $p$-adic situation.
\begin{remark}
	If $\mc L$ is some local system with associated Higgs bundle $(E,\theta)$, then $\mathrm H^n_{\mathrm B}(X;\mc L)$ is isomorphic to the hypercohomoloy
	\[\mathbb H^n\left(X;[E\stackrel\theta\to X\otimes\Omega^1_X\stackrel\theta\to X\otimes\Omega^2_X\to\cdots]\right).\]
\end{remark}

\subsection{\texorpdfstring{$p$}{p}-adic Nonabelian Hodge Theory}
For our main result, we need a site.
\begin{definition}
	Fix a rigid analytic space $X$. Then the \textit{pro\'etale site} consists of cofiltered systems $\{U_i\}\to X$ where each $U_i\to X$ is \'etale and the internal maps are \'etale surjective for sufficiently large indices.
\end{definition}
Roughly speaking, the coverings are given by jointly surjective maps, but there are some set-theoretic issues.
\begin{remark}
	There is a morphism $j\colon X_{\mathrm{pro\acute et}}\to X_{\mathrm{\acute et}}$ of sites.
\end{remark}
\begin{example}
	There is a sheaf $\OO_X$ on the pro\'etale site, which comes from the \'etale site.
\end{example}
\begin{example}
	Similarly, there is a completed sheaf $\widehat{\OO}_X$ which is
	\[\left(\lim_nj^{-1}\OO_X^+/p^n\right)[1/p].\]
	For example, this sheaf takes a perfectoid $\op{Spa}(R,R^+)\to X$ to $R$.
\end{example}
\begin{example}
	There is a sheaf $\mathbb B^+_{\mathrm{dR}}$ which takes perfectoid $\op{Spa}(R,R^+)\to X$ to $A_{\mathrm{int}}(R^+)[1/p]^\land_{\ker\theta}$.
\end{example}
Let's begin by explaining Hodge theory over $C$.
\begin{theorem}[Faltings, Scholze, Conrad--Gabber]
	Fix a smooth proper rigid analytic space $X/C$ admits a filtration $\op{Fil}^\bullet_{\mathrm{HT}}\mathrm H^n_{\mathrm{\acute et}}(X;C)$ for which
	\[\op{gr}^p_{\mathrm{HT}}\mathrm H^n_{\mathrm{\acute et}}(X;C)=\mathrm H^{n-p}(X;\widetilde\Omega^p_X).\]
	Furthermore, this filtration splits after choosing a flat lift of $X$ to $B_{\mathrm{dR}}^+/t^2$.
\end{theorem}
Let's see an example.
\begin{definition}[Higgs--Tate torsor]
	Suppose we have a lift $\mathbb X$ of $X$ to $B^+_{\mathrm{dR}}/t^2$; observe that $\mathbb X$ and $X$ have the same topologies. Notate $\lambda$ as the mapping $X_{\mathrm{pro\acute et}}\to X_{\mathrm{\acute et}}$. We define $\mc L_{\mathbb X}\subseteq\op{Hom}\left(\lambda^{-1}\OO_{\mathbb X},\mathbb B^+_{\mathrm{dR}}/t^2\right)$ to be given by diagrams
	% https://q.uiver.app/#q=WzAsNCxbMCwwLCJcXGxhbWJkYV57LTF9XFxPT197XFxtYXRoYmIgWH0iXSxbMSwwLCJcXG1hdGhiYiBCXitfe1xcbWF0aHJte2RSfX0vdF4yIl0sWzEsMSwiXFx3aWRlaGF0XFxPT19YIl0sWzAsMSwiXFxsYW1iZGFeey0xfVxcT09fWCJdLFswLDEsIiIsMCx7InN0eWxlIjp7ImJvZHkiOnsibmFtZSI6ImRhc2hlZCJ9fX1dLFswLDNdLFszLDJdLFsxLDJdXQ==&macro_url=https%3A%2F%2Fraw.githubusercontent.com%2FdFoiler%2Fnotes%2Fmaster%2Fnir.tex
	\[\begin{tikzcd}[cramped]
		{\lambda^{-1}\OO_{\mathbb X}} & {\mathbb B^+_{\mathrm{dR}}/t^2} \\
		{\lambda^{-1}\OO_X} & {\widehat\OO_X}
		\arrow[dashed, from=1-1, to=1-2]
		\arrow[from=1-1, to=2-1]
		\arrow[from=1-2, to=2-2]
		\arrow[from=2-1, to=2-2]
	\end{tikzcd}\]
	of sheaves of algebras over $\mathbb B^+_{\mathrm{dR}}/t^2$. (There is some deformation-theoretic argument which explains why this is a torsor.)
\end{definition}
\begin{example}
	Suppose $X$ is a curve, and let $\mathbb X$ be a lift of $X$ to $B^+_{\mathrm{dR}}/t^2$. Then we are interested in defining a splitting for
	\[0\to\mathrm H^1(X;\OO_X)\to\mathrm H^1_{\mathrm{\acute et}}(X;C)\to\mathrm H^0(X;\widetilde\Omega^1_X)\to0.\]
	By some comparison theorems, the middle cohomology group is isomorphic to $\mathrm H^1_{\mathrm{pro\acute et}}(X;\widetilde\OO)$, so it is enough to define a splitting $\mathrm H^0(X;\widetilde\Omega^1_X)\to\mathrm H^1_{\mathrm{pro\acute et}}(X;\widehat\OO)$. For this, we send some $\theta\in\mathrm H^0$ to the fiber product $\widehat\OO_X\times_\theta\mathcal L_{\mathbb X}$.
\end{example}
Here is our $p$-adic nonabelian Hodge theory, which is the $p$-adic Simpson correspondence.
\begin{theorem}[Faltings--Hever]
	Let $X$ be a smooth proper rigid space over $C$, and choose a lift $\mathbb X$ to $B^+_{\mathrm{dR}}/t^2$. Then we admit an exponential $\exp\colon C\to(1+\mf m_C)$ and a $\otimes$-equivalence
	\[\op{Vect}(X_{\mathrm{pro\acute et}};\widehat\OO)\to\op{Higgs}(X),\]
	where $\op{Higgs}(X)$ is a(n \'etale) Higgs bundle (defined as earlier). This equivalence is functorial in $\mathbb X$, and
	\[\mathbb H^n\left(X;[E\stackrel\theta\to X\otimes\Omega^1_X\stackrel\theta\to X\otimes\Omega^2_X\to\cdots]\right).\]
\end{theorem}
Of course, the Simpson correspondence is about local systems, so we should explain how we get local systems.
\begin{theorem}[Faltings, Scholze]
	There is a fully faithful $\otimes$-functor $\op{Loc}_C(X_{\mathrm{pro\acute et}})\to\op{Vec}(X_{\mathrm{pro\acute et}};\widehat\OO)$, compatible with cohomology.
\end{theorem}
The moral is that our $p$-adic Simpson correspondence depends on two choices: we must choose a lift $\mathbb X$ and an exponential map $\exp$.

\subsection{The Proof for Line Bundles}
We will spend the rest of the talk explaining the proof of the $p$-adic Simpson correspondence. Let's start with line bundles. To start, we define a Picard sheaf.
\begin{definition}
	Fix a rigid analytic space $X$. Then we let $\op{Rig}_{C,\mathrm{\acute et}}^{\mathrm{sm}}$ to be the category of smooth rigid spaces over $C$, equipped with the \'etale topology. Then we define the \textit{\'etale Picard sheaf} $\op{Pic}_{X,\mathrm{\acute et}}$ by
	\[S\mapsto\frac{\mathrm H^1_{\mathrm{\acute et}}(X\times S;\OO^\times)}{\mathrm H^1_{\mathrm{\acute et}}(S;\OO^\times)}.\]
	There is an analogously defined $\op{Pic}_{X,\mathrm{pro\acute et}}$ which uses $\widehat\OO$ instead of $\OO$.
\end{definition}
\begin{proposition}
	There is a short exact sequence
	\[0\to\mathrm{Pic}_{X,\mathrm{\acute et}}\to\mathrm{Pic}_{X,\mathrm{pro\acute et}}\to\mathrm H^0(X;\widetilde\Omega^1_X)\otimes\mathbb G_a\to0.\]
\end{proposition}
\begin{proof}
	Left-exactness comes from the Leray spectral sequence applied to the composite of sites $X_{\mathrm{pro\acute et}}\stackrel\nu\to X_{\mathrm{\acute et}}\to*$ applied to the sheaf $\widetilde\OO^\times$. (One needs to compute that $\mathrm R^1\nu_*\widehat\OO^\times=\mathrm R^1\nu_*\widehat\OO=\widetilde\Omega^1_X$, where the left equality is by using the exponential, and the right equality is due to Scholze.) Right exactness is harder, and I did not follow it.
\end{proof}
\begin{remark}
	On Lie algebras, we get an exact sequence
	\[0\to\mathrm H^1(X;\OO)\to\mathrm H^1_{\mathrm{pro\acute et}}(X;\widehat\OO)\to\mathrm H^0(X;\widetilde\Omega^1_X)\to0.\]
\end{remark}
We are now in the following situation.
% https://q.uiver.app/#q=WzAsMTAsWzAsMCwiMCJdLFsxLDAsIlxcb3B7UGljfV97WCxcXG1hdGhybXtcXGFjdXRlIGV0fX0oQykiXSxbMiwwLCJcXG9we1BpY31fe1gsXFxtYXRocm17cHJvXFxhY3V0ZSBldH19KEMpIl0sWzMsMCwiXFxtYXRocm0gSF4wKFg7XFx3aWRldGlsZGVcXE9tZWdhXjFfWCkiXSxbNCwwLCIwIl0sWzAsMSwiMCJdLFsxLDEsIlxcbWF0aHJtIEheMV97XFxtYXRocm17XFxhY3V0ZSBldH19KFg7XFxPTykiXSxbMiwxLCJcXG1hdGhybSBIXjFfe1xcbWF0aHJte3Byb1xcYWN1dGUgZXR9fShYO1xcd2lkZWhhdFxcT08pIl0sWzMsMSwiXFxtYXRocm0gSF4wKFg7XFx3aWRldGlsZGVcXE9tZWdhXjFfWCkiXSxbNCwxLCIwIl0sWzAsMV0sWzEsMl0sWzIsM10sWzMsNF0sWzMsOCwiIiwwLHsibGV2ZWwiOjIsInN0eWxlIjp7ImhlYWQiOnsibmFtZSI6Im5vbmUifX19XSxbNyw4XSxbNiw3XSxbNSw2XSxbOCw5XSxbNywyLCIiLDEseyJzdHlsZSI6eyJib2R5Ijp7Im5hbWUiOiJkYXNoZWQifX19XV0=&macro_url=https%3A%2F%2Fraw.githubusercontent.com%2FdFoiler%2Fnotes%2Fmaster%2Fnir.tex
\[\begin{tikzcd}[cramped]
	0 & {\op{Pic}_{X,\mathrm{\acute et}}(C)} & {\op{Pic}_{X,\mathrm{pro\acute et}}(C)} & {\mathrm H^0(X;\widetilde\Omega^1_X)} & 0 \\
	0 & {\mathrm H^1_{\mathrm{\acute et}}(X;\OO)} & {\mathrm H^1_{\mathrm{pro\acute et}}(X;\widehat\OO)} & {\mathrm H^0(X;\widetilde\Omega^1_X)} & 0
	\arrow[from=1-1, to=1-2]
	\arrow[from=1-2, to=1-3]
	\arrow[from=1-3, to=1-4]
	\arrow[from=1-4, to=1-5]
	\arrow[equals, from=1-4, to=2-4]
	\arrow[from=2-1, to=2-2]
	\arrow[from=2-2, to=2-3]
	\arrow[dashed, from=2-3, to=1-3]
	\arrow[from=2-3, to=2-4]
	\arrow[from=2-4, to=2-5]
\end{tikzcd}\]
The bottom-right arrow already admits a splitting $s_{\mathbb X}$ (as discussed earlier), and our goal is to show that top-right arrow admits a splitting. Thus, it is enough to exhibit the middle vertical arrow.
\begin{theorem}[Warner]
	The sheaf $\op{Pic}_{X,\mathrm{\acute et}}$ is representable.
\end{theorem}
It follows that $\op{Pic}_{X,\mathrm{pro\acute et}}$ is representable by the exact sequence above. Thus, we can view the top row is an exact sequence of rigid groups, and the map to the Lie algebra admits a splitting known as the exponential!

One can now compute the functor taking Higgs bundles to line bundles as given by sending the Higgs bundle $(E,\theta)$ to $\nu^*E\otimes_{\widehat\OO}\mc L_\theta$. Here, $\mc L_\theta$ is $\exp(s_{\mathbb X}(\theta))$, where $\exp$ is the exponential map described in the previous paragraph. There is also a description of the inverse functor given by sending $V$ to $\nu_*(V\otimes_{\widehat\OO}\mc L_\theta^{-1})$, where $\theta_V$ is the Higgs bundle corresponding to $V$, and it is not hard to check that these are inverse isomorphisms.
\begin{remark}
	Once we have a splitting of 
	\[0\to\mathrm{Pic}_{X,\mathrm{\acute et}}(C)\to\mathrm{Pic}_{X,\mathrm{pro\acute et}}(C)\to\mathrm H^0(X;\widetilde\Omega^1_X)\otimes\mathbb G_a\to0\]
	on the right, we know that there should be a splitting on the left as well. The functors defined above tell us that this splitting takes $V$ to $\mc L_{\theta_V}$.
\end{remark}

\subsection{Gesturing towards Higher Rank}
We now give some ideas of how to generalize this story to higher rank.
\begin{notation}
	Fix a Higgs bundle $(E,\theta)$ on $X$, and let $B$ denote the image of $\theta\colon\op{Sym}(\widetilde\Omega^1_X)^\lor\to\op{End}(E)$. Then $B$ is a coherent $\OO_X$-algebra, so we set $X'\coloneqq\op{Spa}B$ and gain a finite map
	\[\pi\colon X'\to X.\]
\end{notation}
\begin{remark}
	If $(E,\theta)$ admits an action by $\pi_*\OO_{X'}$, then $E$ becomes a coherent sheaf on $X'$.
\end{remark}
\begin{remark}
	If $\pi$ is \'etale, then $E$ will lift to a line bundle on $X'$. Thus, we are able to apply the line bundle case to produce a $p$-adic Simpson correspondence. However, when $\pi$ is not \'etale, then $\pi_*$ does not preserve being vector bundles on the pro\'etale site. In fact, $\pi_*$ does not even preserve coherence!
\end{remark}
We would like to upgrade the argument in the previous remark to work even if $\pi$ fails to be \'etale. The solution is to write the same proof but base everything on $X$. For example, one defines a more general Picard functor for a coherent algebra $B$ on $X$ by
\[\op{Pic}_{B/X,\mathrm{\acute et}}\colon S\mapsto\frac{\mathrm H^1_{\mathrm{\acute et}}X\times S;B^\times)}{\mathrm H^1_{\mathrm{\acute et}}(S;\op{pr}_{2*}B^\times)}.\]
All the arguments now upgrade: there is a short exact sequence of sheaves as before, and then we want to induce some splitting from group sheaves to their Lie algebras, which is done by choosing an exponential. However, this time the Picard functor is not known to be representable, so one has to work harder to define the exponential maps (e.g., by rigidifying all line bundles). There is a description of the functor $\op{Higgs}(X)\to\op{Vect}(X_{\mathrm{pro\acute et}};\widehat\OO)$ as before. It turns out that describing the inverse functor is harder. Roughly speaking, the problem is the construction of a $B$ when given a vector bundle, but there is a way to do this.

\section{November 18: \texorpdfstring{$p$}{p}-adic Hyperboliciy of Shimura Varieties}
This talk was given by Ananth Shankar at MIT for the BC--MIT number theory semianr. This is joint work with Ben Bakker, Oswal, and Yao.

\subsection{The Complex Story}
Let $Y$ be a complex modular curve. Then the connected components of $Y$ all look like $\mathbb H/\Gamma$, where $\Gamma$ is some arithmetic subgroup.
\begin{theorem}[Picard]
	Suppose that $\Gamma\subseteq\op{PSL}_2(\RR)$ is torsion-free. Then every holomorphic function $f\colon D^\times\to Y$ extends to $D\to Y^{\mathrm{BB}}$, where $(-)^{\mathrm{BB}}$ denotes the Bailey--Borel compactification.
\end{theorem}
\begin{proof}
	Given a correspondence
	% https://q.uiver.app/#q=WzAsMyxbMCwxLCJZIl0sWzEsMCwiXFx3aWRldGlsZGUgWSJdLFsyLDEsIlknIl0sWzEsMF0sWzEsMl1d&macro_url=https%3A%2F%2Fraw.githubusercontent.com%2FdFoiler%2Fnotes%2Fmaster%2Fnir.tex
	\[\begin{tikzcd}[cramped]
		& {\widetilde Y} \\
		Y && {Y'}
		\arrow[from=1-2, to=2-1]
		\arrow[from=1-2, to=2-3]
	\end{tikzcd}\]
	where each arrow is finite \'etale, one can pass the theorem from $Y$ to $Y'$. Using correspondences like this, we may pass to $\Gamma=\Gamma_2$. Then $Y_{\Gamma_2}=\PP^1\setminus\{0,1,\infty\}$. Picard's theorem then tells us that $f\colon D^\times\to\PP^1\setminus\{0,1,\infty\}$ has at most poles at $0$, so we may extend to $D$.
\end{proof}
We are looking for generalizations to Shimura varieties $S$. The fact we will need is that Shimura varieties have connected components which look like $X/\Gamma$, where $X$ is a Hermitian symmetric domain, and $\Gamma$ is some congruence subgroup.
\begin{theorem}[Borel]
	If $\Gamma$ is torsion-free, then every holomorphic map $(D^\times)^a D^b\to S$ extends to $D^a\times D^b\to S^{\mathrm{BB}}$.
\end{theorem}
\begin{corollary}
	Let $V$ be a complex algebraic variety. Then every holomorphic map $V\to S$ is algebraic.
\end{corollary}
\begin{proof}
	Use GAGA.
\end{proof}
\begin{remark}
	Both properties fail when $S$ is replaced by an arbitrary variety. Roughly speaking, the problem with extension is the existence of essential singularities. For example, we can consider the exponential map $\exp\colon\CC\to\CC^\times$.
\end{remark}
We are thus motivated to study the two following properties.
\begin{definition}
	A variety $S$ satisfies \textit{extension} if and only if any map $(D^\times)^a\times D^b\to S$ extends to a map $D^a\times D^b\to\overline S$, for some completion $\overline S$. If $S$ is algebraic, it then satisfies \textit{algebraicity} if and only if any map $V\to S$ from an algebraic variety is in fact algebraic.
\end{definition}

\subsection{The \texorpdfstring{$p$}{p}-adic Statements}
Work of many people (Shimura, Deligne, Borovoi, and Milne) allows one to descend a Shimura variety to be defined over a canonical number field. Thus, we can ask such extension questions over $p$-adic completions. Here is a main theorem.
\begin{theorem}
	Let $S$ be a Shimura variety of abelian type defined over $E$. For any prime $\mf p$ of $\OO_E$ and complete discrete valuation field $K$, any rigid analytic map $f\colon(D^\times)^a\times D^b\to S$ defined over $K$ extends to a map $f\colon D^a\times D^b\to S^{\mathrm{BB}}$.
\end{theorem}
The idea is to reduce to $\mc A_g$. For example, this grants us access to integral models, and certain interesting local systems. There is also an algebraicity corollary, as before.
\begin{corollary}
	Let $S$ be a compact Shimura variety of Hodge type, which we embed into $\mc A_{g,6}$. Then the extension and algebraicity properties hold for the universal abelian scheme over $S$.
\end{corollary}
However, here is our main theorem for today.
\begin{theorem}
	Let $S$ be a nonabelian Shimura variety defined over $E$. Alternatively, we may let $S$ be a geometric perid image. For sufficiently large primes $p$ (relative to $S$), choose a prime $\mf p$ of $E$ above $p$, and fix some complete discretely valued field $K$ containing $E_{\mf p}$. Then any map $S\to D^{\times a}\times D^b\to S$ defined over $K$ extends to a map $D^a\times D^b\to S$, provided that $\im f$ has nonempty intersection with the good reduction locus.
\end{theorem}
Observe that we did not have to compactify $S$ for this result!
\begin{remark}
	The good reduction locus of $S$ means what we expect it means if $S$ were (say) $\mc A_g$. For example, if $S$ is compact, then the good reduction locus is everything.
\end{remark}
\begin{remark}
	The large prime $p$ hypothesis is used many times in the proof. For example, integral models are only known to exist for large $p$. 
\end{remark}
There is once again an algebraicity corollary.

\subsection{Sketch of Proofs}
Let's sketch the argument for $\mc A_g$. We implicitly add enough level structure to avoid stacky issues and so on. Here are the steps.
\begin{enumerate}
	\item One can show that any map $f\colon D^\times\to\mc A_g$ either has image contained in the good reduction locus or contained in the bad reduction locus. The point is to use the N\'eron--Ogg--Shafaverich criterion to test reduction $\ell$-adically.
	\item The various reduction strata are uniformized by Rapoport--Zink spaces. In the event that $f$ has image contained in the good reduction locus, then one can show that it is in the image of some uniformization $j\colon\mathrm{RZ}\to\mc A_g$. (This uses integral models.)
	\item Furthermore, it turns out that $f$ lifts to $\widetilde f\colon D^\times\to\mathrm{RZ}$. (This also uses integral models.)
	\item Taking de Rham cohomology of the universal abelian scheme, one gets a vector bundle with flat connection $(V,D)$, which we can then pull back to $j^*(V,D)$ on $D^\times$ on $\mathrm{RZ}$. It turns out that it admits a trivialization on $\mathrm{RZ}$.
	\item It turns out that $f^*(V,D)$ being trivial extends to $D$. A $p$-adic Riemann hypothesis then shows that $f^*T_p$ extends to $0$, so we get something crystalline at $0$.
	\item Lastly, we use period maps to extend the map $D^\times\to\mathrm{RZ}$ and the map from $D$ to the flag variety (given by the previous step) to a map $D\to\mathrm{RZ}$.
\end{enumerate}
Most of this argument does not work for exceptional type Shimura varieties. However, it is known that one can still get a crystalline local system $\mathbb V_p$ on $S_{E_{\mf p}}$, which then extends to a log Fontaine--Lafaille module on the integral model of the toroidal compactification. One then gets objects on the de Rham site, the crystalline site, and the $\ell$-adic sites. After proceeding as in step 1 above, one proves the following general result.
\begin{theorem}
	Let $\mc S$ be a smooth scheme over the Witt vectors $W$, and let $\mc S^{\mathrm{rig}}$ denote the rigid geometric fiber. Further, suppose that $\mathbb L$ is some crystalline local system. Assume that the associated Fontaine--Lafaille module $\mathbb V_{\mathrm{FL}}$ satisfies one of the following.
	\begin{itemize}
		\item The Kodaira--Spencer morphism is everywhere immersive.
		\item Some bundle associated to $\mathbb V_{\mathrm{FL}}$ is ample.
	\end{itemize}
	Then one has an extension property: every $f\colon D^\times\to\mc S^{\mathrm{rig}}$ extends to $D$.
\end{theorem}
The talk concluded with a sketch of this, but I am already lost anyway.

\section{November 18: Mirror Symmetry and the Breuil--Mezard Conjecture}
This talk was given by Tony Feng at MIT for the BC--MIT number theory seminar.

\subsection{Modular Local Langlands}
We are going to spend most of our time talking about what hte Breuil--Mezard conjecture is. Ahistorically, this is related to the $p$-adic local Langlands correspondence. Let $G$ be some reductive group over $\QQ_p$, and set $k\coloneqq\overline{\FF}_p$.

Roughly speaking, we are supposed to provide a bijection between irreducible representations of $G(\QQ_p)$ over $k$ and Galois representations $\op{Gal}_{\QQ_p}\to\check G(K)$. This was understood for tori by class field theory, and work in the early 2000s extended it to $\op{GL}_2(\QQ_p)$.
\begin{remark}
	Tony had a ``hot take'' that maybe this correspondence does not exist for other groups. For example, we tried to make it work for $\op{GL}_2(\QQ_{p^2})$, but we have not succeeded.
\end{remark}
However, things look better if we work categorically.
\begin{conj}
	There is an emedding of derived categories
	\[D^b\left(\op{Rep}_KG(\QQ_p)\right)\into D^b\left(\op{Coh}\mc X^{\check G}_K\right).\]
	Here, $\mc X^{\check G}_K$ is the moduli space of Galois representations $\rho\colon\mathrm{Gal}_{\QQ_p}\to\check G(K)$. It is called the Emerton--Gee stack.
\end{conj}
\begin{example}
	For $G=\op{SL}_2$, the stack $\mc X$ looks like some union of stacky curves, possibly with nilpotent thickenings. Each stacky curve have two distinguished points, and generically the curve looks like
	\[\begin{bmatrix}
		\eta\chi_{\mathrm{cyclo}}^{\lambda+1} & * \\ & 1
	\end{bmatrix},\]
	where $\lambda$ is fixed, and $\eta$ varies over some unramified characters. The two distinguished points are given by choices of $\eta$ where the representation is irreducible or when it is split. The split and irreducible
\end{example}
However, one does not expect this equivalence to preserve $t$-structures. There is still no cases of the conjecture proven beyond the $G$ listed previously.

\subsection{The Breuil--Mezard Conjecture}
Now, remembering that everything should be derived, we expect a functor $\op{Rep}_KG(\QQ_p)\to\op{Coh}\mc X^{\check G}_K$, so we expect to have a composite
\[\op{Rep}_KG(\ZZ_p)\to\op{Rep}_KG(\QQ_p)\to\op{Coh}\mc X^{\check G}_K.\]
(The left map is given by co-induction.) De-categorifying, there is a homomorphism
\[K_0\op{Rep}_KG(\ZZ_p)\to K_0\op{Coh}\mc X^{\check G}_K.\]
Taking $K_0$ allows us to stop worrying about things being derived.

Observe that $K_0\op{Rep}_KG(\ZZ_p)=K_0\op{Rep}_KG(\FF_p)$ is some free abelian group on the irreducible representations, and one can describe these representations using some kind of highest weight theory$\pmod p$. Thus, the existence of such a map has no content. Instead, we want to impose some compatibility with a characteristic-$0$ story. In other words, we want some kind of commutativity of a diagram
% https://q.uiver.app/#q=WzAsNCxbMCwwLCJLXzBcXG9we1JlcH1fe1xcb3ZcXFFRX3B9RyhcXFFRX3ApIl0sWzAsMSwiS18wXFxvcHtSZXB9X0tHKFxcUVFfcCkiXSxbMSwwLCJLXzBcXG1jIFhee1xcY2hlY2sgR31fe1xcUVFfcH0iXSxbMSwxLCJLXzBcXG1jIFhee1xcY2hlY2sgR31fSyJdLFswLDJdLFsxLDMsIiIsMCx7InN0eWxlIjp7ImJvZHkiOnsibmFtZSI6ImRhc2hlZCJ9fX1dLFswLDFdLFsyLDNdXQ==&macro_url=https%3A%2F%2Fraw.githubusercontent.com%2FdFoiler%2Fnotes%2Fmaster%2Fnir.tex
\[\begin{tikzcd}[cramped]
	{K_0\op{Rep}_{\ov\QQ_p}G(\ZZ_p)} & {K_0\mc X^{\check G}_{\QQ_p}} \\
	{K_0\op{Rep}_KG(\ZZ_p)} & {K_0\mc X^{\check G}_K}
	\arrow[from=1-1, to=1-2]
	\arrow[from=1-1, to=2-1]
	\arrow[from=1-2, to=2-2]
	\arrow[dashed, from=2-1, to=2-2]
\end{tikzcd}\]
where the vertical arrows are$\pmod p$. It turns out to be too hard to work with the top map, and it is too hard to construct the bottom map anyway, so we observe that there is a map $K_0\mc X_{\QQ_p}^{\check G}\to\mathrm H_{\mathrm{top}}(\mc X_{\QQ_p}^{\check G})$, and then we will ask for there to be a commutative diagram as follows.
% https://q.uiver.app/#q=WzAsNCxbMCwwLCJLXzBcXG9we1JlcH1fe1xcb3ZcXFFRX3B9RyhcXFFRX3ApIl0sWzAsMSwiS18wXFxvcHtSZXB9X0tHKFxcUVFfcCkiXSxbMSwwLCJcXG1hdGhybSBIX3tcXG1hdGhybXt0b3B9fShcXG1jIFhee1xcY2hlY2sgR31fe1xcUVFfcH0pIl0sWzEsMSwiXFxtYXRocm0gSF97XFxtYXRocm17dG9wfX0oXFxtYyBYXntcXGNoZWNrIEd9X3tLfSkiXSxbMCwyXSxbMSwzLCIiLDAseyJzdHlsZSI6eyJib2R5Ijp7Im5hbWUiOiJkYXNoZWQifX19XSxbMCwxXSxbMiwzXV0=&macro_url=https%3A%2F%2Fraw.githubusercontent.com%2FdFoiler%2Fnotes%2Fmaster%2Fnir.tex
\[\begin{tikzcd}[cramped]
	{K_0\op{Rep}_{\ov\QQ_p}G(\QQ_p)} & {\mathrm H_{\mathrm{top}}(\mc X^{\check G}_{\QQ_p})} \\
	{K_0\op{Rep}_KG(\QQ_p)} & {\mathrm H_{\mathrm{top}}(\mc X^{\check G}_{K})}
	\arrow[from=1-1, to=1-2]
	\arrow[from=1-1, to=2-1]
	\arrow[from=1-2, to=2-2]
	\arrow[dashed, from=2-1, to=2-2]
\end{tikzcd}\]
The top map will now give us constraints and be computable: it is informed by the classical local Langlands conjectures, $p$-adic Hodge--Tate theory, and automorphy lifting theorems.

Let's explain how the local Langlands correspondence may help us.
\begin{itemize}
	\item On the Galois side, we have invariants given by Hodge--Tate weights $\lambda\in X_*(\check T)^+$ and an inertia type $\tau\colon I_{\QQ_p}\to\check G(\ov\QQ_p)$.
	\item On the automorphic side, there are representations $W(\lambda)\in\op{Rep}_{\ov\QQ_p}G(\ZZ_p)$ and $\op{LLC}(\tau)\in\op{Rep}_{\ov\QQ_p}G(\ZZ_p)$. (These are analogous to fixing the weight and the level.)
\end{itemize}
For example, it turns out that $W(\lambda)\otimes\op{LLC}(\tau)$ should go to the connected component $[\mc X_{\lambda,\tau}]\in\mathrm H_{\mathrm{top}}(\mc X_{\QQ_p}^{\check G})$. We are now ready to state our conjecture.
\begin{conj}[Breuil--Mezard]
	There is a map $K_0\op{Rep}_KG(\FF_p)\to\mathrm H_{\mathrm{top}}(\mc X_K^{\check G})$ such that all $\lambda$ and $\tau$ have
	\[\lim_{p\to0}W(\lambda)\otimes\op{LLC}(\tau)\mapsto\lim_{p\to0}[\mc X_{\lambda,\tau}].\]
	Here, the left-hand limit means we are taking$\pmod p$, and the right-hand limit means that it is flat limit over all $p$.
\end{conj}
\begin{remark}
	As a sanity check, we note that there are finitely many ``variables'': roughly speaking, we merely have to choose some Breuil--Mezard cycle $\mc Z_\sigma$ for each irreducible $\sigma\in\op{Rep}_KG(\FF_p)$. However, there are infinitely many equations given by the constraints, so one can show that there is at most one map.
\end{remark}
\begin{remark}
	This is a confusing statement because it is simultaneously a decategorified version of the local Langlands conjecture, and it is a local version of global reciprocity. In particular, this conjecture turns out to be closely related to $R=\mathbb T$ theorems.
\end{remark}
\begin{example}
	For $G=\op{SL}_2$, irreducible representations of $G(\ov\FF_p)$ turn out to all be reductions from characteristic $0$.
	\begin{itemize}
		\item One finds that $\mc Z_\lambda$ should be $\lim_{p\to0}\mc X_{\lambda,\mathrm{triv}}$.
		\item Alternatively, if we choose level $\Gamma_0(p)$, then we have a decomposition of the reduced representation $\op{Ind}_{B(\FF_p)}^{G(\FF_p)}$, which corresponds to $\mc X_{\lambda,\tau}$ now splitting into two stacky curves.
	\end{itemize}
	It turns out that the two stacky curves above intersect, which corresponds to the existence of some congruence$\pmod p$ between a modular form in level $1$ and a modular form in level $p$. One can also read the $\varepsilon$ conjecture off of these cycles.
\end{example}
\begin{remark}
	Let's give some history.
	\begin{itemize}
		\item The $p$-adic local Langlands conjecture provides a proof for $G=\op{GL}_2(\QQ_p)$.
		\item For $\lambda=(1,0)$, one can do it for $G=\op{GL}_2(K)$ by automorphy lifting methods.
	\end{itemize}
\end{remark}
Here are our main theorems.
\begin{theorem}[Feng--Le Hung]
	The conjecture is true for $G$ unramified and generic $\lambda$ and $\tau$.
\end{theorem}
\begin{theorem}[Feng--Le Hung--Lin]
	The conjecture is true for general $G$ and generic $\lambda$ and $\tau$.
\end{theorem}
The method of the first theorem comes from mirror symmetry. The second theorem still uses mirror symmetry but also a new input.

\subsection{Mirror Symmetry}
There is an analogy between $\QQ_p$ and surfaces $\Sigma$.
\begin{itemize}
	\item $\QQ_p$ has an Emerton--Gee stack $\mc X$, and $\Sigma$ has a character variety $S$. Both sides have some symplectic structure on $\mathrm H^1$.
	\item The embedding $\mc X_{\lambda,\tau}\into\mc X^{\check G}$ looks like Lagrangian in the same way that the character variety $S$ has Lagrangians.
\end{itemize}
Homological mirror symmetry morally allows one to index Lagrangians in a symplectic variety by some dual object (e.g., coherent sheaves), which explains why it may be helpful in our situation.

Even though homological mirror symmetry is mostly unknown, some known cases can help us. Relevant for us is that there is a Lagrangian skeleton (given by the affine springer fiber) embedding into some Hitchin moduli. On one side, affine springer fibers look like unions of $\PP^1$s, so one can imagine matching them up with $\mc X_{\lambda,\tau}$s. On the other side, there is some map $\op{Rep}_KG(\FF_p)\to\op{Coh}^\intercal(G/B)$, so one can use homological mirror symmetry to construct the required map.

Let's say a little about the ramified case because it is funny. Let $G$ be ramiified $U_{2n}$ so that $\mc G^{\mathrm{red}}(\FF_p)=\op{Sp}_{2n}(\FF_p)$. Thus, we are interested in producing a map
\[K_0\op{Rep}_K\op{Sp}_{2n}(\FF_p)\stackrel?\to\mathrm H_{\mathrm{top}}(\mc X^{\op{GL}_{2n}}).\]
We already have a map $K_0\op{Rep}_K\op{Sp}_{2n}(\FF_p)\to\mathrm H_{\mathrm{top}}(\mc X^{\op{SO}_{2n+1}})$, so we are on the hunt for a map
\[\mathrm H_{\mathrm{top}}(\mc X^{\op{SO}_{2n+1}})\stackrel?\to\mathrm H_{\mathrm{top}}(\mc X^{\op{GL}_{2n}}).\]
This is visibly (but not actually) related to twisted endoscopy, and indeed, it takes some of the same inputs as the Fundamental lemma. They called this ``spectral Langlands functoriality'' because it is a functoriality happening on the Galois side (whereas usual Langlands functoriality here would be happening on the automorphic side).

\section{November 20: The Statement of Weil II}
This talk was given by Kenta Suzuki at MIT for the STAGE seminar.

\subsection{The Statement}
We have spent the last many lectures showing the following.
\begin{theorem}[Deligne]
	Fix a smooth projective variety $X$ over $\FF_q$. Then each eigenvalue $\alpha$ of the Frobenius $\mathrm{Frob}_q$ acting on $\mathrm H^i(X_{\ov\FF_q};\QQ_\ell)$ has absolute value $q^{i/2}$.
\end{theorem}
Observe that $\mathrm H^i(X_{\ov\FF_q};\QQ_\ell)$ is $\mathrm R^if_*\QQ_\ell$, where $f\colon X\to\Spec\FF_q$ is the structure morphism. Thus, we may want to find a relative analogue. Already, in the last lecture, we say that $\mathrm H^i(\PP^1;\mc L)$ had some controlled weights, where $\mc L$ was some local system.

Let's recall the notion of weight.
\begin{definition}[weight]
	Fix a scheem $X$ over $\FF_q$ of finite type. Choose a $\ov\QQ_\ell$-sheaf $\mc F$ on $X$ and an isomorphism $\tau\colon\ov\QQ_\ell\to\CC$.
	\begin{listalph}
		\item Then $\mc F$ is \textit{$\tau$-pure of weight $w$} if and only if all eigenvalues $\alpha$ of the Frobenius acting on $\mc F_x$ have magnitude $\left|\tau(\alpha)\right|=q^{(\deg x)(w/2)}$ for all $x\in X(\ov\FF_q)$.
		\item Then $\mc F$ is \textit{$\tau$-mixed} if and only if it admits a filtration by $\tau$-pure.
	\end{listalph}
	Similarly, pure means $\tau$-pure for all $\tau$, and mixed means it admits a filtration of pure sheaves.
\end{definition}
\begin{remark}
	A priori, it seems that being $\tau$-mixed for all $\tau$ does not imply being mixed. It is unclear if these conditions are in fact equivalent.
\end{remark}
\begin{example}
	The Tate twist $\ov\QQ_\ell(1)$ is pure of weight $-2$. This sign convention appears because the geometric Frobenius is the inverse of $(-)^q$, and $(-)^q$ acts by $q$ on $\lim\mu_{\ell^\bullet}$.
\end{example}
Here is our statement.
\begin{theorem}[Weil II] \label{thm:weil-ii}
	Fix a morphism $f\colon X\to Y$ of schemes of finite type over $\FF_q$. If $\mc F$ is a constructible $\ov\QQ_\ell$-sheaf on $X$ which is $\tau$-mixed with weights at most $n$, then for any $i\ge0$, the sheaf $\mathrm R^if_!\mc F$ is $\tau$-mixed with weights at most $n+i$.
\end{theorem}
\begin{remark}
	Here, $f_!\mc F$ is the sheaf whose sections on an open subset $U$ are those sections $s\in\mc F(f^{-1}U)$ with proper support. Equivalently, one can choose a compactification $j\colon X\into X'$ so that $f'\colon X'\to Y$ is proper, and then $f_!=f'_*\circ j_!$.
\end{remark}
As a corollary, let's recover the Riemann hypothesis.
\begin{example}
	If we take $Y=\Spec\FF_q$, then $Rf_!\mc F=\mathrm H^i_c(X_{\ov\FF_q};\mc F)$, which we see is mixed of weights at most $n+i$.
	% , we see $\mathrm H^i_c(X_{\ov\FF_q};\mc F)$
\end{example}
\begin{example}
	If $X$ is smooth and equidimensional of dimension $d$, then Poincar\'e duality provides a perfect pairing
	\[\mathrm H^i(X_{\ov\FF_q};\mc F)\times\mathrm H^{2d-i}(X_{\ov\FF_q};\mc F^\lor)\to\QQ_\ell(-d).\]
	Taking $Y$ to be $\Spec\FF_q$ again, we find that if $\mc F$ has weights at least $n$, then dualizing shows that $\mc F^\lor$ has weights at most $-n$, so $\mathrm H^{2d-i}_c(X_{\ov\FF_q};\mc F^\lor)$ has weights at most $(2d-i)-n$, so the perfect pairing shows that $\mathrm H^i(X_{\ov\FF_q};\mc F)$ has weights at least $n+i$.
\end{example}
\begin{example}
	If $X$ is smooth, proper, and equidimensional, and $\mc F$ is pure of weight $n$, then $\mathrm H^i_c(X_{\ov\FF_q};\mc F)=\mathrm H^i(X_{\ov\FF_q};\mc F)$ is pure of weight $n$ by combining the two corollaries. Notably, $\mc F=\ov\QQ_\ell$ recovers the Riemann hypothesis.
\end{example}

\subsection{Reductions}
We will spend the rest of the talk giving a very sketchy outline of the proof of Weil II. Let's start with some functoriality properties.
\begin{lemma}
	Fix a morphism $f\colon X\to Y$ of schemes of finite type over $\FF_q$.
	\begin{listalph}
		\item If $\mc F$ is pure of weight $n$ on $Y$, then $f^*\mc F$ is pure of weight $n$ on $X$.
		\item Suppose $f$ is finite. If $\mc F$ is pure of weight $n$ on $X$, then $f_*\mc F$ is pure of weight $n$ on $Y$.
		\item Weight is preserved by base-change.
	\end{listalph}
\end{lemma}
\begin{proof}
	For (a), simply note that $(f^*\mc F)_x=\mc F_{f(x)}$ for each geometric point $x$. For (b), simply note that
	\[(f_*\mc F)_y=\bigoplus_{y\in f^{-1}(\{x\}}\mc F_x.\]
	Lastly, for (c), one applies the argument of (b) to the morphism $X_{\FF_{q'}}\to X_{\FF_q}$ for any extension $\FF_q\subseteq\FF_{q'}$.
\end{proof}
\begin{lemma}
	Fix a pure or mixed sheaf $\mc F$ on $X$. Then the same is true for any subquotient of $\mc F$.
\end{lemma}
\begin{proof}
	There is nothing to say for the pure case. In the mixed case, we separate the statement into two steps.
	\begin{itemize}
		\item For any subsheaf $\mc F'\subseteq\mc F$, one can intersect the pure filtration of $\mc F$ with $\mc F'$.
		\item For any quotient $\mc F\onto\mc F''$, one can project the pure filtration of $\mc F$ onto $\mc F''$.
		\qedhere
	\end{itemize}
\end{proof}
\begin{lemma}
	Given an exact sequence
	\[\mc F'\to\mc F\to\mc F''\]
	of sheaves, if $\mc F'$ and $\mc F''$ are mixed, then $\mc F$ is sheaf.
\end{lemma}
\begin{proof}
	By replacing $\mc F'$ with its image in $\mc F$ and $\mc F''$ with the image of $\mc F$, we may pass to a short exact sequence
	\[0\to\mc F'\to\mc F\to\mc F''\to0.\]
	Now, simply glue together the pure filtrations for $\mc F'$ and $\mc F''$ to build a pure filtration of $\mc F$.
\end{proof}
We now do some devissage for \Cref{thm:weil-ii}.
\begin{enumerate}
	\item Weights are preserved by field extensions, so we may extend $\FF_q$ at will.
	\item Note that there is nothing to do if $f$ is quasi-finite because the fibers $(f_!\mc F)_x$ are give by $\bigoplus_{y\in f^{-1}(\{x\})}\mc F_y$.
	\item Given a short exact sequence
	\[0\to\mc F'\to\mc F\to\mc F''\to0\]
	of sheaves on $X$, if we have the theorem for $\mc F'$ and $\mc F''$, then we have it for $\mc F$. Indeed, we simply have to note that the long exact sequence provides us with an exact sequence
	\[\mathrm R^if_!\mc F'\to\mathrm R^i\mc F\to\mathrm R^i\mc F'',\]
	so the left and right being mixed (with the correct weights) implies the same for the middle.
	\item Suppose we have an open subset $j\colon U\subseteq X$ with complement $i\colon Z\into X$. Then we claim that having the theorem for both $\mc F|_U$ and $\mc F|_Z$ yields the theorem for $\mc F$. To see this, one simply uses the short exact sequence
	\[0\to j_!j^*\mc F\to\mc F\to i_!i^*\mc F\to0,\]
	so we use the previous reductino for $j^*\mc F=\mc F|_U$ and $i^*\mc F=\mc F|_Z$.
	\item We may replace $X$ and $Y$ with their reduced subschemes because they have the same \'etale sites.
	\item Because $\mc F$ is constructible, we see that we may replace it by some subquotients in order to assume that it is a local system.
	\item Given morphisms $f\colon X\to Y$ and $g\colon Y\to Z$, if we have the theorem for both $f$ and $g$, then we have it for $g\circ f$. To see this, one uses the Grothendieck spectral sequence
	\[E_2^{pq}=\mathrm R^pg_!(\mathrm R^qf_!\mc F)\Rightarrow\mathrm R^{p+q}(gf)_!\mc F.\]
	The point is that the $E_2$ page is $\tau$-mixed with weights at most $n+p+q$, and then the $E_\infty$ page is simply subquotients of these, so they also have weights at most $n+p+q$. But now the $E_\infty$ page filters $\mathrm R^{p+q}(gf)_!\mc F$, so we conclude that this sheaf is still $\tau$-mixed with weights at most $n+p+q$.
	\item By Noetherian induction, one can use the previous reduction to assume that $f\colon X\to Y$ is smooth of relative dimension $1$ (by fibering by curves). By some Noether normalization and other things, one can further reduce to the case that $f\colon X\to Y$ is a smooth, affine, surjective morphism whose fibers are geometrically connected irreducible smooth curves.
\end{enumerate}

\subsection{Sketch}
To continue, we should recall the definition of a real sheaf.
\begin{definition}
	A sheaf $\mc F$ is \textit{$\tau$-real} if and only if each $x$ has
	\[\tau\det(1-\mathrm{Frob}T;\mc F_x)\in\RR[t]\]
	for all points $x$.
\end{definition}
This is fairly restrictive, but we will be able to use it.
\begin{lemma}
	Fix a local system $\mc G$ which is $\tau$-pure of weight $w$. Then $\mc G$ is the direct summand of a $\tau$-real, $\tau$-pure local system of weight $w$.
\end{lemma}
\begin{proof}
	Consider $\mc G^\lor=\underline{\op{Hom}}(\mc G,\ov\QQ_\ell)$, and choose some $b\coloneqq\tau^{-1}\left(q^{2w}\right)$. Then we may define a local system $\mc L_b$ as being the Galois representation where $\mathrm{Frob}_b$ acts by $b$. Then one can take
	\[\mc F\coloneqq\left(\mc G^\lor\otimes\mc L_b\right)\oplus\mc G.\]
	Indeed, for each eigenvalue $\alpha$ of $\mc G$, we get an additional eigenvalue $\ov\alpha=q^{4w}/\alpha$ from the left summand, so the fact that $\mc F$ is real follows.
\end{proof}
\begin{lemma}
	Let $\mc F$ be a local system on $X$. If $\mc F$ is $\tau$-real, then $\mc F$ is $\tau$-mixed.
\end{lemma}
\begin{proof}
	One uses the Rankin--Selberg method. In other words, we obtain information about $\mc F$ from $\mc F^{\otimes k}$ for large $k$. For sanity, we will only do this in the case that $Y$ is a smooth affine geometrically irreducible curve and that $X$ is a point. Observe that we may reduce to the case that $\mc F$ is irreducible, so we want to show that $\mc F$ is $\tau$-pure.
	
	We will need the following fact: for any real $A\in M_n(\RR)$, one can check that $\det\left(1-(A\otimes\ov A)T\right)^{-1}$ is a power series (in $T$) with nonnegative coefficients. Thus, we find that
	\[\frac1{\tau\det\left(1-\mathrm{Frob}_qT;\mc F_x^{\otimes2n}\right)}\]
	has nonnegative real coefficients for any positive $n$. On the other hand, the Lefschetz fixed point theorem tells us that
	\[\prod_x\frac1{\tau\det\left(1-\mathrm{Frob}_qT;\mc F_x^{\otimes2n}\right)}=\frac{\tau\det\left(1-\mathrm{Frob}_qT;\mathrm H^1_c(\mc F^{\otimes2n}\right)}{\tau\det\left(1-\mathrm{Frob}_qT;\mathrm H^2_c(\mc F^{\otimes2n}\right)}.\]
	Now, the right-hand side absolutely converges for $\left|t\right|\ge q^{1/(n\beta+1)}$, where $\beta$ is an appropriately defined maximal weight. Thus, we get the same convergence for the left-hand side, which lets us bound eigenvalues on the left-hand side because each power series has nonnegative coefficients. (This is basically the same argument as \Cref{thm:spread-out-bound}.)
\end{proof}

\section{November 24: Local Systems and the Simpson Correspondence}
This talk was given by Oakley at Havard for the Mark Kisin seminar.

\subsection{Higgs Bundles from Local Systems}
For today, $C$ is an algebraically closed complete nonarchimedean $p$-adic field. We let $X$ be a smooth proper connected rigid space, and we needed to choose a flat lift $\mathbb X$ of $X$ to $\mathbb B_{\mathrm{dR}}^+/t^2$ and an expoenential $\exp\colon C\to(1+\mf m_C)$. Last time, we exhibited some $\otimes$-equivalence
\[s\colon\mathrm{Vect}(X_{\mathrm{pro\acute et}};\widehat\OO)\to\mathrm{Higgs}(X).\]
We would like to use this to say something about local systems. Last time, we noted that there is a fully faithful embedding $\mathrm{Loc}_C(X_{\mathrm{pro\acute et}})$ into the vector bundles, so today we will be interested in finding the essential image.
\begin{example}
	Consider $X=\PP^1$. Then there is some \'etale cover $\mc M\to\PP^1_C$ (where $\mc M$ is the moduli space of elliptic curves), and its Tate module produces some local system $\mathbb L$. It turns out that this local system is not semistable. The point is that the Higgs bundle $(E,\theta)$ turns out to have $E=\OO(1)\oplus\OO(-1)$, bwhere we have a map $\theta\colon E\to\OO(-1)$ because $\OO(-1)=\OO(1)\otimes\widetilde\Omega^1=E\otimes\widetilde\Omega^1$. However, there is a disagreement in the slopes.
\end{example}
\begin{remark}
	This is basically a minimal computable example: semistability is automatic in rank $1$, and examples are difficult to compute in the non-proper situation.
\end{remark}
Thus, we see that we are looking at too many local systems, so we restrict our category to
\[\op{Rep}_C\pi_1(X;\ov x)\subseteq\op{Loc}_C(X_{\mathrm{pro\acute et}}).\]
Note that these more or less coincide in the complex analytic setting, but they are different in the rigid analytic setting because the left-hand side is using the \'etale fundamental group, which requires our local system to trivialize on a finite \'etale cover.
\begin{corollary}
	There are equivalences
	\[\op{Rep}_C^{\mathrm{uni}}\pi_1(X;\ov x)\cong\op{Loc}_C^{\mathrm{uni}}X_{\mathrm{pro\acute et}}\cong\op{Vect}^{\mathrm{uni}}(X_{\mathrm{pro\acute et}};\widehat\OO)\cong\mathrm{Higgs}^{\mathrm{uni}}(X),\]
	where the unipotent decoration means that we are looking at extensions of the trivial object.
\end{corollary}
\begin{proof}
	There are functors from left to right, and they all preserve $\mathrm H^1$.
\end{proof}
Anyway, here is our answer to the essential image question.
\begin{lemma}
	The inclusion
	\[\op{Rep}_C\pi_1(X;\ov x)\into\op{Vect}(X_{\mathrm{pro\acute et}};\widehat\OO)\]
	has essential image given by the vector bundles which are trivialized on a profinite \'etale cover.
\end{lemma}
\begin{proof}
	Let's just give the two functors. Given a representation $\rho$, one produces the natural vector bundle $\mathbb L_\rho\otimes\widehat\OO$. Conversely, given a vector bundle $V$, we can choose the profinite \'etale cover $\widetilde X$ and take the representation $V(\widetilde X)$. The proof that these are inverse equivalences is a matter of tracking through descent data.
\end{proof}
\begin{theorem}
	Fix a smooth projective curve $X$. Then the essential image of
	\[\op{Rep}_C\pi_1(X;\ov x)\into\op{Higgs}(X)\]
	factors through the semistable Higgs bundle $(E,\theta)$ for which $c_i(E)_\QQ=0$.
\end{theorem}
\begin{proof}
	Recall that there is a short exact sequence
	\[0\to\mathrm{Pic}_{X,\mathrm{\acute et}}\to\mathrm{Pic}_{X,\mathrm{pro\acute et}}\to\mathrm H^0(X;\widetilde\Omega^1)\otimes\mathbb G_a\to0,\]
	which on $\pi_0$ gives identifications $\pi_0\mathrm{Pic}_{X,\mathrm{\acute et}}=\mathrm{Pic}_{X,\mathrm{pro\acute et}}=\ZZ$ given by taking the degree.

	We now claim that $\deg V=\deg s(V)$ and further that $(N,\theta_N)\subseteq(E,\theta)$ implies that there is a pro-\'etale vector bundle $W\subseteq V$ for which $s(W)=(N,\theta_N)$. This latter claim follows from the theorem we have. We will not show this claim.

	The point is that it now suffices to show that $(E,\theta)$ is semistable of degree $0$ if and only if $V=s^{-1}(E,\theta)$ is semistable of degree $0$. For this, one does two cases. If $X=\PP^1$, then all vector bundles are semistable of degree $0$ anyway. Otherwise, $X$ has some genus $g\ge1$, and there is a Riemann--Roch argument which concludes.
\end{proof}
\begin{remark}
	There is an extension beyond curves. The point is that checking semistability with vanishing ratinoal Chern class can be checked after pulling back to every curve. There is some technical input that we need any curve $Y\to X$ for which we have our flat lifts $Y\to\mathbb Y$ and $X\to\mathbb X$, then we need a lifted map $\mathbb Y\to\mathbb X$. This lifting is doable when $\mathbb X$ is smooth.
\end{remark}
As such, we have the following conjecture.
\begin{conj}
	Let $X/\CC_p$ be smooth projective. Then there is an equivalence
	\[\op{Rep}_{\CC_p}\pi_1(X;\ov x)\to\{\text{semistable Higgs bundles }(E,\theta)\text{ with }c_i(E)_\QQ=0\}.\]
\end{conj}
\begin{remark}
	The conjecture is false if we change the ground field $\CC_p$ to any larger field. Approximately speaking, the problem occurs because the residue field of any larger $C$ is no longer a union of finite fields.
\end{remark}
Here are some known cases.
\begin{theorem}[Heuer]
	Let $X/C$ be smooth proper. Then there is an equivalence between one-dimen\-sional continuous representations $C$-representations of $\pi_1(X;\ov x)$ and the category of profinite \'etale-locally trivial Higgs line bundles $(L,\theta)$ on $X_{\mathrm{\acute et}}$.
\end{theorem}
\begin{remark}
	In fact, this latter category was shown to be equivalent to the category of Higgs line bundles $(L,\theta)$ for which $L$ is ``topologically torsion,'' meaning that there is a map $L^{n!}\to1$ for some $n$. In the case of $C=\CC_p$, it is then shown later that being topologically torsion is equivalent to $c_1(L)_\QQ=0$, but this is not the case when $C$ is larger!
\end{remark}
\begin{theorem}[Heuer, \ldots]
	Let $A/C$ be an abelian variety. Then there is an equivalence between the category $\op{Rep}_C\pi_1(A,0)$ and the category of profinite \'etale locally trivially Higgs bundles on $X_{\mathrm{\acute et}}$.
\end{theorem}
\begin{remark}
	This second category is then shown to be equivalent to the category of Higgs bundles $(E,\theta)$ which decompose into a sum of tensor products of unipotent representations and line bundles. So if $C=\CC_p$, then one can show that this latter condition is equivalent to having vanishing rational Chern class again.
\end{remark}

\subsection{Parallel Transport}
We will end the talk by discussing the $p$-adic parallel transport of Deninger and Werner. Our goal is to actually compute the Simpson correspondence. We will only be able to do this for $C=\CC_p$.
\begin{definition}
	Fix a proper scheme $X$ over $\overline{\FF}_p$. Then a vector bundle $E$ on $X$ is \textit{strongly semistable of degree $)$} if and only if all the pullbacks $\mathrm{Frob}_p^{k*}E$ are semistable of degree $0$.
\end{definition}
\begin{definition}
	Fix a formal scheme $\mc X$ over $\op{Spf}\OO_{\CC_p}$. Then a vector bundle $\mc E$ on $\mc X$ is \textit{Deninger--Werner} if and only if $\mc E_{\ov{\FF}_p}$ is strongly semistable of degree $0$.
\end{definition}
\begin{definition}
	Fix a proper rigid space $X$ over $\CC_p$ and a vector bundle $E$ on $X$. Then $E$ is \textit{Deninger--Werner} if and only if there is a formal neighborhood $\mc X$ of $X$ and a lift $\mc E$ of $E$ for which $\mc E$ is Deninger--Werner.
\end{definition}
\begin{theorem}[Deninger--Werner, \ldots]
	Fix a smooth projective curve $X_0$ over $\ov\QQ_p$, and set $X\coloneqq(X_0)_{\CC_p}$. Further, fix a Deninger--Werner Higgs bundles $(E,\theta)$. Then $s^{-1}(E,\theta)$ is some profinite \'etale locally trivial vector bundle.
\end{theorem}
\begin{proof}[Sketch]
	We will only discuss the $\theta=0$ case. Here, one first shows that $\mc E/p$ is trivialized after a finite \'etale cover (up to some Frobenius twist). Then one needs to show this for all powers $\mc E/p^n$ are trivialized by finite \'etale covers by some cohomological argument using the exact sequence
	\[0\to M_n(\mf p\OO_X^+/\mf p^2\OO_X^+)\to\op{GL}_n(\OO_X^+/\mf p^2)\to\op{GL}_n(\OO_X^+/\mf p)\to1.\]
	The point is that all elements in $\mathrm H^1(M_n(\mf p/\OO_X^+/\mf p^2\OO_X^+))$ can pass from pro\'etale cohomology to \'etale cohomology using a comparison theorem. Lastly, once we know that the $\mc E/p^n$s trivialize on finite \'etale covers, it follows that $\mc E$ trivializes on a profinite \'etale cover.
\end{proof}
Thus, we have explained how to produce a representation from a Deninger--Werner Higgs bundles. However, our condition on the Higgs bundle $(E,\theta)$ has only used $E$, so it cannot tell a complete story. It turns out that it is enough to merely be ``potentially'' Deninger--Werner, meaning that our vector bundle is Deninger--Werner after taking a finite cover.
\begin{theorem}[Xu]
	Fix a smooth projective curve $X_0$ over $\ov\QQ_p$, and set $X\coloneqq(X_0)_{\CC_p}$. Then there is an equivalence of categories between $\op{Rep}_{\CC_p}\pi_1(X;\ov x)$ and potentially Deninger--Werner Higgs bundles.
\end{theorem}
\begin{remark}
	These arguments crucially use the fact that we are over $\CC_p$. The fact that we start over $\ov\QQ_p$ seems not so necessary.
\end{remark}

\section{November 25: Arithmetic of Fourier Coefficients}
This talk was given by Naomi Sweeting at MIT for the number theory seminar.

\subsection{Quaternionic Modular Forms}
We are going to talk about $G_2$ eventually, but let's start with $\op{GL}_2$, where the story is Waldspurger's formula.
\begin{theorem}
	Fix a modular form $f$ of weight $2k$ and trivial central character. Then we consider is a Shimura lift $\varphi$ of weight $k+1/2$ (on the group $\widetilde{\op{SL}}_2$); let its Fourier expansion be $\varphi=\sum_na_nq^n$. Then for squarefree $D$, we have
	\[\left|a_D\right|^2=L(1/2,f\otimes\chi_D).\]
\end{theorem}
What is interesting here is that a single object $\varphi$ remembers all quadratic twists of a single modular form.

We are interested in an analogous story where $\widetilde{\op{SL}}_2$ is replaced with $G_2$. As before, we choose a modular form $f$, and it admits a lift $\varphi=\op{GG}(f)$, which is a ``quaternionic'' modular form (on the group $G_2$). (Such an object is predicted by Langlands functoriality.) Let's say something about quaternionic modular forms.
\begin{definition}
	Fix a reductive group $G$ over $\QQ$. Then we define $\mc A(G)\coloneqq L^2(G(\QQ)\backslash G(\AA_\QQ))$, and an \textit{automorphic representation} is an irreducible $G(\AA_\QQ)$-submodule of $\mc A(G)$.
\end{definition}
\begin{remark}
	It turns out that any automorphic representation $\pi$ admits a tensor decomposition $\pi=\bigotimes_v\pi_v$.
\end{remark}
\begin{remark}
	Some of these components are more important than others; for example, $\pi_\infty$ predicts ``arithmeticity'' (e.g., whether we expect to find a Galois representation, integrality properties, etc.). The holomorphic discrete series are the best-behaved in this sense. Note that holomorphic discrete series do not always exist: one needs $G(\RR)/K$ to admit a $G$-invariant complex structure (where $K\subseteq G(\RR)$ is a maximal compact). This is notably equivalent to admitting a Hermitian symmetric domain, and it is related to the (non)existence of Shimura varieties.
\end{remark}
Notably, $G_2$ does not admit holomorphic discrete series, but there is still a distinguished family of ``quaternionic'' discrete series $\{\pi_k\}_{k\ge1}$. We are now allowed to give the following definition, which is due to Pollack.
\begin{definition}
	A function $\varphi\colon G_2(\QQ)\backslash G_2(\AA_\QQ)\to\CC$ is a \textit{quaternionic modular form} of weight $k$ if and only if $\varphi$ generates the lowest $K$-type of $\pi_k$ under the $G_2(\RR)$-action. Here, $K=(\op{SU}(2)\times\op{SU}(2))/\{\pm1\}$ is the maximal compact subgroup of $G_2(\RR)$; the lowest $K$-type is then trivial on one of the $\op{SU}(2)$s and is $k$-dimensional on the other.
\end{definition}
\begin{remark}[Gan--Gross--Savin]
	A quaternionic modular form $\varphi$ (of appropriate level) admit a Fourier expansion whose coefficients $c_A(\varphi)$ are indexed by totally real cubic rings $A$. Here, a ``totally real cubic ring'' $A$ is a ring which is free of rank $3$ over $\ZZ$ and satisfying $A\otimes_\ZZ\RR=\RR^3$.
\end{remark}
\begin{theorem}[Pollack]
	There is basis of quaternionic modular forms with Fourier coefficients in $\QQ^{\mathrm{ab}}$.
\end{theorem}
This is an amazing result: the analogous result for modular forms uses a moduli description of modular curves to tease out some integrality. Anyway, the proof of the above theorem uses an exceptional theta lift from $F_4$. Integrality properties are also unknown.

We are now ready to state our desired conjecture.
\begin{conj}[Gross] \label{conj:gross}
	Fix a modular form $f$ of level $1$, and suppose that we have found a suitable lift $\varphi=\op{GG}(f)$. Then
	\[\left|c_A(\varphi)\right|^2=L(1/2,f\otimes V_E)\op{disc}(A)^{k-1/2},\]
	where $A$ is a maximal order in some field $E/\QQ$, and $V_E$ is defined via the decomposition $\op{Ind}_E^\QQ1=V_E\oplus1$.
\end{conj}
Here is our main result.
\begin{theorem}[Bakic--Horeax--Li-Huerta--S]
	If $f$ is a CM form and $L(1/2,f)\ne0$, then a variant of \Cref{conj:gross} holds.
\end{theorem}
Note that we are only proving a variant because the level of a CM form is always exceeding $1$.

\subsection{Packet Structure}
For our result, we choose $f$ coming from a Hecke character $\chi$ over a quadratic imaginary $K$. Then Arthur's conjecture predicts that $f_\chi$ only lifts to a packet of automorphic representations. Here, a packet consists of some family of automorphic representations whose local components are the same for all but finitely components. Note that this is a rather subtle issue: frequently, just changing a single place very rarely will actually produce another automorphic representation.

Now, for each place $v$, we define $\pi_v^+$ to be the local representation of $G_2(\QQ_v)$ arising from local functoriality. Then we will be able to modify our lifts at some finite set of places $S$ consisting of the places $v$ which are non-split in $K$ and satisfying $\chi_v^2\ne1$; here, there is an additional possible representation $\pi_v^-$, understood somewhat explicitly. Here is what Arthur's conjecture says in this situation
\begin{conj}[Arthur]
	Fix notation as above. Then the lift $\mc A_{\mathrm{GG}}(f_\chi)\subseteq\mc A(G_2)$ is expected to be
	\[\mc A_{\mathrm{GG}}(f_\chi)=\bigoplus_{\varepsilon}\pi^\varepsilon,\]
	where $\varepsilon$ is some tuple of signs $S\to\{\pm\}$ for which $\prod_v\varepsilon_v=\varepsilon(1/2,\chi)\varepsilon(1/2,\chi^3)$. Here, $\pi^\varepsilon=\bigotimes_{v\notin S}\pi_v^+\otimes\bigotimes\pi_v^{\varepsilon_v}$.
\end{conj}
\begin{remark}
	This is known by previous work (with the same authors) when $L(1/2,f_\chi)\ne0$ and $K$ is unramified at $2$.
\end{remark}
\begin{remark}
	Here are a few remarks about the local behavior.
	\begin{itemize}
		\item If $\pi_\infty^-$ is the quaternionic discrete series, then $\pi_\infty^+$ is not. Thus, we will need to take $\varepsilon_\infty=-1$ to have Fourier coefficients.
		\item For $\varphi\in\pi^\varepsilon$, then $c_A(\varphi)=0$ unless $\varepsilon_\ell=\varepsilon(A\otimes\QQ_\ell,\chi_\ell)$ for all $\ell\in S$.
	\end{itemize}
\end{remark}
Here is a more precise version of our main result.
\begin{theorem}
	Assume that $L(1/2,f_\chi)\ne0$, and choose $\varepsilon=(\varepsilon_v)_{v\in S}$ for which $\varepsilon_\infty=-1$ and $\prod_v\varepsilon_v=\varepsilon(1/2,\chi^3)$. Then there is a quaternionic modular form $\varphi\in\pi^\varepsilon$ satisfying the following: for all $A=\OO_E$ for totally real \'etale algebras $E/\QQ$, we have
	\[\left|c_A(\varphi)\right|=\begin{cases}
		L(1/2,f\otimes\chi_E)\op{disc}(A)^{k-1/2} & \text{if }\varepsilon_\ell=\varepsilon(A\otimes\QQ_\ell,\chi_\ell), \\
		0 & \text{otherwise}.
	\end{cases}\]
\end{theorem}

\subsection{Proof Sketch}
The construction of our lifts will come from exceptional theta correspondences. This is basically a correspondence $\Theta$ for automorphic representations from one group $G$ to another group $H$; such a thing exists only when $(G,H)$ forms a dual reductive pair, meaning that there is a map $H\times G\to\widetilde G$ and some $\theta$ functions, and these $\theta$-functions are then used as integral kernels.
\begin{example}
	Classically, one takes $\widetilde G=\widetilde{\op{Sp}}_{2n}$ and $H\times G=\mathrm{Sp}\times\mathrm O$ or $\mathrm U\times\mathrm U$.
\end{example}
\begin{example}
	In the exceptional case, one has finitely many options in the split $\widetilde G$ case.
	\begin{itemize}
		\item There is $G_2\times\op{PGL}_3\to E_6$.
		\item There is $G_2\times\op{PGSp}_6\to E_7$.
		\item Lastly, there is $G_2\times F_4\to E_8$.
	\end{itemize}
\end{example}
Note that the classical lifting embeds $\mathrm{PGL}_2\to\mathrm{PGSp}_6$, and then one takes a $\Theta$-lift to $G_2$. This is a complicated lifting process.

Here is the crux of our calculation.
\begin{idea}
	For this talk, we will use $G_2\times\op{PU}_3\to E_6'$, where $E_6'$ is some quasi-split twist.
\end{idea}
In particular, starting with $f_\chi$ arising from a $\Theta$-lift along $(\op U(1),\op{PU}(3))$, and then another $\Theta$-lift brings us to $G_2$. This is easy enough to do computations with, which is why we are able to prove the theorem.

For example, let's explain how we will approach our Fourier coefficients, following Gan--Gross--Savin. As usual, choose a parabolic $P$, which we give the Levi decomposition $P=MN$; here, $M=\op{GL}_2$. Given some $\varphi\in\mc A(G_2)$, we define the Fourier coefficient $c_\lambda(\varphi)$ as
\[c_\lambda(\varphi)=\int_{[N]}\varphi(n)\overline{\psi_\lambda(n)}\,dn.\]
Notably, characters $\psi_\lambda$ of $N$ turn out to by $\lambda\in\QQ^4$. In order to recover the story with cubic rings, one adds some level structure: notably, if $\varphi$ is invariant under $P(\widehat\ZZ)$, then $c_\lambda(\varphi)$ is supported on $\ZZ^4$, and it depends only on the ring $A_\lambda$, where $A_\lambda$ is some appropriate ring. Lastly, for archimedean reasons, $\varphi_\lambda(\varphi)=)$ vanishes unless $A_\lambda$ is totally real.
\begin{idea}
	Periods for $\Theta(\pi)$ will frequently reduce to periods for $\pi$.
\end{idea}
The reasoning here is that $\Theta$ arises from integral kernels, so periods (which are integrals) can frequently be unwound to the other group. In particular, for $\varphi$ as constructed using $\Theta$-lifting as above from $f_\chi$, one finds that
\[c_\lambda(\varphi)\sim\int_{[T_E]}\rho(t)\,dt,\]
where $E=A_\lambda\otimes_\ZZ\QQ$, and $T_E$ is a chosen torus in $\op{PU}_3$ from $E$, and $\rho$ is the lift from $\chi$ to $\op{PU}(3)$. By the doubling method, one finds that
\[\left|\int_{[T_E]}\rho(t)\,dt\right|^2\sim L(1/2,f_\chi\otimes V_E)L(1/2,f_\chi)\Delta_E^{1/2}.\]
The various $\sim$s turn into some local integrals $I_v(\varphi_v,\lambda)$, and evaluating enough of them completes the proof of the theorem. We remark a few points about the calculation.
\begin{itemize}
	\item The nonarchimedean integrals at $v=\ell$ away from $N$ is only checked to be $1$ when $A\otimes\ZZ_\ell$ is maximal, which explains our maximality condition in the theorem.
	\item The achimedean integrals use explicit Whittaker models.
	\item The nonarchimedean integrals at $v=\ell$ dividing $N$ are only computed at specified test vectors $\varphi_\ell$.
\end{itemize}

\section{December 1: Pan's Proof of a Classicality Result}
This talk was given by Hang at Harvard for the Mark Kisin seminar.

\subsection{The Fontaine--Mazur Conjecture}
For today, we fix a prime $p$. For a field $F$, we let $G_F$ be the absolute Galois group of $F$. We also let $B\subseteq\op{GL}_2(\QQ_p)$ be the upper-triangular Borel subgroup, and we let $\mf g$, $\mf b$, $\mf n$, and $\mf h$ be the usual Lie subalgebras for $\mf g=\mf{gl}(2)$.

Our story starts with the Fontaine--Mazur conjecture.
\begin{conj}
	Let $\rho\colon G_\QQ\to\op{GL}_2(\ov\QQ_p)$ be a continuous irreduible representation, and assume that the following conditions are satisfied for some nonnegative integer $k$.
	\begin{listalph}
		\item Unramified: $\rho|_{G_{\QQ_\ell}}$ is unramified for almost all $\ell$.
		\item Odd: $\det\rho(c)=-1$, where $c$ is the conjugacy class of complex conjugation.
		\item Weights: $\rho|_{G_{\QQ_p}}$ is de Rham of Hodge--Tate weights $(0,k)$.
	\end{listalph}
	Then $\rho$ arises from an eigenform of weight $k$.
\end{conj}
\begin{remark}
	A lot is known about this conjecture, due to many people.
\end{remark}
Here is an approach due to Emerton.
\begin{enumerate}
	\item Given conditions (a) and (b), show that $\rho$ comes from a $p$-adic modular form.
	\item Then prove a classicality result to show that $\rho$ comes from an actual modular form when given (c). Emerton shows this using some local-global compatibility result.
\end{enumerate}
Today, we are going to try to redo the second step without having to appeal to the Langlands conjectures. As such, let's try to be precise about what was proven.
\begin{definition}[completed cohomology]
	Fix a level $K\subseteq\op{GL}_2(\AA_f)$. Then we define the \textit{modular curve} to be
	\[Y_K\coloneqq\op{GL}_2(\QQ)\backslash(\mc H^\pm\times\op{GL}_2(\AA_f))/K.\]
	Its natural compactification is denoted $X_K$. We then define the \textit{completed cohomology} at $p$ to be
	\[\widetilde{\mathrm H}^i(K^p;\QQ_p)\coloneqq\left(\lim_n\colim_{K_p}\mathrm H^i(Y_{K^pK_p}(\CC);\ZZ/p^n\ZZ)\right)\left[\frac1p\right],\]
	and $\widetilde {\mathrm H}^i(K^p;\CC_p)\coloneqq\widetilde{\mathrm H}^i(K^p;\QQ_p)\widehat{\otimes}_{\QQ_p}\CC_p$. When no confusion is possible, $\widetilde{\mathrm H}^1(\CC_p)\coloneqq\colim_{K^p}\widetilde{\mathrm H}^1(K^p;\CC_p)$.
\end{definition}
\begin{remark}
	If we took the limit over $n$ before the colimit over $K_p$, then we would get the cohomology of $Y_{K^p}$ with coefficients in $\ZZ_p$, which sees less than the given definition.
\end{remark}
\begin{remark}
	The cohomology group $\widetilde{\mathrm H}^i(K^p;\CC_p)$ admits actions by $\op{GL}_2(\QQ_p)$, the Galois group, and the Hecke algebra (of level $K^p$).
\end{remark}
Here is our classicality result, due to many people.
\begin{theorem} \label{thm:fontaine-mazur-get-modular}
	Fix an irreducible continuous representation $\rho\colon G_\QQ\to\op{GL}_2(E)$, where $E$ is a finite extension of $\QQ_p$. Suppose that
	\[\op{Hom}_{G_\QQ}\left(\rho,\widetilde{\mathrm H}^1(K^p,\QQ_p)\otimes E\right)\ne0.\]
	\begin{listalph}
		\item If $\rho|_{G_{\QQ_p}}$ is Hodge--Tate of weights $(0,0)$, then $\rho$ arises from an eigenform of weight $1$.
		\item If $\rho|_{G_{\QQ_p}}$ is Hodge--Tate of weights $(0,k)$, then $\rho$ arises from an eigenform of weight $k+1$.
	\end{listalph}
\end{theorem}
\begin{remark}
	The first case is separated because it is largely due to different people, and it is a bit easier because the underlying method (inputting $p$-adic Hodge theory into the Taylor--Wiles method) is easier.
\end{remark}
\begin{remark}
	It is potentially surprising that we have found eigenforms of high weight in the cohomology of modular curves. What is going on here is that we are working with completed cohomology, which allows us to work in $p$-adic families, and it turns out that modular forms of high weight can be $p$-adically approximated by modular forms of weight $2$.
\end{remark}
We will explain a new proof of this theorem.

\subsection{Hodge--Tate Period Maps}
Our story begins with the following result of Scholze, allowing us to work at infinite level.
\begin{theorem}[Scholze]
	Fix some $K^p$, and let $\mc X_{K^pK_p}$ be the adic space associated to $X_{K^pK_p,\CC_p}$.
	\begin{listalph}
		\item Then there is a unique perfectoid space $\mc X_{K^p}$ over $\CC_p$ for which
		\[\mc X_{K^p}=\lim_{K_p}\mc X_{K^pK_p}.\]
		\item There is a $\op{GL}_2(\QQ_p)$-equivariant ``Hodge--Tate'' period map
		\[\pi_{\mathrm{HT}}\colon\mc X_{K^p}\to\op{Flag}\left(\QQ_p^2\right).\]
		Further, $\pi_{\mathrm{HT}}$ is affine and Hecke-equivariant.
	\end{listalph}
\end{theorem}
\begin{remark}
	We will say nothing about the proof, but we will recall the definition of $\pi_{\mathrm{HT}}$. For any elliptic curve $E$, there is a Hodge--Tate filtration
	\[0\to\op{Lie}E^\lor(\CC_p)\to\op{Hom}_{\ZZ_p}(T_pE,\CC_p)\to(\op{Lie}E)^*(\CC_p)(-1)\to0.\]
	On the modular curve, this then idnuces a filtration on $\mc X_{K^p}$ given by
	\[0\to\omega_{K^p}^{-1}\to V\otimes\OO_{\mc X_{K^p}}\to\omega_{K^p}(-1)\to0,\]
	whre $V=\QQ_p^2$, and $\omega_{K^p}$ is the pullback of the Hodge bundle to $\mc X_{K^p}$. The position of $\omega_{K^p}^{-1}$ in $V\otimes\OO_{\mc X_{K^p}}$ at a given point in $\mc X_{K^p}$ defines $\pi_{\mathrm{HT}}$.
\end{remark}
The use $\mc X_{K^p}$ for our purposes is the following interpretation of completed cohomology.
\begin{theorem}[Scholze]
	There are isomorphisms
	\[\widetilde{\mathrm H}^i(K^p;\CC_p)\cong\mathrm H^i(\mc X_{K^p};\OO_{\mc X_{K^p}})\cong\mathrm H^i\left(\op{Flag}\QQ_p^2;\pi_{\mathrm{HT}}\OO_{\mc X_{K^p}}\right)\]
	preserving all the various actions.
\end{theorem}
\begin{remark}
	The left map arises from some almost purity result, apparently. The right map arises from some spectral sequence, which immediately degenerates because $\pi_{\mathrm{HT}}$ is affine.
\end{remark}
We are almost ready to state our main result, but we need a little notation.
\begin{notation}
	Let $\mc B$ be the flag variety $\PP^1=B\backslash\op{GL}_2$. For $x\in\mc B(\CC_p)$, we set $B_x\coloneqq x^{-1}Bx$, and we define $\mf b_x$ and $\mf n_x$ accordingly. We then define
	\[\mf g^\circ\coloneqq\OO_{\mc B}\otimes\mf g,\]
	and we define $\mf b^\circ\subseteq\mf g^\circ$ and $\mc n^\circ$ accordingly.
\end{notation}
Our first main result computes completed cohomology on locally analytic vectors.
\begin{theorem}[Pan]
	For an admissible $\op{GL}_2(\QQ_p)$-representation $V$ over an adic space, we let $V^{\mathrm{la}}$ denote the locally analytic vectors for $\op{GL}_2(\QQ_p)$.
	\begin{listalph}
		\item We have $\mathrm H^i(\PP^1;\OO_{K^p}^{\mathrm{la}})\cong\widetilde{\mathrm H}^i(K^p;\CC_p)^{\mathrm{la}}$.
		\item The algebra $\mf n^\circ$ annihilates $\OO_{K^p}^{\mathrm{la}}$.
	\end{listalph}
\end{theorem}
Our second main result introduces an action.
\begin{theorem}[Pan]
	There is an action $\theta_h$ of $\mf h=\mathrm H^0(\mf b^\circ/\mf n^\circ)$ acting on $\OO_{K^p}^{\mathrm{la}}$ satisfying the following properties.
	\begin{listalph}
		\item The action $\theta_h$ extends to an action by $\op{Sym}(\mf{sl}_2\cap\mf h)$, which then agrees with the action by $Z(U\mf{sl}_2)\subseteq U\mf g$ under the Harish-Chandra isomorphism.
		\item For scalar $z\in\mf g$, $\theta_h|_{z}$ agrees with the expected action of $z$.
		\item Sen operator: the operator $\theta_h(\op{diag}(0,1))$ is the Sen operator on $\widetilde{\mathrm H}^1(K^p;\CC_p)^{\mathrm{la}}$. (Technically, it merely agrees with the Sen operator on finite-dimensionsal Galois-invariant subspaces.)
		\item Define the character $\mu_k\colon\mf b\to\CC_p$ given by $\mu_k\left(\begin{bsmallmatrix}
			a & b \\ & d
		\end{bsmallmatrix}\right)=kd$. Then the $\mu_k$-isotypic component of $\widetilde{\mathrm H}^1(\CC_p)^{\mathrm{la}}$ admits a Hodge--Tate decomposition
		\[\widetilde{\mathrm H}^1(\CC_p)^{\mathrm{la}}_k=N\oplus N',\]
		where $\theta_h$ acts by $(k,0)$ on $N$ and by $(1,k-1)$ on $N'$.
	\end{listalph}
\end{theorem}
\begin{remark}
	There is also a description (via a filtration) of $N$ and $N'$.
\end{remark}
It is not too hard to prove \Cref{thm:fontaine-mazur-get-modular}(a) from here, but we will not explain the proof. Instead, we will content ourselves with giving some taste for the proof of the above main theorem. There are two main ingredients: relative Sen theory, and an explicit description of $\OO_{K^p}^{\mathrm{la}}$. With these two ingredients, the theorem then becomes a calculation.

\section{December 2: Unlikely Intersections with Orthogonality}
This talk was given by Gal Binyamini at Boston College for the BC--MIT number theory seminar.

\subsection{The Pila--Wilkie Theorem}
Our story begins with some logic.
\begin{definition}[$o$-minimal]
	A structure is some nonempty family
	\[\mc S\subseteq\bigsqcup_n\mc P(\RR^n)\]
	which is closed under unions, intersections, complements, linear projections, products, and containing algebraic hypersurfaces. The structure $\mc S$ is \textit{$o$-minimal} if and only if each $A\in\mc S\cap\mc P(\RR)$ is a finite union of points and intervals.
\end{definition}
\begin{remark}
	There are many variations of this definition: one can add more closure properties (e.g., Minkowski sum), one can merely be closed under certain graphs, etc.
\end{remark}
\begin{remark}
	Being $o$-minimal places strong constraints on $\mc S$ because we can consider projections of some $A\in\mc S$ to $\RR$.
\end{remark}
\begin{example}
	The structure $\RR_{\mathrm{alg}}$ constists of the semi-algebraic subsets. It is $o$-minimal.
\end{example}
\begin{example}
	The structure $\RR_{\mathrm{an}}$ is the smallest structure consisting of $\RR_{\mathrm{alg}}$ along with the graphs of analytic functions $[0,1]^n\to\RR$. It is also $o$-minimal. Similarly, we define $\RR_{\mathrm{an},\exp}$ to also be closed under the graph of the exponential $\exp\colon\RR\to\RR$.
\end{example}
Typically, one works with $\RR_{\mathrm{an},\exp}$, but today we will have occasion to work with some larger $o$-minimality.

The relevance to number theory is as follows.
\begin{notation}
	Fix a subset $X\subseteq\RR^n$, we define
	\[X(\QQ,H)\coloneqq\{x\in X\cap\QQ^n:H(x)\le H\}.\]
	% Further,
\end{notation}
\begin{notation}
	Fix a subset $X\subseteq\RR^n$. Then set $X^{\mathrm{alg}}$ to be the union of all positive-dimensional connected semi-algebraic subsets of $X$. The complement $X\setminus X^{\mathrm{alg}}$ is $X^{\mathrm{tr}}$.
\end{notation}
\begin{theorem}[Pila--Wilkie] \label{thm:pila-wilkie}
	Suppose $X\subseteq\RR^n$ is definable in an $o$-minimal structure. For any $\varepsilon>0$, one has
	\[\#X^{\mathrm{tr}}(\QQ,H)\lesssim_{X,\varepsilon}H^\varepsilon.\]
\end{theorem}
\begin{remark}
	One cannot in general improve $H^\varepsilon$.
\end{remark}
\begin{remark}
	One passes to $X^{\mathrm{tr}}$ because semialgebraic subsets can have lots of rational points.
\end{remark}

\subsection{Unlikely Intersections}
The Pila--Wilkie theorem is typically applied via ``unlikely intersections.'' Here is one such application.
\begin{theorem} \label{thm:proto-unlikely}
	Fix an algebraic subset $V\subseteq\left(\CC^\times\right)^n$. If $V$ contains no cosets of positive-dimensional subgroups, then $V\cap\left(\CC^\times\right)^n_{\mathrm{tors}}$ is finite.
\end{theorem}
This result has many proofs. Here is one due to Pila--Zannier, and it generalizes nicely.
\begin{proof}
	We apply \Cref{thm:pila-wilkie}. For technical reasons, assume that $V$ is defined over $\QQ$. Consider the map $\pi\colon[0,1]^n\to(\CC^\times)^n$ defined by
	\[\pi(x_1,\ldots,x_n)\coloneqq(\exp(2\pi ix_1),\ldots,\exp(2\pi ix_n)).\]
	This will be our ``transcendental input'' to \Cref{thm:pila-wilkie}; observe that it covers the torsion of $\left(\CC^\times\right)^n$. Define $X\coloneqq\pi^{-1}(V)$, which we note is definable in $\RR_{\mathrm{an}}$; observe that $V[n]$ can be found in $X(\QQ,H)$ for some $H$ (which is a polynomial in $n$).
	
	It now turns out that $X^{\mathrm{alg}}=\emp$ by the Ax--Schanuel theorem; this uses the hypothesis that $V$ contains no cosets of positive-dimensional subgroups. Now, \Cref{thm:pila-wilkie} implies that
	\[\#V[n]\lesssim_{V,\varepsilon}N^\varepsilon.\]
	We now need to upgrade this into finiteness, which is done via Galois theory. To do this, if $p\in V[n]$, then one can show that there is some $c$ for which
	\[\#\left(\op{Gal}(\ov\QQ/\QQ)\cdot p\right)\ge N^c.\]
	Combining these two bounds implies that $V[n]$ vanishes for sufficiently divisible $n$, and the result follows.
\end{proof}
The sort of claim being proven above is an ``unlikely intersection'' problem: one has some special points and some special subvarieties, and we claim that any controlled subset avoiding the special subvarieties may only have finitely many special points. For example, this is the same sort of result occurs for Shimura varieties.

\subsection{Orthogonal Polynomials}
Today, we will be interested in orthogonal polynomials.
\begin{definition}[orthogonal]
	Fix a weight function $w$ on an interval $I$. A family of polynomials $\{P_n\}_{n\in\NN}$ with rational coefficients is \textit{$w$-orthogonal} if and only if $\deg P_n=n$ for each $n$, $P_n(1)=1$, and
	\[\int_IP_n(x)P_m(x)w(x)\,dx=0\]
	for all $n\ne m$.
\end{definition}
\begin{example}[Legendre]
	The polynomials $\{P_n\}$ are defined with $I=[-1,1]$ and $w\equiv1$.
\end{example}
\begin{example}[Laguere]
	The polynomials $\{L_n\}$ are defined with $I=\RR^+$ and $w(x)=\exp(-x)$.
\end{example}
\begin{example}[Hermite]
	The polynomials $\{H_n\}$ are defined with $I=\RR$ and $w(x)=\exp\left(-x^2\right)$.
\end{example}
\begin{ques}[Stieltjes]
	Consider the family of Legendre orthogonal polynomials $\{P_n\}_{n\in\NN}$.
	\begin{listalph}
		\item Is it possible for $P_n$ and $P_m$ to share roots other than $x=0$?
		\item Is $P_n$ irreducible over $\QQ$ for all $n$?
	\end{listalph}
\end{ques}
\begin{remark}
	Irreducibility is known for the families $\{L_n\}$ and $\{H_n\}$.
\end{remark}
Let's explain why this is an instance of unlikely intersections. For concreteness, we work with the polynomials $\{L_n\}$.
\begin{theorem} \label{thm:lag-unlikely}
	Say that a point $\ell\in\RR^n$ is special if and only if each of its coordinates is a root of $L_n$ for some $n$, and say that a subset $V\subseteq\RR^n$ is special if and only if cut out by the equations $x_i=x_j$ or $x_i=\ell$ for a special point $\ell$. Then an algebraic variety avoiding all special subvarieties admits only finitely many special points.
\end{theorem}
The proof turns out to be essentially the same, but the Galois-theoretic input is more complicated.
\begin{remark}
	The Galois theory could use as input that the Galois group of $L_n$ is $S_n$, which provides a different proof.
\end{remark}
One can do something for Legendre polynomials.
\begin{theorem}
	Fix terminology as \Cref{thm:lag-unlikely} but working with the Legendre polynomials $\{P_n\}$. Then there is some $N$ for which
	\[\#\{x\ne0:P_n(x)=P_m(x)=0\text{ for some }n\ne m\}\le\log(n+m)^N\]
	for any $n$ and $m$.
\end{theorem}
The difficulty between $P_\bullet$s and $L_\bullet$s is having weaker Galois-theoretic input.

\subsection{Orthogonal Subsets}
Our next example will come from harmonic analysis.
\begin{definition}[orthogonal]
	Consider the disk $\mathbb D$. A subset $\Lambda\subseteq\RR^2$ is \textit{orthogonal for $\mathbb D$} if and only if
	\[\int_{\mathbb D}e^{2\pi i(\lambda_1-\lambda_2)\cdot v}\,dv=0\]
	for all distinct $\lambda_1,\lambda_2\in\Lambda$.
\end{definition}
\begin{conj}[Fuglede] \label{conj:fug}
	Consider the disk $\mathbb D$. Then any orthogonal subset $\Lambda$ has $\#\Lambda\le4$.
\end{conj}
To motivate this conjecture (and relate it to unlikely intersections), we introduce Bessel functions.
\begin{definition}[Bessel function]
	The Bessel function $J_1$ is the normalized slow-growing solution to the differential equation
	\[x^2y''+xy'+\left(x^2-1\right)y=0.\]
	It turns out that $J_1$ is defined by some explicit integrals.
\end{definition}
\begin{remark}
	There are also Bessel functions $J_\kappa$ for some $\kappa$, where we replace the differential equation with
	\[x^2y''+xy'+\left(x^2-\kappa\right)y=0.\]
\end{remark}
\begin{remark}
	It turns out that the Fourier transform of $1_{\mathbb D}$ is $J_1(2\pi\norm\lambda)/\norm\lambda$.
\end{remark}
\begin{remark}
	It turns out that
	\[\int_{\mathbb D}e^{2\pi i(\lambda_1-\lambda_2)\cdot v}\,dv=0\]
	is equivalent to $\pi\norm{\lambda_1-\lambda_2}$ being a root of $J_1$. This indicates how we might later relate this to unlikely intersections.
\end{remark}
\begin{notation}
	We define $V_{4,2}\subseteq\RR^6$ to consist of the six-tuple of distances between four ordered points in $\RR^2$. Note that $V_{4,2}$ is an algebraic subset.
\end{notation}
\begin{remark}
	It now turns out that \Cref{conj:fug} is equivalent to having no $J_1$-special points in $V_{4,2}$. This explains why the conjecture may be true: one does not expect to have any algebraic relations between these roots.
\end{remark}
\begin{remark}
	It is known that orthogonal subsets $\Lambda$ for the disk are finite. In fact, it is known that
	\[\#\Lambda\cap[-R,R]^2\lesssim_\varepsilon R^{\varepsilon+3/5},\]
	building on work of many people.
\end{remark}
\begin{theorem} \label{thm:unlikely-bessel}
	Let $V\subseteq\CC^n$ be an algebraic subset with no Bessel special subvarieties. Then the number of special points in $V\cap[-R,R]^n$ is bounded by $c(V,\varepsilon)R^\varepsilon$ for any $\varepsilon>0$.
\end{theorem}
\begin{remark}
	If $n$ is odd, then one can replace $R^\varepsilon$ by $\log R$. It is expected that one can still do this for $n$ even, but it is more technical.
\end{remark}
Let's end the talk by explaining how the previously given orthogonality results relate to unlikely intersections. Let's work with $J_1$ for concreteness.
\begin{theorem}
	There is an $o$-minimal structure $\RR_G$ (which is not $\RR_{\mathrm{an},\exp}$) for which there is a definable map $f_B\colon\RR\to\RR$ such that $f_B$ sends integers to the zeroes of $J_1$.
\end{theorem}
\begin{proof}[Idea]
	The point is that
	\[J_1(x)=e^{ix}G_1(x)+e^{-ix}G_2(x),\]
	where $G_1$ and $G_2$ admit some asymptotic expansions $\sum_ka_kx^{-k}$. Then we can define the graph of $f_B$ as having those pairs $(t,x)$ such that $0\le x-2\pi t\le2\pi$ and
	\[e^{i(x-2\pi t)}G_1(x)+e^{-(x-2\pi t)}G_2(x)=0.\]
	Notably, whenever $t$ is an integer, we find that $J_1(x)=0$.
\end{proof}
From here, one can argue as before using \Cref{thm:pila-wilkie}, pulling back $V$ along $f_B$ and hoping that there are no algebraic subsets. There is a similar approach which works for the Legendre polynomials, using different defining polynomials.

\section{December 2: Unlikely Intersections with Orthogonality}
This talk was given by Jonathan Pila at Boston College for the BC--MIT number theory seminar.

\subsection{The Ax--Schanuel Theorem}
The following conjecture comes from transcendental number theory.
\begin{conj}[Schanuel]
	Fix complex numbers $x_1,\ldots,x_n$ which are $\QQ$-linearly independent. Then
	\[\op{trdeg}_\QQ\QQ(x_1,\ldots,x_n,\exp(x_1),\ldots,\exp(x_n))\ge n.\]
\end{conj}
\begin{example}
	With $n=1$, this is known due to Hermite--Lindermann. It translates into the statement that $\exp(x_1)$ is trascendental when $x_1$ is algebraic.
\end{example}
\begin{example}
	With $x_1,\ldots,x_n\in\ov\QQ$, this is known to Lindermann--Weierstrass. It translates into the statement that the $\exp(x_\bullet)$s are trascendentally independent. In particular, we see that the lower bound $n$ is sharp.
\end{example}
For $n\ge2$ and $x_\bullet$s general, very little is known.

There is a known analogue for functions.
\begin{theorem}[Ax--Schanuel]
	Let $x_1,\ldots,x_n$ be holomorphic functions on $\mathbb D$. Suppose that the functions $x_1,\ldots,x_n$ are $\QQ$-linearly independent modulo constants. Then
	\[\op{trdeg}_\CC\CC(x_1,\ldots,x_n,\exp(x_1),\ldots,\exp(x_n))\ge n+1.\]
\end{theorem}
\begin{example}
	The $n+1$ is sharp. For example, one can take $x_1$ to be the identity function.
\end{example}
\begin{remark}
	One can pass to higher-dimensional disks $\mathbb D$, in which case $n+1$ gets replaced with $n+\dim\mathbb D$. Again, this is sharp by letting the $x_\bullet$s be coordinate functions on $\mathbb D$.
\end{remark}
\begin{remark}
	The most natural setting for this theorem works in differential fields.
\end{remark}
\begin{example}
	To relate this to geometry, one can imagine $(x_1,\ldots,x_n)\colon\mathbb D\to\CC^n$ parameterizing some algebraic curve. In this case, $\op{trdeg}_\CC(x_1,\ldots,x_n)=1$, so it follows that
	\[\op{trdeg}_\CC(\exp(x_1),\ldots,\exp(x_n))\ge n.\]
\end{example}
\begin{example}
	Let's explain the application to \Cref{thm:proto-unlikely}. Suppose that we have an algebraic subvariety $C\subseteq\pi^{-1}(V)$ of positive dimension, we may as well have $\dim C=1$. But then the exponentials of the various parameterizing functions $(x_1,\ldots,x_n)$ for $C$ lie in $V$, so their transcendence degree will be small. It follows by contraposition that some $\QQ$-linear combination of the $x_\bullet$s is a constant in $\CC$, which translates into $V$ containing a coset of a positive-dimensional subtorus of $\left(\CC^\times\right)^n$!
\end{example}

\subsection{Ax--Schanuel Input to Orthogonal Subsets}
Suppose we want to prove \Cref{thm:unlikely-bessel}. We are interested in finding Bessel-special points. Set $H_1(x)\coloneqq e^{ix}G_1(x)$ and $H_2(x)\coloneqq e^{-ix}G_2(x)$ so that $J_1=H_1+H_2$. Then define the subset $Z\subseteq\RR^n$ to consist of those $(t_1,\ldots,t_n)$ for which there is some $(x_1,\ldots,x_n)\in V\cap\left(\RR^+\right)^n$ satisfying the bounds $0\le x_j-2\pi t_j<2\pi$ for all $j$ and
\[e^{-2\pi it_j}H(x_j)+e^{2\pi it_j}H_2(x_j)=0\]
for all $j$. In particular, when $t_j$ is an integer, it follows that $x_j$ is an integer. Thus, the special points in $V$ arise from points in $Z\cap\ZZ^n$, and the bounds $0\le x_j-2\pi t_j<2\pi$ turns out to put these into bijection (due to some separation of the roots of $J_1$). Importantly, $Z$ is definable in $\RR_G$.

We would now like some input like Ax--Schanuel, so suppose that we ahve found some semialgebraic curve $T$ in $Z$; parameterize $T$ by $(t_1,\ldots,t_n)$. Passing to the complexification, we see that $T_\CC$ is contained in some $\widetilde Z$, defined as follows: one has $(t_1,\ldots,t_n)\in\CC^n$ lives in $\widetilde Z$ if and only if one can find $(x_1,\ldots,x_n)\in V$ satisfying
\[e^{-2\pi it_j}H(x_j)+e^{2\pi it_j}H_2(x_j)=0\]
for all $j$. One can now calculate that the transcendence degree of the $t_j$s, their exponentials, the $x_j$s, the $(H_1\circ x_j)$s and the $(H_2\circ x_j)$s have some very small transcedence degree.

We are now approximately ready to state our main theorem.
\begin{notation}
	There are two distinguished linearly independent solutions $z_1$ and $z_2$ to Bessel's differential equation
	\[x^2z''+xz'+\left(x^2-\kappa\right)z=0.\]
	We now set
	\[W_\kappa\coloneqq\begin{bmatrix}
		z_1 & z_2 \\ z_1' & z_2'
	\end{bmatrix}.\]
\end{notation}
\begin{remark}
	It turns out that $\det W(x)=c/x$ for some constant $c$, but this is the only relation: when $\kappa\notin\frac12+\ZZ$,
	\[\op{trdeg}_{\CC(x)}\CC(x,W)=3.\]
	Here, adjoining $W$ means that we are adjoining all its coordinates.
\end{remark}
\begin{theorem}
	Fix everything as above. Choose $\kappa_1,\ldots,\kappa_n\in\CC\setminus(\frac12+\ZZ)$. Let $\{x_1,\ldots,x_n\}$ be non-constant functions in $s\in\mathbb D$, and set $W_i\coloneqq W_{\kappa_i}(x_i)$. Further, choose more non-constant functions $\{t_1,\ldots,t_\ell\}$. Suppose the following conditions are not satisfied.
	\begin{listalph}
		\item The $t_j$s are $\QQ$-linearly independent modulo constants.
		\item For some $j\ne k$, we have $x_j=\pm x_k$ and $\kappa_j\pm\kappa_k\in\ZZ$.
	\end{listalph}
	Then
	\[\op{trdeg}_\CC\CC(t_\bullet,\exp(t_\bullet),x_\bullet,W_\bullet)\ge\ell+3n+1.\]
\end{theorem}
Once again, the relevance of this result for us will be its contraposition, and it allows us to identify the algebraic part, explaining the Ax--Schanuel input to \Cref{thm:unlikely-bessel}. The key inputs to the proof of the above theorem consist of the ``modern'' technology of how to produce Ax--Schanuel for differentially closed fields as well as some understanding of the differential Galois group at play.

\section{December 3: Cohomology of Banach--Colmez Spaces}
The pre-talk and talk were given by Xinyu Zhou as a pre-talk for the Harvard number theory seminar.

\subsection{The \'Etale Site}
We are interested in adic spaces today.
\begin{definition}[finite \'etale]
	A map $\op{Spa}(B,B^+)\to\op{Spa}(A,A^+)$ is \textit{finite \'etale} if and only if $B$ is finite \'etale over $A$ and $B^+$ is the integral closure of $A^+$ in $B$. A map $f\colon X\to Y$ of analytic adic spaces is \textit{finite \'etale} if and only if it locally has this form.
\end{definition}
\begin{definition}[\'etale]
	A map $f\colon X\to Y$ is \textit{\'etale} if and only if each $x\in X$ has an open neighborhood $U$ around $x$ and $V$ around $f(x)$ such that $f$ factors as an open embedding follows by a finite \'etale map.
\end{definition}
\begin{remark}
	For schemes, this definition would be for separated \'etale morphisms, which is the same as some \'etale morphism of the underlying schemes are assumed to be separated.
\end{remark}
\begin{definition}
	Fix an analytic adic space $X$. Then the \textit{\'etale site} consists of those surjective coverings whose maps are \'etale.
\end{definition}
We are interested in the cohomology of local systems on analytic adic spaces. This is expected to be difficult because the \'etale site is more complicated (even locally) as compared to schemes.
\begin{example}
	Fix complete algebraic closure $C$ of $\QQ_p$, let $\OO_C$ be the ring of integers and $\mf m_C$ be the maximal ideal. Then the unit disk $\mathbb D=\op{Spa}(C\langle T\rangle,\OO_C\langle T\rangle)$ has
	\[\mathrm H^1(\mathbb D;\FF_p)\cong\mathrm H^1(\mathbb D;\mu_p)\cong\frac{(1+\mf m_CT\langle T\rangle)}p,\]
	where the last equality follows from Kummer theory. Notably, this surjects onto $\mf m_C/p\mf m_C$, so this dimension is infinite and grows with the cardinality of $C$.
\end{example}
This problem will be solved by using perfectoid spaces.
\begin{enumerate}
	\item To start, we will want our rigid spaces $X$ to be covered locally by perfectoid spaces: namely, for an affine $U\subseteq X$, we want to be able to find an affine $S$ covering $U$ by an inverse limit of \'etale morphisms (which is a pro-\'etale morphism).
	\item We will want the cohomology of affinoid perfect spaces $Y$ to be controlled. More precisely, up to pro-\'etale covering, one can achieve $\mathrm H^i(Y;\FF_p)=0$ for $i>1$. In short, by tilting, one reduces to characteristic $0$, and then one can use Artin--Schreier.
\end{enumerate}

\subsection{The Pro\'etale Site}
Having seen the use of pro-\'etale morphisms, we define our site.
\begin{definition}
	Fix an analytic adic space $X$. Then the \textit{pro\'etale site} $X_{\mathrm{pro\acute et}}$ has its objects which are pro-systems $\{U_i\}$ whose internal maps $U_i\to U_j$ are finite \'etale for large $i$ and $j$. A covering is said to be \textit{pro\'etale} if and only if it is further surjective.
\end{definition}
\begin{remark}
	More intuitively, one can define a pro\'etale site on the category of perfectoid spaces over $\FF_p$ where the coverings literally have their morphisms as limits of \'etale morphisms. Showing this requires showing that rigid analytic spaces are sheaves on this site.
\end{remark}
\begin{remark}
	There is a morphism $X_{\mathrm{pro\acute et}}\to X_{\mathrm{\acute et}}$ of sites.
\end{remark}
\begin{remark}
	For $\FF_p$-local systems, it turns out that the cohomology on the \'etale and pro\'etale sites agree.
\end{remark}
Now that we have our site, we can define some completed structure sheaves.
\begin{definition}
	Fix an analyti adic space $X$. Then we define the structure sheaf $\OO^+/p^n$ as having
	\[\OO^+/p^n(U)\coloneqq R^+/p^n\]
	for any affinoid perfectoid $U=\op{Spa}(R,R^+)$. We then define $\widehat{\OO}^+\coloneqq\lim\OO^+/p^\bullet$ and $\widehat{\OO}\coloneqq\widehat{\OO}^+[1/p]$.
\end{definition}
\begin{remark}
	Fix some $\FF_p$-local system $\mathbb L$. Then there is an Artin--Schreier sequence
	\[0\to\mathbb L\to\mathbb L\otimes\OO^+/p\to\mathbb L\otimes\OO^+/p\to0.\]
\end{remark}
\begin{theorem}[Primitive comparison]
	Let $X$ be a proper rigid analytic space. THen
	\[\mathrm H^i(X;\mathbb L\otimes\OO^+/p)\cong\mathrm H^i(X;\mathbb L)\otimes\OO_C/p.\]
	(This is an almost isomorphism.)
\end{theorem}
The moral is that we can reduce to understanding cohomology for $\OO^+/p$-local systems. With more work, Scholze showed the following.
\begin{theorem}[Scholze]
	FIx a proper rigid analytic space $X$. Then $\mathrm H^i_{\mathrm{\acute et}}(X;\mathbb L)$ is finite for any $\FF_p$-local system $\mathbb L$.
\end{theorem}
\begin{remark}
	There is also a Poincar\'e duality, but it is hard. One way to do this is to construct a $6$-functor formalism for $\OO^+/p$-cohomology and use a version of the Primitive comparison theorem.
\end{remark}

\subsection{Banach--Colmez Spaces}
There are a number of conventions which I do not understand. For example, all maps of $v$-stacks are assumed to be $!$-able, whatever that means. Today, we will give the category $\mathrm{Perf}$ of perfectoid spacaecs over $\FF_p$ the pro\'etale topology.
\begin{definition}[compact]
	A collection $\{f_\bullet\colon X_\bullet\to X\}$ of morphisms is \textit{compact} if and only if any quasicompact $U\subseteq X$ admits $f_1,\ldots,f_n$ for which there are quasicompact $U_i\subseteq X_i$ with
	\[U=f_1(U_1)\cup\cdots f_n(U_n).\]
	A collection in $\mathrm{Perf}_{\mathrm{pro\acute et}}$ is a covering if and only if the maps are pro\'etale and the collection is compact.
\end{definition}
\begin{example}
	There is a tilted structure sheaf $\OO^\flat\colon\op{Spa}(R,R^+)\mapsto R$.
\end{example}
\begin{definition}
	A \textit{diamond} is a sheaf of the form $X/R$ where $X$ is perfectoid and $R$ is a perfectoid equivalence.
\end{definition}
\begin{example}
	Rigid analytic spaces and perfectoid spaces are diamonds.
\end{example}
\begin{example}
	If a group $G$ acts on a space $X$, then $X/G$ is a diamond.
\end{example}
Here is a distinguished perfectoid space.
\begin{definition}[Fargue--Fontaine curve]
	Fix a perfectoid space $S$, and let $\varphi_S$ be the absolute Frobenius. Then we define
	\[X_S\coloneqq(S\times\op{Spd}\QQ_p)/\varphi_S^\ZZ\]
	to be the \textit{Fargue--Fontaine curve} relative to $S$.
\end{definition}
\begin{example}
	There is a natural functor $\Phi$ which takes isocrystals over $\FF_p$ to $\op{Bun}X_S$. We define $\OO(\lambda)\coloneqq\Phi(D_\lambda)$, where $D_\lambda$ is the isocrystal of slope $\lambda$. It turns out that if we write $\lambda=d/r$ in reduced form, then $\op{rank}\OO(\lambda)=r$ and $\deg\OO(\lambda)=d$.
\end{example}
Here are our objects of interest today.
\begin{definition}
	Fix a positive $\lambda$. Then the \textit{negative Banach--Colmez space} $\op{BC}(-\lambda)$ sends a perfectoid $S$ to
	\[\mathrm H^1(X_S;\OO(-\lambda))=\op{Ext}^1_{X_S}(\OO(\lambda),\OO).\]
	This is a pro\'etale sheaf. There is an open subsheaf $\op{BC}(-\lambda)^*$ classifying the semistable extensions.
\end{definition}
\begin{example}
	If $\lambda=-1$, then we see that $\op{BC}(-1)$ will classify extensions
	\[0\to\OO\to\mc E\to\OO(1)\to0.\]
	It turns out that this is either split (with $\mc E=\OO\oplus\OO(1)$) or semistable (with $\mc E=\OO(1/2)$).
\end{example}
\begin{remark}
	There is a strange phemenon that cohomology of pro\'etale $\QQ_p$-local systems on a proper rigid space over $\CC_p$ can be infinite-dimensional over $\QQ_p$. The reason turns out to be that the cohomology comes from $\op{BC}$s, and one can recover finiteness from there even when the $\QQ_p$-dimension is infinite.
\end{remark}

\subsection{Cohomological Calculations}
Here is our main result.
\begin{theorem}
	Fix positive $\lambda$.
	\begin{listalph}
		\item The cohomology $\mathrm H^i(\op{BC}(-\lambda);\OO^\flat)$ is concentrated in degree $i=0$, where it is $\FF_p$.
		\item The cohomology $\mathrm H^i(\op{BC}(-1/r);\OO^\flat)$ is concentrated in degrees $i\in\{0,r+1\}$, where is $\FF_p$ (up to a Tate twist).
	\end{listalph}
\end{theorem}
In short, one starts by noting that $\op{BC}(-\lambda)$ is not a diamond, but one can calculate that $\op{BC}(-1)\times\op{Spd}C$ is some reasonable quotient of $\AA^1_{\CC_p}$ by the translation action of $\QQ_p$. (Something similar is known for $\op{BC}(-1/r)$.) Thus, it more or less remains to extrat out cohomology of $\op{BC}(-1)$, so we need some nice properties of our cohomology.

One can make the site more wild without losing anything.
\begin{definition}[$v$-topology]
	A collection in $\mathrm{Perf}_v$ is a \textit{$v$-covering} if and only if it satisfies compactness.
\end{definition}
\begin{remark}
	It turns out that $\mathrm H^*_v(-;\OO^\flat)\simeq\mathrm H^*_{\mathrm{pro\acute et}}(-;\OO^\flat)$, but it tends to be easier to calculate in the $v$-topology because the local structure is better-behaved.
\end{remark}
\begin{theorem}
	There is a $6$-functor formalism on the (derived) category of (solid) quasicoherent $\OO^\flat$-mod\-ules.
	\begin{listalph}
		\item If $f$ is proper, then $f$ is cohomologically proper (i.e., $f_*=f_!$).
		\item If $f$ is smooth, then $f$ is cohomologically smooth (i.e., $f^!=f^*[2\dim X]$).
	\end{listalph}
\end{theorem}
\begin{remark}
	The necessity of the condensed mathematics here more or less comes from the desire to work with the strange quotient $\AA^1_{\CC_p}/\QQ_p$.
\end{remark}
Our second main result extracts out the properties of this $6$-functor formalism that we will need.
\begin{theorem}
	We work over $\CC_p$ for simplicity.
	\begin{listalph}
		\item K\"unneth isomorphism: for rigid $X$ and $Y$, one has
		\[\mathrm R\Gamma_c(X\times Y;\OO^\flat)\simeq\mathrm R\Gamma_c(X;\OO^\flat)\otimes\mathrm R\Gamma_c(Y;\OO^\flat).\]
		\item Drinfeld's lemma: let $Z\to\op{Spd}\FF_p$ be a $v$-stack. Then there is an isomorpshim
		\[\mathrm R\Gamma(Z\times_{\op{Spd}\FF_p}[\op{Spd}C/\varphi_C];\OO^\flat)\simeq\mathrm R\Gamma(Z;\OO^\flat).\]
	\end{listalph}
\end{theorem}
One now does a long calculation.

\section{December 8: The Goncharov Program}
This talk was given by Ismael Sierra Del Rio at MIT for the algebraic topology seminar.

\subsection{Algebraic \texorpdfstring{$K$}{ K}-Theory}
We are interested in studying the $K$-theory of an infinite field. In particular, we will want to compute the groups $K_*(F)_\QQ$.
\begin{example}
	One has that $K_0(F)$ is $K_0(\mathrm{Vec}F)$, which is $\ZZ$.
\end{example}
\begin{example}
	One can calculate $K_1(F)_\QQ=F^\times_\QQ$. This object admits the following somewhat bizarre presentation: it is the quotient
	\[\frac{\QQ[F^\times]}{([a\cdot b]-[a]\cdot[b])}.\]
\end{example}
\begin{example}
	Matsumoto computed
	\[K_2(F)_\QQ=\frac{\lambda^2_\QQ F_\QQ^\times}{([a]\land[1-a])}.\]
\end{example}
\begin{example}
	One can split $K_3(F)_\QQ$ into the sum of a Milnor part $K_3^{\mathrm M}(F)_\QQ$ and an indecomposable part $K_3^{\mathrm{ind}}(F)_\QQ$.
	\begin{itemize}
		\item The Milnor part is $\land^3F_\QQ^\times/([a]\land[1-a]\land[b])$.
		\item Suslin computed that the indecomposable part is the kernel of a map
		\[\frac{\QQ\langle F^\times\rangle}{\left(\{a\}+\{b\}+\left\{\frac{1-a}{1-ab}\right\}+\left\{\frac{1-b}{1-ab}\right\}+\{1-ab\}\right)}\to\land^2F^\times_\QQ,\]
		where the map sends $\{a\}\mapsto[a]\land[1-a]$.
	\end{itemize}
\end{example}
Though the presentation of $K_3$ looks bizarre, it is part of a general feature.
\begin{theorem}
	One has a weight decomposition
	\[K_d(F)_\QQ=\bigoplus_{m=1}^dK_d^{(m)}(F)_\QQ,\]
	where the Adams operator $\psi^{\ell}$ acts on the $m$th component by $\ell^m$. Furthermore, $K_d^{(1)}=0$ for $d\ge2$.
\end{theorem}
Here, the Adams operations are induced by taking exterior powers of vector spaces.
\begin{remark}
	The Milnor $K$-theory group $K_d^{\mathrm M}(F)_\QQ$ is the top weight piece.
\end{remark}

\subsection{Goncharov Program}
We start with the following conjecture.
\begin{conj}[Beilinson--Deligne]
	For any field $F$, there is a Tannakian category of mixed Tate motives $\mathrm{MTM}(F)$ generated by the Tate motive $\QQ(-1)$. Furthermore,
	\[K_{2m-i}^{(m)}(F)_\QQ=\op{Ext}^i_{\mathrm{MTM}(F)}(\QQ(-m),\QQ(0)).\]
\end{conj}
\begin{remark}
	Being Tannakian implies that one expects to be able to extract a ``motivic'' graded Lie coalgebra $L_\bullet(F)$ so that $\op{MTM}(F)$ consists of representations of $L_\bullet(F)$. The isomorphism to the representation category allows one to realize $\op{Ext}^i_{\mathrm{MTM}(F)}(\QQ(-m),\QQ(0))$ as the $m$th graded piece of $\mathrm H_i(L_\bullet(F))$.
\end{remark}
\begin{remark}
	When $F$ is a number field, the existence of this category is known. One basically passes to the category of Voevodsky motives and then finds the mixed Tate motives living inside.
\end{remark}
Let's make this more explicit. The homology groups $\mathrm H_i(L_\bullet(F))$ come from taking homology of the complexes
% https://q.uiver.app/#q=WzAsMTIsWzEsMiwiTF8xIl0sWzEsMSwiTF8yIl0sWzIsMSwiXFxsYW5kXjJMXzEiXSxbMSwwLCJMXzMiXSxbMSwzLCJpPTEiXSxbMiwzLCJpPTIiXSxbMCwyLCJtPTEiXSxbMCwxLCJtPTIiXSxbMCwwLCJtPTMiXSxbMywzLCJpPTMiXSxbMiwwLCJMXzFcXG90aW1lcyBMXzIiXSxbMywwLCJcXGxhbmReM0xfMiJdLFsxLDJdLFszLDEwXSxbMTAsMTFdXQ==&macro_url=https%3A%2F%2Fraw.githubusercontent.com%2FdFoiler%2Fnotes%2Fmaster%2Fnir.tex
\[\begin{tikzcd}[cramped]
	{m=3} & {L_3} & {L_1\otimes L_2} & {\land^3L_2} \\
	{m=2} & {L_2} & {\land^2L_1} \\
	{m=1} & {L_1} \\
	& {i=1} & {i=2} & {i=3}
	\arrow[from=1-2, to=1-3]
	\arrow[from=1-3, to=1-4]
	\arrow[from=2-2, to=2-3]
\end{tikzcd}\]
and then these homology groups recover the weight pieces of algebraic $K$-theory.

However, this description is not so explicit because $L_\bullet$ was conjured from Tannakian reconstruction. The Goncharov program aims to provide explicit constructions for these $L_i$. For example, $L_3$ is conjectured to be $\QQ\langle F^\times\rangle$ modulo some $22$-term relation. Let's explain wher such relations come from.
\begin{conj}[Goncharov]
	The group $L_m(F)$ is the quotient
	\[\frac{\QQ\langle\op{Li}_{m_1,\ldots,m_k}(a_1,\ldots,a_k):a_1,\ldots,a_k\in F,\sum_im_i=m\rangle}{(\text{relations})},\]
	and the relations can be made explicit.
\end{conj}
\begin{remark}
	The functions $\op{Li}_{m_1,\ldots,m_k}$ are some explicit periods of some motives, and the relations arise from the expected relations of these periods.
\end{remark}
\begin{remark}
	Goncharov also gave a formula for the co-algebra bracket of $L_\bullet$.
\end{remark}
Here is our main result.
\begin{theorem}
	If $F$ is a number field, then $L_m(F)$ exists, it is spanned by the multiple polylogarithm functions, and all the relations are known.
\end{theorem}
\begin{corollary}
	There is an explicit presentation of $K_d(F)_\QQ$.
\end{corollary}
\begin{remark}
	In fact, the ranks of $K_d(F)_\QQ$ were already known, due to Borel.
\end{remark}
\begin{corollary}
	The universality conjecture follows for number fields: all periods coming from mixed Tate motives arise from polylogarithms.
\end{corollary}
\begin{example}
	The special values $\zeta_F(m)$ are periods of mixed Tate motives.
\end{example}

\subsection{The Homology of \texorpdfstring{$\op{GL}_n$}{ GLn}}
We are interested in computing the homology of $F$. To $F$, we associate the $\mathbb E_\infty$-algebra
\[R(F)\coloneqq\bigsqcup_{m\ge1}\op{BGL}_m(F).\]
We are only interested in rational homology, so we are allowed to tensor with $\QQ$ to get $R(F)_\QQ$, which lives in the stable $\infty$-category of $\mathrm{Mod}_\QQ$. Separating out the gradings allows us to define
\[\mathrm H_{m,d}(R(F)_\QQ)\coloneqq\mathrm H_d(\op{BGL}_m(F);\QQ).\]
Algebriac $K$-theory arises from the homotopy groups of $R(F)$, so it now makes sense that we will be interested in the groups $\mathrm H_{m,d}(R(F)_\QQ)$.

Here is one way to approach the calculation: there is a left adjoint to the inclusion from trivial $\mathbb E_\infty$-algebras to all $\mathbb E_\infty$-algebras, and it is named $\mathrm B^{\mathrm{com}}$. Here is the sort of thing which was already known.
\begin{theorem}
	One can calculate
	\[\mathrm H_{m,d}(B^{\mathrm{com}}R(F)_\QQ)=\mathrm H_{d-2m+2}(\op{GL}_m(F);\mathrm{St}^\infty_m(F)_\QQ),\]
	where $\mathrm{St}^\infty_m(F)_\QQ$ is some module related to the Steinberg module.
\end{theorem}
\begin{corollary}
	The homology vanishes in the range $d<2m-2$.
\end{corollary}
\begin{example}
	For $m\ge2$, one can calculate
	\[\mathrm H_0(\op{GL}_m(F);\mathrm{St}^\infty_m(F)_\QQ)=0.\]
\end{example}
Now, it turns out that there is a spectral sequence which computes $K_\bullet(F)_\QQ$ from $\mathrm H_{m,d}(B^{\mathrm{conn}}R(F)_\QQ)$. But there is more structure here: $\Sigma B^{\mathrm{conn}}R(F)_\QQ$ is a Lie coalgebra! It is now reasonable to expect that we can relate this story to mixed Tate motives.
\begin{notation}
	The \textit{Goncharov} Lie coalgebra is
	\[\mc G_\bullet(F)\coloneqq\bigoplus_{m\ge1}\mathrm H_1(\op{GL}_m(F);\mathrm{St}^\infty_m(F)_\QQ).\]
\end{notation}
\begin{conj}
	There is an isomorphism
	\[L_\bullet(F)\cong\mc G_\bullet(F).\]
\end{conj}
\begin{theorem}
	Fix an infinite field $F$.
	\begin{listalph}
		\item The coalgebra $\mc G_m(F)$ is spanned by symbols coming from multiple polylogarithms.
		\item All relations are known, and they are already known.
		\item The cobracket on $\mc G$ is the expected object.
		\item One can compute $\mc G_3(F)$ is the expected $L_3(F)$.
	\end{listalph}
\end{theorem}
\begin{theorem}
	Fix a number field $F$. Then there is a canonical map
	\[\mc G_\bullet(F)\to L_\bullet(F)\]
	which is an isomorphism of Lie coalgebras. This map sends the multiple polylogarithm symbols to each other.
\end{theorem}
\begin{proof}[Idea]
	Construction of the map is a matter of matching relations on both sides. Note that $L_\bullet(F)$ has homology concentrated in degree $1$, so it turns out to be free. On the other hand, the groups $\mathrm H_d(\op{BGL}_m(F)_\QQ)$ are understood due to Borel--Yang (both abstractly and with generating classes), so one can calculate $\mc G_\bullet(F)$ explicitly as well. One then works harder, I guess.
\end{proof}

\section{December 12: Heegner Cycles and \texorpdfstring{$p$}{ p}-Adic \texorpdfstring{$L$}{ L}-functions}
This talk was given by Rui at Johns Hopkins University for the automorphic forms learning seminar.

\subsection{The Main Theorem}
Throughout today, $p$ is a large prime (not dividing $2$, $(2r-1)!$, nor $N\varphi(N)$), $\chi$ is some Dirichlet character of conductor coprime to $N$, $f$ is a newform of weight $2r$ and level $N$, $N$ is a product of primes split in an imaginary quadratic field $K$, and $p=\mf p\overline{\mf p}$ is also split in $K$.
\begin{theorem}
	Suppose that $f$ is ordinary at $p$. If $L(f,\chi,r)\ne0$, then
	\[\dim_F\op{Sel}(V_{f,\chi}/K)=0,\]
	where $V_{f,\chi}$ is the associated Galois representation.
\end{theorem}
\begin{remark}
	Here, $V_{f,\chi}=V_f(r)\otimes\chi$, where $V_f$ is the Galois representation associated to $f$.
\end{remark}
The first ingredient to the proof is a $p$-adic Gross--Zagier formula, which we lift from Bertolini--Darmon--Prasad.
\begin{theorem}
	Fix a newform $f\in S_{2r}(\Gamma_0(N))$, and asume the strong Heegner condition that $N$ is a product of primes split in $K$. Additionally, choose an elliptic curve $A$ with complex multiplication by $\OO_K$, from which one can construct some generalized Heegner cycles $z_{f,\chi}\in\mathrm H^1_f(K;T\otimes\chi)$, where $T\subseteq V$ is some Galois-stable $\ZZ_p$-lattice.

	Now, fix an anticyclotomic character $\psi$ of infinity type $(r,-r)$ and conductor $c_0\OO_K$, and choose some $\phi$ which is the $p$-adic avatar (via $\CC\cong\CC_p$) of an anticyclotomic character of infinity type $(r+j,-j-r)$, where $-r<j<r$ and $p$-power conductor. Then
	\[\frac{\mc L(f)(\hat\phi^{-1})}{\Omega_p^{-2j}}\sim\langle\log\op{loc}_{\mf p}z_{f,\chi},c\rangle,\]
	where $c$ is some explicit class, and $\log$ is the Bloch--Kato logarithm.
\end{theorem}
Let's now sketch the idea.
\begin{enumerate}
	\item The $p$-adic Gross--Zagier formula explains that the infinity types with $j\ge r$ have a $p$-adic $L$-function which interpolate $L$-values.
	\item On the other hand, Waldspurger's formula explains that the region $-r\le j\le r$ has a $p$-adic $L$-function which interpolates pairings from the generalized Heegner cycles.
	\item It turns out that it is the same $L$-function in both points, so one is able to conclude that the $p$-adic $L$-function interpolates some kind of extension of the Heegner cycles beyond their region of definition, to $j\ge r$.
	\item These Heegner cycles in their region of definition are unramified, but when extended, they become ramified, and the previous step explains that there is an $L$-value which predicts if the ramification is actually nontrivial.
	\item The presence of such ramified classes force the Selmer group to vanish.
\end{enumerate}

\section{December 12: Higher Algebraic Geometry}
This talk was given by Germ\'an Stefanich at Johns Hopkins University as a colloqium talk. This is joint work with Peter Scholze, David Ben-Zvi, and David Nadler.

\subsection{Categorification of Algebraic Geometry}
Let's start by reviewing classical algebraic geometry. Fix an algebraically closed field $k$, and then algebraic geometry is interested in the solution sets to some polynomials in affine space. These make the affine algebraic varieties, and they come equipped with a sheaf of $k$-algebras of regular functions.
\begin{theorem}[Hilbert]
	Fix an algebraically closed field $k$. There is an equivalence of categories between the category of affine $k$-varieties and integral $k$-algebras of finite type.
\end{theorem}
It was realized that one can relax working with $k$-algebras.
\begin{theorem}[Grothendieck]
	There is an equivalence of categories between affine schemes and commutative rings.
\end{theorem}
Of course, in algebraic geometry, one is not just interested in affine schemes but rather schemes, which are obtained by carefully gluing affine schemes.
\begin{example}
	The projective line $\PP^1$ admits a covering by two affine lines $\AA^1$, gluing along $\AA^1\setminus\{0\}$.
\end{example}
For our talk, instead of gluing to produce more geometric objects, we will generalize the notion of commutative ring.
\begin{remark}
	Commutative rings came up in our story because our functions formed sheaves of rings. Thus, to work with more general rings, we need to work with more general functions.
\end{remark}
\begin{example}
	Even though the functions on $\PP^1$ are all constants (and hence not very interesting), the category of vector bundles on $\PP^1$ is much more interesting (though also well-understood). Unsururprisingly, it will actually be more convenient to work with quasicoherent sheaves (for categorical reasons).
\end{example}
The above example fits into a more general framework of ``categorification.''
\begin{itemize}
	\item Numbers can be categorified into vector spaces. One recovers the number by taking the dimension.
	\item Functions can be categorified into quasicoherent sheaves. One recovers the function by taking sections.
	\item One can add functions, which becomes sums of sheaves.
	\item Additionally, one can multiply, which become tensor products of sheaves.
\end{itemize}
Under the above framework, we see that the structure sheaf $\OO_X$ can be categorified into $\mathrm{QCoh}(X)$.
\begin{remark}
	In the affine case, this basically corresponds to taking a commutative ring $A$ to $\mathrm{Mod}(A)=\mathrm{QCoh}(\Spec A)$.
\end{remark}
\begin{theorem}
	There is a fully faithful functor from commutative rings to symmetric monoidal categories which sends a ring $A$ to $\mathrm{Mod}(A)$.
\end{theorem}
One may hope to upgrade the algebraic side to make this into an equivalence.

\subsection{Higher Categorification}
It turns out that $\mathrm{QCoh}$ is not enough to distinguish geometry if we move to stacks.
\begin{example}
	Consider the stack $\mathrm B^2\mathbb G_m$, which classifies Azumaya algebras. Then $\OO\left(\mathrm B^2\mathbb G_m\right)$ only has constants, and $\mathrm{QCoh}\left(\mathrm B^2\mathbb G_m\right)$ only has vector spaces. However, there is a $2$-category $\mathrm{2QCoh}\left(\mathrm B^2\mathbb G_m\right)$ (of quasicoherent sheaves of categories) is interesting.
\end{example}
\begin{remark}
	Working with symmetric monoidal $2$-categories is genuinely more interesting.
\end{remark}
This example generalizes to some $\mathrm B^n\mathbb G_m$, so we see that we are going to need symmetric monoidal (i.e., ``stable'') $\infty$-categories.
\begin{definition}[categorical ring]
	A \textit{categorical ring} is a compatible sequence
	\[(A_0,A_1,A_2,\ldots),\]
	where $A_n$ is a symmetric monoidal $n$-category.
\end{definition}
The compatibility refers to some projection functor which takes symmetric monoidal $n$-categories to symmetric monoidal $(n-1)$-categories.
\begin{remark}
	Secretly, all these objects may want to be $\infty$-categories.
\end{remark}
So here is our main definition.
\begin{definition}[gestalten]
	The category of \textit{shapes} is the opposite category of the category of categorical rings.
\end{definition}
\begin{remark}
	Affine schemes do not hav a good theory of gluing: this is the purpose of schemes. However, the category of shapes turns out to have a good theory of gluings. For example, the category of shapes includes the category of quasicoherent quasiseparated schemes.
\end{remark}
\begin{remark}
	Intuitively, a shape consists of the data of a categorical ring $(A_0,A_1,A_2,\ldots)$, where $A_0$ is the functions on the shape, $A_1$ is the sheaves on the shape, and so on.
\end{remark}
It remains to motivate the need for the notion of a shape.
\begin{example}
	There is a moduli ``shape'' of (one-dimensional) topological quantum field theories. Here, a (one-dimensional) topological quantum field theory (over a field $k$) is a symmetric monoidal functor $\mathrm{Cob}\to\mathrm{Vec}(k)$, where $\mathrm{Cob}$ is the $1$-category of cobordisms. It turns out that our shape is given by the shape corresponding to $\mathrm{Cob}$. For example, one can show that $\pi_0$ of the moduli shape of invertible one-dimensional topological quantum field theories is just $\ZZ/2\ZZ$; approximately speaking, this corresponds to the theory of ``superlinear algebra.'' Notably, working in higher dimensions predicts the existence of certain higher superlinear algebras.
\end{example}
For our next example, we note that the category of shapes admits a six-functor formalism. Six-functor formalisms frequently admit an associated cohomology theory: $\mc D$-modules yields de Rham cohomology, \'etale sheaves yields \'etale cohomology, and Betti sheaves yields singular cohomology.
\begin{theorem}
	For every six-functor formalism $\mathrm{Shv}$ and variety $X$, there is a canonical shape $X_{\mathrm{Shv}}$ for which $\mathrm{QCoh}(X_{\mathrm{Shv}})=\mathrm{Shv}(X)$.
\end{theorem}
This explains the bizarre fact that six-functor formalisms frequently look like quasicoherent sheaves on some stack.
\begin{example}
	There is a six-functor formalism $\mathrm{SH}$ of (Voevodsky or Ayoub) motivic sheaves. Then one can view $[\Spec\ZZ]_{\mathrm{SH}}$ as the moduli shape of cohomology theories. It also turns out to parameterize ring stacks.
\end{example}
\begin{example}
	There is a six-functor formalism $\mathrm{IndCoh}$, which is like $\mathrm{QCoh}$ but knows about singularities. If $X$ is smooth, one expects that turns out that $\mathrm{2QCoh}(X_{\mathrm{IndCoh}})$ is an invariant of $T^*X$. One can prove this when $X=V$ is a finite-dimensional vector space.
\end{example}

\end{document}
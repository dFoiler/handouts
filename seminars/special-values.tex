\documentclass{article}
\usepackage[utf8]{inputenc}

\newcommand{\nirpdftitle}{Special Values Seminar}
\usepackage{import}
\inputfrom{../../notes}{nir}
\usepackage[backend=biber,
    style=alphabetic,
    sorting=ynt
]{biblatex}
\setcounter{tocdepth}{2}

\pagestyle{contentpage}

\title{Special Values}
\author{Nir Elber}
\date{Spring 2025}
\usepackage{graphicx}

\begin{document}

\maketitle

\tableofcontents

\section{Special Values of Dirichlet \texorpdfstring{$L$}{L}-Functions}
This talk was given by Rui. Roughly speaking, the style of these sorts of special values results is that someone observes some equalities, then one works out examples, we make a general conjecture, and eventually it is proven.

\subsection{Some Examples}
Let's begin by discussing the simplest $L$-function: the Riemann $\zeta$-function.
\begin{definition}
	The \textit{Riemann $\zeta$-function} $\zeta$ is defined by the series
	\[\zeta(s)\coloneqq\sum_{n=1}^\infty\frac1{n^s}\]
	for $s\in\CC$ such that $\Re s>1$.
\end{definition}
\begin{example}
	Here is what is known about some small special values. Euler showed that $\zeta(s)=\frac{\pi^2}6$, and Ap\'ery showed that $\zeta(3)$ is irrational.
\end{example}
\begin{remark}
	In general, there is a conjecture that the values $\{\pi,\zeta(3),\zeta(5),\ldots\}$ forms an algebrically independent set. Roughly speaking, this is expected by the Grothendieck period conjecture.
\end{remark}
Today, we will be happy working in only slightly larger generality, with Dirichlet $L$-functions.
\begin{definition}[Dirichlet character]
	Fix a positive integer $N$. Then a \textit{Dirichlet character$\pmod N$} is a character $\eta\colon(\ZZ/N\ZZ)^\times\to\CC^\times$. The Dirichlet character $\eta$ is \textit{primitive} if and only if it does not factor through $(\ZZ/D\ZZ)^\times$ for any divisor $D\mid N$. Further, we say that $\eta$ is even (respectively, odd) if and only if $\eta(-1)=1$ (respectively, $\eta(-1)=-1$).
\end{definition}
\begin{definition}[Dirichlet $L$-function]
	Given a Dirichlet character $\eta\pmod N$, we define the \textit{Dirichlet $L$-function} $L(\eta,s)$ by
	\[L(\eta,s)\coloneqq\sum_{n=1}^\infty\frac{\eta(n)}{n^s},\]
	where implicitly $\eta(n)=0$ whenver $\gcd(n,N)>1$.
\end{definition}
\begin{example}
	Let $\eta\colon(\ZZ/2\ZZ)^\times\to\CC^\times$ be the nontrivial character. Then $L(\eta,1)=\frac\pi4$.
\end{example}
\begin{example}
	Let $\eta\colon(\ZZ/8\ZZ)^\times\to\CC^\times$ be the character defined by $\eta(3)=\eta(5)=-1$. This can be proven via a trick. Consider the power series
	\[f(x)\coloneqq x-\frac13x-\frac15x^5+\frac17x^7+\cdots,\]
	from which one finds $f'(x)=1-x^2-x^4+x^6+\cdots=\frac{1-x^2-x^4+x^6}{1-x^8}$. Then one can integrate $f'(x)$ to get
	\[f(x)=\frac{\sqrt2}4\log\left|\frac{x^2+\sqrt 2x+1}{x^2-\sqrt x+1}\right|.\]
	It follows that $L(\eta,1)=\frac{\sqrt2}2\log\left(\sqrt2+1\right)$.
\end{example}
The previous example is an example of the class number formula, and right now it looks like a miracle. To give a taste for what is remarkable here, we note that $1+\sqrt2\in\ZZ[\sqrt2]^\times$ is a fundamental unit. As such, we expect some interesting arithmetic to be going on.

Here is a more general result.
\begin{theorem} \label{thm:baby-quadratic-class-number-formula}
	Suppose $\eta\pmod N$ is a primitive nontrivial Dirichlet character.
	\begin{listalph}
		\item If $\eta$ is even, then for any positive integer $m$, we have
		\[L(\eta,2m)\equiv\pi^{2m}\pmod{\ov\QQ^\times}.\]
		\item If $\eta$ is odd, then for any positive integer $m$, we have
		\[L(\eta,2m-1)\equiv\pi^{2m-1}\pmod{\ov\QQ^\times}.\]
	\end{listalph}
\end{theorem}
The above is an instance of Deligne's conjecture.

For another general result, we note that an even primitive quadratic character $\eta\colon(\ZZ/N\ZZ)^\times\to\{\pm1\}$ has kernel which is an index-$2$ subgroup of $(\ZZ/N\ZZ)^\times\cong\op{Gal}(\QQ(\zeta_N)/\QQ)$, so it corresponds to a quadratic extension $F$ of $\QQ$. In fact, the fact that $\eta$ is even tells us that complex conjugation fixes $F$, so $F$ is totally real. It then turns out that
\[L(\eta,1)\equiv\sqrt{\op{disc}\OO_F}\cdot\log\left|u_F\right|\pmod{\QQ^\times},\]
where $u_F$ is a fundamental unit of $\OO_F$. This also comes from the class number formula, and it is an instance of Beilinson's conjecture.

\subsection{Funtional Equations}
As usual, to write down a suitable functional equation for our $L$-functions, we must add some archimedean factors.
\begin{definition}[completed Dirichlet $L$-function]
	Let $\eta\colon(\ZZ/N\ZZ)^\times\to\CC^\times$ be a primitive Dirichlet character. Then we define $d\in\{0,1\}$ by $\eta(-1)=(-1)^d$ and then
	\[L_\infty(\eta,s)\coloneqq\pi^{-\frac{s+\delta}2}\Gamma\left(\frac{s+\delta}2\right).\]
	Then the \textit{completed Dirichlet $L$-function} is $\Lambda(\eta,s)\coloneqq L_\infty(\eta,s)L(\eta,s)$.
\end{definition}
\begin{remark}
	Recall that $\Gamma(s)$ is defined by
	\[\Gamma(s)\coloneqq\int_0^\infty e^{-t}t^s\,\frac{dt}t\]
	for $s$ such that $\Re s>0$. One also knows that $\Gamma(s)$ admits a meromorphic continuation with understood poles, and it has a functional equation $\Gamma(s+1)=s\Gamma(s)$.
\end{remark}
And here is our functional equation.
\begin{theorem} \label{thm:dirichlet-functional-equation}
	Let $\eta\colon(\ZZ/N\ZZ)^\times\to\CC^\times$ be a primitive Dirichlet character. Then $L(\eta,s)$ admits a meromorphic continuation (with poles only at $s\in\{0,1\}$ only when $\eta$ is trivial) to all $\CC$ and satisfies a functional equation
	\[\Lambda(\eta,s)=\varepsilon(\eta,s)\Lambda\left(\eta^{-1},1-s\right),\]
	where $\varepsilon(\eta,s)$ is some appropriately normalized Gauss sum.
\end{theorem}
We will not prove this today (it is mildly technical). Instead, we will use it to show a partial version of \Cref{thm:baby-quadratic-class-number-formula}. With that said, we will need to do something in the direction of a meromorphic continuation because we will try to understand negative integer values of $L(\eta,s)$.

By expanding out the series, we see that
\[\Gamma(s)L(\eta,s)=\int_0^\infty\sum_{\substack{n\ge1\\\gcd(n,N)=1}}\eta(n)e^{-nt}t^s\,\frac{dt}t=\int_0^\infty\frac1{1-e^{-Nt}}\sum_{n=0}^{N_1}\eta(n)e^{-nt}\,\frac{dt}t.\]
One now plugs into the general machine that produces analytic continuation and functional equations.

\begin{lemma}
	Choose a smooth Schwarz function $f\colon\RR_{\ge0}\to\RR$. Then
	\[L(f,s)\coloneqq\frac1{\Gamma(s)}\int_0^\infty f(t)t^{s}\,\frac{dt}t\]
	has an analytic continuation to all $\CC$ and satisfies $L(f,-n)=(-1)^nf^{(n)}(0)$ for all $n\ge0$.
\end{lemma}
\begin{proof}
	To control singularities, we let $\varphi\colon\RR_{\ge0}\to\RR$ be a smooth bump function satisfying $\varphi|_{[0,1]}=1$ and $\varphi_{[2,\infty)}=0$. Thus, if we expand $f=f_1+f_2$ with $f_1=\varphi f$ and $f_2=(1-\varphi)f$, we see that
	\[\int_0^\infty f_2(t)t^s\,\frac{dt}t=\int_1^\infty f_2(t)t^s\,\frac{dt}t,\]
	and the rapid decay of $f$ grants this term an analytic continuation to all $\CC$, and it even satisfies
	\[L(f_2,-n)=\left(\frac1{\Gamma(-s)}\int_1^\infty f_2(t)t^s\,\frac{dt}t\right)\Bigg|_{s=-n}=0.\]
	Thus, we are allowed to ignore $f_2$ piece. For the $f_1$ part, we inductively integrate by parts. For example, our first integration by parts produces
	\[L(f,s)=\underbrace{\frac1{\Gamma(s)}f_1(t)\frac{t^s}s\bigg|_0^\infty}_0-\frac1{s\Gamma(s)}\int_0^\infty f(t)t\cdot t^s\,\frac{dt}t=-L(f_1',s+1).\]
	Thus, we have moved out $s$ to $s+1$, and we can iteratively produce the needed continuation from the argument above. The result on the special value follows from a computation.
\end{proof}
One can now use the lemma to see that
\[L(\eta,-n)\in\QQ(\eta).\]
Then one can use the functional equation \Cref{thm:dirichlet-functional-equation} to prove \Cref{thm:baby-quadratic-class-number-formula} after tracking everything through. I apologize, but I chose not to write down the details.

\section{The Kummer Congruence and \texorpdfstring{$p$}{p}-Adic Analysis on \texorpdfstring{$\ZZ_p$}{ Zp}}
This talk was given by Mitch. We would like to motivate $p$-adic $L$-functions and prove the Kummer congruences, which are used in their construction.

\subsection{The Kummer Congruence}
Last time, we had an equality of the form
\[\int_{\RR^+}\underbrace{\frac1{1-e^{-Nt}}\sum_{n=1}^{N-1}\eta(n)e^{-nt}}_{f_\eta(t)\coloneqq}\cdot t^s\,\frac{dt}t=\Gamma(s)L(\eta,s),\]
where $\eta\pmod N$ is some primitive Dirichlet character. The moral of the story is that we see that we are taking a Mellin transform of some function $f_\eta(t)$, so it may be interesting to study these functions on their own terms.

For example, if $\eta=1$ is the trivial character, then one finds that
\[tf_1(t)=\frac t{1-e^{-t}}=\sum_{m=0}^\infty B_m\frac{t^m}{m!},\]
where $\{B_m\}_{m\ge0}$ are the Bernoulli numbers. (Indeed, this is a definition of the Bernoulli numbers.) More generally, one can expand
\[tf_\eta(t)=\sum_{m=0}^\infty B_{\eta,m}\frac{t^m}{m!}\]
to define ``twisted'' Bernoulli numbers.

For our special values result, we found an identity
\[L(f,-n)=(-1)^nf^{(n)}(0),\]
where $\Gamma(s)L(f,s)$ refers to the Mellin transform, which eventually implies a special values result
\[L(\eta,-n)=-\frac{(-1)^{n+1}B_{\eta,n+1}}{n+1}\]
after some rearrangement. Parity arguments actually allow us to more or less ignore the sign $(-1)^{n+1}$. Namely, when $n$ is even, then $L(\eta,-n)=0$ for even $n$ (unless $\eta$ is trivial); and when $n$ is odd, then $L(\eta,-n)=0$ for $n\ge1$ odd.

We are now ready to state our Kummer congruences.
\begin{theorem}[Kummer congruence] \label{thm:kummer-cong}
	Fix a nontrivial primitive Dirichlet character $\eta\pmod N$. Fix a prime $p$ coprime to $N$. Choose nonnegative integers $n_1$, $n_2$, and $k$ such that $n_1,n_2\ge k$ and $n_1\equiv n_2\pmod{(p-1)p^{k-1}}$. Then
	\[-\frac{B_{\eta,n_1+1}}{n_1+1}\equiv-\frac{B_{\eta,n_2+1}}{n_2+1}\pmod{p^k}.\]
	If $\eta$ is trivial, then we also need to require $p\nmid(n_1-1)(n_2-1)$.
\end{theorem}
The moral of the story is that the special values of $L(\eta,s)$ (at integers) admit some kind of continuity in $\ZZ_p$. This will motivate us to define a $p$-adic $L$-function which interpolates these values. This interpolation will turn out to be a profittable thing to do, essentially due to Euler systems.
\begin{remark}
	Here is a historical remark. For reasons related to Fermat's last theorem, Kummer was interested in the notion of a ``regular prime.'' Namely, an odd prime $p$ is found to be regular if and only if $p\nmid\#\op{Cl}(\QQ(\zeta_p))$, which turns out to be equivalent to the prime $p$ not dividing any of the numerators of $B_2,B_4,\ldots,B_{p-3}$.
\end{remark}

\subsection{Using the \texorpdfstring{$p$}{p}-Adic \texorpdfstring{$L$}{L}-Function}
Let's begin to describe what a $p$-adic $L$-function should be. Fix a prime $p$ and some (space of) characters $\eta^{(p)}\colon(\ZZ/N\ZZ)^\times\to\CC^\times$ where $\eta^{(p)}\pmod N$ is a Dirichlet character with $p\nmid N$. Additionally, we fix some $\eta_p\colon(\ZZ/p^r\ZZ)^\times\to\CC^\times$, and we would like to ``interpolate'' the values
\[L\left(\eta^{(p)}\eta_p,-n\right)\]
as $n\ge0$ varies. More precisely, we will find that $L$ should be thought of as a measure where $\eta_p$ is an input.

For our construction, we choose some $f_{\eta_p,n}\colon\ZZ_p^\times\to\ov\QQ_p^\times$ given by $f_{\eta_p,n}(a)\coloneqq\eta_p(a)a^n$. Then we will be able to appropriately interpolate with this function.
\begin{remark}
	Note that $f_{\eta_p,n}$ can be thought of as a Galois representation of $\op{Gal}(\QQ(\zeta_{p^\infty})/\QQ)$.
\end{remark}
Now, we note
\[L^{\{p\}}\left(\eta^{(p)}\eta_p,s\right)\coloneqq\prod_{\substack{q\text{ prime}\\\gcd(q,Np)=1}}\frac1{1-\eta^{(p)}(q)q^{-s}}=\begin{cases}
	L\left(\eta^{(p)}\eta_p,s\right) & \text{if }\eta_p\ne1, \\
	L\left(\eta^{(p)}\eta_p,s\right)\left(1-\eta(p)p^{-s}\right) & \text{if }\eta_p=1.
\end{cases}\]
Morally, the $L^{\{p\}}$ product simply removes any problems at $p$, which are relevant while we are working $p$-adically. The interpolation now appeals to the following result.
\begin{theorem} \label{thm:construct-p-adic-l-func}
	Fix a primitive Dirichlet character $\eta^{(p)}\pmod N$ with $p\nmid N$. Then there is a $p$-adic measure $d\mu_{\eta^{(p)}}$ such that for any Dirichlet character $\eta_p\pmod{p^k}$ admits
	\[\int_{\ZZ_p^\times}\eta_p(x)x^n\,d\mu_{\eta^{(p)}}(x)=L^{\{p\}}(\eta^{(p)}\eta_p,-n).\]
\end{theorem}
\begin{remark} \label{rem:p-adic-tate-thesis}
	It is worth comparing this statement to Tate's thesis, where we represent some (completed) $L$-function of a Hecke character $\chi$ as the Mellin transform against a character. The bizarre measure $\mu_{\eta^{(p)}}$ can be seen as incorporating the bizarre prime-to-$p$ parts of the character $\chi$.
\end{remark}
We have not bothered to define $p$-adic integration, but let's explain why this implies \Cref{thm:kummer-cong} first.
\begin{proof}[Proof that \Cref{thm:construct-p-adic-l-func} implies \Cref{thm:kummer-cong}]
	This proof is rather formal. Write $\eta$ as $\eta^{(p)}\eta_p$, where $\eta^{(p)}$ as conductor prime to $p$, and $\eta_p$ has conductor which is a power of $p$. Now, for $n$ large (say, $n\ge k$), we see that
	\[L^{\{p\}}\left(\eta^{(p)}\eta_p,-n\right)=\left(1-\eta(p)p^{-n}\right)L\left(\eta^{(p)}\eta_p,-n\right)\equiv L\left(\eta^{(p)}\eta_p,-n\right)\pmod{p^k}\]
	if $\eta_p$ is trivial, and the statement is still true when $\eta_p$ is nontrivial. Thus, after plugging in our special values result as $-\frac{B_{\eta,n+1}}{n+1}=L(\eta,-n)$, and in light of \Cref{thm:construct-p-adic-l-func}, we would like to show
	\[\int_{\ZZ_p^\times}\eta_p(x)x^{n_1}\,d\mu_{\eta^{(p)}}(x)\stackrel?\equiv\int_{\ZZ_p^\times}\eta_p(x)x^{n_2}\,d\mu_{\eta^{(p)}}(x)\pmod{p^k}\]
	whenever $n_1\equiv n_2\pmod{(p-1)p^{k-1}}$. This last equivalence holds on the level of the integrands because we are looking$\pmod{p^k}$.
\end{proof}

\subsection{Integration}
Let's say something about how $\mu_{\eta^{(p)}}$ functions.
\begin{remark}
	Do note that we are not looking for the usual Haar measure: small cosets receive size $1/p^\bullet$, which is large $p$-adically. Additionally, this will have basically no hope of incorporating the prime-to-$p$ information discussed in \Cref{rem:p-adic-tate-thesis}.
\end{remark}
So let's rebuild some functional analysis so that we can value our measures in $\QQ_p$.
\begin{definition}[Banach space]
	Fix a complete valued $p$-adic field $K$. A \textit{Banach space} over $K$ is a complete normed vector space $B$ over $K$ whose norm $\norm\cdot$ satisfies the triangle inequaltiy
	\[\norm{v_1+v_2}\le\norm{v_1}+\norm{v_2}.\]
\end{definition}
\begin{example}
	Fix a compact topological space $X$. Then the space $C^0(X,\QQ_p)$ of continuous functions $X\to\QQ_p$ is a Banach space over $\QQ_p$. The norm is given by $\norm\cdot_\infty$.
\end{example}
\begin{definition}[orthonormal basis]
	Fix a Banach space $B$ overa complete valued field $p$-adic $K$. Then an \textit{orthonormal basis} is a set $\{e_i\}\subseteq B$ such that $\norm{e_i}=1$ for all $i$, and any vector $v$ admits a unique expansion
	\[v=\sum_ix_ie_i,\]
	which converges in the sense $x_i\to0$ (namely, $\#\{i\in I:\left|x_i\right|\ge\varepsilon\}$ is finite for all $\varepsilon>0$) where $\norm v=\max_i\left|x_i\right|$.
\end{definition}
\begin{remark}
	We are not requiring that $\{e_i\}$ be countable. The condition that $x_i\to0$ also includes a hypothesis that only finitely many of the $x_\bullet$s are nonzero.
\end{remark}
Our key example will be $C^0(\ZZ_p,\QQ_p)$. Here is a nice basis of this space.
\begin{example}
	For nonnegative integers $n\ge0$, define the function $\binom xn\colon\ZZ_p\to\QQ_p$ as the polynomial
	\[\binom xn=\frac{x(x-1)\cdots(x-(n-1))}{n!}.\]
	This is continuous because polynomials are continuous. We also remark that $\norm{\binom xn}_\infty=1$, which can be seen by checking on the dense subset $\ZZ\subseteq\ZZ_p$.
\end{example}
\begin{proposition}
	The functions $\left\{\binom xn\right\}_{n\ge0}$ form an orthonormal basis of $C^0(\ZZ_p,\QQ_p)$.
\end{proposition}
\begin{proof}
	We proceed in steps.
	\begin{enumerate}
		\item We remark that these binomials provide a basis of the polynomial functions $\ZZ\to\ZZ$, which is why we may expect this to be true. Indeed, the idea is to consider finite differences, and binomials appear for reasons related to a finite-difference version of the binomial theorem. Explicitly, for $f\in C^0(\ZZ_p,\QQ_p)$, we define the finite differences $f^{[n]}\colon\ZZ_p\to\QQ_p$ recursively by
		\[\begin{cases}
			f^{[0]}(x)\coloneqq f(x), \\
			f^{[n+1]}(x)\coloneqq f^{[n]}(x+1)-f^{[n]}(x).
		\end{cases}\]
		One can show by induction that
		\[f^{[n]}(x)=\sum_{k=0}^n(-1)\binom nkf(x+n-k),\]
		where the key point is to use Pascal's identity $\binom n{k-1}+\binom nk=\binom{n+1}k$ in the inductive step.

		\item We now take a moment to explain how these finite differences extract coefficients $a_n(f)$ so that $f=\sum_na_n(f)\binom\cdot n$. Indeed, if we already had an expansion $f(x)=\sum_{n\ge0}a_n(f)\binom xn$, then one finds (by induction) that
		\[f^{[k]}(x)=\sum_{n\ge0}a_n(f)\binom x{n-k},\]
		where again the point is to use Pascal's identity, so taking $x=0$ finds $a_n(f)=f^{[n]}(0)$. We remark that this paragraph shows that the expansion of $f$ into binomial coefficients is thus unique.

		\item It remains to show that taking $a_n(f)\coloneqq f^{[n]}(0)$ has $a_n(f)\to0$ as $n\to\infty$ and $f(x)=\sum_{n\ge0}a_n(f)\binom xn$. By scaling, we may assume that $\norm f_\infty=1$, and the explicit formula for $f^{[n]}(0)$ then verifies that $\left|a_n(f)\right|\le1$ for all $n$. In this step, we will check that $a_n(f)\to0$.

		The main claim is that any continuous $f\colon\ZZ_p\to\ZZ_p$ has some $\nu$ such that $\im f^{[p^\nu]}\subseteq p\ZZ_p$. Indeed, because $f$ is continuous, there is $k$ such that the reduction $f\pmod p$ is constant on the cosets of $\ZZ_p/p^\nu\ZZ_p$, which implies that
		\[f^{[p^\nu]}(x)=\sum_{k=0}^{p^\nu}(-1)^k\binom{p^\nu}kf\left(x+p^\nu-k\right)\equiv f\left(x+p^\nu\right)-f(x)\equiv0\pmod p,\]
		as required.

		Applying the claim inductively to the various finite differences of $f$, we see that there is a sequence of integers $\nu_0\le\nu_1\le\cdots$ such that $\im f^{[p^{\nu_\bullet}]}\subseteq p^\bullet\ZZ_p$ for each $\nu_\bullet$. Thus, for $n\ge\nu_\bullet$, we see that $a_n(f)\in p^\bullet\ZZ_p$, completing this step.

		\item We now check that $f(x)=\sum_{n\ge0}a_n(f)\binom xn$. We remark that these are both continuous functions $\ZZ_p\to\ZZ_p$ because we now know that $a_n(f)\to0$ as $n\to\infty$. Thus, it is enough to check the equality on the dense subset $\ZZ\subseteq\ZZ_p$. One can then show equality on $\ZZ$ by induction: one needs to show that
		\[f(n)=\sum_{k=0}^na_n(f)\binom nk\]
		for each $n$, which can be done directly.
		\qedhere
	\end{enumerate}
\end{proof}

\section{More on Measures}
This talk was given by Rui Chen; I was not present for it. Today we set up some measure theory on $\ZZ_p$.

\subsection{The Amice Transform}
From the perspective of the Riesz representation theorem, a measure on $\ZZ_p$ is really just a distribution: it is some way to take controlled (say, compactly supported) functions and produce a number (which is the integral). Here is the corresponding definition over $\ZZ_p$.
\begin{definition}[distribution]
	The space of \textit{distributions} on a profinite group $G$ is the module
	\[\mc D_0(G,\ZZ_p)\coloneqq\op{Hom}_{\mathrm{cont}}\left(C^0(G,\ZZ_p),\ZZ_p\right).\]
	Given some continuous $f\colon G\to\ZZ_p$, we may write $\int_G f\,d\mu$ or $\int_{G}f(x)\,d\mu(x)$ for $\mu(f)$.
\end{definition}
\begin{example}[Dirac distribution]
	Fix some $g\in G$. Then there is a Dirac distribution $\delta_g\in\mc D_0(G,\ZZ_p)$ given by
	\[\int_Gf\,d\delta_g\coloneqq f(g).\]
\end{example}
\begin{remark}
	By continuity, a distribution $\mu\in\mc D_0(\ZZ_p,\ZZ_p)$ is uniquely determined by its values on the binomials $x\mapsto\binom xn$. Indeed, $\mu\mapsto\left(\mu\left(\binom xn\right)\right)\in\ZZ_p^{\ZZ_{\ge0}}$, which we claim is an isomorphism. We will explain this again later, so let's be quick. It is certainly some $\ZZ_p$-linear map, and it is injective because we can expand out any $f\in C^0(\ZZ_p,\ZZ_p)$ as $f(x)=\sum_{n\ge0}a_n(f)\binom xn$ with $a_n(f)\to0$ as $n\to\infty$, thereby yielding
	\[\mu(f)=\sum_{n\ge0}a_n(f)\mu\left(\binom xn\right)\]
	by continuity. Lastly, the above formula successfully defines a distribution no matter how we choose $\mu\left(\binom xn\right)$, which proves surjectivity of the constructed map.
\end{remark}
The previous remark allows us to identify a measure by an infinite tuple of elements in $\ZZ_p$. It will turn out to be convenient to identify such infinite tuples with $\ZZ_p[[T]]$ for the following reason.
\begin{definition}[Amice transform]
	Given $\mu\in\mc D_0(\ZZ_p,\ZZ_p)$, we define the \textit{Amice transform} $A_\mu(T)\in\ZZ_p[[T]]$ by
	\[A_\mu(T)\coloneqq\int_{\ZZ_p}(1+T)^x\,d\mu(x).\]
\end{definition}
\begin{remark}
	Intuitively, we are integrating the measure $\mu$ against the ``character'' $x\mapsto(1+T)^x$, so the Amice transform is a special case of a Fourier transform.
\end{remark}
Let's quickly explain how to interpret $A_\mu(T)$ as an element of $\ZZ_p[[T]]$: the idea is to use the ``binomial theorem'' to expand $(1+T)^x$ as
\begin{align*}
	A_\mu(T) &= \int_{\ZZ_p}(1+T)^x\,d\mu(x) \\
	&= \int_{\ZZ_p}\sum_{n\ge0}\binom xnT^n\,d\mu(x) \\
	&\stackrel*= \sum_{n\ge0}\left(\int_{\ZZ_p}\binom xn\,d\mu(x)\right)T^n \\
	&= \sum_{n\ge0}\mu\left(\binom xn\right)T^n.
\end{align*}
Here, the interchange of the sum and integral $\stackrel*=$ should be understod as a formal operation because we have not previously defined what it means to integrate a formal power series in $\ZZ_p[[T]]$.

\subsection{Distributions via the Iwasawa Algebra}
The above construction is better understood in more generality. Let's begin by reinterpreting $\ZZ_p[[T]]$ in a more canonical way.
\begin{definition}[Iwasawa algebra]
	Fix a profinite group $G$. Then we define the \textit{Iwasawa algebra} as
	\[\ZZ_p[[G]]\coloneqq\limit_{[H:G]<\infty}\ZZ_p[G/H].\]
\end{definition}
\begin{example}
	We compute $\ZZ_p[[\ZZ_p]]$. Indeed, for any $\nu\ge0$, we note that there is a surjection $\ZZ_p[[T]]\onto\ZZ_p[\ZZ_p/p^\nu\ZZ_p]$ given by $T\mapsto[1]-1$, and it becomes a bijection upon passing to the inverse limit. Thus, morally, we ought to view $T\in\ZZ_p[[T]]$ as providing an infinitesimal generator for functions centered at $1\in\ZZ_p$.
\end{example}
\begin{example}
	Fix a profinite group $G=\limit G_\bullet$, we find
	\[\mc D_0(G,\ZZ_p)=\op{Hom}_{\mathrm{cts}}\left(C^0(\limit G_\bullet,\ZZ_p),\ZZ_p\right)=\limit\op{Hom}\left(C^0(G_\bullet,\ZZ_p),\ZZ_p\right).\]
	Now, a distribution on the finite set $G_\bullet$ amounts to assigning a weight in $\ZZ_p$ to each element of $G_\bullet$, so it can be identified with $\ZZ_p[G_\bullet]$ (via the map $g\mapsto\delta_g$, where $\delta_g\in\mc D_0(G_\bullet,\ZZ_p)$ is the indiator of $g$). We conclude that we are looking at $\ZZ_p[[G]]$, and the isomorphism $\ZZ_p[[G]]\to\mc D_0(G,\ZZ_p)$ is given by extending $g\mapsto\delta_g$.
\end{example}
Thus, we see that the Iwasawa algebra provides a natural home for our distributions. For example, $\ZZ_p[[G]]$ has a natural multiplication structure, so we can look for this structure in distributions.
\begin{definition}[convolution]
	Fix a profinite group $G$, and choose two measures $\mu_1,\mu_2\in\mc D_0(G,\ZZ_p)$. Then we define the \textit{convolution} $(\mu_1\star\mu_2)\in\mc D_0(\ZZ_p)$ by
	\[\int_Gf(g)\,d(\mu_1\star\mu_2)(g)\coloneqq\int_G\int_Gf(gh)\,d\mu_1(g)\,d\mu_2(h).\]
\end{definition}
\begin{remark}
	Technically, we ought to check that the function
	\[h\mapsto\int_Gf(gh)\,d\mu_1(g)\]
	is a continuous map $G\to\ZZ_p$. Let $\ell_\bullet\colon G\times C^0(G,\ZZ_p)\to C^0(G,\ZZ_p)$ denote right translation by $h$; then we would like to show that $h\mapsto\mu_1(\ell_h(f))$ is continuous. By composing appropriately, it is enough to check $\ell_\bullet$ is a continuous function, which holds as soon as we give $C^0(G,\ZZ_p)$ the correct topology. (For example, one can take a limit over the finite groups making up $G$.)
\end{remark}
\begin{remark}
	It is not hard to check that convolution makes $\mc D_0(G,\ZZ_p)$ into a ring without identity.
\end{remark}
\begin{proposition}
	Fix a profinite group $G$. The isomorphism $\ZZ[[G]]\to\mc D_0(G,\ZZ_p)$ given by $g\mapsto\delta_g$ is an isomorphism of rings.
\end{proposition}
\begin{proof}
	By taking a limit, it is enough to check this at finite level, allowing us to assume that $G$ is finite. Then any measure is a linear combination of Dirac distributions, so it is enough to check that $\delta_{x}\star\delta_{y}=\delta_{xy}$. Well, we compute
	\begin{align*}
		\int_Gf\,d(\delta_{x}\star\delta_{y}) &= \int_G\int_Gf(gh)\,d\delta_x(g)\,d\delta_y(h) \\
		&= \int_Gf(xh)\,d\delta_y(h) \\
		&= f(xy),
	\end{align*}
	as required.
\end{proof}
Anyway, we close this section by explaining the Amice transform.
\begin{proposition}
	The Amice transform $A_\bullet\colon\mc D_0(\ZZ_p,\ZZ_p)\to\ZZ_p[[T]]$ is the composite of the isomorphisms
	\[\mc D_0(\ZZ_p,\ZZ_p)\cong\ZZ_p[[\ZZ_p]]\cong\ZZ_p[[T]].\]
\end{proposition}
\begin{proof}
	This is a direct computation. In short, it is enough to check this on finite level and then pass to the limit. Now, a measure $\mu$ on $\ZZ_p/p^\nu\ZZ_p$ becomes the polynomial
	\[\sum_{a\in\ZZ_p/p^\nu\ZZ_p}\mu\left(a+p^\nu\ZZ_p\right)[a]\in\ZZ_p[\ZZ_p/p^\nu\ZZ_p],\]
	which then goes to the polynomial
	\[\sum_{a\in\ZZ_p/p^\nu\ZZ_p}\mu\left(a+p^\nu\ZZ_p\right)(1+T)^a,\]
	which is $\int_{\ZZ_p}(1+T)^x\,d\mu(x)$.
\end{proof}

\section{The \texorpdfstring{$p$}{p}-Adic Zeta Function}
This talk was given by me. I did not have time to write up notes.

\section{The Analytic Class Number Formula}
This talk was given by Xin Wei.

\subsection{Artin \texorpdfstring{$L$}{L}-Functions}
Our first topic today is about building $L$-functions attached to Galois representations. For our notation, $F$ will be a number field, and we let $M_{F,f}$ denote the set of its finite places. Let's give some adjectives to our Galois representation.
\begin{definition}
	Fix a Galois representation $\rho\colon\op{Gal}(\ov F/F)\to\op{GL}_n(\ov\QQ_p)$.
	\begin{listalph}
		\item $\rho$ is \textit{unramified} at $v\in M_{F,f}$ if and only if $\rho$ is trivial on the inertia subgroup $I_v$ at $v$.
		\item $\rho$ is \textit{nice} if and only if it is unramified at all but finitely many places $v$ satisfying the following conditions.
		\begin{itemize}
			\item For each $v\in M_{F,f}$ away from $p$, the characteristic polynomial of $\rho(\mathrm{Frob}_v)$ acting on $V^{I_v}$ has algebraic coefficients.
			\item For each $v\in M_{F,f}$ above $p$, the local representation $\rho|_{\op{Gal}(\ov F_v/F_v)}$ is de Rham, and the characteristic polynomial of $\rho(\mathrm{Frob}_v)$ acting on the module $\mathbb D_{\mathrm{pst}}(\rho_v)^{I_v}=(\rho_v|_{L_w}\otimes B_{\mathrm{st}})^{\op{Gal}(L_w/F_v)}$ (coming from $p$-adic Hodge theory) has algebraic coefficients.
		\end{itemize}
	\end{listalph}
\end{definition}
\begin{remark}
	Intuitively, ``de Rham'' means potentially semistable.
\end{remark}
\begin{remark}
	When the image of $\rho$ is finite, one can replace the submodule coming from $p$-adic Hodge theory with just $V^{I_v}$.
\end{remark}
We are now ready to define our local $L$-factors.
\begin{definition}
	Fix a nice Galois representation $\rho\colon\op{Gal}(\ov F/F)\to\op{GL}_n(\ov\QQ_p)$. Then we define
	\[L_v(\rho_v,s)\coloneqq\begin{cases}
		\det\left(1-\rho_v(\mathrm{Frob}_v)q_v^{-s};V^{I_v}\right)^{-1} & \text{if }v\mid p. \\
		\det\left(1-\rho_v(\mathrm{Frob}_v)q_v^{-s};\mathbb D_{\mathrm{st}}(\rho_v)^{I_v}\right)^{-1} & \text{if }\nmid p.
	\end{cases}\]
	Then we define $L(\rho,s)\coloneqq\prod_{v<\infty}L_v(\rho_v,s)$ if it converges for $\Re s$ large enough.
\end{definition}
\begin{remark}
	It is not even known in general if $L(\rho,s)$ always converges if $\Re s$ is large enough, though this is true if $\rho$ comes from geometry (such as if $\im\rho$ is finite). Similarly, little is known about meromorphic continuation or functional equation, where most results are known if one is able to associate $\rho$ to an automorphic form.
\end{remark}
\begin{example}[Dirichlet character]
	For a primitive Dirichlet character $\eta\pmod N$, there is a Galois representation $\widetilde\eta\colon\op{Gal}(\ov\QQ/\QQ)\to\ov\QQ^\times$ given by the following composite.
	\[\arraycolsep=1.4pt\begin{array}{ccccccc}
		\op{Gal}(\ov\QQ/\QQ) &\onto& \op{Gal}(\QQ(\zeta_N)/\QQ) &\simeq& (\ZZ/N\ZZ)^\times &\to& \ov\QQ^\times \\
		\mathrm{Frob}_\ell &\mapsto& \mathrm{Frob}_\ell &\mapsto& \ell^{-1} &\mapsto& \eta(\ell)^{-1}
	\end{array}\]
	Here, one should take $\eta(p)=0$ for $p\mid N$, though this does not matter much. Because we have described explicitly what happens to the Frobenius elements above, we see that $L(\widetilde\eta,s)=L\left(\eta^{-1},s\right)$.
\end{example}
\begin{example}[Tate twist]
	There is a cyclotomic character $\chi_{\mathrm{cyc}}\colon\op{Gal}(\ov F/F)\to\ZZ_p^\times$ given by restriction to $\op{Gal}(F(\mu_{p^\infty})/F)\onto\op{Gal}(\QQ(\mu_{p^\infty})/\QQ)=\ZZ_p^\times$. Expanding out this construction, we see that $\sigma(\zeta)=\zeta^{\chi_{\mathrm{cyc}}(\sigma)}$ for any $\zeta\in\mu_{p^\infty}$. We may write this representation as the Tate twist $\ZZ_p(1)$ in the sequel; then $\ZZ_p(n)\coloneqq\ZZ_p(1)^{\otimes n}$ for $n\ge0$ and $\ZZ_p(n)=\ZZ_p(-n)^\lor$ for $n<0$.
\end{example}
The up-shot of taking Tate twists is that we are able to shift $L$-functions. In particular, for a general Galois representation $\rho\colon\op{Gal}(\ov F/F)\to\op{GL}_{\ov\QQ_p}(V)$ and $v\nmid p$, one can compute
\[L_v(V(n),s)=\det\left(1-\rho(n)(\mathrm{Frob}_v)q_v^{-s}\right)^{-1}=\det\left(1-\rho(n)(\mathrm{Frob}_v)q_v^{-s-n}\right)^{-1}=L_v(V,s+n),\]
and actually the same holds for $v\mid p$.
\begin{example}
	Let's return to our $p$-adic measures $\mu_\eta$ briefly, where we recall that $\eta\colon(\ZZ/N\ZZ)^\times\to\ov\QQ_p^\times$ is some primitive Dirichlet character of conductor prime to $p$. Then for any primitive Dirichlet character $\chi\pmod{p^\nu}$, our measure $\mu_\eta$ satisfied
	\[\int_{\ZZ_p^\times}\chi(x)x^n\,d\mu_\eta(x)=L^{(p)}(\eta,\chi,-n)=L^{(p)}(\widetilde\eta^{-1}\widetilde\chi^{-1}(-n),0),\]
	allowing us to reinterpret the entire story in terms of Galois representations! 
\end{example}
\begin{remark}
	In general, one expects that a nice Galois representation $\rho$ should admit a measure $\mu_\rho$ such that well-behaved $\chi$ have
	\[\int_{\op{Gal}(\ov F/F)^{\mathrm{ab}}_p}\chi(x)\,d\mu_\rho(x)=L^{(p)}(\rho\otimes\chi,0),\]
	where we are integrating over the pro-$p$ part of the abelianization of the Galois group. We are now allowed to only work with $0$ by shifting the value over by enough Tate twists.
\end{remark}
Here are some basic properties.
\begin{proposition}
	The following hold.
	\begin{listalph}
		\item Summation: for nice Galois representations $\rho_1$ and $\rho_2$ of $F$, we have $L(\rho_1\oplus\rho_2,s)=L(\rho_1,s)L(\rho_2,s)$.
		\item Induction: for an extension $F'/F$ of number fields and nice Galois representation $\rho$ of $F'$, we have
		\[L\left(\op{Ind}_{\op{Gal}(\ov\QQ/F)}^{\op{Gal}(\ov\QQ/F')}\rho,s\right)=L(\rho,s).\]
	\end{listalph}
\end{proposition}
\begin{proof}[Sketch]
	Summation is checked place-by-place.  Induction is also checked place-by-place. Away from $p$, this merely relates to some careful tracking through of facts about prime-splitting; at $p$, one needs to work a little harder with the $p$-adic Hodge theory construction.
\end{proof}
\begin{example}
	One can view the Dedekind zeta function $\zeta_F$ as either coming from the trivial representation of $F$ or the induction to $F$ of the trivial representation $1$ of $\QQ$. This is easiest to check when the extension $F/\QQ$ is Galois.
\end{example}

\subsection{The Class Number Formula}
We now state a special values result.
\begin{theorem}
	Fix a number field $F$. Let $\zeta_F(s)$ be the Dedekind zeta function, and consider the following invariants.
	\begin{itemize}
		\item $r_1$ is the number of real embeddings $F\into\RR$.
		\item $r_2$ is the number of complex embeddings $F\into\CC$ counted up to conjugacy.
		\item $h_F$ is the class number of $\OO_F$.
		\item $w_F$ is the number of roots of unity in $\OO_F$.
		\item $\mathrm{Reg}_F$ is the covolume of the unit lattice sitting inside the trace-zero hyperplane of $\RR^{r_1+r_2}$. Namely, one embeds $\OO_F^\times\to\RR^{r_1}\times\RR^{r_2}$ by projecting along the various embeddings $F\into\CC$ and taking logarithms appropriately.
	\end{itemize}
	Then
	\[\lim_{s\to 1}(s-1)\zeta_F(s)=\frac{2^{r_1}(2\pi)^{r_2}\mathrm{Reg}_F h_F}{w_F\sqrt{\left|\op{disc}\OO_F\right|}}.\]
\end{theorem}
\begin{proof}[Sketch]
	Proving this requires one to know something analytic about $\zeta_F(s)$, so the best proofs come from Tate's thesis. In particular, Tate's thesis more or less implies that the residue of $\zeta_F(s)$ at $s=1$ is given by the volume of the norm-$1$ id\'ele class group, which produes the given formula after some effort.
\end{proof}
\begin{remark}
	Using the functional equation for $\zeta_F$, one can show that this is equivalent to
	\[\lim_{s\to0}s^{-r_1-r_2+1}\zeta_F(s)=-\frac{h_F\mathrm{Reg}_F}{w_F}.\]
\end{remark}
We will not say more about the proof, but we will give a few examples for flavor.
\begin{example}
	For $F=\QQ$, the right-hand side is
	\[\frac{2^1(2\pi)^0\cdot1\cdot1}{2\cdot1}=1.\]
	So it remains to show that $(s-1)\zeta(s)\to1$ as $s\to1$. For this, we rouhgly speaking have to know something about a meromorphic continuation of $\zeta(s)$ to $s=1$, so we follow the simplest such proof and write
	\[\zeta(s)=\sum_{n=1}^\infty\frac1{n^s}=\int_1^\infty\frac1{x^s}\,dx+O(1)=\frac1{s-1}+O(1).\]
\end{example}
\begin{proposition}
	We prove the analytic class number formula in the quadratic imaginary case.
\end{proposition}
\begin{proof}
	In this case, $(r_1,r_2)=(0,1)$, and the regulartor is $1$, and $w_F\in\{2,4,6\}$. Thus, the hardest contribution will come from the ideal class group. The idea is to stratify the sum according to ideal classes. For motivation, we begin by working with principal ideals, where we see our sum is
	\[\frac1{\#\OO_F^\times}\sum_{\alpha\in\OO_F\setminus\{0\}}\frac1{\op N\alpha^s}.\]
	Thus, we see that we are basically counting points in the lattice $\OO_F\subseteq\CC$ bounded by some circle. This is approximately the area of the corresponding ellipse, which can be computed by integrating along concentric circles. There are approximately $\frac2{\sqrt{\left|\op{disc}\OO_F\right|}}$ lattice points for each unit square (which can be seen by computing $\OO_F$), so our sum looks like
	\[\frac1{w_F}\int_1^\infty\frac2{\sqrt{\left|\op{disc}\OO_F\right|}}\cdot2\pi R\cdot R^{-2s}\,dR\]
	up to some constant which does not matter much. (To be rigorous, one would need to prove something a little technical about the Gauss circle problem.) This integral eventually collapses to
	\[\frac{2\pi}{w_F\sqrt{\left|\op{disc}\OO_F\right|}}.\]
	For the other ideal classes, one can multiply by an ideal in the inverse class so that we are still basically summing some sublattice of principal ideals, allowing us to appeal to the above argument (after passing to a sublattice of $\OO_F$); the final answer does not change. Summing over ideal classes completes the proof.
\end{proof}
\begin{remark}
	One can do something similar for real quadratic fields by counting points bounded by hyperbolas.
\end{remark}

\end{document}
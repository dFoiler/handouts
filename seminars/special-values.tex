\documentclass{article}
\usepackage[utf8]{inputenc}

\newcommand{\nirpdftitle}{Special Values Seminar}
\usepackage{import}
\inputfrom{../../notes}{nir}
\usepackage[backend=biber,
    style=alphabetic,
    sorting=ynt
]{biblatex}
\setcounter{tocdepth}{2}

\pagestyle{contentpage}

\title{Special Values}
\author{Nir Elber}
\date{Spring 2025}
\usepackage{graphicx}

\begin{document}

\maketitle

\tableofcontents

\section{Special Values of Dirichlet \texorpdfstring{$L$}{L}-Functions}
This talk was given by Rui. Roughly speaking, the style of these sorts of special values results is that someone observes some equalities, then one works out examples, we make a general conjecture, and eventually it is proven.

\subsection{Some Examples}
Let's begin by discussing the simplest $L$-function: the Riemann $\zeta$-function.
\begin{definition}
	The \textit{Riemann $\zeta$-function} $\zeta$ is defined by the series
	\[\zeta(s)\coloneqq\sum_{n=1}^\infty\frac1{n^s}\]
	for $s\in\CC$ such that $\Re s>1$.
\end{definition}
\begin{example}
	Here is what is known about some small special values. Euler showed that $\zeta(s)=\frac{\pi^2}6$, and Ap\'ery showed that $\zeta(3)$ is irrational.
\end{example}
\begin{remark}
	In general, there is a conjecture that the values $\{\pi,\zeta(3),\zeta(5),\ldots\}$ forms an algebrically independent set. Roughly speaking, this is expected by the Grothendieck period conjecture.
\end{remark}
Today, we will be happy working in only slightly larger generality, with Dirichlet $L$-functions.
\begin{definition}[Dirichlet character]
	Fix a positive integer $N$. Then a \textit{Dirichlet character$\pmod N$} is a character $\eta\colon(\ZZ/N\ZZ)^\times\to\CC^\times$. The Dirichlet character $\eta$ is \textit{primitive} if and only if it does not factor through $(\ZZ/D\ZZ)^\times$ for any divisor $D\mid N$. Further, we say that $\eta$ is even (respectively, odd) if and only if $\eta(-1)=1$ (respectively, $\eta(-1)=-1$).
\end{definition}
\begin{definition}[Dirichlet $L$-function]
	Given a Dirichlet character $\eta\pmod N$, we define the \textit{Dirichlet $L$-function} $L(\eta,s)$ by
	\[L(\eta,s)\coloneqq\sum_{n=1}^\infty\frac{\eta(n)}{n^s},\]
	where implicitly $\eta(n)=0$ whenver $\gcd(n,N)>1$.
\end{definition}
\begin{example}
	Let $\eta\colon(\ZZ/2\ZZ)^\times\to\CC^\times$ be the nontrivial character. Then $L(\eta,1)=\frac\pi4$.
\end{example}
\begin{example}
	Let $\eta\colon(\ZZ/8\ZZ)^\times\to\CC^\times$ be the character defined by $\eta(3)=\eta(5)=-1$. This can be proven via a trick. Consider the power series
	\[f(x)\coloneqq x-\frac13x-\frac15x^5+\frac17x^7+\cdots,\]
	from which one finds $f'(x)=1-x^2-x^4+x^6+\cdots=\frac{1-x^2-x^4+x^6}{1-x^8}$. Then one can integrate $f'(x)$ to get
	\[f(x)=\frac{\sqrt2}4\log\left|\frac{x^2+\sqrt 2x+1}{x^2-\sqrt x+1}\right|.\]
	It follows that $L(\eta,1)=\frac{\sqrt2}2\log\left(\sqrt2+1\right)$.
\end{example}
The previous example is an example of the class number formula, and right now it looks like a miracle. To give a taste for what is remarkable here, we note that $1+\sqrt2\in\ZZ[\sqrt2]^\times$ is a fundamental unit. As such, we expect some interesting arithmetic to be going on.

Here is a more general result.
\begin{theorem} \label{thm:baby-quadratic-class-number-formula}
	Suppose $\eta\pmod N$ is a primitive nontrivial Dirichlet character.
	\begin{listalph}
		\item If $\eta$ is even, then for any positive integer $m$, we have
		\[L(\eta,2m)\equiv\pi^{2m}\pmod{\ov\QQ^\times}.\]
		\item If $\eta$ is odd, then for any positive integer $m$, we have
		\[L(\eta,2m-1)\equiv\pi^{2m-1}\pmod{\ov\QQ^\times}.\]
	\end{listalph}
\end{theorem}
The above is an instance of Deligne's conjecture.

For another general result, we note that an even primitive quadratic character $\eta\colon(\ZZ/N\ZZ)^\times\to\{\pm1\}$ has kernel which is an index-$2$ subgroup of $(\ZZ/N\ZZ)^\times\cong\op{Gal}(\QQ(\zeta_N)/\QQ)$, so it corresponds to a quadratic extension $F$ of $\QQ$. In fact, the fact that $\eta$ is even tells us that complex conjugation fixes $F$, so $F$ is totally real. It then turns out that
\[L(\eta,1)\equiv\sqrt{\op{disc}\OO_F}\cdot\log\left|u_F\right|\pmod{\QQ^\times},\]
where $u_F$ is a fundamental unit of $\OO_F$. This also comes from the class number formula, and it is an instance of Beilinson's conjecture.

\subsection{Funtional Equations}
As usual, to write down a suitable functional equation for our $L$-functions, we must add some archimedean factors.
\begin{definition}[completed Dirichlet $L$-function]
	Let $\eta\colon(\ZZ/N\ZZ)^\times\to\CC^\times$ be a primitive Dirichlet character. Then we define $d\in\{0,1\}$ by $\eta(-1)=(-1)^d$ and then
	\[L_\infty(\eta,s)\coloneqq\pi^{-\frac{s+\delta}2}\Gamma\left(\frac{s+\delta}2\right).\]
	Then the \textit{completed Dirichlet $L$-function} is $\Lambda(\eta,s)\coloneqq L_\infty(\eta,s)L(\eta,s)$.
\end{definition}
\begin{remark}
	Recall that $\Gamma(s)$ is defined by
	\[\Gamma(s)\coloneqq\int_0^\infty e^{-t}t^s\,\frac{dt}t\]
	for $s$ such that $\Re s>0$. One also knows that $\Gamma(s)$ admits a meromorphic continuation with understood poles, and it has a functional equation $\Gamma(s+1)=s\Gamma(s)$.
\end{remark}
And here is our functional equation.
\begin{theorem} \label{thm:dirichlet-functional-equation}
	Let $\eta\colon(\ZZ/N\ZZ)^\times\to\CC^\times$ be a primitive Dirichlet character. Then $L(\eta,s)$ admits a meromorphic continuation (with poles only at $s\in\{0,1\}$ only when $\eta$ is trivial) to all $\CC$ and satisfies a functional equation
	\[\Lambda(\eta,s)=\varepsilon(\eta,s)\Lambda\left(\eta^{-1},1-s\right),\]
	where $\varepsilon(\eta,s)$ is some appropriately normalized Gauss sum.
\end{theorem}
We will not prove this today (it is mildly technical). Instead, we will use it to show a partial version of \Cref{thm:baby-quadratic-class-number-formula}. With that said, we will need to do something in the direction of a meromorphic continuation because we will try to understand negative integer values of $L(\eta,s)$.

By expanding out the series, we see that
\[\Gamma(s)L(\eta,s)=\int_0^\infty\sum_{\substack{n\ge1\\\gcd(n,N)=1}}\eta(n)e^{-nt}t^s\,\frac{dt}t=\int_0^\infty\frac1{1-e^{-Nt}}\sum_{n=0}^{N_1}\eta(n)e^{-nt}\,\frac{dt}t.\]
One now plugs into the general machine that produces analytic continuation and functional equations.

\begin{lemma}
	Choose a smooth Schwarz function $f\colon\RR_{\ge0}\to\RR$. Then
	\[L(f,s)\coloneqq\frac1{\Gamma(s)}\int_0^\infty f(t)t^{s}\,\frac{dt}t\]
	has an analytic continuation to all $\CC$ and satisfies $L(f,-n)=(-1)^nf^{(n)}(0)$ for all $n\ge0$.
\end{lemma}
\begin{proof}
	To control singularities, we let $\varphi\colon\RR_{\ge0}\to\RR$ be a smooth bump function satisfying $\varphi|_{[0,1]}=1$ and $\varphi_{[2,\infty)}=0$. Thus, if we expand $f=f_1+f_2$ with $f_1=\varphi f$ and $f_2=(1-\varphi)f$, we see that
	\[\int_0^\infty f_2(t)t^s\,\frac{dt}t=\int_1^\infty f_2(t)t^s\,\frac{dt}t,\]
	and the rapid decay of $f$ grants this term an analytic continuation to all $\CC$, and it even satisfies
	\[L(f_2,-n)=\left(\frac1{\Gamma(-s)}\int_1^\infty f_2(t)t^s\,\frac{dt}t\right)\Bigg|_{s=-n}=0.\]
	Thus, we are allowed to ignore $f_2$ piece. For the $f_1$ part, we inductively integrate by parts. For example, our first integration by parts produces
	\[L(f,s)=\underbrace{\frac1{\Gamma(s)}f_1(t)\frac{t^s}s\bigg|_0^\infty}_0-\frac1{s\Gamma(s)}\int_0^\infty f(t)t\cdot t^s\,\frac{dt}t=-L(f_1',s+1).\]
	Thus, we have moved out $s$ to $s+1$, and we can iteratively produce the needed continuation from the argument above. The result on the special value follows from a computation.
\end{proof}
One can now use the lemma to see that
\[L(\eta,-n)\in\QQ(\eta).\]
Then one can use the functional equation \Cref{thm:dirichlet-functional-equation} to prove \Cref{thm:baby-quadratic-class-number-formula} after tracking everything through. I apologize, but I chose not to write down the details.

\section{The Kummer Congruence and \texorpdfstring{$p$}{p}-Adic Analysis on \texorpdfstring{$\ZZ_p$}{ Zp}}
This talk was given by Mitch. We would like to motivate $p$-adic $L$-functions and prove the Kummer congruences, which are used in their construction.

\subsection{The Kummer Congruence}
Last time, we had an equality of the form
\[\int_{\RR^+}\underbrace{\frac1{1-e^{-Nt}}\sum_{n=1}^{N-1}\eta(n)e^{-nt}}_{f_\eta(t)\coloneqq}\cdot t^s\,\frac{dt}t=\Gamma(s)L(\eta,s),\]
where $\eta\pmod N$ is some primitive Dirichlet character. The moral of the story is that we see that we are taking a Mellin transform of some function $f_\eta(t)$, so it may be interesting to study these functions on their own terms.

For example, if $\eta=1$ is the trivial character, then one finds that
\[tf_1(t)=\frac t{1-e^{-t}}=\sum_{m=0}^\infty B_m\frac{t^m}{m!},\]
where $\{B_m\}_{m\ge0}$ are the Bernoulli numbers. (Indeed, this is a definition of the Bernoulli numbers.) More generally, one can expand
\[tf_\eta(t)=\sum_{m=0}^\infty B_{\eta,m}\frac{t^m}{m!}\]
to define ``twisted'' Bernoulli numbers.

For our special values result, we found an identity
\[L(f,-n)=(-1)^nf^{(n)}(0),\]
where $\Gamma(s)L(f,s)$ refers to the Mellin transform, which eventually implies a special values result
\[L(\eta,-n)=-\frac{(-1)^{n+1}B_{\eta,n+1}}{n+1}\]
after some rearrangement. Parity arguments actually allow us to more or less ignore the sign $(-1)^{n+1}$. Namely, when $n$ is even, then $L(\eta,-n)=0$ for even $n$ (unless $\eta$ is trivial); and when $n$ is odd, then $L(\eta,-n)=0$ for $n\ge1$ odd.

We are now ready to state our Kummer congruences.
\begin{theorem}[Kummer congruence] \label{thm:kummer-cong}
	Fix a nontrivial primitive Dirichlet character $\eta\pmod N$. Fix a prime $p$ coprime to $N$. Choose nonnegative integers $n_1$, $n_2$, and $k$ such that $n_1,n_2\ge k$ and $n_1\equiv n_2\pmod{(p-1)p^{k-1}}$. Then
	\[-\frac{B_{\eta,n_1+1}}{n_1+1}\equiv-\frac{B_{\eta,n_2+1}}{n_2+1}\pmod{p^k}.\]
	If $\eta$ is trivial, then we also need to require $p\nmid(n_1-1)(n_2-1)$.
\end{theorem}
The moral of the story is that the special values of $L(\eta,s)$ (at integers) admit some kind of continuity in $\ZZ_p$. This will motivate us to define a $p$-adic $L$-function which interpolates these values. This interpolation will turn out to be a profittable thing to do, essentially due to Euler systems.
\begin{remark}
	Here is a historical remark. For reasons related to Fermat's last theorem, Kummer was interested in the notion of a ``regular prime.'' Namely, an odd prime $p$ is found to be regular if and only if $p\nmid\#\op{Cl}(\QQ(\zeta_p))$, which turns out to be equivalent to the prime $p$ not dividing any of the numerators of $B_2,B_4,\ldots,B_{p-3}$.
\end{remark}

\subsection{Using the \texorpdfstring{$p$}{p}-Adic \texorpdfstring{$L$}{L}-Function}
Let's begin to describe what a $p$-adic $L$-function should be. Fix a prime $p$ and some (space of) characters $\eta^{(p)}\colon(\ZZ/N\ZZ)^\times\to\CC^\times$ where $\eta^{(p)}\pmod N$ is a Dirichlet character with $p\nmid N$. Additionally, we fix some $\eta_p\colon(\ZZ/p^r\ZZ)^\times\to\CC^\times$, and we would like to ``interpolate'' the values
\[L\left(\eta^{(p)}\eta_p,-n\right)\]
as $n\ge0$ varies. More precisely, we will find that $L$ should be thought of as a measure where $\eta_p$ is an input.

For our construction, we choose some $f_{\eta_p,n}\colon\ZZ_p^\times\to\ov\QQ_p^\times$ given by $f_{\eta_p,n}(a)\coloneqq\eta_p(a)a^n$. Then we will be able to appropriately interpolate with this function.
\begin{remark}
	Note that $f_{\eta_p,n}$ can be thought of as a Galois representation of $\op{Gal}(\QQ(\zeta_{p^\infty})/\QQ)$.
\end{remark}
Now, we note
\[L^{\{p\}}\left(\eta^{(p)}\eta_p,s\right)\coloneqq\prod_{\substack{q\text{ prime}\\\gcd(q,Np)=1}}\frac1{1-\eta^{(p)}(q)q^{-s}}=\begin{cases}
	L\left(\eta^{(p)}\eta_p,s\right) & \text{if }\eta_p\ne1, \\
	L\left(\eta^{(p)}\eta_p,s\right)\left(1-\eta(p)p^{-s}\right) & \text{if }\eta_p=1.
\end{cases}\]
Morally, the $L^{\{p\}}$ product simply removes any problems at $p$, which are relevant while we are working $p$-adically. The interpolation now appeals to the following result.
\begin{theorem} \label{thm:construct-p-adic-l-func}
	Fix a primitive Dirichlet character $\eta^{(p)}\pmod N$ with $p\nmid N$. Then there is a $p$-adic measure $d\mu_{\eta^{(p)}}$ such that for any Dirichlet character $\eta_p\pmod{p^k}$ admits
	\[\int_{\ZZ_p^\times}\eta_p(x)x^n\,d\mu_{\eta^{(p)}}(x)=L^{\{p\}}(\eta^{(p)}\eta_p,-n).\]
\end{theorem}
\begin{remark} \label{rem:p-adic-tate-thesis}
	It is worth comparing this statement to Tate's thesis, where we represent some (completed) $L$-function of a Hecke character $\chi$ as the Mellin transform against a character. The bizarre measure $\mu_{\eta^{(p)}}$ can be seen as incorporating the bizarre prime-to-$p$ parts of the character $\chi$.
\end{remark}
We have not bothered to define $p$-adic integration, but let's explain why this implies \Cref{thm:kummer-cong} first.
\begin{proof}[Proof that \Cref{thm:construct-p-adic-l-func} implies \Cref{thm:kummer-cong}]
	This proof is rather formal. Write $\eta$ as $\eta^{(p)}\eta_p$, where $\eta^{(p)}$ as conductor prime to $p$, and $\eta_p$ has conductor which is a power of $p$. Now, for $n$ large (say, $n\ge k$), we see that
	\[L^{\{p\}}\left(\eta^{(p)}\eta_p,-n\right)=\left(1-\eta(p)p^{-n}\right)L\left(\eta^{(p)}\eta_p,-n\right)\equiv L\left(\eta^{(p)}\eta_p,-n\right)\pmod{p^k}\]
	if $\eta_p$ is trivial, and the statement is still true when $\eta_p$ is nontrivial. Thus, after plugging in our special values result as $-\frac{B_{\eta,n+1}}{n+1}=L(\eta,-n)$, and in light of \Cref{thm:construct-p-adic-l-func}, we would like to show
	\[\int_{\ZZ_p^\times}\eta_p(x)x^{n_1}\,d\mu_{\eta^{(p)}}(x)\stackrel?\equiv\int_{\ZZ_p^\times}\eta_p(x)x^{n_2}\,d\mu_{\eta^{(p)}}(x)\pmod{p^k}\]
	whenever $n_1\equiv n_2\pmod{(p-1)p^{k-1}}$. This last equivalence holds on the level of the integrands because we are looking$\pmod{p^k}$.
\end{proof}

\subsection{Integration}
Let's say something about how $\mu_{\eta^{(p)}}$ functions.
\begin{remark}
	Do note that we are not looking for the usual Haar measure: small cosets receive size $1/p^\bullet$, which is large $p$-adically. Additionally, this will have basically no hope of incorporating the prime-to-$p$ information discussed in \Cref{rem:p-adic-tate-thesis}.
\end{remark}
So let's rebuild some functional analysis so that we can value our measures in $\QQ_p$.
\begin{definition}[Banach space]
	Fix a complete valued $p$-adic field $K$. A \textit{Banach space} over $K$ is a complete normed vector space $B$ over $K$ whose norm $\norm\cdot$ satisfies the triangle inequaltiy
	\[\norm{v_1+v_2}\le\norm{v_1}+\norm{v_2}.\]
\end{definition}
\begin{example}
	Fix a compact topological space $X$. Then the space $C^0(X,\QQ_p)$ of continuous functions $X\to\QQ_p$ is a Banach space over $\QQ_p$. The norm is given by $\norm\cdot_\infty$.
\end{example}
\begin{definition}[orthonormal basis]
	Fix a Banach space $B$ overa complete valued field $p$-adic $K$. Then an \textit{orthonormal basis} is a set $\{e_i\}\subseteq B$ such that $\norm{e_i}=1$ for all $i$, and any vector $v$ admits a unique expansion
	\[v=\sum_ix_ie_i,\]
	which converges in the sense $x_i\to0$ where $\norm v=\max_i\left|x_i\right|$.
\end{definition}
\begin{remark}
	We are not requiring that $\{e_i\}$ be countable. The condition that $x_i\to0$ also includes a hypothesis that only finitely many of the $x_\bullet$s are nonzero.
\end{remark}
Our key example will be $C^0(\ZZ_p,\QQ_p)$. Here is a nice basis of this space.
\begin{example}
	For nonnegative integers $n\ge0$, define the function $\binom xn\colon\ZZ_p\to\QQ_p$ as the polynomial
	\[\binom xn=\frac{x(x-1)\cdots(x-(n-1))}{n!}.\]
	This is continuous because polynomials are continuous. We also remark that $\norm{\binom xn}_\infty=1$, which can be seen by checking on the dense subset $\ZZ\subseteq\ZZ_p$.
\end{example}
\begin{proposition}
	The functions $\left\{\binom xn\right\}_{n\ge0}$ form an orthonormal basis of $C^0(\ZZ_p,\QQ_p)$.
\end{proposition}
\begin{proof}
	We won't bother to write out the proof explicitly, but let's explain why we might expect this. Suppose we have an expansion
	\[f(x)=\sum_na_n(f)\binom xn\]
	already. Then one has a system of equations involving the values of $f$ on integers to solve for the $a_\bullet(f)$s.
\end{proof}

\end{document}
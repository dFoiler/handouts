% !TEX root = ../thesis.tex

\documentclass[../thesis.tex]{subfiles}

\begin{document}

\chapter{Absolute Hodge Classes} % \label{chap:abs-hodge}
In this chapter, we set up some theory surrounding absolute Hodge classes and state their important properties. Our exposition follows \cite{deligne-hodge}.

\section{Review of Cohomology} \label{sec:review-cohom}
In this section, we review all the cohomology theories we will need. Throughout, $X$ is a smooth projective variety, but the meaning of these words may change depending on the context.

\subsection{Betti Cohomology}
We begin with algebraic topology. Fix a smooth projective complex manifold $X$ of dimension $n$, which is a real manifold of dimension $2n$. In this setting, one is able to define sheaf cohomology.

% review of cohomology
% tate twists

\section{The Definition}
% definition of absolute hodge cycles
% construction works
\begin{theorem}[Deligne] \label{thm:hodge-to-abs-hodge}
	
\end{theorem}
\begin{theorem}[Principle B] \label{thm:principle-b}
	
\end{theorem}
\begin{proposition} \label{prop:construct-abs-hodge}
	Construction of absolute Hodge cycle. \cite[Proposition~7.1]{deligne-hodge}.
\end{proposition}

\section{A Category of Motives}
% maybe explain how we get a category of motives

\end{document}
% LTeX: enabled=false

\documentclass[openany]{book}
\usepackage[utf8]{inputenc}

\newcommand{\nirpdftitle}{Senior Thesis}
\usepackage{import}
\inputfrom{.}{nir}
% \usepackage{mathabx}

\renewcommand{\todo}[1][]{}

\pagestyle{contentpage}

\title{Sato--Tate Groups of Generic Superelliptic Curves}
\author{Nir Elber}
\date{Fall 2024}
\rhead{\textit{SATO--TATE GROUPS OF GENERIC CURVES}}

\begin{document}

\maketitle

\nirtableofcontents

\newpage

\setcounter{chapter}{-1}
\chapter{Introduction}

\epigraph{What we didn't do is make the construction at all usable in practice! This time we will remedy this.}
{---Kiran S. Kedlaya, \cite{kedlaya-cft}}

\section{Overview}
Over the past few decades, there has been a growing interest in understanding how the geometry of a space (such as a smooth projective variety) affects its arithmetic.

One example of these effects arises in the form of the Sato--Tate conjecture, which takes an abelian variety $A$ over $\QQ$ and predicts the distribution of the point-counts $\#A(\FF_p)$ (suitably interpreted) as the primes $p$ varies. Here, one finds that the ``geometric'' invariant $\op{End}_\CC(A)$ essentially determines the desired distribution. We refer to \cref{sec:sato-tate} for a more precise discussion, but approximately speaking, the point is that one expects a ``motivic monodromy group'' to control this distribution, and the motivic monodromy group can be computed either in a geometric situation over $\CC$ or understood via such point-counts in an arithmetic situation.

To be slightly more explicit, there are various monodromy groups at play: in the complex analytic situation, there is the Mumford--Tate group $\op{MT}(A)$, and in the $\ell$-adic situation, there is the $\ell$-adic monodromy group $G_\ell(A)$. There are conjectural relations between these, and these conjectures codify the interplay between geometry and arithmetic; for example, the Mumford--Tate conjecture predicts that $\op{MT}(A)_{\QQ_\ell}=G_\ell(A)^\circ$. Ultimately, to understand point-counts, one becomes interested in the groups $G_\ell(A)$, but this group is difficult to compute directly, so it is frequently profittable to compute $\op{MT}(A)$ instead and then use one of the aforementioned conjectures.

In this article, we are interested in the effect of so-called ``exceptional'' geometry on arithmetic, continuing the work of \cite{ggl-fermat}.
% Roughly speaking, we will be interested in computing various monodromy groups attached to an abelian variety $A$: in the complex analytic situation, there is the Mumford--Tate group $\op{MT}(A)$, and in the $\ell$-adic situation, there is the $\ell$-adic monodromy group $G_\ell(A)$.
The exceptional geometry we are interested in concerns exceptional Hodge classes, which are Hodge classes on $A$ (or a power of $A$) which are not generated by an endomorphism of $A$ or the polarization of $A$. The absence of such classes gives control of the geometry of $A$ and hence makes $\op{MT}(A)$ and $G_\ell(A)$ easy to compute. As another application, in the absence of exceptional classes, one knows the Hodge conjecture for all powers of $A$, so exceptional geometry is in some sense ``the enemy'' of proving the Hodge conjecture.

\subsection{Fermat Curves}
Roughly speaking, most abelian varieties do not support exceptional classes, so it requires some effort to find abelian varieties with exceptional classes in nature (and then prove and study their existence!). In \cite{ggl-fermat}, Gallese, Goodson, and Lombardo are able to control exceptional classes in the Jacobians of the ``Fermat'' hyperelliptic curves
\[y^2=x^N+1\]
as $N\ge1$ varies over positive integers. Namely, they are able to write down an algorithm which computes the groups $\op{MT}$ and $G_\ell$ for moderately sized $N$ (say, $N\le100$), and they are able to prove general results in certain cases (such as $N$ prime). It is still true that some $N$ fail to support exceptional classes, such as when $N$ is a prime, but composite $N$ frequently support exceptional geometry, which must be understood to execute the computation.

The present article can be considered a continuation of the work of \cite{ggl-fermat}. For example, the authrors there remark that their methods should be able to be used to compute $\op{MT}$ and $G_\ell$ for the Jacobians of quotients of the smooth projective Fermat curve
\[X_N\colon X^N+Y^N+Z^N=0,\]
which includes the hyperelliptic curves $y^2=x^N+1$ above. This is carried out in \cref{sec:fermat-galois-action}; we note that the main theorem is \Cref{thm:fermat-galois}, where we provide an explicit description of the Galois action on (absolute) Hodge classes in terms of Galois action on certain explicitly computed periods, but we will not give the statement in the introduction because it is somewhat technical.
\begin{remark}
	As an aside, we note that the authors of \cite{ggl-fermat} recourse to more general Fermat hypersurfaces
	\[X_0^N+X_1^N+\cdots+X_m^N=0.\]
	in order to understand powers of the Fermat curve $X_N$. This theory rests on somewhat technical algebraic geometry due to Deligne \cite[Section~7]{deligne-hodge}. In this article, we rebuild the thoery of \cite{ggl-fermat} while only handling powers of $X_N$ directly, allowing us to avoid Deligne's algebraic geometry. The key point is that a careful analysis of the K\"unneth isomorphism allows one to gain the same level of control on the Hodge classes of a power of $X_N$ as one would get with embedding in a Fermat hypersurface. This is carried out in \cref{subsec:classes-on-fermat-curve-power}.
\end{remark}
Having access to more general quotients allows us to see more geometry. To explain one example, we recall the definition of $G_\ell(A)$. Given an abelian variety $A$ defined over a number field $K$, one can use the Galois action on the Tate module $V_\ell A$ of $A$ to define a Galois representation
\[\rho_\ell\colon\op{Gal}(\ov K/K)\to\op{GL}(V_\ell A).\]
Here, $V_\ell A$ turns out to be a vector space ver $\QQ_\ell$ of dimension $2\dim A$. We then define $G_\ell(A)$ to be the smallest algebraic $\QQ_\ell$-subgroup containing the image of $\rho_\ell$. The Mumford--Tate conjecture explains that one expects to recover $G_\ell(A)^\circ$ from the complex geometry of $A$, so it becomes interesting to understand the quotient $G_\ell(A)/G_\ell(A)^\circ$, which we note is finite because $G_\ell(A)$ is an algebraic group. In light of the definition of $G_\ell(A)$, we see that we are interested in the pre-image $\rho_\ell^{-1}(G_\ell(A)^\circ)$; this needs to be a finite-index open subgroup of $\op{Gal}(\ov K/K)$, so there is a finite extension $K_A^{\mathrm{conn}}$ of $K$ such that $\rho_\ell(\sigma)\in G_\ell(A)^\circ$ if and only if $\sigma$ fixes $K_A^{\mathrm{conn}}$.

In \cite[Theorem~7.1.1]{ggl-fermat}, the authors find that their hyperelliptic curves $y^2=x^N+1$ all have $K_A^{\mathrm{conn}}$ to be a multiquadratic extension of $\QQ(\zeta_N)$, and they provide an algorithm to compute it. Further, they find that the prime-power case will always have $K_A^{\mathrm{conn}}=\QQ(\zeta_N)$. One can now ask if one can hope for such control for general quotients of the Fermat curve. Well, \cite[Theorem~7.15]{deligne-hodge} explains that the extension $K_A^{\mathrm{conn}}/\QQ(\zeta_N)$ should always be abelian. However, it turns out that one cannot hope for much more than this.
\begin{example}
	Using the techniques of \cref{sec:fermat-galois-action}, one can show that the Jacobian of the superelliptic curve
	\[y^9=x\left(x^2+1\right),\]
	which is a quotient of the Fermat curve $X^{18}+Y^{18}+Z^{18}=0$, has $K_A^{\mathrm{conn}}=\QQ(\zeta_{18},\sqrt[18]{432})$, which is a degree-$18$ cyclic extension of $\QQ(\zeta_{18})$.
\end{example}
\begin{example}
	The Jacobian of the previous example is not simple. At the cost of having slightly higher dimension, one can show something similar for the Jacobian of $y^{11}=x^2\left(x^2+1\right)$, but now this Jacobian is simple.
\end{example}
In \cref{sec:fermat-galois-action}, we work out the example curve $y^9=x^3+1$ in detail. Here, one does find exceptional classes, but we still have $K_A^{\mathrm{conn}}=\QQ(\zeta_9)$. In the current draft of this article, we do not work out the above two examples because it would require a somewhat lengthy discussion of algebraicity of products of the $\Gamma$-function which has not been included in this first version.

\subsection{Beyond CM}
One aspect of these Fermat curves is that they have so many automorphisms (given by multiplying $X$ or $Y$ by an $N$th root of unity) that their Jacobians have complex multiplication. Complex multiplication aides the computation in a few key ways: in this case, $\op{MT}(A)$ and $G_\ell(A)$ are both tori, thus making them much easier to control. For example, the Mumford--Tate conjecture is known in this case, and there exist algorithms to compute $\op{MT}(A)$ from certain combinatorial data attached to $A$.

As such, to the author's knowledge, the literature does not have an example computation of $G_\ell(A)$ when $A$ does not have complex multiplication and is not fully of Lefschetz type.\footnote{Roughly speaking, ``fully of Lefschetz type'' means that all Hodge classes on $A$ can be explained by endomorphisms and the polarization. In type III, it turns out that these classes do imply the existence of an exceptional class, which is the difference between not supporting exceptional cycles and being ``fully of Lefschetz type.''} In this article, we work out such an example. Admittedly, we do not go far from complex multiplication: where complex multiplication would require $\op{End}(A)\otimes_\ZZ\QQ$ to contain a CM field of dimension $2\dim A$, we work with certain abelian varieties $A$ such that $\op{End}(A)\otimes_\ZZ\QQ$ contains a CM field of dimension $\dim A$. Our limitations are rather technical, and we expect that one can do much better.

As an example difficulty, let's focus on computing $\op{MT}(A)$. Recall that $\op{MT}(A)$ is a connected reductive algebraic group defined over $\QQ$, so we can split up its computation into computing the derived subgroup $\op{MT}(A)^{\mathrm{der}}$ and the neutral component $Z(\op{MT}(A))^\circ$ of the torus. In \cref{sec:center}, we explain how the current arguments used to understand $\op{MT}(A)$ for $A$ with complex multiplication can be used to compute $Z(\op{MT}(A))^\circ$. To explain this result, we pick up some notation: set $F\coloneqq Z(\op{End}(A))$, and then one can diagonalize the action of $F$ on $V\coloneqq\mathrm H^1_{\mathrm B}(A(\CC),\CC)$ to produce a piece of combinatorial data called the ``signature'' $\Phi\colon\op{Hom}(F,\CC)\to\ZZ_{\ge0}$; for brevity, we will set $\Sigma_F\coloneqq\op{Hom}(F,\CC)$. It turns out that one can embed $Z(\op{MT}(A))^\circ$ into the torus $\mathrm T_F\coloneqq\op{Res}_{F/\QQ}\mathbb G_{m,F}$, and our first main result explains how to recover this subtorus.
\begin{restatable*}{corollary}{computezmt} \label{cor:compute-z-mt}
	Fix an abelian variety $A$ over $\CC$ such that $Z(\op{End}(A))$ equals a CM algebra $F$, and define $V\coloneqq\mathrm H^1_{\mathrm B}(A,\QQ)$. Let $\Phi\colon\Sigma_F\to\ZZ_{\ge0}$ be the signature defined in \Cref{lem:hodge-to-signature}. Then $Z(\op{MT}(V))^\circ\subseteq\mathrm T_F$ has cocharacter group equal to the smallest saturated Galois submodule of $\mathrm X_*(\mathrm T_F)=\ZZ[\Sigma_F^\lor]$ containing
	\[\sum_{\sigma\in\Sigma_F}\Phi(\sigma)\sigma^\lor.\]
\end{restatable*}
\begin{remark}
	In fact, a careful reading of the arguments in \cref{sec:center} reveal that we are actually able to compute an explicit power of $Z(\op{MT}(A))$, which technically contains more information. For example, one could provide a sufficient condition for $Z(\op{MT}(A))$ being disconnected.
\end{remark}
It remains to compute $\op{MT}(A)^{\mathrm{der}}$. Under certain simplifying hypotheses given above, we work this out in \Cref{prop:mtc-reldim-2}, which we restate below for convenience. Here $\op L(A)$ is the Lefschetz group, which is intuitively what $\op{MT}(A)$ would be in the absence of exceptional classes.
\begin{restatable*}{proposition}{mtcreldimtwo} \label{prop:mtc-reldim-2}
	Fix a geometrically simple abelian variety $A$ over a number field $K$. Suppose that $F=Z(\op{End}_{\ov K}(A)$ equals a CM field such that $\dim A=\dim F$. Letting $\Phi$ be the corresponding signature, we further suppose that $\Phi(\sigma)=1$ for exactly two $\sigma\in\Sigma_F$. Then we show the Mumford--Tate conjecture holds for $A$, and
	\[\op{MT}(A)^{\mathrm{der}}=\op L(A)^{\mathrm{der}}.\]
\end{restatable*}
\noindent The argument proving \Cref{prop:mtc-reldim-2} achieves something slightly stronger, but it is technical to state and not required for our application. In short, the idea of the proof is to upgrade the fact that the real Lie groups $\op{SU}(2,0)$ and $\op{SU}(1,1)$ are not isomorphic using the Galois action.

Now that we understand $\op{MT}(A)$, we would like to upgrade this to an understanding of $G_\ell(A)$. After the Mumford--Tate conjecture, we (roughly speaking) need to understand the quotient $G_\ell(A)/G_\ell(A)^\circ$, whch \cref{subsec:compute-gl-from-gl0} explains that this amounts to computing the Galois action on certain ``Tate classes.'' Thus, the trick is to not look at a particular Galois representation $\rho_\ell$ but instead a family of them. We can engineer everything so that generic members of the family satisfy the properties needed for the rest of the present subsection to go through. Then our last trick is ensure that some special members of the family are quotients of a Fermat curve, where we know the Galois action! In this way, we can ``transport'' the understanding of the Galois action afforded by the Fermat curves to a generic curve. Here is the toy result we are able to prove.
\begin{restatable*}{theorem}{genericfullst}
	For given $\lambda\in\QQ\setminus\{0,1\}$, define $A$ to be the Jacobian of the proper curve $\widetilde C$ with affine chart $y^9=\left(x^2+x+1\right)(x-\lambda)$. Suppose that $A$ does not have complex multiplication. Then we show $K_A^{\mathrm{conn}}=\QQ(\zeta_9)$, and we compute $\op{ST}(A)$.
\end{restatable*}

\section{Odds and Ends}
In this section, we explain some existential properties of this article.

\subsection{What Is in This Article}
Let's take a moment to explain the layout. In \cref{chap:hodge}, we review all the Hodge theory we will need. Notably, in \cref{sec:center}, we explain the algorithm used to compute the neutral component of the Mumford--Tate group. This chapter ends by reviewing cohomology and discussing absolute Hodge classes.

In \cref{chap:av}, we review everything we need to know about abelian varieties. The ground of the theory is discussed in \cref{sec:av-def}, and then we move on to more specialized topics. For example, \cref{sec:l-adic} discusses the $\ell$-adic representation (and the Tate module in particular), and then we explain how the Galois action on Tate classes is used to compute $G_\ell(A)$ from $G_\ell(A)^\circ$.

Lastly, in \cref{chap:fermat}, we apply the built theory to Fermat curves and their quotients. In particular, we explain how to compute the Galois action on Tate classes by passing to absolute Hodge classes. We then go on to compute the connected monodromy field and $G_\ell(A)$ in a few cases.

\subsection{What Is Not in This Article}
There is a large supply of topics which are not included in the first draft of this article but really should be included in a second. We list them in rough order of importance and provide some of their applications.
\begin{enumerate}
	\item Algorithms to compute products of $\Gamma$.
	\begin{enumerate}
		\item We would be able to compute the connected monodromy fields and monodromy groups of the curves $y^9=x\left(x^2+1\right)$ and $y^{11}=x\left(x^2+1\right)$. An argument with twisting would then allow us to execute the same computations for $y^9=x(x-1)(x-\lambda)$ and $y^{11}=x(x-1)(x-\lambda)$ for generic $\lambda\in\QQ$. The obstruction here is that the period computations are slightly too large to be done by hand.
		\item We would also be able to compute the connected monodromy field for more general Fermat curves and hypersurfaces. It is expected that the connected monodromy field for the Fermat curves $X^p+Y^p+Z^p=0$ is $\QQ(\zeta_p)$. In the presence of these algorithms, one would be able to place strong upper bounds on the connected monodromy field.
	\end{enumerate}
	\item A discussion of Tannakian formalism.
	\begin{enumerate}
		\item Tannakian formalism provides uniform definitions of our monodromy groups $\op{MT}(A)$ and $G_\ell(A)$.
		\item We would be able to discuss pure motives and especially abelian motives using the existing discussion of absolute Hodge classes. For example, this would allow us to include the definition of the motivic Galois group and explain its importance to (for example) the Algebraic Sato--Tate conjecture.
	\end{enumerate}
	\item A discussion of rigid cohomology and Kedlaya's algorithm to compute Frobenius matrices.
	\begin{enumerate}
		\item This would allow us to computationally verify \Cref{thm:fermat-galois}.
		\item This would allow us to form $p$-adic analogues of many parts of our computation, such as \Cref{prop:mtc-reldim-2}.
	\end{enumerate}
\end{enumerate}

% \section{Notation}
% Elements.
% \begin{itemize}
% 	\item $V$ and $W$ are vector spaces, frequently $\QQ$-Hodge structures.
% 	\item $\QQ(n)$ is the Tate twist.
% 	\item $\mathrm H_{\mathrm B}^\bullet$ is Betti cohomology, $\mathrm H_{\mathrm{dR}}^\bullet$ is de Rham cohomology, and $\mathrm H_{\mathrm{\acute et}}^\bullet$ is \'etale cohomology.
% 	\item $\mf g$ and $\mf h$ are Lie algebras.
% \end{itemize}
% Groups.
% \begin{itemize}
% 	\item If $V$ is a $\QQ$-Hodge structure, then $\op{MT}(V)$ and $\op{Hg}(V)$ are the Mumford--Tate and Hodge groups, respectively.
% 	\item $\mathbb S$ is the Deligne torus $\op{Res}_{\CC/\RR}\mathbb G_{m,\RR}$.
% 	\item Given a number field $F$, we define the torus $\mathrm T_F\coloneqq\op{Res}_{F/\QQ}\mathbb G_{m,\QQ}$.
% 	\item Given a CM or totally real number field $F$, we define the subtorus $\mathrm U_F\subseteq\mathrm T_F$ by
% 	\[\mathrm U_F\coloneqq\{x\in\mathrm T_F:x\ov x=1\},\]
% 	where $\ov x$ is complex conjugation when $F$ is CM and the identity when $F$ is totally real.
% 	\item Given an algebraic group $G$, $G^\circ$ denotes the connected component, $Z(G)$ denotes its center, and $G^{\mathrm{der}}$ denotes the derived subgroup.
% \end{itemize}
% Categories.
% \begin{itemize}
% 	\item For a field $F$, $\op{Vec}_F$ is the category of vector spaces over $F$.
% 	\item $\op{HS}_\QQ$ is the category of $\QQ$-Hodge structures.
% \end{itemize}
% Organization is thematic. As such, dependencies are not always strictly linear, though we do our best to not require any content from a later chapter; at times, it is motivational to mention some content from a later chapter, but this is kept to a minimum. Additionally, some omitted proofs may require content from later chapters even if not mentioned.
% notation

\subsection{Acknowledgements}
The author is indebted to many people for the existence of this article. Most importantly, the author is extremely grateful to his advisor Yunqing Tang for many, many patient and enlightening conversations, for example by suggesting the key ideas that went into the main ideas. The author would also like to thank Sug Woo Shin for the opportunity to speak about the Sato--Tate conjecture and helpful conversation to this end. Additionally, the author thanks Hannah Larson for help understanding Hurtwitz spaces. None of this mathematics would have been possible without their constant encouragement.

This article would also not be possible without the extremely welcoming mathematics community at the University of California, Berkeley. In particular, the author would like to thank Jad Damaj, Sophie McCormick, and Zain Shields for diverting conversations, and the author would like to thank Sam Goldberg and Justin Wu for productive conversations.

There are also ennumerable people who have helped the author get to where he is today who were not directly involved in the creation of this article. To name just a few, the author thanks his parents Ron Elber and Virginia Yip for believing in his blooming academic carreer. And most importantly, the author is forever in debt to Hui Sun for constant support and companionship. Without her, the author would be without soul.

\subfile{chaps/hodge}

% \subfile{chaps/abshodge}

\subfile{chaps/av}

\subfile{chaps/satotate}

\subfile{chaps/fermat}

% \chapter{Families of Curves}
% cadoret--aoki on finiteness of cm points
% work out y^9 = x(x-1)(x-lambda)
% clutching
% work out y^11 = x(x-1)(x-lamda)^2

\nirprintbib
\nirprintindex

\end{document}

% y^11 = x^2(x^2-1) is a quotient of x^22 + y^22 = 1: indeed, y^11 = x(x-1) is a quotient of x^11 + y^11 = 1, so we should be able to square the relevant variables
% Jac(y^11 = x(x-1)(x-t)^2) has endomorphism algebra of type iv, relative dimension 2, so any larger endomorphism algebra => CM; conversely, CM of course makes the endomorphism algebra larger. the point is that non-CM is equivalent to having the generic endomorphism algebra.

% compute action on abs hodge cycles on the crystalline = de Rham site
% technically, one needs to match up the bases, which is very possible in the CM case, but it becomes harder in general
% in theory, the sort of classes I am looking at should not actually depend on the choice of basis ...

% - heights on A_g => "most avs have end = Z" by heights
% - => counting special points

% is there a "motivic" way to see the l-independence in the motivic galois group

% field of definition of points on a shimura variety?
% it seems hard to count "generic" AVs (faltings); only heights of special points appear understood
% idea: compare the number of special points on various special subvarieties to produce a sort of density result (heuristically, treat special points as a thin family of all points)
% is equidistribution of special points already known? one has something for quaternionic maybe?
% the result over finite fields may be of some interest? (it should come down to some form of langlands--kottwitz?)

% what was langlands's original idea for honda--tate theory?
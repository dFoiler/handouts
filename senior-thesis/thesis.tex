% LTeX: enabled=false

\documentclass[openany]{book}
\usepackage[utf8]{inputenc}

\newcommand{\nirpdftitle}{Senior Thesis}
\usepackage{import}
\inputfrom{.}{nir}
% \usepackage{mathabx}

\pagestyle{contentpage}

\title{Sato--Tate Groups of Generic Superelliptic Curves}
\author{Nir Elber}
\date{Fall 2024}
\rhead{\textit{SATO--TATE GROUPS OF GENERIC CURVES}}

\begin{document}

\maketitle

\nirtableofcontents

\newpage

\setcounter{chapter}{-1}
\chapter{Introduction}
\section{Notation}
Elements.
\begin{itemize}
	\item $V$ and $W$ are vector spaces, frequently $\QQ$-Hodge structures.
	\item $\QQ(n)$ is the Tate twist.
	\item $\mathrm H_{\mathrm B}^\bullet$ is Betti cohomology, $\mathrm H_{\mathrm{dR}}^\bullet$ is de Rham cohomology, and $\mathrm H_{\mathrm{\acute et}}^\bullet$ is \'etale cohomology.
	\item $\mf g$ and $\mf h$ are Lie algebras.
\end{itemize}
Groups.
\begin{itemize}
	\item If $V$ is a $\QQ$-Hodge structure, then $\op{MT}(V)$ and $\op{Hg}(V)$ are the Mumford--Tate and Hodge groups, respectively.
	\item $\mathbb S$ is the Deligne torus $\op{Res}_{\CC/\RR}\mathbb G_{m,\RR}$.
	\item Given a number field $F$, we define the torus $\mathrm T_F\coloneqq\op{Res}_{F/\QQ}\mathbb G_{m,\QQ}$.
	\item Given a CM or totally real number field $F$, we define the subtorus $\mathrm U_F\subseteq\mathrm T_F$ by
	\[\mathrm U_F\coloneqq\{x\in\mathrm T_F:x\ov x=1\},\]
	where $\ov x$ is complex conjugation when $F$ is CM and the identity when $F$ is totally real.
	\item Given an algebraic group $G$, $G^\circ$ denotes the connected component, $Z(G)$ denotes its center, and $G^{\mathrm{der}}$ denotes the derived subgroup.
\end{itemize}
Categories.
\begin{itemize}
	\item For a field $F$, $\op{Vec}_F$ is the category of vector spaces over $F$.
	\item $\op{HS}_\QQ$ is the category of $\QQ$-Hodge structures.
\end{itemize}
Organization is thematic. As such, dependencies are not always strictly linear, though we do our best to not require any content from a later chapter; at times, it is motivational to mention some content from a later chapter, but this is kept to a minimum. Additionally, some omitted proofs may require content from later chapters even if not mentioned.
% notation

\subfile{chaps/hodge}

\subfile{chaps/abshodge}

\subfile{chaps/av}

% \subfile{chaps/satotate}

\subfile{chaps/fermat}

\chapter{Families of Curves}
% cadoret--aoki on finiteness of cm points
% work out y^9 = x(x-1)(x-lambda)
% clutching
% work out y^11 = x(x-1)(x-lamda)^2

\nirprintbib
\nirprintindex

\end{document}

% y^11 = x^2(x^2-1) is a quotient of x^22 + y^22 = 1: indeed, y^11 = x(x-1) is a quotient of x^11 + y^11 = 1, so we should be able to square the relevant variables
% Jac(y^11 = x(x-1)(x-t)^2) has endomorphism algebra of type iv, relative dimension 2, so any larger endomorphism algebra => CM; conversely, CM of course makes the endomorphism algebra larger. the point is that non-CM is equivalent to having the generic endomorphism algebra.

% compute action on abs hodge cycles on the crystalline = de Rham site
% technically, one needs to match up the bases, which is very possible in the CM case, but it becomes harder in general
% in theory, the sort of classes I am looking at should not actually depend on the choice of basis ...